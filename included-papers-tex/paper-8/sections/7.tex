\subsection{Conclusions and Future Work}

This paper presents Questgram [Qg], a quest generation tool with a mixed-initiative approach integrated into the Evolutionary Dungeon Designer. Questgram lets a human designer co-create an overarching quest that fits in a dungeon level, as the dungeon is being developed in the designer. Both map and quest are designed in parallel and with the suggestions provided by EDD. Quests make use of Doran and Parberry's quest structure as production rules in a grammar so that all quests are well-formed with respect to the grammar and the level landscape. We show results from a two-fold evaluation, an expressive range analysis, and a user study. 

The expressive range analysis shows several dominant quest actions and structures, though all types of actions could be generated at a wide range of quest lengths. The mixed-initiative approach was positively met by the user study participants, along with the manual creation. However, automatic creation and automatic suggestions received a mixed response, mainly because of a random placement of position on the actions and the use of abstract quest actions. The tool's overall response was positive, and a majority of the participants reported increased creativity while using the tool. Many participants expressed its usability to gain inspiration, as a solution to inspiration blockages, and as a resource-efficient tool for game developers to use. None of the testers noticed the dominance of some actions detected in the expressive range analysis.

This inspirational use points towards the need to explore other fundamental and more useful ways to establish effective MI-CC workflows where systems can adapt and be effectively employed and used. For instance, some interesting future paths would be to explore the creation of more adaptive collaboration that considers the designer's style or to give more autonomy to the AI to have more participation in the creative process and its effects. Within this, one interesting area is the one of eXplainable AI for Designers~\citepeighth{p8Zhu2018-XAIDesignersMICC} where the goal is to achieve system explainability to improve the collaboration and interaction between human and AIs.

 %how to give more autonomy to the AI system and increasing the trustworthiness on the system how to give more autonomy to the AI system 
%improving MI-CC systems


%For instance, Zhu et al.~\citepeighth{p8Zhu2018-XAIDesignersMICC} proposes 


%For instance, different initiative levels where the AI has more participation, varying autonomy where   Zhu et al.~\citepeighth{p8Zhu2018-XAIDesignersMICC}

%used  than MI-CC points towards a 
%Likewise, this also shows the need for exploring other fundamental and useful ways to establish human-AI collaboration, specifically MI-CC tools that effectively enables the proactive collaboration and creation. Some preliminary study showed that when adding NPC roles and adapting Questgram to it, helps the users contextualize and create quests more alike common quests in RPG. Yet, users used even less the suggested actions as it was seem as "help for novices," rather than a way to create quests between both. This points towards the need to explore other fundamental and more useful ways to establish MI-CC workflows where systems can adapt and create ~\citepeighth{p8Zhu2018-XAIDesignersMICC}. , but reducing even more the use of the suggested actions, 


This research sets the first step toward intertwined story and level mixed-initiative generation on EDD, and future work could be to incorporate quest elements as input in the level generation process so that quests and levels reciprocally influence their generative processes~\citepeighth{p8Liapis2019-OrchestratingGames}. Furthermore, adding semantic evaluation on the generated suggestions would allow Questgram to generate interconnected quests that make sense as subsequent parts of an overarching story plot involving game elements, as well as adding a natural-language generation layer to enhance quests with the automatic generation of detailed descriptions and narratives. Another interesting future research would be to create designer models to adapt quest suggestions to the designer's particular style~\citepeighth{p8Liapis2013-designerModel,p8alvarez2020-designerpersonas}. Finally, more extensive user studies will be conducted to analyze further the tool's usability and intuitiveness. 



% Additionally,  Larger user studies would be conducted to analyze the usability and intuitiveness of the tool interface continuously.


% map locations, NPCs, items, and other game elements. Another interesting future line would be to add a natural-language generation layer to enhance quests with the automatic generation of detailed descriptions and narratives.  




