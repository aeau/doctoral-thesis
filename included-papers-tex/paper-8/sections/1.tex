\subsection{Introduction}

Defining quests and related concepts have been the focus of considerable research, where quests have been related to tasks, challenges, rewards, or as a storytelling device adding nuances to what a quest is~\cite{p8Doran2011-questsMMORPGs,Breault2021-CONANQuestGen,Trenton2010-questpatterns,yu2020quest}. Most games have some quest driving the game's plot and gameplay. Adventure games, action-adventure games, and role-playing games (RPG) are among the main genres using quests~\cite{p812-howard2008quests}, where most of these genres take place or containing some type of dungeon such as \emph{The Legend of Zelda}, \emph{Skyrim}, or \emph{The Binding of Isaac}. Dungeons as game content can be defined as a single level or set of levels containing enemies, treasures, hidden passages, puzzles, decorations, or Non-playable characters (NPC), thus creating space that allows the player to explore the unknown areas~\cite{p8Dahlskog2015-patternsDungeonsGens}. Dungeons are a popular level design, especially within PCG~\cite{p8shaker_procedural_2016,Yannakakis2018,Liapis2020-pcgWorkshop}, where it has been present ever since the 1970s in games such as \emph{Rogue}.

The increasing usage of Procedural Content Generation (PCG) in both research and industry~\cite{p8Liapis2020-pcgWorkshop,1-Lavender2016TheZD} has shown successful results regarding the efficiency of the game development process~\cite{p8Pedersen2010-modelingPlayerExp} but also to generate a big amount of variation in games, increasing their replayability~\cite{p8Smith2014-understandingPCG}. PCG can generate game content quickly such as missions and levels~\cite{p8dormans2011generating}, content adapted to players~\cite{p8hastings_evolving_2009}, or data-driven generation~\cite{p8Green2018,summerville2018procedural}. Narrative and quest generation as objectives and goals has also been the focus of PCG~\cite{p8Mason2019-LumeStoryGeneration,flodtol2020-WIPMakeSenseDungs,ammanabrolu2019-towardQuestGeneration}, where the aim has been to capture and use quest concepts and patterns to approach the generation, such as the work by Trenton et al.~\cite{p8Trenton2010-questpatterns}, Kreminski and Wardrip-fruin~\cite{p8Kreminski2018-SketchingStorylets}, Smith et al.~\cite{p8Smith2011-situatingQuests} or Doran and Parberry~\cite{p8Doran2011-questsMMORPGs}. 

Nevertheless, much of the content is still best made by humans, especially when subjective evaluations are needed~\cite{p8shaker_procedural_2016}. To cope with this, one could use a mixed-initiative approach.
% Mixed-initiative focuses on the human-machine collaboration where both take part in the solution~\cite{p8novick97-mixedInit}. 
Mixed-initiative Co-creativity (MI-CC) was introduced by Yannakakis et al., where both human and AI co-create and -design some game facet with a proactive initiative~\cite{p8yannakakis2014micc}. MI-CC has been explored mainly for level design in tools such as the Sentient Sketchbook~\cite{p8Liapis2013-sentientsketchbook}, Tanagra~\cite{p8smith_tanagra:_2011}, Morai Maker~\cite{p8guzdial-lvldsg-aiide-2018}, or the Evolutionary Dungeon Designer (EDD)~\cite{p8Alvarez2020-ICMAPE}. EDD lets the user design interconnected room in a dungeon while receiving room suggestions adapted to their creation. Questgram is implemented in EDD, taking advantage of its level design capabilities and mixed-initiative approach.

% EDD is a mixed-initiative level design tool that lets the user design interconnected room in a dungeon while a MAP-Elites algorithm variant, IC MAP-Elites, generate and suggests adapted rooms. Questgram is implemented in EDD, taking advantage of its level design capabilities and mixed-initiative approach.

This research takes the quest analysis, quest patterns, and quest grammars identified by Doran and Parberry~\cite{p8Doran2011-questsMMORPGs}, implements it in EDD, adapts it in a mixed-initiative approach for the creation of quest sequences, and extends it to work with level generation. The designer can create quests by adding manually available quest actions in a quest sequence and receive suggestions from the quest grammar that they might use to continue the quest, replace some part of the quest, or get inspiration to continue their quest. The available quest actions are related to the current dungeon layout. If modifying such dungeon renders invalid current quest parts, the designer is prompted to fix the quest manually or using actions suggested by the system. The system was evaluated quantitatively by assessing the diversity and incidence of quest actions, and qualitatively through a user study evaluating the experience, usability and suitability of the system.




% A world without a narrative to create meaning and make sense of the events that happen or a narrative without a world or space to take place creates a pointless system in most cases. This aligns with the view of Aarseth, who links the space to the quest in games being both level design and quests dependant on each other~\cite{p8aarseth2005hunt}. A similar point is presented by Ashmore and Nitsche in relation to the games' interactivity as they discuss that a generated level without depth and context lacks interest for the final user~\cite{p830-ashmore2007-questGeneratedWorld}. Conveniently, DEHn and lebowitz and kybartas and bidarra

% Narrative and quest generation has also been the focus of PCG, where it is in the most parts linked to the generation of levels

% Quests and it's definition has been the focus...
% Defining quests and related concepts have been the focus of multiple research, where quests have been related to tasks, challenges, rewards, or as a storytelling device adding nuances to what a quest is~\cite{p8Doran2011-questsMMORPGs,31-breault2018let,Trenton2010-questpatterns,yu2020quest}. Most games have some type of quest driving the game's plot and gameplay. Adventure games, action-adventure games, and role-playing games (RPG) are among the main genres using quests~\cite{p812-howard2008quests}, where most of these genres take place or containing some type of dungeon such as \emph{The Legend of Zelda}, \emph{Skyrim}, or \emph{The Binding of Isaac}. Dungeons as game content can be defined as a single level or set of levels that are set in an underground complex and is connected through an overworld with cities or a wilderness. Dungeons can contain enemies, treasures, hidden passages, puzzles, decorations, and Non-playable characters (NPC), thus creating space that allows the player to explore the unknown areas~\cite{p8Dahlskog2015-patternsDungeonsGens}. Dungeons are a popular level design, especially within PCG~\cite{p8shaker_procedural_2016,Yannakakis2018,Liapis2020-pcgWorkshop}, where it has been present in popular games ever since the 1970s with games such as \emph{Rogue}.


% Quests have been defined by Horward as "... a conceptual bridge that can help to join together many two-part or binary pairs [...] these include game and narrative, gaming and literature, technology and mythology and meaning and action~\cite{p812-howard2008quests}”. Howard argues that quests unify both meaning and action, where meaning hails from strategic actions with thematic, narrative and personal implications, and actions being those that are meaningful for the player on the level of ideas, personal ambitions, benefits to society, and spiritual authenticity~\cite{p812-howard2008quests}. Aarseth describes quests as concrete and attainable goals, and such can be hierarchic, concurrent, serial, or a combination of those. Further, Aarseth describes three basic quest types~\emph{Time-},~\emph{Place-}, and~\emph{Objective-oriented}, which can also be combined to form seven different quest types~\cite{p8aarseth2005hunt}. 


% Similar definitions have been introduced relating quests to tasks, challenges, rewards, and as a storytelling mechanic have been introduced before, which adds more nuance~\cite{p89-Doran2011-questsMMORPGs,31-breault2018let,Trenton2010-questpatterns,yu2020quest}, . 


% Similarly, Doran and Parberry~\cite{p89-Doran2011-questsMMORPGs}, Trenton et al.~\cite{p8Trenton2010-questpatterns}, Breault et al.~\cite{p831-breault2018let} ~\cite{p817-SoaresdeLima_HierarchicalgenQuests}


% Narrative and quest generation has been the focus of recent research in PCG, where the aim has been to capture quest concepts and quest patterns to approach quest generation such as the work by Trenton et al.~\cite{p8Trenton2010-questpatterns}, Kreminski et al.~\cite{p8Kreminski2018-SketchingStorylets} or Doran and Parberry~\cite{p89-Doran2011-questsMMORPGs}. 


% On a bigger perspective, research on quest generation using PCG has been conducted by Parberry and Doran~\cite{p89-Doran2011-questsMMORPGs}, Braualt et al~\cite{p831-breault2018let}, and Dormans ~\cite{p810-dormans2011generating}. This paper researches quest generation in a mixed-initiative procedural content generator.
%but none have been through mixed-initiative, thus making this research a starting point for a new research area to be explored within both PCG, mixed-initiative and narratives. 
% As one challenge of PCG in games is that it could ultimately lead to uninspiring and uncreative content~\cite{p827-AraujoGameWorldsCreativity}, our research contributes to shifting from an automatic procedurally generated content to mixed authorship with the creative collaboration in focus.

% However, many parts are still best made by humans, especially when subjective evaluations are needed~\cite{p84-shaker_procedural_2016}. To cope with this, one could use a mixed-initiative approach. Mixed-initiative focuses on the human-machine collaboration where both take part in the solution~\cite{p85-novick97-mixedInit}. Mixed-initiative Co-creativity (MI-CC) was introduced by Yannakakis et al. where both human and AI co-create and -design some game facet with a proactive initiative~\cite{p86-yannakakis2014micc}.


% EDD lets the user manually design interconnected rooms with a set of generic tiles and uses a MAP-Elites algorithm variant, IC MAP-Elites, to generate and suggest levels to the designer. Questgram is implemented in EDD making use of its level design capabilities and mixed-initiative approach.



% We implement our tool, the Questgram, in EDD, which lets the user manually design interconnected rooms with a set of generic tiles. EDD primary uses a variant of the MAP-Elites algorithm, IC MAP-Elites, to generate a vast amount of levels adapted to the designer's current design, evaluated by its composition, and several designer selected feature dimensions, and suggested to the designer to continue or replace their creation.

% forfor multiple genres for the creation of strategy games~\cite{p8Liapis2013-sentientsketchbook}, the c

% MI-CC has been explored 

% Examples of previous game development tools using mixed-initiative co-creation (MI-CC) are Sentiment Sketchbook~\cite{p85-novick97-mixedInit}, Tangara~\cite{p87-smith_tanagra:_2011} and the evolutionary dungeon designer (EDD)~\cite{p88-alvarez2019empowering}. EDD is a mixed-initiative dungeon designer for making dungeons for adventure games. The tool lets the user manually design rooms with enemies, treasures, chambers and an overall world structure that connects the different rooms. For the computing party of MI-CC, EDD uses evolutionary computations to procedurally generate content suggestions. The two parties collaborate and the evolutionary algorithms provide the user with alternatives, based on symmetry or similarity criteria. Currently, EDD does not have any narrative or story, which limits EDD’s potential, functionality and creative usage for the user. Narrative and quests are important to bring meaning into a game for the player as presented by Howard~\cite{p812-howard2008quests}. 



% The increasing usage of Procedural Content Generation (PCG) in both research and industry~\cite{p81-Lavender2016TheZD} has shown successful results regarding the efficiency of the game development process~\cite{p82-Pedersen2010-modelingPlayerExp} but also to generate a big amount of variation in games, increasing their replayability. endless variations of a game, therefore making games “infinitely” replayable~\cite{p83-Shaker2012EvolvingEvolution}. For example animation and the environment is taking a large part of a development budget~\cite{p82-Pedersen2010-modelingPlayerExp}, which PCG can produce solutions efficiently. PCG can generate game content quickly, however some parts are still best made by humans~\cite{p84-shaker_procedural_2016}. One solution is using a mixed-initiative approach. Which focuses on taking turns that are negotiated rather than determined by a single party regarding the modality of interaction~\cite{p85-novick97-mixedInit}, in the case of game development, where the computer and human co-creates a solution to a problem. However, the two actors' contributions do not need to be the same~\cite{p86-yannakakis2014micc}. 

% Examples of previous game development tools using mixed-initiative co-creation (MI-CC) are Sentiment Sketchbook~\cite{p85-novick97-mixedInit}, Tangara~\cite{p87-smith_tanagra:_2011} and the evolutionary dungeon designer (EDD)~\cite{p88-alvarez2019empowering}. EDD is a mixed-initiative dungeon designer for making dungeons for adventure games. The tool lets the user manually design rooms with enemies, treasures, chambers and an overall world structure that connects the different rooms. For the computing party of MI-CC, EDD uses evolutionary computations to procedurally generate content suggestions. The two parties collaborate and the evolutionary algorithms provide the user with alternatives, based on symmetry or similarity criteria. Currently, EDD does not have any narrative or story, which limits EDD’s potential, functionality and creative usage for the user. Narrative and quests are important to bring meaning into a game for the player as presented by Howard~\cite{p812-howard2008quests}. 

% On a bigger perspective, research on quest generation using PCG has been conducted by Parberry and Doran~\cite{p89-Doran2011-questsMMORPGs}, Braualt et al~\cite{p831-breault2018let}, and Dormans ~\cite{p810-dormans2011generating}. This paper researches quest generation in a mixed-initiative procedural content generator.
% %but none have been through mixed-initiative, thus making this research a starting point for a new research area to be explored within both PCG, mixed-initiative and narratives. 
% As one challenge of PCG in games is that it could ultimately lead to uninspiring and uncreative content~\cite{p827-AraujoGameWorldsCreativity}, our research contributes to shifting from an automatic procedurally generated content to mixed authorship with the creative collaboration in focus. 

% This links described by Howard

% The genres where quests appear are mainly adventure games, action-adventure games, and role-playing games (RPG)~\cite{p812-howard2008quests}. Both RPGs and adventure games have a rich storyline, detailed characters, and involve exploration~\cite{p813-aarseth2005hunt}. Dungeons as game content can be defined as a single level or set of levels that are set in an underground complex and is connected through an overworld with cities or a wilderness~\cite{p814-Dahlskog2015-patternsDungeonsGens}. The cities can act as a replenishment zone where the player can do trades using found items or item upgrades. Dungeons can contain enemies, treasures, hidden passages, puzzles~\cite{p814-Dahlskog2015-patternsDungeonsGens}, decorations, and Non-playable characters (NPC)~\cite{p84-shaker_procedural_2016}, thus creating space that allows the player to explore the unknown areas~\cite{p814-Dahlskog2015-patternsDungeonsGens}. Dungeons are a popular level design, especially within PCG~\cite{p84-shaker_procedural_2016,13-aarseth2005hunt,15-Yannakakis2018}, where it has been present in popular games ever since the 1970s~\cite{p816-Dahlskog2015-patternsDungeonsGens}. 

% Moreover, Aarseth defines quests as “a game that depends on mere movement from position A to position B"~\cite{p813-aarseth2005hunt} and Howard defines it as “A quest is a middle term, a conceptual bridge that can help to join together many two-part or binary pairs [..] these include game and narrative, gaming and literature, technology and mythology and meaning and action”~\cite{p812-howard2008quests}. This link between quest narrative, quests, and quest games has been proposed by Howard~\cite{p812-howard2008quests}, arguing that quests unify both meaning and action. The meaning hails from strategic actions with thematic, narrative and personal implications, and actions being those that are meaningful for the player on the level of ideas, personal ambitions, benefits to society, and spiritual authenticity. Quests can be linked together like a chain to advance in the game’s story further~\cite{p816-Dahlskog2015-patternsDungeonsGens} and give structure by limiting the player’s available choice through providing access to certain areas only in a specific order(s)~\cite{p816-Dahlskog2015-patternsDungeonsGens}, thus making the game designer take control of the players’ agenda~\cite{p813-aarseth2005hunt}. Besides being a fundamental element for narrative progression~\cite{p817-SoaresdeLima_HierarchicalgenQuests}, quest affects the amount of space needed for a game’s landscape~\cite{p810-dormans2011generating,13-aarseth2005hunt}. Aarseth describes three basic quest types~\emph{Time-},~\emph{Place-}, and~\emph{Objective-oriented}, which can also be combined to form seven different quest types.

% While many of the tools 

% This research takes the quest analysis, quest patterns, and quest grammars identified by Doran and Parberry~\cite{p8Doran2011-questsMMORPGs}, implements it in EDD, adapts it in a mixed-initiative approach for the creation of quest sequences, and extends it to work with level generation. The designer is able to create quests by adding manually available quest actions in a quest sequence, and at the same time they get suggestions from the quest grammar that they might use to continue the quest, replace some part of the quest, or get inspiration to continue their quest. The available quest actions are related to the current dungeon layout. If modifying such dungeon renders invalid current quest parts, the designer is prompted to fix the quest, either manually or by using actions suggested by the system. Therefore, we also present a simple but clear step towards intertwining level design and narrative facets with the aim of holistic PCG~\cite{p8Liapis2019-OrchestratingGames}. The system was evaluated through mixed-methods. First, it was evaluated quantitatively to assess the diversity and incidence of quest actions in multiple quest sequence generation in a sample dungeon. Second, a user study evaluated the usability, it was preliminary evaluated through a user study, where users used the tool in different modes and assessing its function, usability, and suitability.

% present a step towards holistic PCG~\cite{p8Liapis2019-OrchestratingGames} by intertwining level design  and narrative 

% Our quest tool is implemented in EDD, a research framework for mixed-initiative content generation with a focus on level design.

%Discuss about the evaluation
%Introduce what the user can do, and how it is linked with the level part!


% implements it in 

% and extends it into a mixed-initiative approach

% This paper takes previous research by Parberry and Doran~\cite{p89-Doran2011-questsMMORPGs} on classification of RPG quests, and uses it as the basis for integrating a mixed-initiative narrative generator on EDD. This approach does not only generate quests in a completely automatized fashion, as in~\cite{p89-Doran2011-questsMMORPGs} and~\cite{p831-breault2018let}, but lets users participate in the creative process by adding specific quest actions, such as “GOTO”, ”KILL”, “STEAL”~\cite{p89-Doran2011-questsMMORPGs}, that can be extended upon user request with automatically generated actions. These procedurally generated actions use grammars to articulate the action generation with production rules, inspired by both Dorman’s~\cite{p820-Dormans_AdventuresinLvlDes} work on action-adventure games and Parberry and Doran~\cite{p89-Doran2011-questsMMORPGs}.  The generated quests will display the different series of quests available to the user in a list. The tool scans the room, thus eliminating quests that have a clear pre-condition, such as for the “KILL” action an enemy must have been placed. The user has the option to either use the generated actions, to manually place out different actions or use a combination of both to build up a full quest.

% PCG has several advantages such as  games can be produced faster and cheaper, thus making it possible for smaller teams without resources of large companies to create content rich games [27].