\documentclass[conference]{IEEEtran}
\IEEEoverridecommandlockouts
% The preceding line is only needed to identify funding in the first footnote. If that is unneeded, please comment it out.
\usepackage{cite}
\usepackage{amsmath,amssymb,amsfonts}
\usepackage{algorithmic}
\usepackage{graphicx}
\usepackage{textcomp}
\usepackage{xcolor}
\def\BibTeX{{\rm B\kern-.05em{\sc i\kern-.025em b}\kern-.08em
    T\kern-.1667em\lower.7ex\hbox{E}\kern-.125emX}}
\begin{document}

\title{Towards Designer Modeling through Design Style Clustering\\
\thanks{The Evolutionary Dungeon Designer is part of the project \textit{The Evolutionary World Designer}, supported by The Crafoord Foundation.}
}
% \author{\IEEEauthorblockN{Alberto Alvarez}
% \IEEEauthorblockA{\textit{Department of Computer Science\\and Media Technology} \\
% \textit{Malmö University}\\
% Malmö, Sweden \\
% alberto.alvarez@mau.se}
% \and
% \IEEEauthorblockN{Jose Font}
% \IEEEauthorblockA{\textit{Department of Computer Science\\and Media Technology} \\
% \textit{Malmö University}\\
% Malmö, Sweden \\
% jose.font@mau.se}
% \and
% \IEEEauthorblockN{Julian Togelius}
% \IEEEauthorblockA{\textit{Department of Computer Science and Engineering\\Tandon School of Engineering} \\
% \textit{New York University}\\
% New York, NY, USA \\
% julian@togelius.com}
% }

%JULIAN, WHAT IF WE USE THIS FORMAT INSTEAD?
%SOUNDS GOOD, either works
\author{\IEEEauthorblockN{Alberto Alvarez\IEEEauthorrefmark{1}\IEEEauthorrefmark{3}, Jose Font\IEEEauthorrefmark{1}\IEEEauthorrefmark{5}, and Julian Togelius\IEEEauthorrefmark{2}}
\IEEEauthorblockA{\IEEEauthorrefmark{1}Department of Computer Science and Media Technology\\
Malmö University, Sweden\\
Email: \IEEEauthorrefmark{3}alberto.alvarez@mau.se,
\IEEEauthorrefmark{5}jose.font@mau.se}
\IEEEauthorblockA{\IEEEauthorrefmark{2}Department of Computer Science and Engineering, Tandon School of Engineering\\
New York University, USA\\
Email: julian@togelius.com}
}

\maketitle
% \IEEEpeerreviewmaketitle

\begin{abstract}
We propose modeling designer style in mixed-initiative game content creation tools as archetypical design traces. These design traces are formulated as transitions between design styles; these design styles are in turn found through clustering all intermediate designs along the way to making a complete design. This method is implemented in the Evolutionary Dungeon Designer, a prototype mixed-initiative system for roguelike games. We present results both in the form of design styles for rooms, which can be analyzed to better understand the kind of rooms designed by users, and in the form of archetypical sequences between these rooms. We further discuss how the results here can be used to create style-sensitive suggestions. Such suggestions would allow the system to be one step ahead of the designer, offering suggestions for the next phase, assuming that the designer will follow one of the archetypical design traces.
\end{abstract}

% \begin{IEEEkeywords}
% CHANGE THIS, component, formatting, style, styling, insert
% \end{IEEEkeywords}

\begin{IEEEkeywords}
Procedural Content Generation, Mixed-Initiative Co-Creativity, Unsupervised Learning, Computer Games
\end{IEEEkeywords}

\subsection{Introduction}

There exists a plethora of games\footnote{For instance, currently there are more than 68k games in steam \url{https://store.steampowered.com/search/?category1=998}.}, with diverse genres and each containing a different set of gameplay mechanics, audio, level, graphic, and narrative facets. The creation and combination of these facets make game development a hard task, commonly involving a diverse group of developers~\citeptwelvth{p12Blow2004-gamesHard}. Likewise, the generation of these facets in conjunction has been categorized as one of the biggest and most challenging tasks within computational creativity~\citeptwelvth{p12Liapis2014-gameCreativity,p12Liapis2019-OrchestratingGames}. However, games share common elements and underlying narratives, but it is non-trivial how to identify these, how to define and analyze these games structurally, or what type of common underlying structures exist; pointed out as well by~\citeptwelvth{p12aarseth2018-ontologicalMeta,p12vozaru_game_2022}.

Among the different facets, narrative stands out in games as it helps to create meaning, make sense of situations, and make games [stories] recognizable~\citeptwelvth{p12mateas2003-facade,p12Aarseth2012-Narrativetheory,p12kybartas2016survey,p12flodtol2020-WIPMakeSenseDungs}. Narrative structures can be used to describe how an experience or story is to be developed as argued by Barthes~\citeptwelvth{p12Barthes75-introStructNarr}, and to create an abstract representation based on the narrative structure instead of a temporal and partially-ordered sequence of events~\citeptwelvth{p12Szilas2003-structuralModelsIDtension}. Common narrative structures used in many domains are Aristotle's drama structure, which subdivides a story into \textit{exposition}, \textit{climax}, and \textit{resolution} or Propp's analysis on the morphology of russian folktale, which revealed a common structure among them, denoted as Propp's 31 ``narremes''~\citeptwelvth{p12propp1975-morphology}.

This paper presents \emph{TropeTwist}, a preliminar system that uses Tropes~\citeptwelvth{p12Thompson2018-usingTropesNarrativeEvents,p12tropesSimpsons} extracted from TvTropes~\citeptwelvth{p12tvtropes,p12periodicTable} as patterns and fundamental units, which when combined can compose structures further representing other composed tropes. Common narrative structures can be identified and defined using \emph{TropeTwist}. TropeTwist can define generic aspects of a story, leading to the identification of events, roles, and narrative elements, as well as a novel way to form narratives. As a proof-of-concept, we built, analyzed, and described structurally three game examples shown in figure~\ref{fig:teaserfig}, top row.

%We propose graph grammars as indirect encoding of narrative graphs and the use of the Multi-dimensional Archive of Phenotypic Elites (MAP-Elites)~\citeptwelvth{p12Mouret2015-MAPElites} to generate novel variations (shown in figure~\ref{fig:teaserfig}, bottom row) using the proof-of-concept examples as roots. Simultaneously, we propose metrics to evaluate the coherence, cohesion, and interestingness of the resulting narrative graphs, which we use as comparison between hand-made and generated narrative graphs. Our preliminary results show that by using MAP-Elites, it is possible to vary the structure in such a way that more interesting structures appear while retaining coherence. 

We propose graph grammars as indirect encoding of narrative graphs and the use of the Multi-dimensional Archive of Phenotypic Elites (MAP-Elites)~\citeptwelvth{p12Mouret2015-MAPElites} to generate novel variations (shown in figure~\ref{fig:teaserfig}, bottom row) using the proof-of-concept examples as roots. Simultaneously, we propose metrics to evaluate the resulting narrative graphs' coherence, cohesion, and interestingness. Our preliminary results show that we can produce more interesting structures retaining coherence based on our metrics. 

% We propose indirectly encoding these as graph grammars and using the Multi-dimensional Archive of Phenotypic Elites (MAP-Elites)~\citeptwelvth{p12Mouret2015-MAPElites} to generate 

% We Further, using graph grammars and evolutionary algorithms (EA) and these graphs as targets and roots, we show preliminary results on how to generate and evaluate novel variations using the Multi-dimensional Archive of Phenotypic Elites (MAP-Elites)~\citeptwelvth{p12Mouret2015-MAPElites}. We evaluate the proof-of-concept narrative graphs and the generated ones based on their inter



% In our system, narrative structures serve as an abstraction layer, which can be used to create more generic aspects of a story, and at the same time, we can leverage its ambiguity as the structure is not necessarily bounded to specific partially-ordered events. As a proof-of-concept, we built, analyzed, and described structurally three game examples shown in figure~\ref{fig:teaserfig}, top row. Further, using graph grammars and evolutionary algorithms (EA) and these graphs as targets and roots, we show preliminary results on how to generate and evaluate novel variations using the Multi-dimensional Archive of Phenotypic Elites (MAP-Elites)~\citeptwelvth{p12Mouret2015-MAPElites}. We evaluate the proof-of-concept narrative graphs and the generated ones based on their inter

% This could allow a designer to create the structure of what they intend the narrative to be rather than focusing on how it will occur, how to communicate it to the player, or the specific plot points. As a proof-of-concept, we built, analyzed, and described structurally three game examples shown in figure~\ref{fig:teaserfig}, top row. Further, using graph grammars and evolutionary algorithms (EA) and these graphs as targets and roots, we show preliminary results on how to generate and evaluate novel variations.

% In our system, narrative structures serve as an abstraction layer, which can be used to create more generic aspects of a story, and at the same time, we can leverage its ambiguity as the structure is not necessarily bounded to specific partially-ordered events. This could allow a designer to create the structure of what they intend the narrative to be rather than focusing on how it will occur (i.e., syuzhet), how to communicate it to the player (i.e., discourse), or the specific plot points (i.e., fabula). As a proof-of-concept, we built, analyzed, and described structurally three game examples shown in figure~\ref{fig:teaserfig}, top row. Further, using graph grammars and evolutionary algorithms (EA) and these graphs as targets and roots, we show preliminary results on how to generate and evaluate novel variations.

\begin{figure*}
    \centering
    \includegraphics[width=\textwidth]{figures/figure-1_extra8.png}
       \caption{Proof-of-concept narrative structures of existing games (top row) created using TropeTwist with the available nodes (table~\ref{tab:tropes}). Bottom row shows exemplar elites generated with MAP-Elites using the respective top row narrative structure as root. Color matching squares, lines, and triangles denote different meso-patterns in the structures. Squares and triangles are the start and end of a meso-pattern, respectively.}
       \label{fig:teaserfig}
\end{figure*}
\subsection{Previous work}

\subsubsection{Mixed-Initiative Co-Creativity}
Similar to user or player modeling, designer modeling for content creation tools (CAD and MI-CC tools) was suggested by Liapis et al~\citepfifth{p5Liapis2013-designerModel}, where it is proposed the use of designers models that capture their styles, preferences, goals, intentions, and interaction processes. In their work, they suggest methods, indications, and advice on how each part can be model to be integrated into a holistic designer model, and how each game facet can use and benefit from designer modeling. Moreover, in \citepfifth{p5Liapis2014-designerModelImpl} the same authors discuss their implementation of designer modeling and the challenges of integrating all together in their MI-CC tool, Sentient Sketchbook, which had a positive outcome on the adaptation of the tool towards individual “artificial” users.

Furthermore, Lehman et al \citepfifth{p5lehman2016creative} presented Innovation Engines that combine the capabilities and advantages of machine learning and evolutionary algorithms to produce novel 3D graphics with the use of Compositional Pattern-Producing Networks (CPPN) evolved with MAP-Elites, and evaluated by the confidence a deep neural network had on the models belonging to a specific object category.

\subsubsection{Procedural Content Generation via Machine Learning}
Summerville et al. \citepfifth{p5summerville2018procedural} define Procedural Content Generation via Machine Learning (PCGML) as the generation of game content by models that have been trained on existing game content. The main approaches to PCGML are: autonomous content generation, content repair, content critique, data compression, and mixed-initiative design.

\begin{figure}
\includegraphics[width=\textwidth]{fig1.jpg}
\caption{Screenshot of the dungeon editor screen in EDD, displaying a sample dungeon composed by five rooms.} \label{p5fig1}
\end{figure}

In the latter case and, as appointed by Treanor et al. \citepfifth{p5treanor2015ai}, AI may engage with a human user participating in the creation of content, so that new gameplay emerges from this shared construction. This emerging relationship between the user and the AI system, when implemented through a trained machine learning algorithm, has the potential to reduce user frustration, error, and training time. This is due to the capacity of a machine learning solution to adapt to the design preferences of the user that interacts with the MI-CC tool by learning from the user-generated dataset of previous choices.

\subsubsection{The Evolutionary Dungeon Designer}

The Evolutionary Dungeon Designer (EDD) is an MI-CC tool for designers to build 2D dungeons. EDD allows designers to manually edit the overall dungeon and its composing rooms (see Figure \ref{p5fig1}), as well as to use procedurally generated suggestions either as inspiration to work on or as a finished design (see Figure \ref{p5fig2}). Both options fluently alternate during the creation process by means of a workflow of mutual inspiration, through which all manual editions performed by the user are fed into the underlying continuous Evolutionary Algorithm, accommodating them into the procedural suggestions. A detailed description of EDD and its features can be found in~\citepfifth{p5Alvarez2018a,p5Alvarez2018,p5Baldwin2017a,p5Baldwin2017}.

Subsequent user studies \citepfifth{p5Alvarez2018,p5Baldwin2017} carried out with game designers on EDD raised the following areas of improvement: (1) the designers struggled with EDD’s capability of understanding the designer’s intentions and preserving custom designs; (2) the tool was unable to generate aesthetically pleasing suggestions since the fitness function only accounted for functionality, but not aesthetics, of design patterns; (3) the designers wanted to keep certain manual editions from being altered by the procedural suggestions.  

With the aims of addressing these limitations as well as fostering the user's creativity with quality-diverse proposals, EDD was improved with the Interactive Constrained MAP-Elites (IC MAP-Elites) \citepfifth{p5alvarez2019empowering}, an implementation of MAP-Elites into the continuous evolutionary process in EDD. With this addition, the user drives the generation of procedural suggestions by modifying at any moment the areas of the search space where the evolution should put the focus on. This is done by selecting among the available dimensions: symmetry, similarity, design patterns, linearity, and leniency. Additionally, the designers have now the chance to limit the search space by locking map areas and thus preserving manually edited content.

This paper contributes by building on top of EDD's IC MAP-Elites, adding a data-driven Designer Preference Model that adapts and personalizes the design experience, as well as balances the expressivity of the tool and the controllability of the designer over the tool. Other researchers have pursued a similar goal by biasing the search space through having the user perform a manual selection after every given number of generations~\citepfifth{p5Picbreeder-Secretan2008,p5Liapis2012-adaptiveVisual,p5Novelty-Lehman2011}. Nevertheless, this approach leads to an increase in user fatigue by repeatedly asking for user input and thus, stalling the evolutionary process until such input is received. Moreover, this staged process seems incompatible with the dynamic reciprocal workflow of MI-CC tools, where the focus is on the designer proactively creating content rather than passively browsing a set of suggestions.

\begin{figure}[t]
\includegraphics[width=\textwidth]{fig2.jpg}
\caption{The room editor screen in EDD. The top-right pane shows the suggestions provided by the IC MAP-Elites algorithm. Below are the six top-raked suggestions by the Designer Preference Model. The left pane contains the manual edition features.} \label{p5fig2}
\end{figure}

The remaining sections of the paper are structured as follows: Section 3 describes the data-driven Designer Preference Model; Section 4 presents the initial experimental results, and Section 5 discusses the results and future lines of research of this novel approach.

%%Check the section references!
% % Please add the following required packages to your document preamble:
% \usepackage{graphicx}
\begin{table}[]
\centering
\begin{tabular}{|l|lll|}
\hline
        & AIv1       & AIv2        & AIv3        \\ \hline
Leniency        & 0.56±0.07  & 0.62±0.09   & 0.57±0.08   \\
Linearity        & 0.91±0.02  & 0.92±0.02   & 0.91±0.01   \\
MesoPat       & 0.15±0.05  & 0.13±0.07   & 0.12±0.05   \\
SpatialPat    & 0.35±0.1   & 0.41±0.11   & 0.34±0.09   \\
Symmetry   & 0.43±0.11  & 0.35±0.18   & 0.35±0.12   \\
$W_{dens}$ & 0.27±0.09  & 0.26±0.08   & 0.21±0.05   \\
$W_{spar}$ & 0.21±0.05  & 0.19±0.03   & 0.15±0.01   \\
$E_{dens}$ & 0.24±0.07  & 0.27±0.06   & 0.3±0.06    \\
$E_{spar}$ & 0.22±0.05  & 0.32±0.05   & 0.35±0.06   \\
$T_{dens}$ & 0.37±0.13  & 0.28±0.09   & 0.34±0.07   \\
$T_{spar}$ & 0.36±0.11  & 0.3±0.1     & 0.37±0.05   \\ \hline
Steps      & 39.25±6.38 & 84.31±14.85 & 76.75±17.02 \\ \hline
\end{tabular}
\caption{Summary of the created rooms filtered by the AI version used. All values are the average of all the created rooms using the specific AI version. The first five values relates to the MAP-Elites dimensions, then the fitness of the rooms, the density and sparsity values for wall (W), enemies (E), and treasures (T), and finally the avg. steps taken to design a room.}
\label{tab:AIavgValues}
\end{table}

\subsection{Experiment Setup}

We conducted a user study to explore the user experience of using different levels of AI agency, the different design characteristics, and the relationship between the human designer and the AI. We collected both quantitative data on the AI's impact on the co-designed end product and qualitative data through think-a-loud and semi-structured interviews regarding the users' experience when interacting with the AI. The interview structure is inspired by the pyramid model, meaning the interviews will begin with specific questions, and gradually have more open questions, which naturally allows for a discussion towards the end. This model is chosen to support the variation of subjects the interview is desired to cover, as well as support natural transitions between the questions and their openness. The questions and user study procedure can be found in Appendix A. 

%The interviews are semi-structured, meaning it includes both closed and more open questions, and depending on the discussion and answers, some questions might be omitted.

% We collected quantitative data regarding what impact the AI had on the co-designed end product, and how the human designer interacted with the AI's contributions. Likewise, we collected qualitative data through recorded think-aloud observations and semi-structured interviews regarding the users' experience, and possibly catch certain remarks of frustration or appreciation of their digital colleague that can be valuable for the discussion of the relationship between the co-creators. 


%We collected the following data:

%\begin{itemize}
%    \item \textbf{Audio Recordings:} 
%\end{itemize}

Eight participants tested our tool with game design and level design experience. One participant was a professional game designer with eight years of professional experience (first participant), and seven participants were third-year Game Development students. They all had an individual digital session, where we shared our screen, and they took remote control to conduct the study. Participants accepted to participate, signed consent forms, and then received a short introduction describing the experiment and its steps. The participants were then asked to design two contiguous rooms in a dungeon, repeating this process for each of the AI variants and expressing their design decisions verbally whenever they felt like it. After using the tool, the participants were interviewed, focusing on and covering an overarching understanding of the user experience, particularly in terms of creativity and interaction with the AI.

% the relationship that occurs between the AI and human designer (See Appendix B). 

For all the sessions, human designers could place up to 12 tiles, and the AI could place as many tiles as the human placed. The AI could contribute only in a rectangular area surrounding the tiles the human designer recently contributed with, including a margin of 1 tile. This choice is made to support a responsive and collaborative behavior of the AI that builds on the human designer's contribution.


% The locations available for the AI to contribute in for each turn are limited to a rectangular area surrounding the tiles the human designer recently contributed with, including a margin of 1 tile. This choice is made to support a responsive and collaborative behavior of the AI that builds on the human designer's contribution.

% The margin for the contribution area is set to 1, as it was found during experimentation that any margin bigger than this is likely perceived as the AI contributing to other areas than the ones the human is focused on, because of the default size of the room being relatively small.

%  as this enables the designer to contribute with an adequate amount of tiles during their turn and create representable structures

% The rooms produced during the user study are displayed in Figure 6, 7 and 8. Rooms with red borders are infeasible, meaning there are unreachable tiles. The UI displays a warning when this happens, and the AI can repair this during its turn, however the resulting rooms that are infeasible are a result of the human designer creating unreachable areas, and then immediately selecting to go to the World Editing view, before pressing "End Turn". 
% Each participant created two rooms for each version of the AI. Participant 1 created Room 1 and Room 2 for all version, Participant 2 created Room 3 and Room 4 for all version, etc. All of the participants had the option to adjust the sizes of the rooms in the World Editing view before entering the Room Editing view, however none of them did, and therefore all of the resulting rooms are of the default size. The designer also has the option to change the location of the hero and the doors. The location of the hero was only moved twice in all of the session, and the doors where never moved. 






%, the participant will take part in an interview. The questions, and their order, are planned out and designed to cover an overarching understanding of the user experience, in particular in terms of creativity, and the relationship that occurs between the AI and human designer (See Appendix B). 




%The participants were asked to repeat this for each of the AI-initiatives. was asked to repeatEach pair of room 

% then asked to complete three tasks, each regarding

%The users were then asked to complete three tasks that covered the tool's functionality and the AI-initiatives, respectively for each task. The tasks were 


%and different approaches to creating quests. The tasks were to 1) manually create a quest, 2) automatically create a quest, and 3) create a quest through mixed-initiative. They were also asked to create a dungeon that suited their preferences and objectives before creating quests. The questionnaire consisted of 17 closed-ended questions, and the rest were open-ended. The interview began with a questionnaire with six questions about the users' background and experience within game development and finish with questions about their experience and opinions on the tool. Both the questionnaire and interview followed guidelines described by



%The participants used the tool



\subsection{Story Designer}

% Me he tomado la libertad de darle el nombre de Story Designer a esta nueva herramienta en EDD. Te parece? Sobre todo es para simplificar cómo nos referimos a él

% \begin{figure*}[t!]
%     \centering
%     \includegraphics[width=\textwidth]{figures/main-help.png} %temp
%     \caption{The Story Designer screen in EDD.}
%     \label{fig:story-screen}
% \end{figure*}

% \begin{itemize}
% \item Brief intro to the tool's purpose. %done
% \item Subsection about story structures with TVtropes, node types, connection types, examples %done
% \item Subsection with EDD's story structure creation view. %done
% \item Subsection with representing and evolving with Graph Grammars in EDD 
% \end{itemize}

% \begin{table*}[ht]
% \centering
\caption{General consensus on EDD's features} \label{p1tab:consensus}
\resizebox{0.8\textwidth}{!}{
\begin{tabularx}{\textwidth}{|p{0.2\textwidth}|p{0.99\textwidth}|}
\cline{1-2}

Description & Participants’ Consensus \\\cline{1-2}
World Grid  of the dungeon                   & Its purpose of establishing an illusion of a fully realized dungeon is somewhat achieved. However, limitations exist with how it defines feasibility, a dungeon’s starting point, and the entrances, which disrupts the designers’ decisions.                                                                                                                                                                                   \\\cline{1-2}
World View                                  & The world view’s usefulness for the most part could not be established, other than for the purpose of going to the suggestions view (which was already seldom during the user study) and having a closer look at the entire dungeon without any distractions. Some participants preferred features to be already in the room view’s minimap, and some wanted to see more specific functionalities within the world view itself. \\\cline{1-2}

Enabling and  \newline disabling rooms                & As the user study restricted participants to create 3x3 dungeons, this feature for the most part has been neglected. This is also in part because of its accessibility only being in the world view, which proved to be an inefficient view in general. However, its use for bigger dungeon sizes later on was appreciated, especially for more intricate design purposes.                                                      \\\cline{1-2}
Suggestions View                            & Similarly to enabling and disabling rooms, it was quite difficult to encourage the use of this functionality due to the world view’s inefficient usability. However, this could also be due to the dungeon’s small size, as some participants expressed high interest in using more suggestions with larger dungeon sizes.                                                                                                      \\\cline{1-2}
Minimap  \newline  navigation                      & The minimap proved to be a strong tool not only for navigation purposes, but also for supporting design decisions and choices. The directional buttons were rarely used, but their room previews were helpful in emphasizing the current room’s connection to adjacent rooms without looking at the minimap. On the other hand, this lowered the usability of the world view.                                                   \\\cline{1-2}
Parameters                                       & The parameters were, in general, lacking. They served to be important in decision-making when choosing a suggested map in room view, but there were still doubts on their accuracy and sufficiency when providing information about the generated suggestions.                                                                                                                                                                       \\\cline{1-2}
Generated maps for  \newline  suggestions in room view & Suggestions in the room view proved to be very helpful in supporting the whole design process as they primarily acted as inspirations for the users. The most prominent comment among the users is the preference of having more control on how suggestions should be generated depending on different types of parameters.                                                                                                     \\\cline{1-2}
Design \newline  patterns& The patterns’ visualization was, in general, lacking and not self-explanatory. Some participants have expressed interest in using patterns as a parameter in the generation of suggestions.                                                                                                                                                                                                                                     \\\cline{1-2}
Dark theme                                  & EDD’s dark theme for the user interface received a positive response as it makes working with the program easier.
	\\ \cline{1-2}
\end{tabularx}
}
\end{table*}

% \begin{table}
% \begin{center}
% {\caption{Best performing setups based on their internal validation and visualization of clustered data points.}\label{table:setups}}
% \resizebox{\textwidth}{!}{
% \begin{tabular}{ccccccc}
% \hline
% \rule{0pt}{12pt}
% Algorithm&Data&K&$\Diamond$&$\Box$&$\bigtriangleup$ 
% \\ 
% \hline
% \\[-6pt]
% K-Means & Tiles-PCA & 9 & 0.43 & 0.73 & 9438.233 \\ 
% K-Means & Tiles-PCA & 12 & 0.41 & 0.77 & 9436.928 \\
% K-Means & Dimensions-PCA & 12 & 0.43 & 0.73 & 7738.343 \\
% Agglomerative single & Combined-PCA & 6 & 0.51 & 0.43  & 38.833 \\ 
% Agglomerative avg. & Dimensions-PCA & 6 & 0.44 & 0.67 & 3463.567 \\ 
% \hline
% \\[-6pt]
% \multicolumn{6}{l}{$\Diamond$ Silhouette Score\ \
% $\Box$ Davies Bouldin Index\ \
% $\bigtriangleup$ Calinski-Harabasz Index}
% \end{tabular}
% }\end{center}
% \end{table}

% \begin{itemize}
%     \item Brief intro to the tool's purpose.
%     \item Subsection about TropeTwist briefly explaining how tropes are used and the trope patterns. story structures with TVtropes, node types, connection types, examples done \checkmark
%     \item Subsection with EDD's story structure creation view, and workflow. \checkmark
%     \item Add info on the development from questgram with specific NPCs and the questview changed (although, that last part is not relevant). Also add the info on how to actually use this system.
%     \item Brief subsection with how graph gramamrs work and how we represent these graphs for the EA. Not super in detail because all of that is from tropetwist. 
%     \item The things that are extended in this version is the amount of dimensions, and how the elites are shown! (besides actually showing all of this!)
%     \item Dimensions should be addressed in the final subsection. 
%     \item How elites are shown and all of that goes to the workflow subsection.
% \end{itemize}


%\begin{figure*}[t!]
%    \centering
%    \includegraphics[width=\textwidth]{figures/current_GUI.png}
%    \caption{The Story Designer screen in EDD.}
%    \label{fig:story-screen}
%\end{figure*}
 
% \begin{table*}[ht]
% \centering
\caption{General consensus on EDD's features} \label{p1tab:consensus}
\resizebox{0.8\textwidth}{!}{
\begin{tabularx}{\textwidth}{|p{0.2\textwidth}|p{0.99\textwidth}|}
\cline{1-2}

Description & Participants’ Consensus \\\cline{1-2}
World Grid  of the dungeon                   & Its purpose of establishing an illusion of a fully realized dungeon is somewhat achieved. However, limitations exist with how it defines feasibility, a dungeon’s starting point, and the entrances, which disrupts the designers’ decisions.                                                                                                                                                                                   \\\cline{1-2}
World View                                  & The world view’s usefulness for the most part could not be established, other than for the purpose of going to the suggestions view (which was already seldom during the user study) and having a closer look at the entire dungeon without any distractions. Some participants preferred features to be already in the room view’s minimap, and some wanted to see more specific functionalities within the world view itself. \\\cline{1-2}

Enabling and  \newline disabling rooms                & As the user study restricted participants to create 3x3 dungeons, this feature for the most part has been neglected. This is also in part because of its accessibility only being in the world view, which proved to be an inefficient view in general. However, its use for bigger dungeon sizes later on was appreciated, especially for more intricate design purposes.                                                      \\\cline{1-2}
Suggestions View                            & Similarly to enabling and disabling rooms, it was quite difficult to encourage the use of this functionality due to the world view’s inefficient usability. However, this could also be due to the dungeon’s small size, as some participants expressed high interest in using more suggestions with larger dungeon sizes.                                                                                                      \\\cline{1-2}
Minimap  \newline  navigation                      & The minimap proved to be a strong tool not only for navigation purposes, but also for supporting design decisions and choices. The directional buttons were rarely used, but their room previews were helpful in emphasizing the current room’s connection to adjacent rooms without looking at the minimap. On the other hand, this lowered the usability of the world view.                                                   \\\cline{1-2}
Parameters                                       & The parameters were, in general, lacking. They served to be important in decision-making when choosing a suggested map in room view, but there were still doubts on their accuracy and sufficiency when providing information about the generated suggestions.                                                                                                                                                                       \\\cline{1-2}
Generated maps for  \newline  suggestions in room view & Suggestions in the room view proved to be very helpful in supporting the whole design process as they primarily acted as inspirations for the users. The most prominent comment among the users is the preference of having more control on how suggestions should be generated depending on different types of parameters.                                                                                                     \\\cline{1-2}
Design \newline  patterns& The patterns’ visualization was, in general, lacking and not self-explanatory. Some participants have expressed interest in using patterns as a parameter in the generation of suggestions.                                                                                                                                                                                                                                     \\\cline{1-2}
Dark theme                                  & EDD’s dark theme for the user interface received a positive response as it makes working with the program easier.
	\\ \cline{1-2}
\end{tabularx}
}
\end{table*}

% \begin{table}
% \begin{center}
% {\caption{Best performing setups based on their internal validation and visualization of clustered data points.}\label{table:setups}}
% \resizebox{\textwidth}{!}{
% \begin{tabular}{ccccccc}
% \hline
% \rule{0pt}{12pt}
% Algorithm&Data&K&$\Diamond$&$\Box$&$\bigtriangleup$ 
% \\ 
% \hline
% \\[-6pt]
% K-Means & Tiles-PCA & 9 & 0.43 & 0.73 & 9438.233 \\ 
% K-Means & Tiles-PCA & 12 & 0.41 & 0.77 & 9436.928 \\
% K-Means & Dimensions-PCA & 12 & 0.43 & 0.73 & 7738.343 \\
% Agglomerative single & Combined-PCA & 6 & 0.51 & 0.43  & 38.833 \\ 
% Agglomerative avg. & Dimensions-PCA & 6 & 0.44 & 0.67 & 3463.567 \\ 
% \hline
% \\[-6pt]
% \multicolumn{6}{l}{$\Diamond$ Silhouette Score\ \
% $\Box$ Davies Bouldin Index\ \
% $\bigtriangleup$ Calinski-Harabasz Index}
% \end{tabular}
% }\end{center}
% \end{table}

Story Designer is a new system integrated in EDD, which presents a visual interface for mixed-initiative narrative structure generation. It makes extensive use of the TropeTwist system as foundation to build narrative graphs and assess them by identifying trope patterns. The user manually designs a story structure by adding and interconnecting nodes in a graph, which seeds an evolutionary algorithm (EA) that generates story structure suggestions that can be incorporated into the user's design. This continuous co-creative design process implements the Interactive Constrained MAP-Elites (IC MAP-Elites) approach presented in~\cite{p11alvarez_empowering_2019}, providing quality-diverse suggestions across several feature-dimensions.

Story Designer is interconnected with the level design facet in EDD. This means that the narrative graphs that can be developed and that can be generated and suggested are constrained by the content that exists in the levels. For instance, if the designer adds two NPCs besides the Hero, then the system could at most, use three character nodes to represent them, or if the designer adds a boss enemy and a quest item, this would mean that the boss enemy could be represented as one of the villain nodes (e.g., Enemy, Big Bad, or Dragon) and the quest item as a possible Plot Device.

\subsubsection{TropeTwist}

TropeTwist~\cite{p11alvarez_tropetwist_2022} is a system that uses tropes~\cite{p11lewis_governing_2018,garcia-sanchez_simpsons_2021,richmond_tv_2004,harris_periodic_2016}, narrative conventions easily recognizable by the audience, as patterns that combine to compose narrative structures. These structures define generic aspects of a story, leading to the identification of events, roles, and other relevant narrative elements arranged as nodes in an interconnected narrative graph. By having all this elements in a graph, entire narratives are encoded using graph grammars, to then procedurally generate novel narrative variations by means of a MAP-Elites algorithm that considers several narrative evaluation metrics, such as interestingness, coherence, and cohesion. 

Nodes in a narrative graph represent tropes. Interconnected tropes create other composite tropes and patterns, that can be identified as subgraphs of a complete narrative graph. These patterns can be \textbf{micro-patterns} encapsulating a single trope node, \textbf{meso-patterns}, often composed by more than one micro-pattern with a specific meaning, and \textbf{auxiliary patterns}, identifying structural gaps in the graph. For a detailed definition of all tropes and patterns, please refer to~\cite{p11alvarez_tropetwist_2022}. Here we present a comprehensive summary:

\begin{itemize}
    \item Micro-patterns are the fundamental narrative unit in the system, encapsulating tropes in building blocks to create complex narrative structures. These are classified into structure patterns (SP), that articulate the story elements (i.e. Conflict), character patterns (CP) (i.e. heroes and villains), and plot device patterns (PDP), that move the story forwards towards a particular goal (i.e. the MacGuffin).
    \item Meso-patterns may emerge from the combination of micro-patterns and other meso-patterns, denoting spatial, semantic, and usability relationship within the narrative graph.
    \begin{enumerate}
        \item The \emph{Conflict Pattern (ConfP)} ties a conflict node to two other nodes representing both parties in a conflict (i.e. HERO $\rightarrow$ CONFLICT $\rightarrow$ EMP, a hero is in conflict with the Empire).
        \item The \emph{Derivative Pattern (DerP)} defines relations of entailment between other nodes, called derivatives. These derivatives acquire a local and temporal order, and a causal relationship. I.e the former conflict connected to EMP $\diamondsuit$--- DRA $\diamondsuit$--- NEO, means that the hero engages the Empire, which entails both a conflict with the Dragon (\emph{DRA}) and the appearance of the Chosen One (\emph{NEO}).
        \item The \emph{Reveal Pattern (RevP)} connects two independent CPs as one, meaning that character A was, in fact, always character B, and vice-versa. This pattern turns all existing conflicts between them into \emph{fake} conflicts.
        \item The \emph{Active Plot Device Pattern (APD)} triggers a PDP and integrates it in the the narrative, since PDP are passively described and lack any start condition.
        \item \emph{Plot Points (PP)} are key discrete narrative events. The derivatives within a \textit{DerP}, the source of a reveal pattern, as well as active plot devices are considered plot points.
        \item A \emph{Plot Twist (PT)} identifies those plot points that could change the natural flow of the narrative. I.e. in EMP $\diamondsuit$--- DRA $\diamondsuit$--- NEO, NEO is identified as a plot twist since its nature (heroic) is opposed to that of the first node EMP (villainous), which alters the natural order of the connecting derivative pattern.
    \end{enumerate}
    \item Auxiliary patterns spot and encapsulate those areas in the graph that don't contain meaningful narrative information. \textit{Nothing} highlights nodes that are not identified or part of any meso-pattern; whereas \textit{Broken Link} marks outgoing connections from any node that do not lead to any pattern.
\end{itemize}



\subsubsection{Workflow}


Story Designer is integrated in EDD as a separate view (Figure \ref{fig:story-screen}) that can be accessed anytime from the dungeon editor. The use starts with a minimal sample narrative graph HERO $\rightarrow$ CONFLICT $\rightarrow$ ENEMY in the manual edition pane (center). This graph can be extended by adding nodes from the node context menu that pops up with a right-click on an empty space. Node are arranged by type for the sake of clarity, and an option to automatically re-arrange the graph is shown at the end of the menu. Right-clicking on an existing node border will pop up the edge context menu, that allows the user to create a new connection or to delete the selected node. Existing connections are deleted by left-clicking on them.

In a way similar to EDD's room editor \cite{p11alvarez_empowering_2019}, as the user edits the narrative graph manually, this graph is fed into the underlying evolutionary algorithm that procedurally generates on the fly alternative narrative graphs in the suggestions pane (right). The top-right corner shows the feature-dimension matrix, whose cells are colored depending on the fitness of the fittest elite contained in it, ranging from dark red (no elite yet), to dark green (optimal fitness). The elite in the selected cell of the matrix is displayed in the bottom-right corner. Hovering the mouse above a cell displays its elite's graph above the selected one, which allows the user to compare several graphs at a glance.    

%\subsubsection{Building stories with tropes}

%In storytelling, a trope \cite{p11garcia-sanchez_simpsons_2021} is a convention or figure of speech that is assumed by the storyteller to be easily recognizable by the audience. TV tropes is an online wiki and repository that compiles, curates, and describes several thousands of tropes in many sorts of media, such as television, films, literature, and games \cite{p11richmond_tv_2004}. As exemplified by \cite{p11harris_periodic_2016}, tropes can be interconnected in graph-like structures, called story molecules, to succinctly depict the story behind a narrative in any common media.

%Story Designer elaborates on the concept of story molecule as a means to represent stories using graph-like structures of interconnected tropes, called narrative graphs. Table \ref{tab:tropes} shows all the tropes that can be added as nodes to a narrative graph, represented by their symbol. Nodes are depicted with shapes specific to their trope type: heroes (rectangle), conflicts (diamond), enemies (hexagon), and plot devices (circle).

%Nodes in a narrative graph are necessarily interconnected by either unidirectional or bidirectional edges (with one or both arrow heads), or by entailment edges (with a single diamond head). Given nodes A and B, A $\diamondsuit$--- B reads as "A entails B", whereas A $\rightarrow$ B denotes a relationship from A to B, and B $\rightarrow$ A the opposite. A $\leftrightarrow$ B denotes a reflexive relationship between A and B. As an example, HERO $\rightarrow$ CONFLICT $\rightarrow$ EMP denotes a hero who is in conflict against an empire-type enemy, whereas HERO $\leftrightarrow$ CONFLICT denotes a hero who is in conflict with herself. EMP $\diamondsuit$--- DRA $\diamondsuit$--- NEO, denotes an empire that entails a dragon enemy that, once beaten, will lead to the appearance of a chosen one hero.



%\subsubsection{Trope Patterns}

%\begin{itemize}
%    \item This has to be reduced considerably to simply introduce the terms but link the reader to the TropeTwist paper! (Add it to Arxiv)!
%    \item Micro: Micro-patterns are the fundamental unit in the system which aims at categorizing different sets of the individuals patterns that are shown in table~\ref{tab:tropes}. Micro-patterns are the basic building block which when connected together allows the detection of meso-patterns.  (AND WRITE THEM)
%    \item Meso: If micro-patterns are the fundamental units to construct narratives in StoryDesigner, then Meso-patterns are the features~\cite{p11dahlskog_multi-level_2014} that emerge in the narrative from dynamically combining micro-patterns and in some occasions these with meso-patterns. Meso-patterns are composite patterns, always composed by more than one pattern denoting some spatial, semantic, and usability relationship within the narrative graph. Micro-, meso-, and macro-patterns have been used to generalize the generation of content and reduce the burden and complexity of generators mainly in regards to evaluation and encoding of the content, as well as a tool to compare generated or human-authored content. For StoryDesigner and the creation of narratives, we have identified a subset of Tropes (extracted from TVTropes) that require (or work as) the combination between more fundamental units. For instance, the reveal pattern relates to the "Good all along" or "evil all along" tropes from TVTropes. (AND WRITE THEM)
%    \item AUXILIARY: Denotes problems in the graph, and sub-optimal and impractical nodes and connections within a graph. (AND WRITE THEM)
%\end{itemize}

%\subsubsection{TropeTwist}

%------------------------------

% \subsubsection{Trope Patterns}

%\subsubsection{Tropes as Patterns}

% Given the nature of tropes as recognizable parts of stories
% The system analyzes the possible tropes to us

% All patterns calculate their quality, which is then used in different ways to estimate the quality and fitness of narrative structures.

% In most of the patterns that will be described we calculate and use two general qualities, which are indicated when used. The first quality is the $generic_{qual}(pattern)$, which uses the occurrence of the specific pattern within the edited graph by the user and is calculated as:

% \begin{equation}
%     quantity_{qual}(pattern) = pattern \sim \mathcal{N}(\mu,\,\sigma^{2})\,.
% \end{equation}

% where...

% The second general quality used to calculate the quality of the tropes is the $quantity_{qual}(pattern, trope)$, which uses the occurrence of a trope of a specific pattern class within the tested graph, and is calculated as:

% \begin{equation}
%     repetition_{qual}(pattern, trope) = \dfrac{\sum_{i=0}^{\left | patterns \right |}}{\left | pattern \right |}
% \end{equation}

% where $p$ is defined as the specific patterns within  $p = pattern \in AllPatterns$,

% \paragraph{Micro-Patterns}

% Micro-patterns are the fundamental unit in the system which aims at categorizing different sets of the individuals patterns that are shown in table~\ref{tab:tropes}. Micro-patterns are the basic building block which when connected together allows the detection of meso-patterns. 

% \paragraph{Structure Pattern}

% An structure pattern (SP) is any type of trope that would give some structural definition to a narrative, whether this being a conflict, specific act, or a part in a dramatic arc (e.g., climax). In Story Designer, the only type of structure trope is the \textsc{conflict} (C) trope, which represents the most basic structural interaction. The quality $SP_{qual}$ is calculated as the normalized linear combination of:

% \begin{equation}
%     SP_{qual} = generic_{qual}(SP) + involvement_{qual}(SP)
% \end{equation}

% \paragraph{Hero Pattern}

% Hero patterns (HP) are good characters within the narrative that could potentially be the main hero of the narrative or complementary roles. In Story Designer, \textsc{HERO}, \textsc{5MA}, \textsc{NEO}, \textsc{SH} are classified as HP. HPs are commonly used as source or targets (or both) of other patterns and in a few special occasions to denote a relation to another character. The quality $HP_{qual}$ is calculated as the normalized linear combination of:

% \begin{equation}
%     HP_{qual} = generic_{qual}(HP) + quantity_{qual}(HP) + involvement_{qual}(HP)
% \end{equation}

% \paragraph{Villain Pattern}

% Villain patterns (VP) are by nature, evil characters within the narrative that could be the antagonist of the narrative, an elite enemy, or a faction\footnote{Having more than one villain in the narrative structure effectively establishes the bosses as \textit{Disc-One Final Boss} \url{https://tvtropes.org/pmwiki/pmwiki.php/Main/DiscOneFinalBoss}}. In Story Designer, \textsc{ENEMY}, \textsc{EMP}, \textsc{BAD}, \textsc{DRA} are classified as VP. As the opposite of the HPs, VPs properties and usages are mirrored. The quality $VP_{qual}$ is calculated as the normalized linear combination of:

% \begin{equation}
%     VP_{qual} = generic_{qual}(VP) + quantity_{qual}(VP) + involvement_{qual}(VP)
% \end{equation}

% \paragraph{Plot Device Pattern}

% Plot device patterns (PDP), are described as the element within the narrative moves it forward, in the shape of a goal, object, or dramatic element to be used or encountered by any of the characters. The quality $PDP_{qual}$ is calculated as the normalized linear combination of:

% \begin{equation}
%     PDP_{qual} = generic_{qual}(PD) + quantity_{qual}(PDP)
% \end{equation}


% \paragraph{Meso-Patterns}

% If micro-patterns are the fundamental units to construct narratives in StoryDesigner, then Meso-patterns are the features~\cite{p11dahlskog2014-multimultilevel} that emerge in the narrative from dynamically combining micro-patterns and in some occasions these with meso-patterns. Meso-patterns are composite patterns, always composed by more than one pattern denoting some spatial, semantic, and usability relationship within the narrative graph. Micro-, meso-, and macro-patterns have been used to generalize the generation of content and reduce the burden and complexity of generators mainly in regards to evaluation and encoding of the content, as well as a tool to compare generated or human-authored content. For StoryDesigner and the creation of narratives, we have identified a subset of Tropes (extracted from TVTropes) that require (or work as) the combination between more fundamental units. For instance, the reveal pattern relates to the "Good all along" or "evil all along" tropes from TVTropes

% %In StoryDesigner and for the creation of narratives, we have identif
% %Meso-patterns are features 
% %Meso-patterns are composite patterns, which combine or effectively utilizes several micro-patterns or other meso-patterns to produce 
% %Meso-patterns are the combination
% % Micro-patterns are the fundamental unit in the system which aims at categorizing different sets of the individuals patterns that are shown in table~\ref{tab:tropes}. Micro-patterns are the basic building block which when connected together allows the detection of meso-patterns. 

% \paragraph{Conflict Pattern} Relates to the conflict between two character micro-patterns (HPs or VPs). A conflict $C$ is formally defined as $\langle s, C, t \rangle$ where s and t are a character micro-pattern (i.e., either hero pattern or villain pattern). $C$ denotes the general conflict trope, which can be simple conflict (c), conflict against nature (cona), or conflict against society (coso). A conflict meso-pattern do not necessarily describe a conflict between two different characters, as conflicts can be self-conflicts expressed as a bi-directional connection between the character trope and conflict trope. This effectively makes $s$ and $t$ the same. 

% Moreover, a conflict pattern is either~\textsc{explicit} or~\textsc{implicit} as depicted in figure~\ref{fig:conflict_pattern_a} and~\ref{fig:conflict_pattern_b}. \textsc{Explicit Conflicts} are the ones that are explicitly encoded in the graph and directed from a source $s$ to a target $t$ passing through the conflict trope $C$. On the other hand, \textsc{Implicit Conflicts} relates to the conflicts from a target $t$ (or derivatives) to a source $s$ (or derivatives) that are not encoded in the graph. For instance, if a hero has an explicit conflict with an enemy, but the enemy does not, that produces an implicit conflict from enemy to hero. Conflicts are the main way of representing some type of structure and groups within StoryDesigner, and as seen before in the micro-pattern definition, they play a crucial role in their quality. The quality $ConfP_{qual}$ is calculated as the normalized linear combination of:

% \begin{equation}
%     ConfP_{qual} = generic_{qual}(ConfP) + quantity_{qual}(ConfP) + involvement_{qual}(ConfP)
% \end{equation}

% % and must be opposite to each other (e.g., if $s$ is a hero pattern, then $t$ must be a villain pattern.
% % \paragraph{Implicit Conflict Pattern}
% \paragraph{Composite Conflict Pattern} very simple, works as a grouper for a set of conflict patterns united by the same \textit{structure pattern}. This means, that if the same structure pattern is responsible (or is used) to establish $n$ amount of conflicts, then there will be $n$ amount of conflict patterns, and they will all be encapsulated within one composite conflict pattern. This is beneficial when we need to check and evaluate properties of the conflicts, such as the amount of conflicts associated to one single structure pattern (burdening) or the amount of self-conflicts. Due to this nature, CCP do not calculate a quality per se, rather their quality is cumulative from all the $ConfP$ they contain normalized by the amount of $ConfP$. 
% %there will be one CCP 

% \paragraph{Derivative Pattern} Derivatives defines a relationship between tropes connected by "entails" connections ($\diamondsuit$---). Derivatives start from a root micro-pattern and continue until no more "entail" connections are encounter, effectively establishing a hierarchy from the root to the derivatives. By design, the patterns within a derivative pattern have a local and temporal order, and a causal relationship. For instance, in the example given at the start of this section (EMP $\diamondsuit$--- DRA $\diamondsuit$--- NEO); having a conflict or engaging with the \emph{EMP}, entails both the conflict with \emph{DRA} and the appearance of \emph{NEO}. In this context, this means that only by "beating", "winning against", or any other way of overcoming the \emph{DRA} we will make \textit{narrative space} for \emph{NEO} to appear - as a new hero or the evolution of another. The \textit{narrative space} is crucial as it helps building up situations and plot points, and create meaningful interactions. The quality of a derivative pattern $Derp_{qual}$ is calculated based on its quantity on the target narrative graph, the balance of derivatives within this pattern using a gaussian bell centered on the avg. in the tested narrative graph ($\theta$), and the diversity of the derivatives.

% \begin{equation}
%     Derp_{qual} = generic_{qual}(Derp) + 
%     \theta + 
%     \frac{\sum_{i=0}^{|Derp_{der}|}i_{base}}{|Derp_{der}|}
% \end{equation}

% % , and have, by design, a local temporal order and causal 

% \paragraph{Reveal Pattern} A reveal connects two independent characters as one, meaning that character A was in fact always character B, and vice-versa. This pattern serves for creating confusion and surprise within a narrative structure, as for instance, a villain could have been in fact "Good All Along"\footnote{https://tvtropes.org/pmwiki/pmwiki.php/Main/GoodAllAlong}. In practice, a reveal pattern is identified as a villain or hero connected with a uni-directional connection ($\rightarrow$) to a hero or villain, respectively. As a consequence, all existing conflicts between factions would become \emph{fake}. $rev_{qual}$ is calculated based on its quantity on the target narrative graph, the amount of total reveals in the tested narrative graph in relation to characters, and the amount of generated fake conflicts given this specific reveal.

% \begin{equation}
%     rev_{qual} = generic_{qual}(rev) + repetition_{qual}(rev) + \Bigg( 1.0 - \frac{\sum_{i=0}^{|conf|}    \begin{cases}
%         1,& \text{if } rev \in x_{i}\\
%         0,              & \text{otherwise}
%     \end{cases}}{|ng_{conf}|} \Bigg)
% \end{equation}

% % the balance of derivatives within this pattern using a gaussian bell centered on the avg. in the tested narrative graph ($\theta$), and the diversity of the derivatives. 

% % Reveal patterns connect and relate a character with an opposing, as both are the same but in different factions.

% \paragraph{Active Plot Device Pattern} Plot device patterns by themselves only describe a goal or target to be achieved within a narrative. However, they do not operationalize and integrate those within the narrative. Once an PDP is in use by connecting to another micro-pattern and having other micro-patterns connected to it, then they do key contributions to the narrative. Almost all narrative make use of a plot device to move forward the story; thus, in Story Designer PDPs get utilize quickly as well. In practice, an \textit{APD} is identified as PDPs that contains at least any one connection to them, and optionally, one single output connection. These limitation is added to limit the effect of a PDP wihthin a narrative. The $APD_{qual}$ is measured based on the amount of APDs within the target graph, and the APD's usability, calculated based on the amount of incoming connections and output normalized using a gaussian bell centered on the half size of the tester narrative graph defined as $\theta$. Usability is calculated like this to penalize APDs that do not use as many connections as possible but neither be the only APD in place.

% \begin{equation}
%     APD_{qual} = generic_{qual}(APD) + \theta
% \end{equation}


% \paragraph{Plot Elements} 

% Plot elements are treated as composite patterns and meso-patterns, but rather than pointing some underlying trope such as \emph{reveal}, they describe key elements within the current narrative structure. Plot elements can then be used to assess semantically the narrative graph and to inform other systems about main events within the structure.

% % They are technically composite patterns as the meso-patterns, but are linked to plot information and description.

% \paragraph{Plot Points} Plot points are key events happening within the narrative graph, identified as discrete moments given some pattern. At the moment, we consider as plot points, the derivatives within a \textit{Derivative pattern}, the reveal pattern's source, and the plot devices that are \textit{Activate plot Device}. Quality $PP_{qual}$ is measured based on the amount of PPs within the target narrative graph, and the amount of PPs within the tested narrative graph, in relation to the nodes within it.

% \begin{equation}
%     PP_{qual} = generic_{qual}(PP) + quantity_{qual}(PP)
% \end{equation}

% \paragraph{Plot Twist} Plot twists take advantage of plot points to identify those that could have a bigger impact, surprise a player, or change drastically the narrative. In practice, \emph{PTs} consider the source of \textit{RevP}, derivatives from \textit{DerP} that are a different micro-pattern than the root, and \textit{APDs} that are connected to other \textit{APDs}. For instance, reusing the same example of EMP $\diamondsuit$--- DRA $\diamondsuit$--- NEO, given that NEO is a different micro-pattern than root EMP, this will be identified as a \textit{Plot Twist} as it alters the "natural" order in the derivative. PTs quality ($PT_{qual}$) is based on the amount of PTs within a target narrative graph, the PT's degree of involvement in the narrative, and the balance of PTs in function of the PPs in the tested narrative graph. When a PT is related to a \textit{RevP}, involvement is calculated as how much the structure changes based on that (i.e., how many fake conflicts are created). When it is related to \textit{DerP}, involvement is calcualted as how different the pattern is and its order within the derivatives. Finally, when it is related to \textit{APD}, involvement is based on how usable the \textit{APD} is within the narrative based on incoming and outgoing connections.

% \begin{equation}
%     PT_{qual} = generic_{qual}(PT) + involvement_{qual}(PT, assoc_{pat}) + \frac{|ng_{pt}|}{|ng_{pp}|}
% \end{equation}

% \paragraph{Auxiliary Patterns}

% Denotes problems in the graph, and sub-optimal and impractical nodes and connections within a graph.

% \paragraph{Nothing} Nodes that are not assigned or identified within any of the meso-patterns, are categorized as \textit{Nothing} within Story Designer. This means that their contribution to the overall narrative is none-existent; thus, pointing towards "degenerated" narratives. 
% \paragraph{Broken Link} A node might be useful within a narrative but not all of its connections. Therefore, unused connections within a graph are assigned a \textit{Broken Link} pattern to identify those that do not contribute to a coherent and interesting narrative structure.

% \paragraph{Micro-Patterns}

% \paragraph{Structure Pattern}
% \paragraph{Hero Pattern}
% \paragraph{Villain Pattern}
% \paragraph{Plot Device Pattern}

% \paragraph{Meso-Patterns}

% \paragraph{Conflict Pattern} Relates to the conflict between two character micro-patterns. A conflict $C$ is formally defined as $\langle s, C, t \rangle$ where s and t are a character micro-pattern (i.e., either hero pattern or villain pattern). $C$ denotes the general conflict trope, which can be simple conflict (c), conflict against nature (cona), or conflict against society (coso). A conflict meso-pattern do not necessarily describe a conflict between two different characters, as conflicts can be self-conflicts expressed as a bi-directional connection between the character trope and conflict trope. This effectively makes $s$ and $t$ the same. 

% Moreover, a conflict pattern is either~\textsc{explicit} or~\textsc{implicit}. 

% % and must be opposite to each other (e.g., if $s$ is a hero pattern, then $t$ must be a villain pattern.
% % \paragraph{Implicit Conflict Pattern}
% \paragraph{Composite Conflict Pattern}
% \paragraph{Derivative Pattern}
% \paragraph{Reveal Pattern}


% \paragraph{Plot Elements}

% \paragraph{Plot Points}
% \paragraph{Active Plot Device Pattern}
% \paragraph{Plot Twist}

% \paragraph{Auxiliary Patterns}

% \paragraph{Nothing}
% \paragraph{Broken Link}




\subsubsection{Evolving narrative structures with Graph Grammars}

The underlying evolutionary algorithm in Story Designer is an adapted version of IC MAP-Elites~\cite{p11alvarez_interactive_2020} to evolve grammars. In Story Designer, an individual's phenotype is a narrative graph, and its encoding genotype is a graph grammar. A graph grammar is a context-free grammar whose productions add, remove, and modify nodes and edges to a graph. 

An individual's genotype is the production rules of the grammar, which are deterministic i.e., a production rule (or pattern) only matches one production. Given that the graph grammar does not need to be applied sequentially until terminal nodes are reached, every individual does a random sampling of the rules in their genotype to produce \emph{recipes}. \emph{Recipes} simply describe the order of rules to be applied (sequentially) and the amount of times they will be applied. \emph{Recipes} do not have repetitions within them i.e., if rule 1 is added at step 2, subsequent addition would simply add to the amount of times that rule will be applied at step 2. The internal parts of the EA works exactly as in TropeTwist, but now it is extended to use all the capabilities of IC MAP-Elites, namely, the continuous adaptive evolution aspect~\cite{p11alvarez_tropetwist_2022,alvarez_interactive_2020}.

%\begin{table}[]
\caption{Level constraints used per Experiment. Constraints were chosen based on the maximum amount of elements needed to design the narrative structure in the system.}
\begin{tabular}{l|llll}
Constraining elements & Exp. 1 & Exp. 2 & Exp. 3 & Exp. 4 \\ \hline
Heroes                & 2            & 2            & 4            & 2            \\
Enemies               & 2            & 2            & 1            & 2            \\
Quest Items           & 2            & 3            & 1            & 2           
\end{tabular}
\label{tab:level-constraints}
\end{table}

%Graph grammars do not apply rules sequentially; instead, every individual does a random sampling of the rules in their genotype to produce \emph{recipes} to generate a narrative graph. \emph{Recipes} describe the rules' order and repetition. \emph{Recipes} do not have repetitions within them, i.e., if rule 1 is added at step 2, subsequent addition would simply add to the number of times that rule will be applied at step 2. 

%The EA worksThe EA works exactly as 


%Our implementation uses the tropes listed in Table \ref{tab:tropes} as nodes, and the three available connection types as edges. Figure~\ref{fig:gen2phen} shows a sample complete process from an individual's genotype (i.e., rules) to the phenotype (i.e., narrative graph).

%Figure XXX shows a sample individual's genotype (grammar) and phenotype (narrative graph).

% individual's genotype is the production rules of the grammar, which are deterministic i.e., a production rule (or pattern) only matches one production. Given that the graph grammar does not need to be applied sequentially until terminal nodes are reached, every individual does a random sampling of the rules in their genotype to produce \emph{recipes}. \emph{Recipes} simply describe the order of rules to be applied (sequentially) and the amount of times they will be applied. \emph{Recipes} do not have repetitions within them i.e., if rule 1 is added at step 2, subsequent addition would simply add to the amount of times that rule will be applied at step 2. Given that production rules can grow without limit and are always minimum 2, the amount of recipes can escalate quickly (e.g., if we have 5 rules, the permutations are calculated as $5!$, resulting in 120 possible recipes without counting the amount of times the rule can be applied). Therefore, we limit the system and all experiments to $10$ recipes regardless of the chromosome size.

%Maybe show first the fitness evaluation and then go and jump to the dimensions.

Moreover, thanks to continuous evolution, the EA constantly incorporates the most recent version of the user’s design to the population of individuals in the corresponding cell of the feature-dimension matrix. The designer can switch between dimensions at any given time, as well as changing their granularity. IC MAP-Elites manages two different populations within each cell: a feasible and an infeasible one. Individuals move across cells when their dimension values change, or between the feasible and infeasible population according to their fulfillment of the feasibility constraint. Narrative graphs are deemed infeasible if they are not fully connected (i.e., all nodes can be reached from an arbitrary starting point), if there exist a conflict pattern within the graph with more than one self-conflict, or if level design constraints are enabled, the narrative graph violates any level design constraint. Infeasible individuals are evaluated (equation~\ref{eq:inf_fitness} in a weighted sum ($w_{0}=0.5, w_{1}=0.25, w_{2}=0.25$) based on how close they are to being fully connected and to remove inadequate self-conflicts, while trying to maximize the graph's cohesion.

%\begin{equation}
%\label{eq:cohesion_fitness}
%f(cohesion) = w_{0} *  \frac{\#NOT_{pat} + \#BROL_{pat}}{|V(NG)| \times 2} + w_{1} *  \frac{\#NOT_{pat} + \#BROL_{pat}}{\#micro_{pat} \times 2} 
%\end{equation}

\begin{multline}
\label{eq:inf_fitness}
f(infeasible) = w_{0} \times f(cohesion) + w_{1} \times \frac{\#!reachable_{V(NG)}}{|V(NG)|} 
\\ + w_{2} \times  \frac{\#!valid_{NG(self_conf)}}{|V(NG)|}
\end{multline}

Generated narrative graphs that are deemed feasible, are evaluated on their coherence (equation~\ref{eq:coherence_fitness}), which is used to assess how correct, coherent, and in general, syntactically correct the narrative graphs are. Coherence aims at maximizing an equally weighted sum between cohesion and consistency (eq.~\ref{eq:consistency_fitness}). Cohesion refers to the link between elements that hold together to form some group, which in Story Designer means the minimization of auxiliary patterns (\textit{Nothing} and \textit{Broken Link}) within the narrative graph. Consistency means that the narrative graph should be regular and free of contradictions, aiming at maximizing the quality of micro-patterns and minimize contradictions created by meso-patterns (contradictions can affect the consistency fitness up to $w_{0}=0.3$). For a more detailed explanation of how the EA works internally and the different fitness functions, we refer to the TropeTwist paper~\cite{p11alvarez_tropetwist_2022}.

%.. Thus, we calculate $f(consistency)$ as the collective quality of micro-patterns (regarded as fundamental units that should be somewhat regular and useful) normalized by the micro-pattern count $|micropat|$, and the goodness of explicit and implicit conflicts based on the amount of fake conflicts that exist. In effect, with consistency we aim at minimizing contradictions created by meso-patterns that affect micro-patterns and improving the general quality of the four micro-patterns.

%A consistent NG should be regular and free of contradictions. Thus, we calculate $f(consistency)$  (eq.~\ref{eq:consistency_fitness}) as the collective quality of micro-patterns since they are the building blocks, and conflicts' goodness based on the number of fake conflicts. In effect, with consistency, we aim to maximize the quality of micro-patterns and minimize contradictions created by meso-patterns.

%With cohesion, we focus on minimizing the   calculated as a equally weighted sum that aims at maxizming

%- Cohesion: Linking between words to hold together the text (how related the tropes are??)
 %*       --> Probably we can use something like, if a pattern is "nothing" there are cohesion problems?

%Coherence is calculated as a weighted sum ($w_{0}=0.7, w_{1}=0.3$)


%In addition to the feature-dimensions, all individuals are evaluated according to the following fitness function:

%\begin{equation}
%\label{eq:consistency_fitness}
%f(consistency) = w_{0} \times \frac{\sum_{i=0}^{|ng_{micro}|} i_{qual}}{|ng_{micropat}|} +  \\ 
%w_{1} \times \frac{|ng_{fakeConfP}|}{|ng_{confP}|} 
%\end{equation}

\begin{equation}
\label{eq:consistency_fitness}
f_{consistency} = \frac{\sum_{i=0}^{len(ng_{micro})} i_{qual}}{len(ng_{micropat})} -  \\ 
w_{0} \times \frac{len(ng_{fakeConfP})}{len(ng_{confP})} 
\end{equation}

\begin{equation}
\label{eq:coherence_fitness}
f(coherence) = f(consistency) + (1.0 - f(cohesion))
\end{equation}

\subsubsection{Behavior Dimensions for Graph Grammars}

Dimensions in MAP-Elites are a key component for the search space to be delimited, and are identified as those aspects of the individuals that can be calculated in the behavioral space, and that are independent of the fitness calculation. In Story Designer, the designer is able to pick two dimensions at a time to facilitate visualization, and all dimensions, when needed, are limited using a threshold $\delta = 5$. TropeTwist implemented \textit{Interestingness} and \textit{Step} as behavior dimensions when using MAP-Elites to generate novel narrative graphs. \textit{Step} is calculated as the Levenshtein distance between two narrative graphs, taking into account the amount of nodes and connections and their type (eq. \ref{eq:StepDim}). \textit{Interestingness} make use of the APDs, Plot Points, and Plot Twists that are present in a narrative graph to assess an approximate semantic evaluation since those represent some type of variation in the graph (eq.~\ref{eq:interesting_fitness}). Given \textit{Interestingness} is a highly subjective measurement, we rely on those patterns since they calculate their quality based on the current narrative graph and the one being edited by the designer.

\begin{equation}
\label{eq:StepDim}
D_{step} =  \frac{lev_{a,b} (|a|, |b|)}{\theta}
\end{equation}

\begin{equation}
\label{eq:interesting_fitness}
D_{int} = w_{0} \times \frac{APD_{q}}{\#APD} + w_{1} \times \frac{\#PP_{q}}{\#PP} +  w_{2} \times \frac{PT_{q}}{\#PT}ng
\end{equation}

Furthermore, we have extended TropeTwist with five more dimensions relevant to the narrative structure design process, to give more choice to designers and experiment with other dimensions in the search space:


%Furthermore, This EA is wrapped by an IC MAP-Elites \cite{p11alvarez} that evaluates each phenotype (narrative graph) according to the a set of feature-dimensions that assess different aspects of the narrative graphs:

%Given that the patterns active plot devices, plot points, and plot twists can have a high impact in the narrative graph and depend on several other patterns and specific structures, we limit all of then with threshold $\delta = 5$.

\begin{figure*}[t]
    \centering
    \includegraphics[width=\textwidth]{figures/example-experiments.png}
    \caption{Narrative graphs used for the experiments, constructed and designed in Story Designer. When experiment 4 is discussed, the narrative graph referred is Experiment 4.5 as that is the design's final step.}
    \label{fig:examples}
\end{figure*}

\textbf{Diversity (div).} Diversity measures the variety of [base] trope types within a narrative structure. Currently, there exist four base trope types, \textit{Hero} (h), \textit{Villain} (v), \textit{Structure} (s), \textit{Plot Devices} (pd). Diversity takes into account the tropes that also extend these base tropes. Thus, $D_{div}$ collects all tropes within a graph, and increase a counter for each of the base trope type ($NG_{base} = h, v, s, pd \in NG$), normalized by the max amount of base trope types depicted in Eq.~\ref{eq:diversityDim}:

%\textbf{Diversity (div).} Diversity measures the variety of [base] trope types within a narrative structure. Currently, there exist four base trope types, \textit{Hero} (h), \textit{Villain} (v), \textit{Structure} (s), \textit{Plot Devices} (pd). Diversity takes into account the tropes that also extend these base tropes shown in table~\ref{tab:tropes}. Thus, $D_{div}$ collects all tropes within a graph, and increase a counter for each of the base trope type ($NG_{base} = h, v, s, pd \in NG$), normalized by the max amount of base trope types depicted in Eq.~\ref{eq:diversityDim}:

\begin{equation}
\label{eq:diversityDim}
D_{div} =  \frac{NG_{base}}{\#Trope_{base}}
\end{equation}

%Equation~\ref{eq:diversityDim} shows  is calcula
%A measurement of the variety of trope types included in the graph's nodes in relation to its size.

\textbf{Conflict (confs).} Since we already calculate all patterns within a narrative graph, conflict simply calculates the amount of \emph{explicit} conflict patterns ($\#NG_{c} = C_{exp} \in allPatterns$) that exist within a narrative graph normalized by a conflict threshold $\omega = 5$. We use $\omega$ to avoid stimulating the generation of narrative graphs with a massive amount of conflicts, which could create noise in the evolution and focus on the conflicts rather than other tropes and patterns. $D_{confs}$ is then calculated as $\frac{\#NG_{c}}{\omega}$.

%Equation~\ref{eq:ConfsDim} shows the final calculation of $D_{confs}$.

%\begin{equation}
%\label{eq:ConfsDim}
%D_{confs} =  \frac{\#NG_{c}}{\omega}
%\end{equation}

\textbf{Plot points (pp).} Plot points measures the amount of plot points within a narrative graph ($\#NG_{pp} = pp \in allPatterns$) and normalize it by $\delta$. Given that plot points are dynamically assessed based on other patterns and combination of tropes, we limit the dimension with $\delta$ to avoid losing coherence in favor of generating more plot points. $D_{pp}$ is calculated as $\frac{\#NG_{pp}}{\delta}$.

\textbf{Plot Twist (pt).} Plot twist measures the amount of plot twists within a narrative graph ($\#NG_{pt} = pt \in allPatterns$) and normalize it by $\delta$. Plot twists relate to special situations within a narrative graph where the somewhat abrupt change in the tropes or combination of tropes could alter the narrative and create a surprise effect. Therefore, we limit the dimension with $\delta$ to avoid ``degenerating" narratives with too many twists. $D_{pp}$ is calculated as $\frac{\#NG_{pt}}{\delta}$.

\textbf{Plot devices (pd).} Plot devices measure the amount of active plot devices within a narrative graph ($\#NG_{pd} = apd \in allPatterns$) and normalize it by $\delta$. Plot devices create targets and goals within a narrative, and active plot devices operationalize these in the narrative graph associating them with multiple tropes; thus, similar to \emph{pt}, we limit with $\delta$ to avoid ``degenerating" the narrative. $D_{pd}$ is calculated as $\frac{\#NG_{pd}}{\delta}$.



%How many plot devices are included in the graph in relation to its size

%therefore, we leverage from the \emph{plot point}, \emph{plot twist}, and \emph{active plot device} patterns and the ratio of fake conflicts to measure the \emph{int} of the generated narrative graphs.

%\textbf{Interestingness (int).} With interesting, we aim at measuring the semantic quality of a narrative graph. A narrative graph can be syntactically correct and coherent yet do not have a good (or existent) semantic quality and do not evoke interest for designers and players. Therefore, we leverage from the \emph{plot point}, \emph{plot twist}, and \emph{active plot device} patterns to measure the \emph{int} of the generated narrative graphs. Given that \emph{int} combines these three qualities designers might not be interested in highly\footnote{Interesting-boring qualities are subjective measurements, thus what is highly interesting in our system might not necessarily be for a designer, which is why (in part) we leverage on the narrative graph created by the designer to measure and evaluate patterns and their quality.} interesting generated narrative graphs as they could degrade their narrative objective. Furthermore, the nature of \emph{int} creates pressure on the fitness function since the incidence of the three above-mentioned patterns could (if overused) "degenerate" the narrative; thus, decreasing its coherence. $D_{int}$ is calculated as the weighted sum ($w_{0}=0.4, w_{1}=0.2, w_{2}=0.4$) of the cumulative quality of \emph{plot twists} and \emph{active plot devices} within a narrative graph normalized by their count, and the \emph{plot point} count normalized by the amount of nodes in the graph (equation~\ref{eq:interesting_fitness}).



%\begin{equation}
%\label{eq:fitness}
%f(narrative) = w_{0} * f(interesting) + w_{1} * f(coherence)
%\end{equation}

%The fitness function encompasses a weighted sum between \textit{interest} and \textit{coherence}.



% % Please add the following required packages to your document preamble:
% \usepackage{graphicx}
\begin{table}[]
\centering
\begin{tabular}{|l|lll|}
\hline
        & AIv1       & AIv2        & AIv3        \\ \hline
Leniency        & 0.56±0.07  & 0.62±0.09   & 0.57±0.08   \\
Linearity        & 0.91±0.02  & 0.92±0.02   & 0.91±0.01   \\
MesoPat       & 0.15±0.05  & 0.13±0.07   & 0.12±0.05   \\
SpatialPat    & 0.35±0.1   & 0.41±0.11   & 0.34±0.09   \\
Symmetry   & 0.43±0.11  & 0.35±0.18   & 0.35±0.12   \\
$W_{dens}$ & 0.27±0.09  & 0.26±0.08   & 0.21±0.05   \\
$W_{spar}$ & 0.21±0.05  & 0.19±0.03   & 0.15±0.01   \\
$E_{dens}$ & 0.24±0.07  & 0.27±0.06   & 0.3±0.06    \\
$E_{spar}$ & 0.22±0.05  & 0.32±0.05   & 0.35±0.06   \\
$T_{dens}$ & 0.37±0.13  & 0.28±0.09   & 0.34±0.07   \\
$T_{spar}$ & 0.36±0.11  & 0.3±0.1     & 0.37±0.05   \\ \hline
Steps      & 39.25±6.38 & 84.31±14.85 & 76.75±17.02 \\ \hline
\end{tabular}
\caption{Summary of the created rooms filtered by the AI version used. All values are the average of all the created rooms using the specific AI version. The first five values relates to the MAP-Elites dimensions, then the fitness of the rooms, the density and sparsity values for wall (W), enemies (E), and treasures (T), and finally the avg. steps taken to design a room.}
\label{tab:AIavgValues}
\end{table}

\subsection{Experiment Setup}

We conducted a user study to explore the user experience of using different levels of AI agency, the different design characteristics, and the relationship between the human designer and the AI. We collected both quantitative data on the AI's impact on the co-designed end product and qualitative data through think-a-loud and semi-structured interviews regarding the users' experience when interacting with the AI. The interview structure is inspired by the pyramid model, meaning the interviews will begin with specific questions, and gradually have more open questions, which naturally allows for a discussion towards the end. This model is chosen to support the variation of subjects the interview is desired to cover, as well as support natural transitions between the questions and their openness. The questions and user study procedure can be found in Appendix A. 

%The interviews are semi-structured, meaning it includes both closed and more open questions, and depending on the discussion and answers, some questions might be omitted.

% We collected quantitative data regarding what impact the AI had on the co-designed end product, and how the human designer interacted with the AI's contributions. Likewise, we collected qualitative data through recorded think-aloud observations and semi-structured interviews regarding the users' experience, and possibly catch certain remarks of frustration or appreciation of their digital colleague that can be valuable for the discussion of the relationship between the co-creators. 


%We collected the following data:

%\begin{itemize}
%    \item \textbf{Audio Recordings:} 
%\end{itemize}

Eight participants tested our tool with game design and level design experience. One participant was a professional game designer with eight years of professional experience (first participant), and seven participants were third-year Game Development students. They all had an individual digital session, where we shared our screen, and they took remote control to conduct the study. Participants accepted to participate, signed consent forms, and then received a short introduction describing the experiment and its steps. The participants were then asked to design two contiguous rooms in a dungeon, repeating this process for each of the AI variants and expressing their design decisions verbally whenever they felt like it. After using the tool, the participants were interviewed, focusing on and covering an overarching understanding of the user experience, particularly in terms of creativity and interaction with the AI.

% the relationship that occurs between the AI and human designer (See Appendix B). 

For all the sessions, human designers could place up to 12 tiles, and the AI could place as many tiles as the human placed. The AI could contribute only in a rectangular area surrounding the tiles the human designer recently contributed with, including a margin of 1 tile. This choice is made to support a responsive and collaborative behavior of the AI that builds on the human designer's contribution.


% The locations available for the AI to contribute in for each turn are limited to a rectangular area surrounding the tiles the human designer recently contributed with, including a margin of 1 tile. This choice is made to support a responsive and collaborative behavior of the AI that builds on the human designer's contribution.

% The margin for the contribution area is set to 1, as it was found during experimentation that any margin bigger than this is likely perceived as the AI contributing to other areas than the ones the human is focused on, because of the default size of the room being relatively small.

%  as this enables the designer to contribute with an adequate amount of tiles during their turn and create representable structures

% The rooms produced during the user study are displayed in Figure 6, 7 and 8. Rooms with red borders are infeasible, meaning there are unreachable tiles. The UI displays a warning when this happens, and the AI can repair this during its turn, however the resulting rooms that are infeasible are a result of the human designer creating unreachable areas, and then immediately selecting to go to the World Editing view, before pressing "End Turn". 
% Each participant created two rooms for each version of the AI. Participant 1 created Room 1 and Room 2 for all version, Participant 2 created Room 3 and Room 4 for all version, etc. All of the participants had the option to adjust the sizes of the rooms in the World Editing view before entering the Room Editing view, however none of them did, and therefore all of the resulting rooms are of the default size. The designer also has the option to change the location of the hero and the doors. The location of the hero was only moved twice in all of the session, and the doors where never moved. 






%, the participant will take part in an interview. The questions, and their order, are planned out and designed to cover an overarching understanding of the user experience, in particular in terms of creativity, and the relationship that occurs between the AI and human designer (See Appendix B). 




%The participants were asked to repeat this for each of the AI-initiatives. was asked to repeatEach pair of room 

% then asked to complete three tasks, each regarding

%The users were then asked to complete three tasks that covered the tool's functionality and the AI-initiatives, respectively for each task. The tasks were 


%and different approaches to creating quests. The tasks were to 1) manually create a quest, 2) automatically create a quest, and 3) create a quest through mixed-initiative. They were also asked to create a dungeon that suited their preferences and objectives before creating quests. The questionnaire consisted of 17 closed-ended questions, and the rest were open-ended. The interview began with a questionnaire with six questions about the users' background and experience within game development and finish with questions about their experience and opinions on the tool. Both the questionnaire and interview followed guidelines described by



%The participants used the tool



\subsection{Designer Personas} \label{p6section:results}

%We  trained  3  models  for  each  representation  and  problem configuration. To analyze these models, we collected 40 generated levels for each model.  To generate the levels, 40 different random level layouts were generated, the models were then tasked with modifying these random layouts into good levels.   We  analyzed  the  final  modified  levels  using  different change percentages, ranging from0%to100%, where thepercentage represents the fraction of tiles the agent is allowedto change during inference

%To understand the typical progress of designers and validate the clustering, we visualize how typical design sessions traverse the various clusters. These trajectories  are  then  clustered  to  find  a  small  handful of designer personas.

\begin{figure*}[t]
    \centering
     \subfloat[\textsc{Architectural-focus}\label{p6subfig-1:dummy}]{%
       \includegraphics[width=0.45\textwidth]{figures/1.png}
     }
     \hfill
     \subfloat[\textsc{Goal-oriented}\label{p6subfig-2:dummy}]{%
       \includegraphics[width=0.45\textwidth]{figures/2.png}
     }\hfill
    %  \medskip
     \subfloat[\textsc{Split central-focus}\label{p6subfig-3:dummy}]{%
       \includegraphics[width=0.45\textwidth]{figures/3.png}
     }
     \hfill
     \subfloat[\textsc{Complex-balance}\label{p6subfig-4:dummy}]{%
       \includegraphics[width=0.45\textwidth]{figures/4.png}
     }
    
    \caption{Examples of each of the archetypical paths from one of the frequent sequences used to create the clusters. To the left of each subfigure, we present each key step in the trajectory i.e. when the design entered a new cluster. (a) presents the \textsc{Architectural-focus} archetypical path where the focus is firstly on creating the structural design of the rooms; the design process jumps back and forth suddenly to cluster 10 (one of the possible branches) due to the designer adding a boss, and removing it immediately. (b) presents the \textsc{Goal-oriented} archetypical path where the design focus on a minimal structure complexity and mix between adding structural changes and enemies/treasures. (c) shows the \textsc{Split central-focus} archetypical path where, intentionally, the designer creates a center obstacle with a boss and build around it. Finally, (d) presents the \textsc{Complex-balance} archetypical path; the design focuses on building complex, uncommon structures first and then add some goal to it with enemies and treasures, taking advantage of the spaces.}
    \label{p6fig:archetypical-examples}
\end{figure*}


Once we created, evaluated, and labeled the clusters, we were able to cluster and visualize the paths of a typical design session. Figure \ref{p6fig:paths-designers} presents an example of the design sessions, where we cluster each step of the design. This sequential process revealed that there is an interesting continuity between clusters, even capturing when a designer probably applied one of the procedural suggestions due to bigger steps in the design style clusters. Further, through this process, we could understand the progress of designers in their design process and represent their trajectory in relation to the traversed clusters rather than individual editions.

\subsubsection{Unique Trajectories}

Using the clusters in Figure \ref{p6fig:all-clusters}, we clustered the design session of all the $180$ designs and collected the unique trajectories that arose from traversing the various clusters. These unique trajectories varied in the starting point, length, and end-point, however, when analyzing the trajectories we identified common patterns among them. They had a similar shape as the following $Unique=$\{0\textgreater8\textgreater4\textgreater7\textgreater10\}, where the first and last element of the sequence are respectively, the starting- and end-points, with all the unique intermediate steps in between.

To gather the common patterns from the trajectories, we applied the Generalized Sequential Pattern (GSP) algorithm, which locates frequent subsequences in the analyzed trajectories. For instance, given three trajectories (a) \{5\textgreater1\textgreater3\textgreater11\textgreater9\}, (b) \{5\textgreater1\textgreater3\textgreater11\textgreater4\, and (c) \{0\textgreater1\textgreater3\textgreater11\}, none of these is a perfect match in its entirety, but GSP can spot that subsequences \{1\textgreater3\textgreater11\}, \{1\textgreater3\}, \{3\textgreater11\}, among others, appear with frequency $= 3$.

%(2) obtain only 1 pattern ($\{5>1>3>11\}$) with frequency=2, if searching from starting points. Finally, using GSP, we find $\{5>1>3>11\}$, $\{1>3>11\}$, $\{1>3\}$, $\{3>11\}$

%We collected these unique trajectories, and 
Furthermore, after doing a preliminary analysis, we identified some steps that we classified as ``border designs'': steps that are borderline between two clusters. These \textit{border designs} disrupted the sequence pattern mining by creating noise in the unique trajectories, specifically when these \textit{border designs} entered a different cluster for just a few steps. %we categorize them as when these "unique" noisy steps were brief.
Therefore, we filtered them out by applying a threshold $\theta = 3$, so that all subsequences inside one cluster with less than $\theta$ steps are removed from the main sequence. I.e, the sample trajectory \{0\textgreater0\textgreater0\textgreater0\textgreater8\textgreater8\textgreater8\textgreater6\textgreater8\} turns into \{0\textgreater8\} instead of \{0\textgreater8\textgreater6\textgreater8\}. Through this, we were able to reduce the noise and the search space, obtaining more meaningful and frequent patterns.

\subsubsection{Archetypical Paths through Style Space}

%Figure \ref{p6fig:finalPaths} shows the archetypical paths taken by designers when creating rooms. Represented as arrows to denote direction, 

%From all the collected unique trajectories, we identified 4 main archetypical paths, which are the ones taken most frequently by designers either as their full path or as the initial path. In Figure \ref{p6fig:finalPaths}, it is shown the archetypical paths, represented as thicker arrows to denote direction, that represent the taken by designers when creating rooms. 

In Figure \ref{p6fig:finalPaths}, we present the archetypical paths, represented as thicker arrows to denote direction, which show the most frequent paths taken by designers either through their whole design process or as the initial meaningful steps. From all the collected unique trajectories, we have identified 4 main archetypical paths, labelled, \textsc{Architectural-focus}, \textsc{Goal-oriented}, \textsc{Split central-focus}, and \textsc{Complex-balance}. In addition, we have numbered each cluster for easier visualization and referencing. 

Moreover, in the figure, it can also be observed thinner purple arrows pointing to different clusters from several of the clusters that are part of the main paths. These are \textit{possible branches} presented in the unique trajectories and added based on their frequency. Through these possible branches, the design of an archetypical session, can vary and extended or deviate the final design. Each archetypical path is defined and explained as follows: 

\paragraph{Architectural-focus}The path followed by this archetype focuses first on designing the architecture of the room with walls. Through this, the design focuses on shaping the visual patterns, chambers, and corridors to give a clear space for adding goals and objectives with enemies and treasures. The sequence is denoted with a green arrow in Figure \ref{p6fig:finalPaths}, and following the sequence \{0\textgreater8\textgreater3\textgreater7\}.

\paragraph{Goal-oriented}Design processes following this archetypical path, create the rooms in a more standard way, combining simpler symmetric wall structures with distributed placement of enemies and treasures. Thus, rather than focusing extensively on an individual part of the room, the rooms have an initial structure and then they are populated with some specific goal-in-mind. The sequence is denoted with a red arrow in Figure \ref{p6fig:finalPaths}, and following the sequence \{0\textgreater8\textgreater6\}.

%Thus, rather than focusing on an individual part of the room until satisfied, the rooms have some initial structures that are populated and continue through an iterative process between these steps.% rooms go through an iterative process of adding have some initial structures that are

\paragraph{Split central-focus}This archetypical path focuses on designing rooms with obstacles placed in the center of the room in the shape of enemies, treasures, or wall structures that clearly split the room into different areas. The design process is less organized than the other archetypes since it searches to achieve the split goal with any of the available tiles. The sequence is denoted with a black arrow in Figure \ref{p6fig:finalPaths}, and following the sequence \{0\textgreater5\textgreater6\}.
%, since the middle step is cluster 5 ("Separating and populating chambers"), which relates to rooms which are expected since the cluster 5 ("Separating and populating chambers") relate to rooms that  as specific structural shapes are not necessary. 

\paragraph{Complex-balance}This archetypical path focuses on building complex symmetric shapes with a clear objective for the player and adapting the spaces with a balance of enemies and treasures. In general, the rooms created following this path are more unique and typically balanced. %   with that adapt well. The process is quite 
The sequence is denoted with a blue arrow in Figure \ref{p6fig:finalPaths}, and following the sequence \{8\textgreater3\textgreater6\}.

Furthermore, using these archetypical paths, we can then categorize certain clusters as key clusters or being more relevant than others based on their contribution to the paths, their frequency, and their usage. Most of the paths go through or end in cluster 6 (``Balancing and optimizing'') and cluster 8 (``Main architectural patterns''), which relate to rooms that have a more explicit mix between corridors and small chambers and more clear architecture. The rooms in those clusters are or shaped as end rooms, as in the case of cluster 6, or architecturally shaped to be “optimized” to a specific goal e.g. a dense bordered room. Similarly, most of the sequences start from cluster 0 ("Initial room shapes"), with $134$ out of the $180$ designs, which correlates to the type of designs encountered in that clusters. Thus, it is understandable that most of the archetypical paths pass through any of these three clusters. 

Nevertheless, it is the steps in-between what creates a clear differentiation between the archetypical paths, which is the benefit of observing the design process as a whole in the clustered room style space. For instance, in fig.~\ref{p6fig:finalPaths}, it can be observed that \textsc{Split central-focus} starts in the same cluster as three other paths, and tentatively ends in the same cluster as three other. However, the designs following \textsc{Split central-focus} are more different to the other trajectories, since it enters a cluster that is denser with several tile types in principle, and where designers seem to have a clearer goal.
% With this, we can further understand why \textsc{Split central-focus} is more different to the other trajectories, since it enters a cluster that is "less organized" in principle. 

%Furthermore, we can also observe that certain clusters are key steps for most paths because they are or a frequent starting or ending point. Most of the paths go through or end in cluster 6 ("Balancing and optimizing") and cluster 8 ("Main structural patterns"), which relate to rooms that have a more explicit mix between corridors and small chambers and a more clear structure, thus, it is understandable since the rooms in those clusters are or shaped as end rooms, as in the case of cluster 6, or structurally shaped to be “optimized” to a specific goal (E.g. dense bordered room, maze-like, more challenging, etc.). However, the distribution of endpoints is quite even, and meanwhile, cluster 6 and cluster 11 ("Chamber separation, Forced enemy encounter") are the most frequent ending points with $36$ and $25$ out of $180$ design processes, the rest of clusters are quite close.

%Similarly, most of the sequences start from cluster 0 ("Initial room shapes"), with $134$ out of the $180$ designs, which correlates to the type of designs encountered in those clusters.

Moreover, in figure~\ref{p6fig:archetypical-examples}, we present examples of each of the designer personas by visualizing the sequence of steps done in representative design sessions, showing how these paths would look like in practice. Each visualization of a designer persona has the key design steps to the left, where each image is in a sequence: the first is the first edition of the designer, the last is the final edition, and the in-between represent entering a new room style cluster. 

In (a), it is shown the \textsc{Architectural-focus}, where the designer first created the border of the room with a clear chamber division. As the designer adds and subsequently removes the boss, the design jumps to cluster 10, which is one of the possible branches, adding a high challenge. In (b), it is shown the \textsc{Goal-Oriented}, where the designer sketched the main shape of the room followed by alternating between enemies, treasures, and walls to design the goal of the player within the room. In this example, the designer ends the design close to cluster 9, with a disorganized placement of tiles and a less aesthetical room, but forming small choke areas balancing the placement of enemies and treasures.

In (c), it is shown the \textsc{Split Central-focus}, where the designer directly started by adding a boss in the center of the room and using this as a reference point, shaped the rest of the room. In (d), it is shown the \textsc{Complex-balance}, where the designer focused on creating an uncommon structure and followed by adding enemies and treasures symmetrically, with clear individual areas for the player to approach.

% It is not surprising to focus on the center as it 

Finally, further analyzing figure~\ref{p6fig:archetypical-examples}, it can also be observed an interesting dual tendency of the designers in the archetypical paths. This dual tendency is to either focus on the aesthetic configuration of the room based on what is perceived in the editor exemplified the personas: \textsc{Architectural-focus} and \textsc{Split central-focus}, and to focus on the player experience exemplified the personas: \textsc{Goal-oriented} and \textsc{Complex-balance}. Nevertheless, both are not mutually exclusive, instead this illustrates adequately the dualistic role the designer has when using the tool and designing rooms. That of creating an aesthetically pleasing object as it is seen in the editor, and that of creating an experience.% However, this is not mutually exclusive. Instead, it shows 


% Furthermore, when analyzing how the different design sequences were clustered and forming the designer personas, we observed an interesting dual tendency of the designers. This dual tendency is to either focus on the aesthetic configuration of the room based on what is perceived in the editor through the personas: \textsc{Architectural-focus} and \textsc{Split central-focus}, and to focus on the player experience through the personas: \textsc{Goal-oriendted} and \textsc{Complex-balance}. This exemplified quite good 
% dualistic role 
% When forming the designer personas, and analyzing how different design sequences


%where the designer focused on creating the shape of the room before adding any enemyfirst created the border of the room

% Finally, in Figure \ref{p6fig:archetypical-examples}, we present examples of each of the archetypical paths to show how would these paths look like in practice, further supporting our findings and path definitions. 

% The archtypical paths a dual tendency of the designers to either go for a strategy that reflects their perception of the level from the editor - like the aesthetic configurations of it, instead of the experiential ones. for example the ones that had a split central focus and a structural focus (which btw maybe i would change to architectural focus). and then there's the ones that have a focus on the player experience like the goal oriented and complex behavior ones. i think this split reflex a very nice dualistic role that the designer has in front of the editor - that of creating an aesthetically pleasing object, as they see it in the editor, and that of creating an experience.
% , and exmplified
%That build around



% Notice that not all the clusters have connectionthat based on the unique trajectories, where designers decided to move towards other areas. 

%it is shown the archetypical paths, represented as thicker arrows to denote direction, that represent the taken by designers when creating rooms. 
% moving from 6 to 11 or from 3 to 11 is a fairly frequent step, thus making it and cluster 6, key points.
% In fact, ending the design at 7 is not that common, thus, 



% the red cluster with $95$ out of the $180$, and from the purple cluster with $49$ out of the $180$, which correlates to the type of designs encountered in those clusters, mainly emptier rooms with initial sketches and shapes. 

% Most of the paths start in cluster 0 ("Initial room shapes") 

% From the figure, we can extract the following clusters as key steps for most of the patterns: "light green", "light blue", "red", and "purple". 

% It can be observed that most of the paths end or go through the "light green" and "light blue" clusters. Both relate to rooms that have a more clear structural pattern and more explicit mix between corridors and small chambers, thus, is understandable since the rooms in those clusters are or shaped as end rooms or structurally shaped to be “optimized” to a specific goal (E.g. dense bordered room, maze-like, more challenging, etc.). Quantitatively, most of the sequences end up in those clusters, $64$ out of the $180$ end in cluster 6 (light blue) and $52$ out of the $180$ end in cluster 11 (light green).


% Similarly, most of the sequences start from the red cluster with $95$ out of the $180$, and from the purple cluster with $49$ out of the $180$, which correlates to the type of designs encountered in those clusters, mainly emptier rooms with initial sketches and shapes. 


% \begin{figure*}
% \centerline{\includegraphics[width=0.7\textwidth]{figures/path-examples-2.png}}
% \caption{Examples of each of the archetypical paths from one of the frequent sequences used to create the clusters. To the left of each subfigure, we present each key step in the trajectory i.e. when the design entered a new cluster. (a) presents the \textsc{Structural focus} archetypical path where the focus is firstly on creating the structural design of the rooms; the design process jumps back and forth suddenly to cluster 10 (one of the possible branches) due to the designer adding a boss, and removing it immediately. (b) presents the \textsc{Goal-oriented} archetypical path where the design focus on a minimal structure complexity and mix between adding structural changes and enemies/treasures. (c) shows the \textsc{Split central-focus} archetypical path where intentionally, the designer creates a center obstacle with a boss and build around it. Finally, (d) presents the \textsc{Complex-balance} archetypical path; the design focuses on building complex uncommon structures first and then add some goal to it with enemies and treasures, taking advantage of the spaces.} \label{p6fig:archetypical-examples}
% \end{figure*}



% \begin{subfigure}[t]{0.33\textwidth}
%         \centering
%         \includegraphics[width=0.95\textwidth]{figures/1.png}
%         \caption{Linearity-\#MesoPatterns}
%     \end{subfigure}%
%     \begin{subfigure}[t]{0.33\textwidth}
%         \centering
%         \includegraphics[width=0.95\textwidth]{figures/1.png}
%         \caption{Linearity-\#MesoPatterns}
%     \end{subfigure}%
%     \begin{subfigure}[t]{0.33\textwidth}
%         \centering
%         \includegraphics[width=0.95\textwidth]{figures/1.png}
%         \caption{Linearity-\#MesoPatterns}
%     \end{subfigure}%
%     \begin{subfigure}[t]{0.33\textwidth}
%         \centering
%         \includegraphics[width=0.95\textwidth]{figures/1.png}
%         \caption{Linearity-\#MesoPatterns}
%     \end{subfigure}%

%From the figure, it can be observed that most of the paths end or go through the "light green" and "light blue" clusters. Both relate to rooms that have a more clear structural pattern and more explicit mix between corridors and small chambers, thus, is understandable since the rooms in those clusters are or shaped as end rooms or structurally shaped to be “optimized” to a specific goal (E.g. dense bordered room, maze-like, more challenging, etc.). Further, most of the sequences end up


%Most of the rooms end there, 64 end in cluster 6 (light blue) and 52 end in cluster 11 (light green), so it makes sense that they are key steps in most of the subsequences. Further, most of the rooms start in cluster 0 (red) – 95 rooms – or in cluster 8 (purple) – 49 rooms, which make them key steps as well

%Most of the sequences with $95$ out of the $180$ starting in the red cluster, and $49$ out of the $180$ starting in the purple cluster. which correlates to the type of designs encountered in those clusters, mainly emptier rooms with initial sketches and shapes. 


%Info about the trajectories, 1) variation (length), 2) starting cluster, 3) end-point cluster.


% \begin{itemize}
%     \item[\textbf{DONE:}] Present 3 representative examples where we cluster each step of the design process. Perhaps I should create a room that would go through all the clusters?
%     \item[\textbf{DONE:}] Explain that we did this for all the 180 designs, and we collected the unique trajectories along the clusters, reducing the dimensionality of each step to each cluster.
%     \item[\textbf{DONE:}] Due to border designs (step that are in the border between 2 different clusters), we applied a threshold to reduce the noise those inputs could have when clustering the trajectories of the designer.
%     \item[\textbf{DONE:}] this data (the sequences) were then applied the GSP algorithm, a subsequence frequent pattern mining algorithm, to extract the frequent patterns in the sequences (including subsequences within the sequence).
%     \item This resulted in the following trajectories, which can also be observed in Figure X.
%     \item 
% \end{itemize}

\subsection{Conclusions}



% \begin{itemize}
%     \item Discussion on what does this archetypical design trajectories mean?
%     \item how to use them? next steps into integrating this into a system. To use this in a search-based approach as objectives for the generation to move towards the directions where (according to our archetypical design trajectories) the designers will move towards in their design process. Perhaps I could also bring the discussion from the workshop-paper for HC-AI.
%     \item discussion on creativity? is the output or the process where the actual creativity is outputted? Compare using end-design clustering to using sequences to cluster.
%     \item Discussion on how PCGRL relates to this type of work? --> Perhaps this is something for the background instead.
% \end{itemize}{}


% This paper presents a step towards designer modelling in a MI-CC environment by providing an implementation of designer personas as archetypical trajectories through style space, as a means to characterize several representative and frequent design styles together. 

%This paper presents a novel approach and meaningful steps towards designer modeling in an MI-CC environment. By providing an implementation of designer personas as archetypical trajectories through style space, we show that 

This paper presents a novel approach and meaningful steps towards designer modeling through an experiment on archetypical design trajectories analysis in an MI-CC environment. Through this, we characterize several representative design styles as designer personas. We have first run and compared several clustering setups to find the best partitioning of the design style using the edition sequences of the collected $180$ unique rooms, ending in $8196$ data points, and resulting in a set of twelve cohesive, coherent, and meaningful clusters. We have then mapped these $180$ design sequences in terms of these clusters, applying frequent sequence mining to find four frequent and unique designer styles, with related common sub-styles. As a result, we have presented a roadmap of design styles over a map of data-driven design clusters. 

%This paper presents a step towards designer modeling through an experiment on archetypical design trajectories analysis in an MI-CC environment, as a means to characterize several representative design styles as designer personas. We have first run and compared several clustering setups to find the best partitioning using the edition sequences of the collected $180$ unique rooms, ending in $8196$ data points, and resulting in a set of twelve cohesive, coherent, and meaningful clusters. We have then mapped these $180$ design sequences in terms of these clusters, applying frequent sequence mining to find four frequent unique designer styles, with related common sub-styles. As a result, we have presented a roadmap of design styles over a map of data-driven design clusters. %The examples in Figure \ref{p6fig:archetypical-examples}, help us to clarify 

%  namely the \textsc{Designer Personas}

% Our work draws on the ideas, concepts, and goals and concepts proposed by Liapis et al. when introducing the Designer Modeling as a model to capture multiple designer's processes. A prototype of such was implemented in the sentient sketchbook~\citepsixth{p6Liapis2014-designerModelImpl}, where it is proposed the use of interactive evolution by biasing the search space in favor of hand-crafted features of the design. we propose an alternative and novel route to designer modeling through clustering the design space and the room style based on the collected data. Moreover, we differ as well on the type of level design, being the sentient sketchbook a tool for strategy games~\citepsixth{p6liapis_generating_2013}, while EDD a tool for adventure and rogue-like games~\citepsixth{p6Alvarez2020-ICMAPE}. These differences strengthen the importance and usefulness of designer modeling, and highlight the holistic and generic properties of this designer-centric perspective.

Designer modeling was proposed as an approach to capture multiple designer's processes to create a better workflow by Liapis et al.~\citepsixth{p6Liapis2013-designerModel}, and our work draws on many of their ideas, concepts, and goals. Furthermore, a prototype of such was implemented in the sentient sketchbook~\citepsixth{p6Liapis2014-designerModelImpl}, where it is proposed different approaches to model style, process, and goals based on choice-based evolution and the designer's current design to adapt the provided suggestions accordingly. We propose an alternative route to designer modeling through clustering the design space and the room style based on the collected data. Moreover, we differ in the type of level design, being the sentient sketchbook a tool for strategy games~\citepsixth{p6liapis_generating_2013}, while EDD is a tool for adventure and rogue-like games~\citepsixth{p6Alvarez2020-ICMAPE}. These differences strengthen the importance and usefulness of designer modeling and highlight the holistic and generic properties of this designer-centric perspective and its possibilities.

% Designer modeling in computer-aided design tools was proposed by Liapis et al.~\citepsixth{p6Liapis2013-designerModel} as an approach to capture multiple designer's processes to create a better workflow, and a prototype of such was implemented in the sentient sketchbook~\citepsixth{p6Liapis2014-designerModelImpl}. While our work drags on many of the concepts, ideas, and goals described by Liapis et al., we propose an alternative route to designer modeling through clustering the design space

% In their work, they propose the use of hand-crafted

% Our work drags on many of the concepts, ideas, and goals described in~\citepsixth{p6Liapis2013-designerModel}, but we propose an alternative route to designer modeling through clustering the design space and the room style based on the collected data. In contrast 

% Their work propose the use of interactive evolution by biasing the search space in favor of hand-crafted features of the design akin to~\citepsixth{p6Alvarez2020-DesignerPreference}. However, we propose an alternative and novel route to designer modeling through clustering the design space and the room style based on the collected data. Moreover, we differ as well on the type of level design, being the sentient sketchbook a tool for strategy games~\citepsixth{p6liapis_generating_2013}, while EDD a tool for adventure and rogue-like games~\citepsixth{p6Alvarez2020-ICMAPE}. Applying the idea of designer modelling to both genres, not only shows the importance and usefulness of designer modeling but also the holistic and generic view 

% These differences strengthen the importance and usefulness of designer modeling, and highlight the holistic and generic properties of this designer-centric perspective.

% % might be interesting to discuss this.
% While the approach described in this paper is applied in a tool for creating zelda-like dungeon games~\citepsixth{p6tloz}, the approach can be reused and extended to other domains 

These contributions allow us to better understand, cluster, categorize and isolate designer behavior. This is very valuable for mixed-initiative approaches, where a clear virtual model of the designer's style allows us to better drive the search process for procedurally generating content that is valuable for the designer. Designer personas have the potential to be used in many different scenarios. For instance, as objectives for a search-based approach to enable a more style-sensitive system, to evaluate the fitness of evolutionary generated content or to train PCG agents via Reinforcement Learning~\citepsixth{p6khalifa2020-pcgrl}. 

Moreover, recognizing the designers' current style and the path taken so far, which would indicate a possible designer persona, could open the possibility for recognizing their intentions, preferences, and goals. This traced roadmap of designer personas could let a content generator anticipate a designer's next moves without heavy computational cost, just by identifying her current location on the map and offering content suggestions that lie in the most promising clusters to be visited next. Conversely, it could also identify designers who do not follow a certain path, i.e. deviating from the pattern, trying to understand their objective through their design style.

% Finally, in our work, we did not observe any type of cross-path i.e. a design going from one path to another. We believe that this is due to the level at which we observe the archetypical paths. However, preliminary analysis on the dataset used in this paper and as expected, the design process of designed rooms within the same dungeon does follow different paths, and sometimes even crossing each other. This opens an interesting and exciting area to explore a wider layer, taking rooms as a set of archetypical paths taken by designers. Observing the paths taken in previous and future rooms, and the dungeon as a whole, as briefly introduced in section~\ref{p6sec:designStyle}, to understand the designers' intentions and goals when they proceed to create a new room is a promising future step to take with the current system. 

%  i.e. room-wise, as the designer typically would design the room with a set of goals

Finally, it is also important to observe the nature of the previous and future rooms created by a designer. Observing the dungeon as a whole, as briefly introduced in section~\ref{p6sec:designStyle}, to understand the designers' intentions and goals when they proceed to create a new room is a promising future step to take with the current system. 

% Furthermore, the designer personas addresses the dynamic-dynamic system vs. dynamic-static system open question raised by Alvarez and Font~\citepsixth{p6Alvarez2020-DesignerPreference}, which relates to the challenge of adapting a system to the a ever-changing designer. With the use of the archetypical paths, the model is not anymore adapting and moving through the solution space with the designer, rather the designer traverse through an already clustered space. 

% With the use of the archetypical paths, we can not only identify the current designer persona the designer is following but we can also adapt and anticipate to what they might end up doing. 

% Furthermore, the designer personas addresses an open question raised by Alvarez and Font \citepsixth{p6Alvarez2020-DesignerPreference}, related to the challenges  using a dynamic-dynamic system vs. a dynamic-static system. The authors describe the dynamic-dynamic system as a system where both designer and AI-system move through the solution space, with the AI-system constantly trying to adapt to the designer. They concluded that the main challenge correspond to designers constantly concept drifting resulting in them continuously changing their decisions. Instead, the authors proposed the use of a dynamic-static system, where the model is not anymore adapting and moving through the solution space with the designer, rather the designer traverse through an already clustered space. With the use of the archetypical paths, we can not only identify the current designer persona the designer is following but we can also adapt and anticipate to what they might end up doing. 


%and conclude that the main challenges in

% Moreover, this traced roadmap of designer personas could let a content generator anticipate a designer's next moves without heavy computational cost, just by identifying her current location on the map and offering content suggestions that lie in the most promising clusters to be visited next. %Further, one could also be able to identify designers that do not follow a certain path i.e. deviating from the pattern, and try to understand through their design style their objective.





% From the $180$ unique rooms, we extracted and used the edition sequence of each of the rooms, from their initial design to the more elaborated end-design, to compose a richer dataset that could capture the design process of a designer rather than focusing on the end-point. Through this, we ended up using $8196$ data points in our dataset.

% We have first run and compared s


% through experimenting with 

% This paper presents a step towards designer modelling by providing a prototype implementation of designer personas as archetypical trajectories through style space. These archetypical paths

% This paper presents an experiment on archetypical design trajectories analysis in a MI-CC environment, as a means to characterize several representative design styles as designer personas. We have first run and compared several clustering setups to find the best partitioning, resulting into a set of twelve cohesive, coherent, and meaningful clusters. We have then mapped almost 200 complete design sequences in terms of these clusters, applying sequence mining to find four frequent unique designer styles, with related common sub-styles. As a result, we have presented a roadmap of design styles over a map of data-driven design clusters. 



% be used as goal for other systems were anticipating a design or creating a synthetic objective might be more complicated. We envision that these designer personas can be used within 


\section*{Acknowledgment}

The Evolutionary Dungeon Designer is part of the project \textit{The Evolutionary World Designer}, supported by The Crafoord Foundation.

\bibliographystyle{IEEEtran}
\bibliography{references.bib}
\vspace{12pt}

\end{document}
