\graphicspath{{included-papers-tex/paper-13/}}

%\includedPaper{\textsc{paper vi - designer modeling through design style clustering}}{\textsc{paper vi - designer modeling through design style clustering}}{Alberto Alvarez, Jose Font, and Julian Togelius}

\includedPaper{\textsc{paper xii - Towards AI as a \\ Creative Colleague in Game Level Design}}{\textsc{paper xii - Towards AI as a Creative Colleague in Game Level Design}}{Tinea Larsson, Jose Font, and Alberto Alvarez}

\normalfont
\textbf{\textsc{ABSTRACT}}

In Mixed-Initiative Co-Creative tools, the human is mostly in control of what will and can be created, delegating the AI to a more suggestive role instead of a colleague in the co-creative process. Allowing more control and agency for the AI might be an interesting path in co-creative scenarios where AI could direct and take more initiative within the co-creative task. However, the relationship between AI and human designers in creative processes is delicate, as adjusting the initiative or agency of the AI can negatively affect the user experience. In this paper, different degrees of agency for the AI are explored within the Evolutionary Dungeon Designer (EDD) to further understand MI-CC tools. A user study was performed using EDD with three varying degrees of AI agency. The study highlighted elements of frustration that the human designer experiences when using the tool and the behavior in the AI that led to possible strains on the relationship. The paper concludes with the identified issues and possible solutions and suggested further research.

\textbf{\textsc{PUBLISHED IN}}

Proceedings of The 18th AAAI Conference on Artificial Intelligence and Interactive Digital Entertainment (AIIDE), AAAI, 2022

\section*{TOWARDS AI AS A CREATIVE COLLEAGUE IN GAME LEVEL DESIGN}

\subsection{Introduction}

% How can we best build a system that lets a human designer collaborate with procedural content generation (PCG) algorithms to create useful and novel game content? 
Collaboration between AI and humans to co-design and co-create content is a significant challenge and the main focus of Mixed-Initiative Co-Creativity (MI-CC), which is the joint effort by a human user and AI to create content together~\cite{p13yannakakis_mixed-initiative_2014,liapis_can_2016}. In an MI-CC environment, designers can unleash their creativity while the computer ensures playability, measures quality, and potentially inspires them towards more creative designs. These systems' objectives are to foster creativity and provide seamless proactive collaboration, ultimately enabling a mutually beneficial collaboration. The AI role has been categorized depending on the computer agency and initiative: nanny, pen-pal, coach, and colleague~\cite{p13lubart_how_2005}. For an AI to be a colleague, it would have to intervene in the human process and take initiatives directly affecting the end product and creative process.

% that splits into four categories sorted by an ascending degree of computer agency over the creative process~\cite{p13roles}:  For an AI to be a true colleague, it would have to intervene the human's process and take initiatives which directly affect the end product. Morai Maker, an AI-driven Game Level Editor for Super Mario Bros-style games~\cite{p13morai-maker}, is an example of this. The human designer and the AI are equal co-creators that take turns to directly affect the final output, and they both have to work together by adapting to each other's design decisions.

% MI-CC is the joint effort by a human user and AI to create content together in a digital environment~\cite{p13MI-CC}, where a designer can unleash their creativity while the computer ensures playability, measures quality, and potentially inspires them towards more creative designs~\cite{p13tanagra,sentient-sketchbook,morai-maker}. 

 


% However, there needs to be an understanding between the human designer and the AI system about what needs to be designed, ideally even a shared goal.

% The role of the AI in mixed-initiative systems is usually relegated to the background giving the human control over tasks, goals, and processes to achieve the final design. Nevertheless

% Approaches such as designer modeling~\cite{p13liapis_designer_2013,alvarez_designer_2022} or player modeling~\cite{p13yannakakis_experience-driven_2011,holmgard_automated_2019} could help reducing the gap.

% There are

% Human-AI collaboration has been 

Morai Maker is an AI-driven Level Editor for Super Mario Bros-style games~\cite{p13guzdial_friend_2019}, which aims at having an AI as a colleague, with an equal role as the human designer, both adapting to each other. The Evolutionary Dungeon Designer (EDD) is a mixed-initiative design tool for rogue-like dungeon games~\cite{p13alvarez_fostering_2018}. EDD uses an evolutionary algorithm (MAP-Elites) to constantly generate finished rooms for the user to pick and replace their design based on the user's manual designs. The AI does not have any definitive control over the design decisions. Rather it suggests content adapted to the designer's current design, and the designer has the option not to incorporate the AI in their creations~\cite{p13alvarez_empowering_2019}. Nevertheless, it seems relevant to explore how other degrees of AI agency could affect the resulting co-creative process in terms of frustration, constraints, efficiency, or diversity, compared to when two humans create together. This comes with potential issues derived from altering the AI's agency; that human creativity can be dampened by restrictions in the creative process~\cite{p13yannakakis_mixed-initiative_2014}.

% The Evolutionary Dungeon Designer (EDD) is a mixed-initiative design tool for rogue-like dungeon games~\cite{p13eddy1}. EDD uses an evolutionary algorithm (MAP-Elites) to, based on the user's manual designs, constantly generate finished rooms for the user to pick and replace their designincorporate into the design process. This matches the nanny paradigm~\cite{p13roles}, which solely supports with suggestions and shows helpful information about the rooms~\cite{p13eddy2}. The AI does not have any definitive control over the design decisions, and the designer has the option to not incorporate the AI in its creations~\cite{p13eddy2}. Nevertheless, it seems relevant to explore how other degrees of computer agency could affect the resulting co-creative process in terms of novelty, surprise, efficiency, and diversity, as compared to when two humans create together. This comes with potential risks derived from altering the initiative of the AI, that human creativity can be dampened by restrictions and frustration in the creative process~\cite{p13MI-CC}.

% To enable a colleague relationship between human and AI in EDD, the AI would have to take initiatives in the design process~\cite{p13roles}. 

This paper explores how AI with varying degrees of agency affects the human users' design process in EDD. Three different versions of the tool are developed with varying degrees of the AI's control over the design process. These versions are then examined in a user study, and the results are analyzed to understand further the colleague relationship between humans and AI in MI-CC systems. The study also analyzes the degree of support these three AI companions have on lateral thinking, which is a vital part of the creative process. By assessing the three variants of agency, it is possible to compare the differences in the resulting creative relationships between the designer and AI, identifying factors that affect the designer's creative process in terms of frustrating elements, perceived limitations, and adaptation to their creative colleague.


% \textbf{RQ1: How does adjusting the control of the design decisions in mixed-initiative systems affect the human user's creativity?}

% \begin{itemize}
%     \item \textbf{RQ1.1} What are the effects on the human designer's design goal during the process?
    
    
%     \item \textbf{RQ1.2} What are the effects on the human designer's perception of limitations or frustration?
    
    
% \end{itemize}



% Answering this research question aspires to provide understanding of the human's interaction with mixed-initiative systems and how it affects the human's experience in different degrees of creative freedom.


% \textbf{RQ2: How can the three degrees of initiative be explored to asses the support of the AI in terms of the human's creative process?}

% \begin{itemize}
%     \item \textbf{RQ2.1} How do they support lateral thinking and the introduction of new ideas?
    
    
%     \item \textbf{RQ2.2} How does the designer respond to the AI's differing degrees of initiative?
    
    
% \end{itemize}




\subsection{Related Work}

% \subsubsection{Procedurally Generated Game Content}
% Procedural Content Generation (PCG) is a method of creating content through intelligent algorithms ~\cite{p13PCG}. By using computers to create content from manually created assets and rules, large amounts of game content can be produced efficiently, while also promoting replayability as it can produce endless variations of content~\cite{p13PCG}. As a result of the benefits to this method, there are many domains for PCG. One of the more well-established uses of PCG in games is terrain-generation, for example in the game Minecraft~\cite{p13minecraft}, or the Diablo-series~\cite{p13diablo}. Other uses of PCG include quest-generation, such as in the game Legends of Aethereus~\cite{p13Legends-of-Aethereus}, and procedurally generated narrative and story, such as in the game Dwarf Fortress~\cite{p13dwarf-fortress}.

% Autonomy \cite{p13negrete-yankelevich_co-creativity_2020}

% Reflection \cite{p13kreminski_reflective_2021}

% Interactiveness~\cite{p13deterding_mixed-initiative_2017}

% \item Also important to bring that paper about mixed-initiative tools from FDG 2021, and~\cite{p13partlan_design-driven_2021}
%     \item the paper of the three pillars for mixed-initiative.~\cite{p13lai_towards_2020}
    
    




% was described by Novick and Sulton as a multi-factor model that combines: choosing the task, choosing the agent in control and how the interaction is established, and choosing the expected outcome from the collaboration~\cite{p13novick_what_1997}. 

% Mixed-initiative Co-Creativity focuses on the joint effort between humans and AI with proactive initiative to tackle creative tasks not only with the assisstance of AI 

% Mixed-Initiative Co-Creativity poses that humans can not only be assisted through the support of AI, but possibly also advances the humans creativity~\cite{p13yannakakis_mixed-initiative_2014}.

% The proposed question is not new and has been approached by different disciplines, under the~\acrfull{mi} paradigm.~\acrshort{mi} refers to the collaboration between \emph{human} and \emph{computer} where both have some proactive initiative to solve some task.~\acrshort{mi} can be seen as a multi-agent collaboration scenario, where the interaction should be flexible, allowing for a continuous negotiation of initiative and leverage on each other's strengths to solve a task~\cite{p13allen_mixed-initiative_1999}. \emph{Initiative} was described by Novick and Sulton as a multi-factor model that combines: choosing the task, choosing the agent in control and how the interaction is established, and choosing the expected outcome from the collaboration~\cite{p13novick_what_1997}. 

% Moreover, Horvitz discussed such a question in terms of Intelligent User Interfaces~\cite{p13birnbaum_compelling_1997}, describing mixed-initiative systems and interfaces as a more natural collaboration in a user interface that emerges from intertwining human control and manipulation, and automation~\cite{p13horvitz_uncertainty_1999}. Horvitz presented several principles of mixed-initiative interaction and its challenges, many of which still exist~\cite{p13horvitz_principles_1999}, mainly describing this interaction as conversation systems between AI and humans~\cite{p13horvitz_computational_1999}. Moreover, Yannakakis et al. introduced the~\acrfull{micc} paradigm for the co-creation of creative content, where both AI and humans alternate in the initiative to co-design and solve tasks~\cite{p13yannakakis_mixed-initiative_2014}. Their work describes key findings and discussions for how MI-CC does not only help human designers solve tasks, but also fosters their creativity through an interactive feedback loop and lateral thinking~\cite{p13liapis_can_2016,liapis_computational_2014,alvarez_fostering_2018}.

% \subsubsection{Mixed-Initiative Co-Creativity}



MI-CC focuses on tackling tasks between humans and AI with proactive initiative, where AI does not only assist humans but could also collaborate with them, leveraging on both their strengths~\cite{p13yannakakis_mixed-initiative_2014,allen_mixed-initiative_1999}. \emph{Initiative} is a multi-factor model combining: choosing the task, the agent in control and how the interaction is established, and the expected outcome~\cite{p13novick_what_1997}. In this work, both humans and AI have the same task, and the interaction is established as turn-based, each taking discrete control. The outcome is expected to vary as AI agency increases since larger constraints are added for the human that might need to adapt towards those. 

Some MI-CC systems enable different collaborative approaches, which are considered in this paper. Tanagra~\cite{p13smith_tanagra_2011} is a design tool for platform levels where the system takes as input and constraints the current user's design and creates content fulfilling gaps around it. Morai Maker~\cite{p13guzdial_co-creative_2018} is an MI-CC tool where the human designer and AI take turns to design Super Mario Bros. levels. The AI adds content in its turn, which can be maintained or erased by the human designer, which the AI learns to adapt to through reinforcement learning. Furthermore, Lode Encoder~\cite{p13bhaumik_lode_2021} explores a creative collaboration where the human is constrained by only being able to use AI-generated content, which they need to choose to compose their design. This shows an unusual collaboration that users expressed as a playful, game-like creative process.

\begin{table*}[ht]
% \centering
\caption{General consensus on EDD's features} \label{p1tab:consensus}
\resizebox{0.8\textwidth}{!}{
\begin{tabularx}{\textwidth}{|p{0.2\textwidth}|p{0.99\textwidth}|}
\cline{1-2}

Description & Participants’ Consensus \\\cline{1-2}
World Grid  of the dungeon                   & Its purpose of establishing an illusion of a fully realized dungeon is somewhat achieved. However, limitations exist with how it defines feasibility, a dungeon’s starting point, and the entrances, which disrupts the designers’ decisions.                                                                                                                                                                                   \\\cline{1-2}
World View                                  & The world view’s usefulness for the most part could not be established, other than for the purpose of going to the suggestions view (which was already seldom during the user study) and having a closer look at the entire dungeon without any distractions. Some participants preferred features to be already in the room view’s minimap, and some wanted to see more specific functionalities within the world view itself. \\\cline{1-2}

Enabling and  \newline disabling rooms                & As the user study restricted participants to create 3x3 dungeons, this feature for the most part has been neglected. This is also in part because of its accessibility only being in the world view, which proved to be an inefficient view in general. However, its use for bigger dungeon sizes later on was appreciated, especially for more intricate design purposes.                                                      \\\cline{1-2}
Suggestions View                            & Similarly to enabling and disabling rooms, it was quite difficult to encourage the use of this functionality due to the world view’s inefficient usability. However, this could also be due to the dungeon’s small size, as some participants expressed high interest in using more suggestions with larger dungeon sizes.                                                                                                      \\\cline{1-2}
Minimap  \newline  navigation                      & The minimap proved to be a strong tool not only for navigation purposes, but also for supporting design decisions and choices. The directional buttons were rarely used, but their room previews were helpful in emphasizing the current room’s connection to adjacent rooms without looking at the minimap. On the other hand, this lowered the usability of the world view.                                                   \\\cline{1-2}
Parameters                                       & The parameters were, in general, lacking. They served to be important in decision-making when choosing a suggested map in room view, but there were still doubts on their accuracy and sufficiency when providing information about the generated suggestions.                                                                                                                                                                       \\\cline{1-2}
Generated maps for  \newline  suggestions in room view & Suggestions in the room view proved to be very helpful in supporting the whole design process as they primarily acted as inspirations for the users. The most prominent comment among the users is the preference of having more control on how suggestions should be generated depending on different types of parameters.                                                                                                     \\\cline{1-2}
Design \newline  patterns& The patterns’ visualization was, in general, lacking and not self-explanatory. Some participants have expressed interest in using patterns as a parameter in the generation of suggestions.                                                                                                                                                                                                                                     \\\cline{1-2}
Dark theme                                  & EDD’s dark theme for the user interface received a positive response as it makes working with the program easier.
	\\ \cline{1-2}
\end{tabularx}
}
\end{table*}

% \begin{table}
% \begin{center}
% {\caption{Best performing setups based on their internal validation and visualization of clustered data points.}\label{table:setups}}
% \resizebox{\textwidth}{!}{
% \begin{tabular}{ccccccc}
% \hline
% \rule{0pt}{12pt}
% Algorithm&Data&K&$\Diamond$&$\Box$&$\bigtriangleup$ 
% \\ 
% \hline
% \\[-6pt]
% K-Means & Tiles-PCA & 9 & 0.43 & 0.73 & 9438.233 \\ 
% K-Means & Tiles-PCA & 12 & 0.41 & 0.77 & 9436.928 \\
% K-Means & Dimensions-PCA & 12 & 0.43 & 0.73 & 7738.343 \\
% Agglomerative single & Combined-PCA & 6 & 0.51 & 0.43  & 38.833 \\ 
% Agglomerative avg. & Dimensions-PCA & 6 & 0.44 & 0.67 & 3463.567 \\ 
% \hline
% \\[-6pt]
% \multicolumn{6}{l}{$\Diamond$ Silhouette Score\ \
% $\Box$ Davies Bouldin Index\ \
% $\bigtriangleup$ Calinski-Harabasz Index}
% \end{tabular}
% }\end{center}
% \end{table}

This paper uses EDD as the tool to explore AI agency and control. EDD is an MI-CC system where designers can create interconnected rooms composing a dungeon~\cite{p13alvarez_empowering_2019}. As designers create their content, the AI constantly suggests content adapted to the designer's design using the Interactive Constrained MAP-Elites (IC MAP-Elites). We make extensive use of IC MAP-Elites to generate rooms that are adapted to the target room. In~\cite{p13alvarez_interactive_2020}, the authors show that IC MAP-Elites can generate high-performing and diverse rooms from different targets and using different dimension combinations. Its adaptiveness and stability, two necessary properties, were assessed with continuously edited rooms in~\cite{p13alvarez_assessing_2021}, showing that the designer has a positive effect and can steer the algorithm with their design.

%In EDD, designers can interact in different ways with the algorithm by locking tiles to preserve their changes and changing feature dimensions and their granularity to narrow the search in interesting generative spaces~\cite{p13alvarez_assessing_2018,alvarez_interactive_2020}.

%This paper makes use of EDD and the IC MAP-Elites to generate rooms that are adapted to the target room. In~\cite{p13alvarez_interactive_2020}, the authors show that IC MAP-Elites can generate high-performing and diverse rooms from different targets and using different dimension combinations. Its adaptiveness and stability, two necessary properties, were assessed with continuously edited rooms in~\cite{p13alvarez_assessing_2021}, showing that the designer has a positive effect and can steer the algorithm with their design. Given that we do not know what dimensions the designer might be interested on, we use all seven dimensions in the search and weight each of them equally. 





% which designers can interact with by changing dimensions,
% In EDD, the designer can create interconnected rooms composing a 

% InEDD is an MI-CC system for building adventure game dungeons. A human designer creates rooms interconnected in a graph structure that composes a dungeon. The computer constantly inputs the human designs to procedurally generate sets of suggested rooms on-the-fly, by means of an Interactive Constrained MAP-Elites (IC MAP-Elites) evolutionary algorithm driven by level design patterns. The designer can incorporate and further edit those suggestions at will, and also interacts with the evolutionary system by locking tiles and fine-tuning some hyper-parameters of the IC MAP-Elites algorithm.

% A detailed description of all EDD's features, including the use of design patterns, its implementation of the IC MAP-Elites algorithm, and its designer modeling module can be found in~\cite{p13eddy1, MAP-Elites-eddy, eddy2}.

% Mixed-Initiative Co-Creativity poses that human designers can not only be assisted through the support of AI, but possibly also advances the humans creativity~\cite{p13yannakakis_mixed-initiative_2014}. Both computer and human contribute to the creative tasks with their own perks, as they fluently inspire each other and inflict on each other's creative process. 

% The computer can present the human with creative choices made with the element of randomness, which presents the human with ideas that the human was possibly unlikely to make, and therefore provide unexpected mutations to the game content~\cite{p13MI-CC}. As described by Yannakakis et al., some of the main aspects of how we define human creativity will benefit greatly from what a computer can provide in a digital creative environment~\cite{p13MI-CC}. Yannakakis et. al. describes how one major part of creativity, lateral thinking, benefits greatly from computer-made randomness as follows:

% \textit{"The random stimulus principle of lateral thinking relies on the introduction of a foreign conceptual element with the purpose of disrupting preconceived notions and habitual patterns of thought, by forcing the user to integrate and/or exploit the foreign element in the creation of an idea or the production of a solution. Randomness within lateral thinking is the main guarantor of foreignness and hence of stimulation of creativity."}~\cite{p13MI-CC}

% Mixed-Initiative systems is a powerful tool to enable efficient and innovative content generation, but requires further research to explore how to optimize its full potential. 

% Notable research projects covering this subject include Tanagra~\cite{p13smith_tanagra_2011}, which is a MI-CC game design tool to create 2-dimensional platform style game levels. Tanagra's level generator can either autonomously create levels in absence of human input with a PCG-algorithm, or respond to a human designer's input, as well as display helpful information about the level such as patterns, and ensuring that the level is playable. When a human designer edits a generated level, the AI will keep those edits when suggesting the following generated content.

% Morai Maker is another example of a research project within MI-CC. Morai Maker is a game level design tool for the game Super Mario Bros., where human and AI create in a turn-based manner~\cite{p13guzdial_co-creative_2018}. The AI in the system uses PCGML to generate content.
% A study that was performed on Morai Maker in 2019 examined how human users reacted to the system having three different generation algorithms, and discussed the resulting creative experiences, as well as the resulting relationship they developed with the computer in their creative processes. The study concluded that, although users experienced varying satisfaction with the additions the AI contributed with, the users demonstrated a willingness to adapt their behaviour to the agent. 

% % This conclusion is of great relevance to this thesis project, as it supports the idea that MI-CC can possibly achieve a more equal collaboration and creativity with a human-AI team. The turn-based procedure of human-AI interaction used in Morai Maker~\cite{p13morai-maker} has also inspired this thesis project to use a similar tactic, to enable both the computer and the human to influence the end product, and striving towards an equal influence in the end result. 


% Lode Encoder is an AI-assisted tool for creating levels using procedural content generation via machine learning for the game Lode Runner~\cite{p13bhaumik_lode_2021}. This tool explores a creative collaboration where the human is constrained by only being able to use AI-generated content. The human user is presented with AI-generated levels, and has the choice to add elements from the generated rooms into their creation, refresh the generation, or use an AI-controlled "wand tool" to attempt to repair broken ladders or platforms to increase the amount of playable levels being created. The tool Lode Encoder started out as a completely autonomous level generator. This version of the tool presented a problem, the generated levels where of unsatisfactory quality, except for if they were overfit on the data. The results from the overfitted data resulted in generated levels of high quality, but would always result in a level almost identical to an existing level in the training data. This lead to the idea of the wand tool, by using the results of an overfitted network to repair the generated levels that where of poor quality. With this idea in mind, the tool was redeveloped as a MI-CC tool. The study shows an unusual creative collaboration to take place, that users expressed as a playful, game-like creative process. 

\subsubsection{AI Roles and Adaptability}

% It is important to highlight the role of the computer in this interaction. 

Lubart discusses four different roles a computer might take to promote creativity; \emph{computer as nanny}: management of creative work; \emph{computer as pen-pal}: communication service between collaborators; \emph{computer as coach}: Using creative enhancement techniques; and \emph{computer as colleague}: partnership between computer and humans~\cite{p13lubart_how_2005}. This is further explored by Guzdial et al. where designers perceived the AI collaborator with more or less value depending on their desired role for the AI, varying between: \emph{friend}, \emph{collaborator}, \emph{student}, or \emph{manager}~\cite{p13guzdial_friend_2019}.

% Paramount is the role of the computer agent in this interaction, as it would help establish the boundaries of the interaction, what is expected, and how creativity could be fostered. Lubart analyzed this interaction and examined the different ways computers could be involved in creative work to promote creativity. In his work, he proposed four roles: \emph{computer as nanny}: management of creative work; \emph{computer as pen-pal}: communication service between collaborators; \emph{computer as coach}: Using creative enhancement techniques; and \emph{computer as colleague}: partnership between computer and humans~\cite{p13lubart_how_2005}. Recently, this was explored by Guzdial et al., where designers perceived the AI collaborator with more or less value depending on their desired role for the AI, varying between: \emph{friend}, \emph{collaborator}, \emph{student}, or \emph{manager}~\cite{p13guzdial_friend_2019}.

%  The relationship that occurs between co-creators in a MI-CC tool is arguably one of the most important element within a system that aims to foster creativity.

Establishing different roles such as colleague and collaborator might require some user model within the system. Designer modeling, as defined by Liapis et al.~\cite{p13liapis_designer_2013}, is a way to classify and predict a designer's style, goals, preferences, and processes. Preference models~\cite{p13alvarez_learning_2020,liapis_adapting_2012} have been built based on designers' choices and used as surrogate models to evaluate further generated content. Similarly, using the designers' creation, the designers' processes and styles could be modeled to inform other systems and adapt the generated content~\cite{p13liapis_designer_2014,alvarez_designer_2022,halina_threshold_2022}.

% modeling the designers processes and styles~\cite{p13liapis_designer_2014,alvarez_designer_2022,halina_threshold_2022} using the designers creations 

% Designer models regarding preferences~\cite{p13alvarez_learning_2020,liapis_adapting_2012} or  have been explore

% Designer Model

% Alvarez et. al. has implemented a novel method of designer modeling within EDD, which successfully identified four frequent and unique designer styles, with related common sub-styles~\cite{p13designer-modelling-edd}. The study showed promise of this method being used to create an AI-colleague within EDD~\cite{p13designer-modelling-edd}. It is possible that an AI that can recognize a designers persona, style, preferences and goals, and adapting its behaviour to support these attributes, can result in a well-performing and interesting co-creator~\cite{p13designer-modelling, designer-modelling-edd}.

% Designer modeling, Lubart, allen, Novick, Guzdial, Liapis (aesthetic models), check the other paper 

% Lubart~\cite{p13lubart_how_2005} describes four roles for a computer system in a collaborative task. Nanny, pen-pal, coach, and colleague. 

% Guzdial~\cite{p13guzdial_friend_2019}

% Can be asymmetric, focus on completely different tasks, and can be dynamic? maybe that we leave for Dynamicity

% Designer Modeling~\cite{p13alvarez_designer_2022,liapis_designer_2013,liapis_designer_2014,halina_threshold_2022}

% Adaptive aesthetic models~\cite{p13liapis_adapting_2012}





% The relationship that occurs between co-creators in a MI-CC tool is arguably one of the most important element within a system that aims to foster creativity. Founding a valuable and co-creative relationship between AI and human is a major challenge, and further developing designs of AI is necessary to work towards perfecting this relationship. Designer modeling, as defined by Liapis et. al.~\cite{p13designer-modelling}, is a way to classify and predict a designer's style, goals, preferences, processes, and their definitions. As proposed by Alvarez et. al.~\cite{p13designer-modelling-edd}, this concept can be used to create an adaptive and collaborative colleague in a MI-CC environment. Alvarez et. al. has implemented a novel method of designer modeling within EDD, which successfully identified four frequent and unique designer styles, with related common sub-styles~\cite{p13designer-modelling-edd}. The study showed promise of this method being used to create an AI-colleague within EDD~\cite{p13designer-modelling-edd}. It is possible that an AI that can recognize a designers persona, style, preferences and goals, and adapting its behaviour to support these attributes, can result in a well-performing and interesting co-creator~\cite{p13designer-modelling, designer-modelling-edd}.

% With the high level of initiative of the AI in a collaborative relationship, similarly to the one in Lode Encoder, human creativity can be challenged and explored in a new setting, to further our understanding of the possibilities of human-AI collaboration.


% \subsubsection{Interactive Constrained MAP-Elites}
% The Multi-dimensional Archive of Phenotypic Elites (MAP-Elites) algorithm is a quality-diversity search algorithm which presents a comprehensive view of high-performing solutions according to specifically defined dimensions of the search space~\cite{p13MAP-Elites}. MAP-Elites produces a variety of high-performing solutions, with different attributes, allowing diverse results with the possibility to easily investigate specific attributes and combinations of attributes. The resulting elites are represented in a map, where each cell contains one solution~\cite{p13MAP-Elites-eddy}. One axis of the map corresponds to one dimension of the search space. When a new generation is started, an offspring is generated based on one or more existing solutions. The offspring is then evaluated according to its feature dimensions. If the new solution is placed in a cell with an existing solution, the one with the highest fitness value occupies the cell. The result is a map of solutions ordered in the best performing solution found in the particular feature dimensions.

% The program that will be the subject of the study, EDD, uses an evolutionary PCG-algorithm to generate content and Interactive Constrained MAP-Elites (IC MAP-Elites) to search for the best performing generated content~\cite{p13MAP-Elites-eddy}. The user of EDD can edit content, and the user's choices are then fed into the algorithm, making the MAP-Elites interactive. 
% The existing MAP-Elites system in EDD will be used in this thesis project, as a foundation for the decision making of the AI to determine what tile to place and where. By doing so, the thesis project can effectively produce an intelligent agent, and with that allow more resources to be put into the theoretical framework and case study.

% \subsubsection{The Evolutionary Dungeon Designer}

% The Evolutionary Dungeon Designer EDD is an MI-CC system for building adventure game dungeons. A human designer creates rooms interconnected in a graph structure that composes a dungeon. The computer constantly inputs the human designs to procedurally generate sets of suggested rooms on-the-fly, by means of an Interactive Constrained MAP-Elites (IC MAP-Elites) evolutionary algorithm driven by level design patterns. The designer can incorporate and further edit those suggestions at will, and also interacts with the evolutionary system by locking tiles and fine-tuning some hyper-parameters of the IC MAP-Elites algorithm.

% A detailed description of all EDD's features, including the use of design patterns, its implementation of the IC MAP-Elites algorithm, and its designer modeling module can be found in~\cite{p13eddy1, MAP-Elites-eddy, eddy2}.

% \begin{table}[]
% \begin{tabular}{p{0.25\linewidth}| p{0.65\linewidth}}
% Feature          & Definitions                                                                                                                  \\ \hline
% Similarity (Sim)       & Aesthetic (tile-by-tile) similarity between a generated level and the designer's design.                         \\ \hline
% Inner Similarity (IS) & Different tiles' sparsity and density similarity between a generated level and the designer's design.     \\ \hline
% Symmetry         & Room's aesthetic symmetry.                                                                                 \\ \hline
% Leniency (Len)        & Challenge based on enemies and treasures.\\ \hline
% Linearity (Lin)       & Paths that exist connecting entry points in a level.  \\ \hline
% \#Meso-Patterns (Meso) & Amount of meso-patterns that exist within a level. This is a discrete dimension rather than continuous.\\ \hline
% \#Spatial-Patterns (Spa) & Amount of spatial-patterns that exist within a level. \\ \hline
% \end{tabular}
% \caption{Level design based dimensions used in EDD with IC MAP-Elites.}
% \label{tab:dimensions}
% \end{table}

% The Evolutionary Dungeon Designer (EDD) ~\cite{p13eddy1, eddy2} is a MI-CC tool to co-create 2D dungeons in the style of the seminal game \emph{The Legend of Zelda}~\cite{p13tloz}. Designers manually edit the dungeon structure as well as the interior of every room in it. EDD constantly offers tailored room suggestions on the fly that designers may decide to incorporate to their designs at any moment. 

% In EDD, the system analyzes the level-design patterns (i.e., micro- and meso-patterns) that exist in each room, calculating and utilizing their quality to assess rooms. Micro-patterns are the building blocks in a design, which in EDD are categorized as \textit{spatial micro-patterns}: chamber, corridor, intersections, connector; and \textit{inventorial micro-patterns}: enemy, treasure, and door. On the other hand, Meso-patterns are defined as the relation between micro-patterns or other meso-patterns, and by the composition between inventorial micro-patterns and spatial micro-patterns. Meso-patterns are used to identify structures in the room that join together a set of micro-patterns and can be: \textit{ambush}, \textit{guard chamber}, \textit{treasure chamber}, and \textit{guarded treasure}. All patterns are shown in figure~\ref{fig:basecomponents}, and further information and discussion can be found in~\cite{p13Baldwin2017,Alvarez2018}.

% The Evolutionary Dungeon Designer (EDD) is an MI-CC game design tool for designing  dungeons~\cite{p13eddy1, eddy2}. The user is first met by the dungeon-editing view, where the dungeon is represented by a graph-structure, displaying rooms and their connections. In this view, the user can add rooms of any size between 3 x 3 and 20 x 20 tiles, and connectors between rooms in the form of door-tiles. The smallest valid dungeon consists of two rooms with one two-way connection between them. The designer marks one of the rooms as the initial room to enable the program to calculate the dungeons feasibility. If a room or door is unreachable from the initial room, they will be highlighted in red, to alert the designer. The user can also select the option "Start with our suggestions" which will present six procedurally generated suggestions the user can start with, and edit to their preferences.
% % \begin{figure}[!h]
% %     \includegraphics[width=\columnwidth]{images/Artifact/world_view new.png}
% %     \caption{The world editing view.}
% % \end{figure}


% To edit a room in the dungeon, the user double-clicks a room in the dungeon-editing view. In the room editing view, there is a pane of editing options on the left side of the screen. This pane displays the options available for manually editing the room by brush painting with one of the available tiles. The available tile-brushes are floor, wall, treasure, enemy or boss. There are three brush sizes, single tile, three-tile cross-shape, or five-tile cross shape. By control-clicking, the user can bucket paint all adjacent tiles of the same type. The middle of the view displays the currently edited room, also referenced as the target room. The right side of the room-editing view displays a matrix with suggested room designs. These suggestions have been procedurally generated through an evolutionary algorithm, and then selected by the IC MAP-Elites genetic algorithm~\cite{p13MAP-Elites-eddy}. The IC MAP-Elites runs continuously while the user is in the room-editing view, and updates the matrix with elites throughout. The evolutionary process is fed with the target room, and with that, all changes to the target room affect the generated suggestions. Additionally, the user has the option to paint locks on tiles in the target room to preserve specific tiles in all procedurally generated suggestions. The user has the option to apply one of the suggested room designs, restart the evolutionary generation, or editing the target room manually. The user can press "Show Patterns" to toggle the visibility of meso-patterns (ambush, guard chamber, treasure chamber and guarded treasure) in the target room, and can also toggle the visibility of locks drawn in the room by clicking "Show Locks". The button "Go To World Grid" takes the user back to the world editing view.


% % \begin{figure*}
% %     \includegraphics[width=\textwidth]{images/Artifact/empty room editing view.png}
% %     \caption{The modified room editing view in the tool.}
% % \end{figure*}

% The generating algorithm uses an interactive evolution, meaning the user input has influence in the generation of content. To supply the user with the most useful suggestions out of all the generated rooms, the program uses a MAP-Elites algorithm, which uses multidimensional discretization of the generated content to present the resulting highest rated candidates of each dimension to the user ~\cite{p13MAP-Elites, MAP-Elites-eddy}. 
% There are seven dimensions used for the MAP-Elites in EDD, representing a specific attribute of rooms~\cite{p13MAP-Elites-eddy}, described below.  


% \textbf{Symmetry}

% Symmetric structures tend to be more visually appealing for the user, and is often used by human designers to distribute content evenly over the canvas. Symmetry is assessed by non-passable tiles (i.e. walls), and evaluated along the X-axis, the Y-axis and the diagonals ~\cite{p13MAP-Elites-eddy}.


% \textbf{Similarity}

% Similarity is calculated by comparing tile by tile with the target room. Similarity is used to present the user with generated rooms that vary from their design, while also preserving aesthetic choices~\cite{p13MAP-Elites-eddy}.


% \textbf{Number of Meso-Patterns}

% The meso-patterns in a room displays the encounters the theoretical player could experience in that room . The considered patterns are treasure rooms, guard rooms, and ambushes . By identifying patterns in sets of tiles, such as a treasure surrounded by walls with an enemy guarding it being identified as a guard room, the number of meso-patterns can represent the utility of a room ~\cite{p13MAP-Elites-eddy}. 


% \textbf{Number of Spatial Patterns}

% The Spatial Patterns that can be identified are corridors, chambers, connectors and nothing . A high amount of spatial patterns correlates with a room that is subdivided well with walls, and has many elements other than floor tiles, which indicates a more unique and useful room ~\cite{p13MAP-Elites-eddy}.


% \textbf{Linearity}

% Linearity represents the number of traversable paths that exist between the doors in the room . This correlates to what kind of gameplay the designer is trying to accomplish, for example a more maze-like approach might be accomplished by having the player go through other rooms to reach the doors, while a more combat-based approach might be accomplished by high linearity with many guarding enemies~\cite{p13MAP-Elites-eddy}.


% \textbf{Inner Similarity}

% Inner similarity compares the micro-patterns of the generated room with the target room . The density and sparsity of enemies, treasures and walls is calculated to identify similarity in micro-patterns, to present the user with generated rooms that have similar design choices. This differs from the evaluation of Similarity which focuses on aesthetic similarity, as Inner Similarity focuses on the similarity in utility and functional design choices~\cite{p13MAP-Elites-eddy}.


% \textbf{Leniency}

% Leniency represents the difficulty in a room at any given point. The amount of treasures and enemies in a room, as well as their density and sparsity, together with how safe doors are, are considered in the equation. A room with low leniency has few enemies, or enemies placed sparsely, and a balanced amount of rewarding treasure. Leniency corresponds to the design of a room in terms of reward and risk~\cite{p13MAP-Elites-eddy}.


% \subsubsection{Designer Modeling}

% The relationship that occurs between co-creators in a MI-CC tool is arguably one of the most important element within a system that aims to foster creativity. Founding a valuable and co-creative relationship between AI and human is a major challenge, and further developing designs of AI is necessary to work towards perfecting this relationship. Designer modeling, as defined by Liapis et. al.~\cite{p13designer-modelling}, is a way to classify and predict a designer's style, goals, preferences, processes, and their definitions. As proposed by Alvarez et. al.~\cite{p13designer-modelling-edd}, this concept can be used to create an adaptive and collaborative colleague in a MI-CC environment. Alvarez et. al. has implemented a novel method of designer modeling within EDD, which successfully identified four frequent and unique designer styles, with related common sub-styles~\cite{p13designer-modelling-edd}. The study showed promise of this method being used to create an AI-colleague within EDD~\cite{p13designer-modelling-edd}. It is possible that an AI that can recognize a designers persona, style, preferences and goals, and adapting its behaviour to support these attributes, can result in a well-performing and interesting co-creator~\cite{p13designer-modelling, designer-modelling-edd}.


%\subsubsection{AI as colleague}
%text~\cite{p13roles} % Will maybe add later, but here I will describe the relationship between human and AI in co-creating. How does one create a colleague relationship? How is it used? Examples of Co-Creative colleagues in Game Design Tools. Examples of other uses.

\subsection{Room Style Clustering}

% \begin{enumerate}
%     \item Information on the user studies and how we develop the clusters.
%     \item collected data through 2 user studies
%     \item transformed the data into 5 datasets
%     \item data reduction through PCA and T-SNE. Cluster with K-means, K-Medoids, agglomerative clustering and DBSCAN. Evalulated through internal indices (silhouette index, DB index, and CH-index) 
% \end{enumerate}

\begin{figure*}[ht!]
\centerline{\includegraphics[width=\textwidth]{figures/process-steps.png}}
\caption{The stages of the design style clustering development: (1) Data was first collected through two user studies. (2) Then, using the design sequences, the data was processed into five different datasets, one using the room images, a second using the tiles information, and three using tabular information. (3) A data reduction technique was applied to different datasets, and then they were clustered and internally evaluated. (4) The clusters were formed, picked from the best performing methods, and labeled based on the data points within each cluster. The cluster were evaluated by visualizing how a typical design session traverse the various clusters, and K-Means (K=12) was chosen as the final approach. (5) Finally, using this final approach all the sequences were clustered and archetypical paths were identified.%(5) The final approach,  K-Means (K=12) was evaluated by visualizing how a typical design session traverse the various clusters. Finally, the sequences were clustered by the final approach and archetypical paths were identified.
} \label{p6fig:approach-steps}
\end{figure*}

\begin{figure*}[t]
\centerline{\includegraphics[width=\textwidth]{figures/final-cluster.png}}
\caption{Best resulting cluster set. K-Means (K=12), using the \textbf{Tiles} Dataset. While it scores slightly less in the internal indices that other setups, a qualitative analysis successfully gives us more granularity by subdividing the main bottom clusters, to label and cluster the design process of designers. Sample rooms belonging to each cluster are displayed on the right, next to the total number of rooms in the cluster.} \label{p6fig:all-clusters}
\end{figure*}

% \begin{figure*}[b]
% \centerline{\includegraphics[width=13cm]{figures/representative cluster-steps.png}}
% \caption{Examples of a step by step edition sequence of a design session and it's clustering. To the left, we present the actual sequence and steps of one of the rooms in the dataset and to the right is the actual trajectory of the design in the cluster space. Numbered and in black, it is shown how each step of the design process is clustered by our approach} \label{p6fig:paths-designers}
% \end{figure*}

% \begin{figure}[h]
% \centerline{\includegraphics[width=8cm]{figures/representative-cluster-steps-alter.png}}
% \caption{Example of a step by step edition sequence of a design session and it's clustering. At the top, we present the actual sequence and steps of one of the rooms in the dataset, in a $4\times7$ grid, starting at the top left with the first edition. At the bottom, it is the actual trajectory of the design in the cluster space. Numbered and in black, it is shown how each step of the design process is clustered by our approach} \label{p6fig:paths-designers}
% \end{figure}

\begin{figure*}
\centerline{\includegraphics[width=\textwidth]{figures/representative-clusters-small.png}}
\caption{Example of a step by step edition sequence of a design session and it's clustering. At the left, we present the actual sequence and steps of one of the rooms in the dataset, in a $4\times7$ grid, starting at the top left with the first edition. At the bottom, it is the actual trajectory of the design in the cluster space. Numbered and in black, it is shown how each step of the design process is clustered by our approach} \label{p6fig:paths-designers}
\end{figure*}

This paper presents an approach and fundamental steps towards the implementation of designer personas: an analysis of designer style clustering to isolate archetypical paths that can be later be used to build ML surrogate models of archetypal designers. Such models would adapt to the dynamic designer during the mixed-initiative creative process by being placed in the solution space, allowing the designer to traverse such space of models as she drifts through the many dimensions of her creative process.

The proposed system builds on top of EDD's Designer Preference Model and preliminary results \citepsixth{p6Alvarez2020-DesignerPreference}, expanding it to classify the designers' designs based on clusters developed using previously hand-made design sequences by expert and non-expert designers. Figure \ref{p6fig:approach-steps} illustrates our approach in five sequential stages, from data collection to experimentation and results. The first four stages are explained in the following subsections, whereas Section~\ref{p6section:results} shows the experimental results.

\subsubsection{Data Collection}

We conducted two user studies where participants were tasked with freely designing a dungeon in EDD and the rooms that compose it with no further restrictions, using all the available tiles i.e. floor, wall, treasure, enemy, and boss tiles. All participants were introduced to the tool before the design exercise. User-generated data was gathered during the complete design session, creating a new data entry every time the designer edited the dungeon. In total, we had $40$ participants, $25$ of these (i.e. NYU participants) were industry or academic researchers within the Games and AI field, and the other $15$ (i.e. MAU participants) were game design students. This resulted in a diverse dataset composed of $180$ unique rooms like the ones depicted in Figure~\ref{p6fig:approach-steps}, that was pre-processed and clustered in the subsequent stages. 

\subsubsection{Dataset pre-processing}

From the $180$ unique rooms, we extracted and used the edition sequence of each of the rooms, from their initial design to the more elaborated end-design, to compose a richer dataset that could capture the design process of a designer rather than focusing on the end-point. Through this, we ended up using $8196$ data points in our dataset.
%just the end-point. We ended up using $8196$ rooms
Moreover, five different copies of the dataset were created to analyze and compare the performance of the clustering stage using the following image pre-processing methods:

\begin{enumerate}
\setcounter{enumi}{0}
    \item \textbf{Room:} No pre-processing. Room images are fed into the next stage as they were created by the designer, with a resolution of $1300\times 700\times3$, corresponding to width, height, and RGB ($3$ color channels).
    
    \item \textbf{Tiles:} Each room tile type is mapped to a single-color pixel and the rooms are simplified to a pixel-tile based representation, as shown in the second stage of Figure \ref{p6fig:approach-steps}. The dimensions are downscaled to $13\times 7\times3$.
    %Each room tile is simplified to a single-color pixel, as shown in the second stage of Figure \ref{p6fig:approach-steps}, downscaled to $13\times 7\times3$.
    
    \item \textbf{Dimensions:} Each room is described by its five IC-MAP-Elites feature dimension values, excluding the similarity scores: \textsc{Linearity}, \textsc{Leniency}, \textsc{\#MesoPatterns}, \textsc{\#SpatialPatterns}, and \textsc{Symmetry}. A complete description of these features can be found in~\citepsixth{p6Alvarez2020-ICMAPE}.
    
    \item \textbf{Inner Content:} Each room is described by $12$ values, related to the count, sparsity, and density of the enemy, treasure, floor, and wall tiles contained in it.
    
    \item \textbf{Combined:} A combination of the \textbf{Dimensions} and \textbf{Inner Content} methods.
\end{enumerate}

\subsubsection{Clustering and Analysis}

To run all setups, data reduction algorithms, clustering algorithms, and do the internal evaluation of the clusters, we used scikit-learn machine learning toolset~\citepsixth{p6scikit-learn}. To obtain the best set of clusters, we ran different setups with the above datasets. The data was reduced to two meaningful dimensions with two different data reduction algorithms, Principal Component Analysis (PCA) and T-Distributed Stochastic Neighbor Embedding (T-SNE). For both data reduction algorithms, we fit the algorithms with each individual dataset, setting to two principal components and in the case of T-SNE using PCA as initializing algorithm, and transforming the data into a new dataset \emph{pca\_dataset} and \emph{tsne\_dataset} per dataset. Each two-dimensional point in the new datasets represents a step in the sequences described above.%Likewise, for the T-SNE, we fit the algorithm with each individual dataset, setting the parameters to two principal components and using PCA as initializing algorithm, and then transformed the data into a new dataset \textit{tsne_dataset} per dataset.

Moreover, all the resulting datasets were then clustered using \textsc{K-Means, K-Medoids, Agglomerative clustering}, and \textsc{DBSCAN}. K-Means was initialized using the standard k-means++ implemented in scikit-learn, which initialize all centroids distant from each other. K-Medoids was initialized similarly, using the standard k-medoids++, and tested using the \emph{cosine}, \emph{euclidean}, and \emph{manhattan} distances. Agglomerative clustering is a hierarchical clustering approach using a bottom-up approach implemented in scikit-learn using four different linkage criteria for comparing data points: \emph{Ward}, \emph{Complete}, \emph{Average}, and \emph{Single}. Finally, DBSCAN cluster points based on density separated by low-density areas; thus, DBSCAN automatically finds $k$ based on two parameters, $\epsilon$ describing the maximum distance between points and \emph{min\_samples} describing the minimum amount of samples within a group to be considered a cluster. K-Means, K-Medoids, and Agglomerative clustering were tested using multiple $K$ values ranging from 3 to 13, and DBSCAN was tested with several $\epsilon$ values ranging from 0.3 to 1.0, and \emph{min\_samples} ranging from 2 to 9.

%, testing with $K$ values ranging from 3 to 13 for the first three ones, and several $\epsilon$ values for DBSCAN.

%testing different minimum distance between data points ($\epsilon$) and the minimum amount of data points within a cluster to be considered a dense region for DBSCAN.

Since we lack a labeled dataset (i.e. ground truth) for cluster validation, we evaluated the results from all setups using the internal indices below, as well as manually inspecting the rooms composing the resulting clusters.

%Since in our approach lacks a labeled dataset (i.e. ground truth) for cluster validation, 

\begin{itemize}
\item \textbf{Silhouette Score:} The Silhouette Score shows how similar a data point is to the cluster it is associated with, through calculating the difference between the $\overline{distance}$ from the point to the points in the nearest cluster and the $\overline{distance}$ to the points in the actual cluster. The value is bounded from -1 to +1, with values closer to +1 indicating a good separation of the clusters, and closer to -1 meaning that some points might belong to another cluster.
\item \textbf{Davies-Bouldin Index:} The DB-index is the ratio between the within-cluster distances and between-clusters distances. With this, we can have an insight into the average similarity of clusters with their closest cluster. The value is bounded from 0 to +1, with values closer to 0 relate to clusters that are farther apart from each other and less dispersed, thus, this index is more crucial when we have more dense representations.
\item \textbf{Calinski-Harabasz Index:} The CH-index is another index related to the density of the clusters and how well separated they are. The score is the ratio between the within-cluster dispersion (compactness) and the between-cluster dispersion (separation). The CH-index is positively unbounded, and the higher the score the better.
\end{itemize}

\subsubsection{Cluster Labelling}

\begin{table}
\begin{center}
{\caption{Best performing setups based on their internal validation and visualization of clustered data points.}\label{p6table:setups}}
\resizebox{0.9\textwidth}{!}{
\begin{tabular}{ccccccc}
\hline
\rule{0pt}{12pt}
Algorithm&Data&K&$\Diamond$&$\Box$&$\bigtriangleup$ 
\\ 
\hline
\\[-6pt]
K-Means & Tiles-PCA & 9 & 0.43 & 0.73 & 9438.233 \\ 
K-Means & Tiles-PCA & 12 & 0.41 & 0.77 & 9436.928 \\
K-Means & Dimensions-PCA & 12 & 0.43 & 0.73 & 7738.343 \\
Agglomerative single & Combined-PCA & 6 & 0.51 & 0.43  & 38.833 \\ 
Agglomerative avg. & Dimensions-PCA & 6 & 0.44 & 0.67 & 3463.567 \\ 
\hline
\\[-6pt]
\multicolumn{6}{l}{$\Diamond$ Silhouette Score\ \
$\Box$ Davies Bouldin Index\ \
$\bigtriangleup$ Calinski-Harabasz Index}
\end{tabular}
}\end{center}
\end{table}

Table~\ref{p6table:setups} shows the best performing setups according to their internal indices scores. The clusters in these setups were manually inspected in order to detect the qualitative features that better define them. 

When using the \textbf{Dimensions} and \textbf{Combined} datasets, the clusters do perform good, if not better, in certain indices than when using the \textbf{Tiles} dataset. However, when analysing the resulting setups, they were missing a clearer relation between the clustered rooms, which was exacerbated when analysing sequences and paths on these setups, where they missed continuity between clusters.

Conversely, given that we are creating tile-based rooms and dungeons, the features were more representative for the \textbf{Tiles} dataset, which when used, generally performed well in the evaluated internal indices, and the produced clusters meaningfully separate the data. Further, as it will be presented in Section \ref{p6section:results}, when clustering sequences and analyzing the cluster path of the designs, there exist a continuity between designs that supports its usability. Figure \ref{p6fig:all-clusters} shows the best-resulting cluster set found among all the experiments run.



% As expected, the \textbf{Tiles} dataset generally performed well in the evaluated internal indices, and the produced clusters meaningfully separated the data.
% better information and were meaningfully group together.  relation each of the features have with 

% As expected, the \textbf{Tiles} representation have good results across the 3 indices, 

% JOSÉ: I leave it here waiting for the final results from Alberto

%Moreover, We noticed that the agglomerative approach results in very specific clusters alongside a quite broad cluster consisting of unrelated data points, regardless of the $K$ used. These setups scored well in the different indices but fail to accurately partition the space in relevant groups.

%Moreover, there are recurrent clusters between the different setups but when using more clusters like in figure~\ref{p6fig:all-clusters} (b), we can have more granularity when partitioning the space, improving the separation of more related data points. In figure~\ref{p6fig:all-clusters}(b) we present several rooms that have been clustered together matching the labeling of the clusters. In the figure, there is a clear correlation between the designs and the labels of their respective cluster, and an interesting continuity between the final clusters.

% In Figure~\ref{p6fig:all-clusters}, we present the final and selected approach for clustering room styles using K-Means (K=12) and the \textbf{Tiles} dataset reduced with the PCA algorithm. To the right, next to each color in the legend, we have different representative rooms that belong to the clusters, in their respective color, and have been clustered together. Furthermore, besides the local relation between clusters, there exists a layered division among group of clusters in the y-axis, where the bottom clusters relate more to architectural pattern complexity, from very empty rooms to mazes. The middle clusters focus on populating the rooms with enemies and treasures, creating the actual goals of the room and balancing the challenge. Finally, the top clusters are composed of dense rooms where the enemy and treasure addition do not necessarily need to follow any clear objective. 


% The clusters's label are plotted on top of each of the clusters, describing in general, the content that is within them. These cluster 

In the figure, we have plotted on top of the clusters the labels describing in general, the content that is within them. The following is a description of the clusters and the rooms that were clustered together:

\textbf{0. Empty-Initial rooms:} %This cluster contains $1797$ data points, and 
This cluster relates mostly to the initial designs made by the designers. These designs are from completely empty rooms to initial work-in-progress structures.

\textbf{1. Maze-like complex architecture:} This cluster to the extreme of the architectural patterns complexity layer, relates to more highly-linear, confined and maze-like rooms.% with more structure on what is possible. %The cluster contains $269$ data points.

\textbf{2. Dense, less organized:} This cluster contains rooms that still have a certain objective but are moving towards more disorganized distributions of micro-patterns in relation to their density. %This cluster contains $410$ data points.

\textbf{3. architectural complexification:} %This cluster contains $841$ data points, and 
This cluster relates mostly to the complexification of wall structures by having dense wall chunks, representative architectural patterns, or symmetrical patterns.

\textbf{4. Dense, full range leniency:} Focusing on density as the other two clusters within the same layer, this cluster relates to rooms that are in the full range of leniency from very rewarding, treasure rooms to very challenging boss rooms. %This cluster contains $144$ data points.

\textbf{5. Separating and populating chambers:} This cluster relates to the process of separating rooms into distinct chambers, focusing on the center of the room, and starting to populate rooms with enemies and treasures. %The cluster contains $781$ data points.

\textbf{6. Balancing and optimizing:} This cluster contains a mix between corridors and chambers within rooms with a focus on balancing rooms and optimizing their design towards certain goals. %The cluster contains $638$ data points.

\textbf{7. Bordered rooms with deeper architectural development:} This cluster relates mostly to rooms with an added wall border by the designer, and where the focus is to shape chambers and develop more visual structures.

\textbf{8. Main architectural shapes:} Similar to other clusters within the same layer, this cluster relates to the development and definition of main architectural patterns that are somewhat symmetric.

\textbf{9. Dense, disorganized micro-patterns:} This cluster clusters the extreme rooms that contain a high density of tiles, other than floor-tiles, without a clear structure or objective for the player.

\textbf{10. High challenge, clear goal:} This cluster relates to well-shaped rooms with clear wall structures and goals, towards more challenge. 

\textbf{11. Chamber separation with forced enemy encounter:} This cluster relates to rooms that are in the process of a clear segmentation into corridors and chambers, and that enforce to some extent, enemy encounters for the player. 

Furthermore, besides the local relation between clusters, the clusters are implicitly divided in three layers on the Y-axis. From bottom to top, (a) architectural patterns complexity, relating to clusters composed of rooms with clearer or complex shapes done with walls, from empty rooms to mazes. (b) Goal creation, enemy/treasure balance, with clusters comprehending the strategic addition of enemies and treasures to establish objectives in the room for the player. In terms of EDD, these rooms are composed of more meso patterns. And (c), over-population, which relates to clusters filled with less organized and dense rooms where the enemy and treasure addition do not necessarily need to follow any clear objective. Identifying the designer in such layer, and the path they have taken to get there could show meaningful information in the design process. For instance, the intentions of the designer, in what phase of the design process she is at the moment i.e. trying the tool or observing how the tool reacts or scraping her current goal towards a new goal within the room. 

% \begin{enumerate}
% \setcounter{enumi}{-1}
% \item[0] \textbf{Empty/Initial rooms:}
% \item \textbf{Complex wall mazes:} 
% \item \textbf{Dense, less structure} 
% \item \textbf{Structural complexification:} 
% \item \textbf{Dense, full range leniency} 
% \item \textbf{Separating and populating chambers:} 
% \item \textbf{Balancing and optimizing rooms:} 
% \item \textbf{Bordered rooms with deeper structural development:}
% \item \textbf{Development of main structural shapes:} 
% \item \textbf{Dense, unstructured:} 
% \item \textbf{Challenging rooms with clear goal:} 
% \item \textbf{Chamber separation with forced enemy encounter:} 
% \end{enumerate}



\begin{figure*}[t!]
\centerline{\includegraphics[width=\textwidth]{figures/resulting-paths-FINAL.png}}
\caption{Final and common designer trajectories. With thick arrows it is presented the archetypical paths, calculated using the frequencies of subsequences from $180$ diverse rooms. Each color represent a unique trajectory; with green the \textsc{Architectural-focus}, with red the \textsc{Goal-oriented}, with black the \textsc{Split central-focus}, and with blue the \textsc{Complex-balance}. Finally, thinner purple arrows extending from clusters traversed by the archetypical paths show the multiple possible branches that an archetypical path can deviate or extend to.} \label{p6fig:finalPaths}
\end{figure*}

% \begin{figure*}[t]
% \centerline{\includegraphics[width=16cm]{figures/final-cluster.png}}
% \caption{Best resulting cluster sets. (a) is K-Means (K=9), and (b) is K-Means (K=12), both are using the \textbf{Tiles} Dataset. While (b) performs slightly worst in the internal indices, when inspecting the qualitative features, it successfully subdivides the main bottom clusters which grants us with more granularity to label and cluster the design process of designers.} \label{p6fig:all-clusters}
% \end{figure*}

% \begin{figure}[t]
% \centerline{\includegraphics[width=8cm]{figures/cluster-figure-updated.png}}
% \caption{Overview of how the design style clustering would be used and integrated into the evaluation of the suggestions provided to the user. %is used and integrates into the evaluation of the suggestions provided to the user.
% } \label{p6fig:cluster}
% \end{figure}

% \begin{figure}[t]
% \centerline{\includegraphics[width=8cm]{figures/all_clusters.png}}
% \caption{All the selected clusters to be labeled and analyzed. In order, each of the clustering approaches correspond to the setups presented in table~\ref{p6table:setups}. (e) Shows extra information on the designs that were clustered together, and which were used to label the respective cluster.} \label{p6fig:all-clusters}
% \end{figure}

% \begin{figure}[b]
% \centerline{\includegraphics[width=8cm]{figures/hand-made-clusters.png}}
% \caption{Example of hand-made room designs used to create the clusters. only (a) and (b) belong to the same clusters} \label{p6fig:handMadeClustered}
% \end{figure}

% \begin{figure*}[t]
% \centerline{\includegraphics[width=18cm]{figures/approach_steps.png}}
% \caption{Rooms at generation $2090$ targeting Number of spatial-patterns (X) and Symmetry (Y). Each cell displays (top-right) the fitness of the optimal individual in its related feasible population. }
% \label{p6figs:approachSteps}
% \end{figure*}

% \begin{figure}[t]
% \centerline{\includegraphics[width=8cm]{figures/cluster-figure-updated.png}}
% \caption{Example of a figure caption.}
% \label{p6fig:implementationClusters}
% \end{figure}
% Please add the following required packages to your document preamble:
% \usepackage{graphicx}
\begin{table}[]
\centering
\begin{tabular}{|l|lll|}
\hline
        & AIv1       & AIv2        & AIv3        \\ \hline
Leniency        & 0.56±0.07  & 0.62±0.09   & 0.57±0.08   \\
Linearity        & 0.91±0.02  & 0.92±0.02   & 0.91±0.01   \\
MesoPat       & 0.15±0.05  & 0.13±0.07   & 0.12±0.05   \\
SpatialPat    & 0.35±0.1   & 0.41±0.11   & 0.34±0.09   \\
Symmetry   & 0.43±0.11  & 0.35±0.18   & 0.35±0.12   \\
$W_{dens}$ & 0.27±0.09  & 0.26±0.08   & 0.21±0.05   \\
$W_{spar}$ & 0.21±0.05  & 0.19±0.03   & 0.15±0.01   \\
$E_{dens}$ & 0.24±0.07  & 0.27±0.06   & 0.3±0.06    \\
$E_{spar}$ & 0.22±0.05  & 0.32±0.05   & 0.35±0.06   \\
$T_{dens}$ & 0.37±0.13  & 0.28±0.09   & 0.34±0.07   \\
$T_{spar}$ & 0.36±0.11  & 0.3±0.1     & 0.37±0.05   \\ \hline
Steps      & 39.25±6.38 & 84.31±14.85 & 76.75±17.02 \\ \hline
\end{tabular}
\caption{Summary of the created rooms filtered by the AI version used. All values are the average of all the created rooms using the specific AI version. The first five values relates to the MAP-Elites dimensions, then the fitness of the rooms, the density and sparsity values for wall (W), enemies (E), and treasures (T), and finally the avg. steps taken to design a room.}
\label{tab:AIavgValues}
\end{table}

\subsection{Experiment Setup}

We conducted a user study to explore the user experience of using different levels of AI agency, the different design characteristics, and the relationship between the human designer and the AI. We collected both quantitative data on the AI's impact on the co-designed end product and qualitative data through think-a-loud and semi-structured interviews regarding the users' experience when interacting with the AI. The interview structure is inspired by the pyramid model, meaning the interviews will begin with specific questions, and gradually have more open questions, which naturally allows for a discussion towards the end. This model is chosen to support the variation of subjects the interview is desired to cover, as well as support natural transitions between the questions and their openness. The questions and user study procedure can be found in Appendix A. 

%The interviews are semi-structured, meaning it includes both closed and more open questions, and depending on the discussion and answers, some questions might be omitted.

% We collected quantitative data regarding what impact the AI had on the co-designed end product, and how the human designer interacted with the AI's contributions. Likewise, we collected qualitative data through recorded think-aloud observations and semi-structured interviews regarding the users' experience, and possibly catch certain remarks of frustration or appreciation of their digital colleague that can be valuable for the discussion of the relationship between the co-creators. 


%We collected the following data:

%\begin{itemize}
%    \item \textbf{Audio Recordings:} 
%\end{itemize}

Eight participants tested our tool with game design and level design experience. One participant was a professional game designer with eight years of professional experience (first participant), and seven participants were third-year Game Development students. They all had an individual digital session, where we shared our screen, and they took remote control to conduct the study. Participants accepted to participate, signed consent forms, and then received a short introduction describing the experiment and its steps. The participants were then asked to design two contiguous rooms in a dungeon, repeating this process for each of the AI variants and expressing their design decisions verbally whenever they felt like it. After using the tool, the participants were interviewed, focusing on and covering an overarching understanding of the user experience, particularly in terms of creativity and interaction with the AI.

% the relationship that occurs between the AI and human designer (See Appendix B). 

For all the sessions, human designers could place up to 12 tiles, and the AI could place as many tiles as the human placed. The AI could contribute only in a rectangular area surrounding the tiles the human designer recently contributed with, including a margin of 1 tile. This choice is made to support a responsive and collaborative behavior of the AI that builds on the human designer's contribution.


% The locations available for the AI to contribute in for each turn are limited to a rectangular area surrounding the tiles the human designer recently contributed with, including a margin of 1 tile. This choice is made to support a responsive and collaborative behavior of the AI that builds on the human designer's contribution.

% The margin for the contribution area is set to 1, as it was found during experimentation that any margin bigger than this is likely perceived as the AI contributing to other areas than the ones the human is focused on, because of the default size of the room being relatively small.

%  as this enables the designer to contribute with an adequate amount of tiles during their turn and create representable structures

% The rooms produced during the user study are displayed in Figure 6, 7 and 8. Rooms with red borders are infeasible, meaning there are unreachable tiles. The UI displays a warning when this happens, and the AI can repair this during its turn, however the resulting rooms that are infeasible are a result of the human designer creating unreachable areas, and then immediately selecting to go to the World Editing view, before pressing "End Turn". 
% Each participant created two rooms for each version of the AI. Participant 1 created Room 1 and Room 2 for all version, Participant 2 created Room 3 and Room 4 for all version, etc. All of the participants had the option to adjust the sizes of the rooms in the World Editing view before entering the Room Editing view, however none of them did, and therefore all of the resulting rooms are of the default size. The designer also has the option to change the location of the hero and the doors. The location of the hero was only moved twice in all of the session, and the doors where never moved. 






%, the participant will take part in an interview. The questions, and their order, are planned out and designed to cover an overarching understanding of the user experience, in particular in terms of creativity, and the relationship that occurs between the AI and human designer (See Appendix B). 




%The participants were asked to repeat this for each of the AI-initiatives. was asked to repeatEach pair of room 

% then asked to complete three tasks, each regarding

%The users were then asked to complete three tasks that covered the tool's functionality and the AI-initiatives, respectively for each task. The tasks were 


%and different approaches to creating quests. The tasks were to 1) manually create a quest, 2) automatically create a quest, and 3) create a quest through mixed-initiative. They were also asked to create a dungeon that suited their preferences and objectives before creating quests. The questionnaire consisted of 17 closed-ended questions, and the rest were open-ended. The interview began with a questionnaire with six questions about the users' background and experience within game development and finish with questions about their experience and opinions on the tool. Both the questionnaire and interview followed guidelines described by



%The participants used the tool



\subsection{Conclusions}



% \begin{itemize}
%     \item Discussion on what does this archetypical design trajectories mean?
%     \item how to use them? next steps into integrating this into a system. To use this in a search-based approach as objectives for the generation to move towards the directions where (according to our archetypical design trajectories) the designers will move towards in their design process. Perhaps I could also bring the discussion from the workshop-paper for HC-AI.
%     \item discussion on creativity? is the output or the process where the actual creativity is outputted? Compare using end-design clustering to using sequences to cluster.
%     \item Discussion on how PCGRL relates to this type of work? --> Perhaps this is something for the background instead.
% \end{itemize}{}


% This paper presents a step towards designer modelling in a MI-CC environment by providing an implementation of designer personas as archetypical trajectories through style space, as a means to characterize several representative and frequent design styles together. 

%This paper presents a novel approach and meaningful steps towards designer modeling in an MI-CC environment. By providing an implementation of designer personas as archetypical trajectories through style space, we show that 

This paper presents a novel approach and meaningful steps towards designer modeling through an experiment on archetypical design trajectories analysis in an MI-CC environment. Through this, we characterize several representative design styles as designer personas. We have first run and compared several clustering setups to find the best partitioning of the design style using the edition sequences of the collected $180$ unique rooms, ending in $8196$ data points, and resulting in a set of twelve cohesive, coherent, and meaningful clusters. We have then mapped these $180$ design sequences in terms of these clusters, applying frequent sequence mining to find four frequent and unique designer styles, with related common sub-styles. As a result, we have presented a roadmap of design styles over a map of data-driven design clusters. 

%This paper presents a step towards designer modeling through an experiment on archetypical design trajectories analysis in an MI-CC environment, as a means to characterize several representative design styles as designer personas. We have first run and compared several clustering setups to find the best partitioning using the edition sequences of the collected $180$ unique rooms, ending in $8196$ data points, and resulting in a set of twelve cohesive, coherent, and meaningful clusters. We have then mapped these $180$ design sequences in terms of these clusters, applying frequent sequence mining to find four frequent unique designer styles, with related common sub-styles. As a result, we have presented a roadmap of design styles over a map of data-driven design clusters. %The examples in Figure \ref{p6fig:archetypical-examples}, help us to clarify 

%  namely the \textsc{Designer Personas}

% Our work draws on the ideas, concepts, and goals and concepts proposed by Liapis et al. when introducing the Designer Modeling as a model to capture multiple designer's processes. A prototype of such was implemented in the sentient sketchbook~\citepsixth{p6Liapis2014-designerModelImpl}, where it is proposed the use of interactive evolution by biasing the search space in favor of hand-crafted features of the design. we propose an alternative and novel route to designer modeling through clustering the design space and the room style based on the collected data. Moreover, we differ as well on the type of level design, being the sentient sketchbook a tool for strategy games~\citepsixth{p6liapis_generating_2013}, while EDD a tool for adventure and rogue-like games~\citepsixth{p6Alvarez2020-ICMAPE}. These differences strengthen the importance and usefulness of designer modeling, and highlight the holistic and generic properties of this designer-centric perspective.

Designer modeling was proposed as an approach to capture multiple designer's processes to create a better workflow by Liapis et al.~\citepsixth{p6Liapis2013-designerModel}, and our work draws on many of their ideas, concepts, and goals. Furthermore, a prototype of such was implemented in the sentient sketchbook~\citepsixth{p6Liapis2014-designerModelImpl}, where it is proposed different approaches to model style, process, and goals based on choice-based evolution and the designer's current design to adapt the provided suggestions accordingly. We propose an alternative route to designer modeling through clustering the design space and the room style based on the collected data. Moreover, we differ in the type of level design, being the sentient sketchbook a tool for strategy games~\citepsixth{p6liapis_generating_2013}, while EDD is a tool for adventure and rogue-like games~\citepsixth{p6Alvarez2020-ICMAPE}. These differences strengthen the importance and usefulness of designer modeling and highlight the holistic and generic properties of this designer-centric perspective and its possibilities.

% Designer modeling in computer-aided design tools was proposed by Liapis et al.~\citepsixth{p6Liapis2013-designerModel} as an approach to capture multiple designer's processes to create a better workflow, and a prototype of such was implemented in the sentient sketchbook~\citepsixth{p6Liapis2014-designerModelImpl}. While our work drags on many of the concepts, ideas, and goals described by Liapis et al., we propose an alternative route to designer modeling through clustering the design space

% In their work, they propose the use of hand-crafted

% Our work drags on many of the concepts, ideas, and goals described in~\citepsixth{p6Liapis2013-designerModel}, but we propose an alternative route to designer modeling through clustering the design space and the room style based on the collected data. In contrast 

% Their work propose the use of interactive evolution by biasing the search space in favor of hand-crafted features of the design akin to~\citepsixth{p6Alvarez2020-DesignerPreference}. However, we propose an alternative and novel route to designer modeling through clustering the design space and the room style based on the collected data. Moreover, we differ as well on the type of level design, being the sentient sketchbook a tool for strategy games~\citepsixth{p6liapis_generating_2013}, while EDD a tool for adventure and rogue-like games~\citepsixth{p6Alvarez2020-ICMAPE}. Applying the idea of designer modelling to both genres, not only shows the importance and usefulness of designer modeling but also the holistic and generic view 

% These differences strengthen the importance and usefulness of designer modeling, and highlight the holistic and generic properties of this designer-centric perspective.

% % might be interesting to discuss this.
% While the approach described in this paper is applied in a tool for creating zelda-like dungeon games~\citepsixth{p6tloz}, the approach can be reused and extended to other domains 

These contributions allow us to better understand, cluster, categorize and isolate designer behavior. This is very valuable for mixed-initiative approaches, where a clear virtual model of the designer's style allows us to better drive the search process for procedurally generating content that is valuable for the designer. Designer personas have the potential to be used in many different scenarios. For instance, as objectives for a search-based approach to enable a more style-sensitive system, to evaluate the fitness of evolutionary generated content or to train PCG agents via Reinforcement Learning~\citepsixth{p6khalifa2020-pcgrl}. 

Moreover, recognizing the designers' current style and the path taken so far, which would indicate a possible designer persona, could open the possibility for recognizing their intentions, preferences, and goals. This traced roadmap of designer personas could let a content generator anticipate a designer's next moves without heavy computational cost, just by identifying her current location on the map and offering content suggestions that lie in the most promising clusters to be visited next. Conversely, it could also identify designers who do not follow a certain path, i.e. deviating from the pattern, trying to understand their objective through their design style.

% Finally, in our work, we did not observe any type of cross-path i.e. a design going from one path to another. We believe that this is due to the level at which we observe the archetypical paths. However, preliminary analysis on the dataset used in this paper and as expected, the design process of designed rooms within the same dungeon does follow different paths, and sometimes even crossing each other. This opens an interesting and exciting area to explore a wider layer, taking rooms as a set of archetypical paths taken by designers. Observing the paths taken in previous and future rooms, and the dungeon as a whole, as briefly introduced in section~\ref{p6sec:designStyle}, to understand the designers' intentions and goals when they proceed to create a new room is a promising future step to take with the current system. 

%  i.e. room-wise, as the designer typically would design the room with a set of goals

Finally, it is also important to observe the nature of the previous and future rooms created by a designer. Observing the dungeon as a whole, as briefly introduced in section~\ref{p6sec:designStyle}, to understand the designers' intentions and goals when they proceed to create a new room is a promising future step to take with the current system. 

% Furthermore, the designer personas addresses the dynamic-dynamic system vs. dynamic-static system open question raised by Alvarez and Font~\citepsixth{p6Alvarez2020-DesignerPreference}, which relates to the challenge of adapting a system to the a ever-changing designer. With the use of the archetypical paths, the model is not anymore adapting and moving through the solution space with the designer, rather the designer traverse through an already clustered space. 

% With the use of the archetypical paths, we can not only identify the current designer persona the designer is following but we can also adapt and anticipate to what they might end up doing. 

% Furthermore, the designer personas addresses an open question raised by Alvarez and Font \citepsixth{p6Alvarez2020-DesignerPreference}, related to the challenges  using a dynamic-dynamic system vs. a dynamic-static system. The authors describe the dynamic-dynamic system as a system where both designer and AI-system move through the solution space, with the AI-system constantly trying to adapt to the designer. They concluded that the main challenge correspond to designers constantly concept drifting resulting in them continuously changing their decisions. Instead, the authors proposed the use of a dynamic-static system, where the model is not anymore adapting and moving through the solution space with the designer, rather the designer traverse through an already clustered space. With the use of the archetypical paths, we can not only identify the current designer persona the designer is following but we can also adapt and anticipate to what they might end up doing. 


%and conclude that the main challenges in

% Moreover, this traced roadmap of designer personas could let a content generator anticipate a designer's next moves without heavy computational cost, just by identifying her current location on the map and offering content suggestions that lie in the most promising clusters to be visited next. %Further, one could also be able to identify designers that do not follow a certain path i.e. deviating from the pattern, and try to understand through their design style their objective.





% From the $180$ unique rooms, we extracted and used the edition sequence of each of the rooms, from their initial design to the more elaborated end-design, to compose a richer dataset that could capture the design process of a designer rather than focusing on the end-point. Through this, we ended up using $8196$ data points in our dataset.

% We have first run and compared s


% through experimenting with 

% This paper presents a step towards designer modelling by providing a prototype implementation of designer personas as archetypical trajectories through style space. These archetypical paths

% This paper presents an experiment on archetypical design trajectories analysis in a MI-CC environment, as a means to characterize several representative design styles as designer personas. We have first run and compared several clustering setups to find the best partitioning, resulting into a set of twelve cohesive, coherent, and meaningful clusters. We have then mapped almost 200 complete design sequences in terms of these clusters, applying sequence mining to find four frequent unique designer styles, with related common sub-styles. As a result, we have presented a roadmap of design styles over a map of data-driven design clusters. 



% be used as goal for other systems were anticipating a design or creating a synthetic objective might be more complicated. We envision that these designer personas can be used within 


%  d) shows a step by step process when using AIv2. For a, b, and c, 1, 3, 5, 9, 11, relate to participants 1, 2, 3, 5, and 6, respectively.

\begin{figure*}[h!]
 \includegraphics[width=\textwidth]{images/rooms-designed-2.png}
 \caption{Sample step by step process when using AIv2. Human's turn is top row, while the AI's turn is the bottom row.}
 \label{fig:exampleDesignProcess}
\end{figure*}

\subsection{Discussion}
\label{sec:discussion}
% This section contains an analysis of the results presented in Section VI, as well as a discussion to connect the results to the theoretical framework and the specific aspects relating to the research questions. 


\subsubsection{Willingness to include the AI in the design process}

All participants expressed an interest and willingness to see what the AI could come up with to design the rooms, emphasizing that they either considered or incorporated the ideas brought forward by the AI, which further supports other MI-CC research conclusions~\cite{p13guzdial_friend_2019,bhaumik_lode_2021}. However, many participants expressed multiple frustrating factors, and based on figure~\ref{fig:human-ai-contribution}.a, when given the opportunity, the human designers did not include much of the AI's contribution but might have provided some ideas that either influenced or were part of the final design. Figure~\ref{fig:human-ai-contribution} also displays that as the AI got more agency, the AI tiles at the end design increased. This can be due to frustration expressed by participants where they didn't agree with the AI design and ended up handing over the control to the AI and giving up, to some extent, their aspiration to create. The lower meso- and spatial-patterns combined with the lower symmetry in AIv3 are also part of the issue. The fewer patterns that exist mean that the rooms are less structured, which, combined with no way to correct these, could result in designers feeling that the levels are more ``random".


%in general, this would make designers less 

 %as th, except for Symmetry, which reduces as the AI gains more agency. Similarly, patterns are reduced when using high agency, which shows  

%As expected, the amount of tiles remaining at the end design increased as the AI got more initiative.


%All of the participants expressed an interest and willingness to see what the AI could come up with. Many of the participants expressed that they either considered the ideas brought forward by the AI, or incorporated them. This further strengthens the conclusion made in other studies exploring MI-CC Game Design tools, that human designers are interested and willing to using MI-CC to create game content. However, as many of the participants also expressed multiple frustrating factors when using the tool,Additionally, the quantitative data further suggests that when proposed with the option, many of the human designers where in fact not including a lot of the AI's contributions in their end products.


%As displayed in Figure 6 with the resulting rooms, and Figure 9 with the representation of tiles in the resulting rooms, the human designers where disinclined to incorporate the AI-placed tiles in the final product when using AIv1. As displayed in Figure 9, the human designers generally did not place down many of the suggestions that the AI made in AIv1. This suggests that the version of the AI with low initiative did not have great influence on the final design, however the AI might have provided some ideas that either influenced or was part of the end product.
%When comparing Figure 12, 13 and 14, displaying the divisions between the tiles in the resulting rooms, one the general pattern can be identified, namely, as the AI gains initiative, the AI has an increasing impact on the final product. As many participants expressed that the AI made creations that they did not agree with, and that they handed over the control to the AI and gave up their aspiration to create, the end product consisting of a majority AI-placed tiles is likely because of this frustration. 



\subsubsection{Variants of the users}

Most participants used the tool similarly except for participants 1 and 6. Participant 1 didn't want to incorporate the AI's contributions, as it can be seen in fig.~\ref{fig:human-ai-contribution}, explicitly stating that ``... I don't think level design is a good place for an AI that has more control than the human... The little details that I love in level design would never be created by an AI. Nice little references, or easter-eggs, or how humans get inspired by simple things..." On the other hand, participant 6 recurrently incorporated many of the AI's tiles. When using AIv2 and AIv3, they pressed ``End Turn" repeatedly to find out what the AI would be able to create, commenting ``I want to see if it can create something cool." However, while their approach was completely different, they both agreed that they would prefer the AI to create a complete room, and they could polish it from there.

% , which is the common approach in MI-CC~\cite{p13sentient-sketchbook,eddy1}.

%Out of the eight participants there were two in particular, who used the tool in very different ways from each other. Participant 1, who created room 1 and 2, had a strong inclination to not incorporate the AI's contributions. This is suggested in Figure 12 and Figure 13, where the final product consisted of over 95\% human-placed tiles when using AIv1 and AIv2, as seen as bar 1 and 2 in both charts.
%Participant 1 is a professional game designer, and has 8 years of professional experience within the field. They expressed the following:

%\textit{"I only see the usage of the Medium and High in the purpose of typical PCG-tools, like populating areas fast and efficiently. I don't think level design is a good place for an AI that has more control than the human. I think there are other areas where it fits better, such as PCG, creating big arrays of content where randomness is something you want, enemy clusters et cetera." ... "The little details that I love in level design would never be created by an AI. Nice little references, or easter-eggs, or how humans get inspired by simple things like walking in London and seeing a cool building, you want to put that into a level. I think AI can handle high-level stuff, like compositions of shapes and light and stuff, but that the details should be left to the human."}


%Participant 6, a third year Game Development-student, who created room 11 and 12, included a lot of the AI-placed tiles in their final products when using AIv2 and AIv3, as displayed in Figure 13 and Figure 14. Room 11 with AIv2, as shown as bar 11 in Figure 13, was the only room where all tiles placed in the end product where AI-placed tiles. This is also shown in Figure 7, as Room 11. Multiple times while using AIv2 and AIv3, Participant 6 pressed "End Turn" repeatedly, to find out what the AI would create. One of those times, the participant commented: 

%\textit{"I want to see if it can create something cool." }


%In the interview, when answering Question 11, Participant 6 suggested that a possible improvement to the tool could be displaying complete generated rooms, which the human can choose from and then edit a set amount of tiles in. This suggestion is very similar to the original state of EDD, which was used to modify for this project, works.  
%Interestingly, despite the differences in how  Participant 1 and Participant 6 used the tool, the way Participant 1 described their opinions of a better use of the tool, as an efficient way to generate rooms with the human polishing some final touches, is similar to how Participant 6 actually used the tool. 
%Additionally, as Participant 6 expressed interest in a tool similar to the original state of EDD, to efficiently create rooms with minor influence from the human, however control in the form of selecting out of the generated rooms, and polishing the rooms, this also coincides with the opinions of Participant 1.


\subsubsection{Frustrating factors and Constraints}

% When using EDD, the participants expressed multiple frustrating factors.
The participants expressed multiple frustrating factors within the tool. The main point was the repetitive behavior of overwriting the human tiles with floor tiles, removing their ideas without contributing with anything of value, and the human designer feeling forced to move on from those positions and contribute somewhere else in the room. This was exacerbated when using AIv2 as, unlike AIv1, the AI placed down the tiles rather than suggesting, and unlike AIv3, the human designer still had the option to overwrite the AI-placed tiles. The human designers assign value to each tile type as they provide different aspects in level design; for instance, participants often placed enemies and treasures close to each other, possibly to create a risk and reward in the level. Figure~\ref{fig:exampleDesignProcess} shows one example of a participant creating a room using AIv2, step by step. The human designer's first contribution includes a long continuous wall and a boss. When the AI has its first turn, it contributes with floor tiles overwriting the designer's tiles. Towards the end of the design, the human designer places a boss tile in the bottom left corner that the AI overwrites with floor tiles twice before the human designer gives up and finishes the room without a boss. This sample creation process shows that the AI tried to steer the room towards more leniency and open areas, which contradicted the human's goal. 

Another main frustrating factor was the loss of control experienced by human designers when co-creating with the AI, especially AIv3. Participants expressed that the AI's decisions limited them and were forced to work around what the AI designed. As the AI gained control, they felt their creative process got increasingly constrained. This aligns with the Lode Encoder study~\cite{p13bhaumik_lode_2021}, where the participants expressed frustrations with completing a playable level, as they were forced to rely on the AI to generate the option they wanted in the final stages of the level creation. Further, when using AIv3, the number of positions that the human has available decreases with every turn, while the AI can continue to place tiles on any position, which unavoidably limits the human's control over the final design.

Most of the participants felt frustrated and constrained as the AI gained more control over the design process. Additionally, all participants suggested removing the turns and constraints of the number of tiles per turn to improve the tool. Three of the eight participants expressed that adjusting the AI's role to one of an assistant to the human designer would improve their creative experience. 

%A few of the designers expressed a dislike of the turn-based creative process, as well as the constraint of maximum amount of tiles per turn. These participants expressed that they felt they were not collaborating with the AI, but rather working against each other and fighting for control over the design.  

%The participants expressed a dislike of the constraining nature of AIv3, where they are limited by the AI’s decisions, and that they were forced to work around what the AI designed. In the study of the MI-CC tool Lode Encoder, by Bhaumik et. al, the human designers where constrained to only use the content that the AI generated~\cite{p13lode-encoder}. The participants in the study of Lode Encoder expressed frustrations with completing a playable level, as they where forced to rely on the AI to generate the option they wanted in the final stages of the level creation. The increasing frustration some participants felt as the level creation proceeded when using Lode Encoder, is notably similar to the increasing frustrations the participants of this study expressed when using AIv3. The participants grew discontented as they where increasingly constrained by the AI's creations.


%When using AIv3, one particular constraint is very present. As the amount of positions that the human can place new tiles in decreases with every turn, while the AI can continue to place tiles on any position, there is an unavoidable limitation for the human designer. Many of the participants found this version of AI and this constraint particularly frustrating, and expressed that they gave up and let the AI control the design after few turns. This also is supported by the data, displayed in Figure 14, which shows that the resulting rooms from using AIv3 generally consisted of a majority of tiles placed by the AI. 




%The participants expressed multiple frustrating factors within the tool. One of the most commonly mentioned factor occurred when using AIv2. Many participants experienced that the AI used its turn to overwrite the walls, treasures, enemies and boss-tiles the human just placed down with floor-tiles. 

%When using AIv2, this behaviour in the AI was the most prevalent, as unlike AIv1, the AI placed down tiles during its round outside of the human designers control, and unlike AIv3, the human designer still had the option to overwrite the AI-placed tiles. In this version of the AI, this often created a frustrating back and forth for the human designers, where the AI spend its turn "removing" what the human designer had just placed, and the human designer spends their following turn replacing the same tiles again. This often was repeated until the human designer gave up, or the MAP-elites algorithm had finished a new run and the AI made different decisions.


% \begin{figure*}[h]
%     \centering
%     \includegraphics[width=\textwidth]{images/steps.png}
%     \caption{Step by step process of one participant using AIv2 to create a room. The top row shows what the human contributed with during the human's turn, the bottom row display what the AI responded with during its turn.}
%     \label{fig:my_label}
% \end{figure*}

%Figure 20 shows one example of a participant, Participant 5 creating his first room using AIv2, step by step. The human designer's first contribution includes a long continuous wall, as well as a boss-tile. When the AI has its first turn it contributes with floor tiles overwriting the long wall, and the boss tile. Towards the end of the design, the human designer places a boss tile in the left bottom corner, that the AI overwrites with floor tiles twice, before the human designer gives up and exits the room editing view without a boss in the room.


%The participants perceived the AI's seeming preference of overwriting human-placed tiles with floor-tiles as the AI removing their ideas, without contributing with anything of value. This suggests that human designers assign a value to each type of tile, as they provide different aspects of game level design. For example, participants seemed to value boss-tiles highly, as they provide a big challenge. The participants often placed enemies and treasures close to each other, possibly to create a risk and reward in the level. The AI however, does not value the tiles this way, and therefore is more prone to placing floor-tiles than the human designer is.
%As displayed in Figure 18, the most common types of tile to place as a human designer in descending order is wall, treasure, enemy, floor and finally boss. Comparing this data to the AI-placed tiles, displayed in Figure 19, the AI is considerably more prone to placing floor tiles, as it is the most common type of tile placed by the AI.
%As only 5\% of tiles placed by human designers where floor-tiles, compared to 61\% of the tiles placed by AI, the perception of an AI-agent with a fondness of overwriting human-placed tiles with floor-tiles is very likely accurate.


%Another notable difference in the tiles that the different co-creators placed is that the AI never placed a boss tile.  Moreover, many participants expressed that the AI placed floor-tiles where they had placed a boss-tile, actively removing the boss. As shown in Figure 6, 7 and 8, many of the participants where prone to creating one normal room as their first room, and secondly one boss-room, for each version of the AI. Also displayed in Figure 6, 7 and 8, is the pattern that the amount of boss-tiles in the resulting rooms decreased with an increase of initiative from the AI. This further correlates with the notable disagreement between AI and human designers in terms of valuing tiles differently, and how it affected boss-tiles in particular. 


%One other main frustrating factor that was brought up was the loss of control that the human designers experienced when co-creating with the AI with high initiative, AIv3. This was many of the participants motivation to name AIv3 as the version they liked the least. The participants expressed a dislike of the constraining nature of AIv3, where they are limited by the AI’s decisions, and that they were forced to work around what the AI designed. In the study of the MI-CC tool Lode Encoder, by Bhaumik et. al, the human designers where constrained to only use the content that the AI generated~\cite{p13lode-encoder}. The participants in the study of Lode Encoder expressed frustrations with completing a playable level, as they where forced to rely on the AI to generate the option they wanted in the final stages of the level creation. The increasing frustration some participants felt as the level creation proceeded when using Lode Encoder, is notably similar to the increasing frustrations the participants of this study expressed when using AIv3. The participants grew discontented as they where increasingly constrained by the AI's creations.

%\subsubsection{Constraints}

%Many of the participants expressed that as the AI gained control, they felt increasingly constrained in their creative process. When using AIv2, some participants felt constrained when the AI repeatedly overrode human-placed tiles with floor, and the human designer felt forced to move on from those positions and contribute somewhere else in the room. Another constraint that arose from this behaviour in the AI, was that many participants expressed that the AI did not allow them to place boss-tiles, and the human designer therefore was constrained to not create a boss room, as they had initially had intended. 


%When using AIv3, one particular constraint is very present. As the amount of positions that the human can place new tiles in decreases with every turn, while the AI can continue to place tiles on any position, there is an unavoidable limitation for the human designer. Many of the participants found this version of AI and this constraint particularly frustrating, and expressed that they gave up and let the AI control the design after few turns. This also is supported by the data, displayed in Figure 14, which shows that the resulting rooms from using AIv3 generally consisted of a majority of tiles placed by the AI. 


%A few of the designers expressed a dislike of the turn-based creative process, as well as the constraint of maximum amount of tiles per turn. These participants expressed that they felt they were not collaborating with the AI, but rather working against each other and fighting for control over the design.  



\subsubsection{The concept of a well performing, high agency co-creator}

%Since there is a willingness from human designers to incorporate AI into creative processes, and there exist distinct advantages of using MI-CC, it is important to identify why this tool did not provide a satisfactory creative experience. 



Most participants showed a willingness to incorporate AI into the creative process, contributing with new ideas or performing services such as ensuring feasibility. Yet they were reluctant to incorporate higher agency AI, which suggests that the AI needs to be aligned with their goals, intentions, and procedures, i.e., have an accurate designer model~\cite{p13liapis_designer_2013}. Within EDD and level design tools, multiple practical improvements are to be made. Five participants described the AI's behavior as random and unpredictable, especially when overwriting human-made structures and contributions. The AI currently calculates the most common tiles in the positions of the contribution area and contributes with the tiles of highest occurrence among a set of generated elites. This contradicts how human designers perceive the design importance of tiles, valuing higher usable tiles rather than floors. The AI could then weigh higher those and the combined structures they create. Additionally, the AI could favor unedited areas before overriding human-placed tiles to support rather than override. 

Another important point is that all AI versions are static in the design process, which means that the AI follows the same procedure regardless of the agency level. The collaboration do change in the design process (e.g., suggest or directly placing tiles, or if AI tiles can be removed) but other aspects and parameters do not change or adapt to designers. These parameters are connected to the overarching design of the AI rather than the AI's agency, which might have affected the designers' perception. For instance, given that the AIv3 tiles could not be replaced, changing the amount of tiles, rectangular area, or its adaptability in regards to what the designer had created thus far could be beneficial.

%Adaptation could take place by using designer modeling to tailor the changes. We could also analyze the designers' trajectory in relation to the behavioral dimensions. Then, we could map how the room has changed regarding the dimensions and what type of goal the designer might be going towards, and bias the KNN distance metric towards that.

%We could also or by analyzing the designer' current trajectory in the behavior space. Their trajectory in relation to the behavioral dimensions, we could map how the room has changed regarding the dimensions and what type of goal the designer might be going towards. Then, KNN could bias the neighborhood towards those.

%The AI does collaborate differently in the design process (e.g., how to place tiles or 

Furthermore, the AI seemed to break apart walls and open up sub-rooms. This is possibly a result of the Linearity and Meso-Pattern dimensions in the MAP-Elites algorithm. The resulting elites of the generated rooms with the highest linearity will have the highest amount of traceable paths. Many participants seemed to want to create rooms with long walls, sub-rooms, paths that required the player to encounter enemies, and common aesthetical features such as symmetry. Analyzing the path designers are taking in these dimensions could better inform the search for content and the generation of elites, so the content is adapted to those preferences. However, adapting these dimensions might be counterproductive for the other dimensions, as symmetric rooms might not create balanced rooms regarding the Leniency dimension as it might not be considered as important. Another approach could be to incorporate designer modeling. By identifying possible design goals or design styles of the human co-creator, the AI can adjust its decisions and behavior to offer different levels of support depending on the human designer's behavior~\cite{p13liapis_designer_2013,alvarez_designer_2022}. 

%Furthermore, the AI seemed to break apart walls and open up sub-rooms. This is possibly a result of the Linearity and Meso-Pattern dimensions in the MAP-Elites algorithm. The resulting elites of the generated rooms with the highest linearity will have the highest amount of traceable paths. Many participants seemed to want to create rooms of low linearity, with long walls, sub-rooms, and paths that required the player to encounter enemies. Likewise, symmetry was sought by most designers, which is a common aesthetical feature. Analyzing the path designers are taking in these dimensions could better inform the search for content and the generation of elites, so the content is adapted to those preferences. However, adapting these dimensions might be counterproductive for the other dimensions, as symmetric rooms might not create balanced rooms regarding the Leniency dimension as it might not be considered as important. Another approach could be to incorporate designer modeling. By identifying possible design goals or design styles of the human co-creator, the AI can adjust its decisions and behavior to offer different kinds of levels of support depending on the human designer's behavior~\cite{p13liapis_designer_2013,alvarez_designer_2022}. 



%The majority of the participants preferred the version of the AI with low initiative, AIv1. The second most preferred version of AI was AIv2. However, as there is a willingness from human designers to incorporate AI into creative processes, and there are distinct advantages to use MI-CC in game level design, it is important to identify why this tool did not provide a satisfactory creative experience, as well as how this can be used in further research to allow our understanding and development of MI-CC to be further developed.

%A majority of the participants felt frustrated and constrained as the AI gained more control over the design process. Additionally, all participants suggested removing the turns and constraints of amount of tiles per turn to improve the tool. Three of the eight participants expressed that adjusting the AI's role to one of an assistant to the human designer would improve their creative experience. Although a willingness to incorporate an AI agent who displays new ideas, or performs services such as ensuring feasibility has been identified among the majority of the participants, they are seemingly more reluctant to incorporate an AI of higher initiative and influence into their creative process. This suggests that for human designers to have a satisfactory experience with a MI-CC tool with an AI of high initiative and definitive impact on the design, the AI must achieve a level of behaviour which the human agrees with.


%As this study explicitly aimed towards exploring the limitations and possibilities of a MI-CC-tool with an AI with high initiative and definitive impact on the creations, it is of importance to attempt to identify possible solutions to this problem. To develop an artificially intelligent co-creator who performs well, possibly human-like, makes valuable and satisfactory decisions that the human designer appreciate and wants to incorporate, is arguably a challenging task.


%However, within the scope of EDD and this specific study, there are multiple possible improvements to be made.
%Many of the participants, five out of eight, described the behaviour of the AI as random and unpredictable. One of the issues with the behaviour of the AI, as expressed by the participants, is its proneness to override walls, treasures, enemies and bosses with floor-tiles, especially human-placed tiles of these types.
%One possible improvement of this is to rework the choice of tiles to contribute within the AI-component. The AI currently calculates the most common tiles in the positions of the contribution area, and contributes with the tiles of highest occurrence. As previously discussed, this contradicts how the human designers view the tiles and their importance in the room, as humans seem to value bosses, enemies or treasures higher than floors or walls. By adjusting this part of the AI to not equate all tiles as being of the same importance in the design, but rather favoring to change floors over bosses for example, the AI could possibly behave more human-like.
%Additionally, the AI could favor to place tiles on top of untouched positions before overriding the human-placed tiles, to further support the human's creation before suggesting changes.


% Furthermore, some participants expressed that the AI placed floor tiles on top of wall tiles, seemingly to break apart longer walls and open up sub-rooms. This is possibly a result of the dimensions linearity and meso-patterns in the MAP-elites. The resulting elites of the generated rooms with the highest linearity will have the highest amount of traceable paths. Many participants seemed to want create rooms of low linearity, with long walls, sub-rooms, and paths which required the player to encounter enemies. One other notable possible preference among the participants was a seeming attraction to symmetry (See Figure 6, 7 and 8). Human attraction to symmetry is very common, and the fact that many designers seem to aim towards creating symmetric rooms is not unexpected. The dimension representing symmetry in the MAP-elites is valued equally to the other dimensions, which contradicts the suggested human favoring of this particular attribute.

% By adjusting the importance or influence of the different dimensions, the AI will contribute according to these preferences. For example, by valuing symmetry as ten times as important as all other dimensions, the AI would very likely only create symmetric rooms, however will likely not create balanced rooms as the leniency-dimension would not be considered as significant. To adjust the importance of specific dimensions in an effort to improve the AI, one would have to experiment and evaluate how to adjust these variables to create the most well-performing and human-like agent.
% Liapis et. al. performed research related to this idea, by exploring the adaptation of visual aesthetics depending on the user, with the help of AI~\cite{p13adapting-models-visual-aesth}. In their study, the explored how 2D spaceships could be generated by ML-PCG according to the user's tastes and preferences, and how humans valued the different aesthetic attributes. The study concluded that this could be a valuable strategy to create visually satisfying content through AI, for varying types of human designers~\cite{p13adapting-models-visual-aesth}.


% Another example of how the AI could be improved to make more human-like and predictable decisions is by reworking the AI to incorporate designer modelling. By identifying possible design goals or design styles of the human co-creator, the AI can adjust their decisions and behaviour to offer different kinds, or levels of, support depending on the human designer's behaviour. As the study performed by Alvarez et. al. discussed, this area requires more research, however shows great promise as a way to develop well performing AI in MI-CC tool~\cite{p13designer-modelling}. As of now, Alvarez et. al. are currently working on implementing and evaluating designer modelling in EDD. When it is implemented, it will likely be a relatively easy incorporated improvement to the AI of this version of EDD.
















\subsection{Conclusion}

This study explored the limitations and possibilities of an MI-CC-tool with an AI with a varied agency. We aimed at doing an initial exploratory study with static parameters, resulting in baselines to analyze what can be done and how designers experienced the system. This, in turn, opens up and continues the discussion towards AI collaborating as a colleague and enabling alternative ways to foster creativity (e.g., constraining the design space such as in~\cite{p13bhaumik_lode_2021}). Our study showed that AI gaining control over the design results in frustration and feeling constrained. Constraints are not bad per se, as they can be a way to foster creativity~\cite{p13boden_creative_2004,acar_creativity_2019}, but they need to be placed in a way that the human designer might feel inspired, motivated, or supported to continue the design. Human designers had to adapt towards those imposed goals instead of the other way around, which creates an unwanted dynamic when human designers perceive the AI's behavior as erratic, random, and without clear objectives.

 %One limitation of our study is that there are Our study is limited by static changes across versions with no adaptable parameter based on what the designer creates that could have influenced this unwanted dynamic.

%We have aimed at doing an initial explorative study with static changes and no adaptable parameters. With these parameters and naïve baselines, the extent of what can be done and how designers experienced the system could be initially explored and discussed. In turn, this opens up and continues the discussion towards AI collaborating as a colleague and enabling alternative ways to foster creativity (e.g., constraining the design space).

% One of the main challenges within MI-CC is to develop an AI co-creator which performs well, and makes valuable and satisfactory decisions that the human designer appreciate and wants to incorporate. This study explicitly aimed towards exploring the limitations and possibilities of an MI-CC-tool with an AI with varied initiative and definitive impact on the creations, which showed several challenges such as lose of control and perceived behavior by the AI. Our study showed that AI gaining control over the design results in frustration and feeling constrained. Constrains are not fundamentally bad as they can be a way to foster creativity~\cite{p13boden_creative_2004,acar_creativity_2019}, but constraints need to be placed in a way that the human designer might feel inspired, motivated, or supported to continue the design. As a result of the AI gaining more initiative, human designers had to adapt towards those imposed goals instead of the other way around, which creates an unwanted dynamic when human designers perceive the AI's behavior as erratic, random, and without clear objectives.

% Moreover, the study identified possible issues that may arise in MI-CC design tools. As the advantages of MI-CC and the interest from creators to use MI-CC systems has been further confirmed in this study, the importance of further research to improve these systems is evident. This study focused on adjusting the control to explore how the human designer reacted to another type of AI, one that has a more equal role to the human designer. 

%Improving an equally influential AI co-creator (i.e., collaborator) is important in MI-CC.

Many of the results pointed to a general preference for an AI with a more supportive role in collaborative tools. One approach could be to have a hybrid model between what is presented in this paper and other typical MI-CC systems that focus more on suggesting final designs. The AI could take parameters from the human designer, such as an area in a room, amount of tiles, or an attribute that the human designer would like to increase in the room, but still maintain their design, effectively constraining the AI to find creative ways to achieve its goals. In EDD, designers can lock tiles to not be changed by the AI, which is something to be experimented with. Although this would give the human designer a slightly higher degree of influence on the end product compared to the AI, the constraints of how many tiles can be locked, or possibly what types of tiles can be locked, can be experimented with to adjust the relationship between AI and human designer. Currently, the search is steered, to some extent, by the designers' design, but in future work, we could bias the search even more towards interesting areas based on the creation process and the trajectories they are taking in behavior dimension space. 

Additionally, using designer models is a feasible approach. By predicting design goals, adapting to phases of the design process, or identifying certain design styles and adapting to the human designer, a responsive and adaptive, intelligent, and human-like artificial co-creator could be developed. This could allow for an AI that adapts to the human designer and performs well enough that the frustrations and feelings of constraints are minimal or perceived as less prevalent as the designs turn out more similar to what the human desired.

% For example, the human designer could place down tiles freely, and then order the AI to increase the symmetry in the room, and the AI would then edit tiles to reach a certain level of symmetry. This would likely be easily implemented for all the current dimension in the MAP-elites algorithm, already present in EDD.

% \subsubsection{Improving an Equally Influential AI Co-Creator}


% The results in the study show that an AI co-creator, that has equal control as the human designer or more, is easily perceived as frustrating and constraining. Because the MI-CC shows great promise for efficiently creating game content, and furthering the human's creativity, it is still important to try to develop an AI that can co-create with the human in a more satisfactory way. One example of how this can be achieved is to give the human designer the option to lock a limited amount of tiles, disabling them from being overwritten by the AI. Although this would give the human designer a slightly higher degree of influence on the end product compared to the AI, the constraints of how many tiles can be locked, or possibly what types of tiles can be locked, can be experimented with to adjust the relationship between AI and human designer. Alternatively, this can be used to evaluate the overlap of roles in the AI as an assistant and an equal colleague, and how the human reacts to the differences in the resulting relationships. 


% An additional example of how an AI co-creator of high quality can be created is by incorporating designer modelling into the AI. By predicting design goals, adapting to phases of the design process, or identifying certain design styles and adapting to the human designer, a responsive and adaptive, intelligent and human-like artificial co-creator could be developed. This could allow for an AI that adapts to the human designer, and performs well enough that the frustrations and feelings of constraints are minimal, or perceived as less prevalent as the designs turn out more similar to what the human desired.


% \textbf{AI as an Assistant in Mixed-Initiative Co-Creative Systems}

    
% Many research projects have already explored the AI collaborator as an assistant to the human designer. This study focused on adjusting the control to explore how the human designer reacted to another type of AI, one that has a more equal role to the human designer. Many of the results pointed to a general preference of an AI that has a more supportive role in collaborative tools.
% One example of what can be explored in to cover this area of further research is implementing the AI co-creator with another purpose, namely to be a creative assistant to the human designer. 
% Within EDD for example, this new version of the tool could work by implementing AI that takes parameters from the human designer, such as an area in a room, amount of tiles, or an attribute that the human designer would like to increase in the room. For example, the human designer could place down tiles freely, and then order the AI to increase the symmetry in the room, and the AI would then edit tiles to reach a certain level of symmetry. This would likely be easily implemented for all the current dimension in the MAP-elites algorithm, already present in EDD.

    
% \textbf{Improving an Equally Influential AI Co-Creator}


% The results in the study show that an AI co-creator, that has equal control as the human designer or more, is easily perceived as frustrating and constraining. Because the MI-CC shows great promise for efficiently creating game content, and furthering the human's creativity, it is still important to try to develop an AI that can co-create with the human in a more satisfactory way. One example of how this can be achieved is to give the human designer the option to lock a limited amount of tiles, disabling them from being overwritten by the AI. Although this would give the human designer a slightly higher degree of influence on the end product compared to the AI, the constraints of how many tiles can be locked, or possibly what types of tiles can be locked, can be experimented with to adjust the relationship between AI and human designer. Alternatively, this can be used to evaluate the overlap of roles in the AI as an assistant and an equal colleague, and how the human reacts to the differences in the resulting relationships. 


% An additional example of how an AI co-creator of high quality can be created is by incorporating designer modelling into the AI. By predicting design goals, adapting to phases of the design process, or identifying certain design styles and adapting to the human designer, a responsive and adaptive, intelligent and human-like artificial co-creator could be developed. This could allow for an AI that adapts to the human designer, and performs well enough that the frustrations and feelings of constraints are minimal, or perceived as less prevalent as the designs turn out more similar to what the human desired.



% it is of importance to attempt to identify possible solutions to these challenges. 

% This section concludes the paper by summarizing the identified answers to the research questions, as well as introducing future work that may improve related studies, based on the results, analysis and discussion in previous sections. 


% \textbf{RQ1:  How does adjusting the control of the design decisions in mixed-initiative systems affect the human user's creativity?}


% The results show that humans feel constrained and frustrated as the AI co-creator gains control over the design process. Mixed-Initiative systems are suitable to provide new ideas for humans in creative tools, and this was further confirmed by the descriptions of the participants experiences with the tool. However, as the AI gain control over the design, the human sometimes looses interest in being creative and lets the AI take over the creative process.


% \textbf{What are the effects on the human designer's design goal during the process?}

    
% Almost all of the human designers felt forced to adapt their design goals as a result of the AI's decisions, but only when the AI had equal to or more control over the design than the human. 

    
% \textbf{What are the effects on the human designer's perception of limitations or frustration?}

    
% Almost all participants felt constrained and limited in their creativity. This contributed to frustration when using the tool, and had negative effects on the perception of the creative process. Some reacted to the constraints with a decreased interest to design thee room ,and let the AI design alone instead. Some adapted their design style to more iterative style, where the design goal was no longer important, but focus was steered towards creating something satisfying only during the current turn instead.

    



% \textbf{RQ2: How can the three degrees of initiative be explored to asses the support of the AI in terms of the human's creative process?}


% The degrees of control used in this study showed that in this tool, humans generally prefer an AI with low initiative, over one with equal to or higher than the human co-creator. As the AI gained more influence on the design process, the human designers felt increasingly stumped in their creativity. 
% For a human designer to feel satisfied with an AI co-creator of higher initiative, one hypothesis may be that the AI has to have an advanced intelligent behaviour, which is adaptive to the human designer. The results from this study supports the speculation that it is more likely that humans are more willing to collaborate with AI that does exceed the human in terms of control.


% \textbf{How do they support lateral thinking and the introduction of new ideas?}

    
% The low and medium level of initiative show potential as supporters of lateral thinking. The Highest level of control does introduce new ideas, however it also proposes limitations and constraints to the human designer that often makes the human feel creatively constrained.

    
    
% \textbf{How does the designer respond to the AI's differing degrees of initiative?}

    
% The results suggest a willingness to adapt to the different levels of control of the AI. The responses to the AI with a low degree of initiative where generally positive, and most participants preferred this version as it provided interesting suggestions and ideas without having a definitive influence on the design. 
% Responses to the AI with the medium degree of initiative where generally frustrating, specifically with the behaviour of the AI and not the concept of an AI that contributes with editable content. The AI with the highest degree of initiative made the designers feel constrained and limited in their design.






% \subsubsection{Future Work}

% % \emph{Suggestions of Improvements (Q11)}


% % All participants suggested removing the constrain of turns and amount of tiles per turn. Three participants suggested that the AI could be used more as an assistant to the human designer, by giving the AI parameters such as area or type of tiles to place. Three participants suggested using a more intelligent or human-like AI-agent. Examples of how the AI could be improved included valuing the types of tiles similarly to how the human designer does, learning from what the human contributes with and adapting its behaviour, and possibly attempting to predict the design goal that the human designer has. One participant answered that they would like to have the AI suggest complete generated rooms, that the human designer can then polish or edit freely. One participant had a related suggestion, which is a final step of the design process when the room is complete, where the human can edit an additional set amount of tiles, giving the human a chance to overwrite some AI-placed tiles.

% % \vspace{3mm}

% The study has identified possible issues that may arise in MI-CC game level design tools. As the advantages of MI-CC and the interest from creators to use MI-CC systems has been further confirmed in this study, the importance of further research to improve these systems is evident. 


% \textbf{AI as an Assistant in Mixed-Initiative Co-Creative Systems}

    
% Many research projects have already explored the AI collaborator as an assistant to the human designer. This study focused on adjusting the control to explore how the human designer reacted to another type of AI, one that has a more equal role to the human designer. Many of the results pointed to a general preference of an AI that has a more supportive role in collaborative tools.
% One example of what can be explored in to cover this area of further research is implementing the AI co-creator with another purpose, namely to be a creative assistant to the human designer. 
% Within EDD for example, this new version of the tool could work by implementing AI that takes parameters from the human designer, such as an area in a room, amount of tiles, or an attribute that the human designer would like to increase in the room. For example, the human designer could place down tiles freely, and then order the AI to increase the symmetry in the room, and the AI would then edit tiles to reach a certain level of symmetry. This would likely be easily implemented for all the current dimension in the MAP-elites algorithm, already present in EDD.

    
% \textbf{Improving an Equally Influential AI Co-Creator}


% The results in the study show that an AI co-creator, that has equal control as the human designer or more, is easily perceived as frustrating and constraining. Because the MI-CC shows great promise for efficiently creating game content, and furthering the human's creativity, it is still important to try to develop an AI that can co-create with the human in a more satisfactory way. One example of how this can be achieved is to give the human designer the option to lock a limited amount of tiles, disabling them from being overwritten by the AI. Although this would give the human designer a slightly higher degree of influence on the end product compared to the AI, the constraints of how many tiles can be locked, or possibly what types of tiles can be locked, can be experimented with to adjust the relationship between AI and human designer. Alternatively, this can be used to evaluate the overlap of roles in the AI as an assistant and an equal colleague, and how the human reacts to the differences in the resulting relationships. 


% An additional example of how an AI co-creator of high quality can be created is by incorporating designer modelling into the AI. By predicting design goals, adapting to phases of the design process, or identifying certain design styles and adapting to the human designer, a responsive and adaptive, intelligent and human-like artificial co-creator could be developed. This could allow for an AI that adapts to the human designer, and performs well enough that the frustrations and feelings of constraints are minimal, or perceived as less prevalent as the designs turn out more similar to what the human desired.








\appendix
\subsection{Appendix A}
\subsubsection{Interview procedure and questions}

\emph{Setting up a user study session}


\begin{enumerate}
     \item The conductor starts a meeting in Zoom.
     \item The conductor explains the steps to the participant.
     \item When consent of recording the session is acquired by the conductor, the conductor starts recording.
     \item The conductor starts the tool and shares the screen.
     \item The conductor allows the participant to control the conductors machine via Zoom remote control.
     \item The conductor instructs the participant to perform the test.
 \end{enumerate}

\paragraph{Instructions}

The task is to design at least two rooms in a dungeon world, with each variant of the AI.

\begin{itemize}
    \item Step 1: Choose the LOW level of the AI and click “Create World”.

\item Step 2: You are now in the World Editing View. Edit as you please. To enter the room editing view, double click the room you wish to edit.

\item Step 3: You are now in the Room Editing View. Use the brushes on the left to edit the room. Click “End Turn” to end your turn, and let the AI contribute. 

\begin{itemize}
    \item AI-placed tiles are tinted purple. AI-suggested tiles are tinted green.
    \item If you are in the LOW variation of AI, click the green suggestions you’d like to place. Click continue when you want to have your turn again.
    \item To go back to the World Editing View, click “Go To World View”.
\end{itemize}

\item Step 4: When you feel satisfied with your creation, tell me so. Restart the program. Start over at step 1 for the next AI version until all three are used once.

\end{itemize}


\paragraph{Interview Questions}

\begin{itemize}
    \item Q1: Which of the three versions of AI did you prefer?
Why?

\item Q2: Which of the three versions of AI did you find least appealing? 
Why?


\item Q3:  How would you describe the creative experience?


\item Q4: What is your perception of the AI’s behaviour? 


\item Q5: Did you feel your creativity was constrained when using any of the three AIs?


\item Q6: Did you adapt to the different AI versions? 
In what ways?


\item Q7: Did you perceive that the AI adapted to you?
In what ways?


\item Q8: How would you describe the relationship between designer and the AI?


\item Q9: How did the AI’s decisions affect your creative process?


\item Q10: How did the different versions affect your design goals?


\item Q11: What do you think is missing or needs to be improved for an AI as the one of the HIGH-version (with high initiative) to be used in collaborative tools?
\end{itemize}




\bibliographystylepthirteenth{ieeetr}
\bibliographypthirteenth{included-papers-tex/paper-13/zotero-references.bib}