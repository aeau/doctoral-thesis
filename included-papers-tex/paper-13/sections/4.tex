% Please add the following required packages to your document preamble:
% \usepackage{graphicx}
\begin{table}[]
\centering
\begin{tabular}{|l|lll|}
\hline
        & AIv1       & AIv2        & AIv3        \\ \hline
Leniency        & 0.56±0.07  & 0.62±0.09   & 0.57±0.08   \\
Linearity        & 0.91±0.02  & 0.92±0.02   & 0.91±0.01   \\
MesoPat       & 0.15±0.05  & 0.13±0.07   & 0.12±0.05   \\
SpatialPat    & 0.35±0.1   & 0.41±0.11   & 0.34±0.09   \\
Symmetry   & 0.43±0.11  & 0.35±0.18   & 0.35±0.12   \\
$W_{dens}$ & 0.27±0.09  & 0.26±0.08   & 0.21±0.05   \\
$W_{spar}$ & 0.21±0.05  & 0.19±0.03   & 0.15±0.01   \\
$E_{dens}$ & 0.24±0.07  & 0.27±0.06   & 0.3±0.06    \\
$E_{spar}$ & 0.22±0.05  & 0.32±0.05   & 0.35±0.06   \\
$T_{dens}$ & 0.37±0.13  & 0.28±0.09   & 0.34±0.07   \\
$T_{spar}$ & 0.36±0.11  & 0.3±0.1     & 0.37±0.05   \\ \hline
Steps      & 39.25±6.38 & 84.31±14.85 & 76.75±17.02 \\ \hline
\end{tabular}
\caption{Summary of the created rooms filtered by the AI version used. All values are the average of all the created rooms using the specific AI version. The first five values relates to the MAP-Elites dimensions, then the fitness of the rooms, the density and sparsity values for wall (W), enemies (E), and treasures (T), and finally the avg. steps taken to design a room.}
\label{tab:AIavgValues}
\end{table}

\subsection{Experiment Setup}

We conducted a user study to explore the user experience of using different levels of AI agency, the different design characteristics, and the relationship between the human designer and the AI. We collected both quantitative data on the AI's impact on the co-designed end product and qualitative data through think-a-loud and semi-structured interviews regarding the users' experience when interacting with the AI. The interview structure is inspired by the pyramid model, meaning the interviews will begin with specific questions, and gradually have more open questions, which naturally allows for a discussion towards the end. This model is chosen to support the variation of subjects the interview is desired to cover, as well as support natural transitions between the questions and their openness. The questions and user study procedure can be found in Appendix A. 

%The interviews are semi-structured, meaning it includes both closed and more open questions, and depending on the discussion and answers, some questions might be omitted.

% We collected quantitative data regarding what impact the AI had on the co-designed end product, and how the human designer interacted with the AI's contributions. Likewise, we collected qualitative data through recorded think-aloud observations and semi-structured interviews regarding the users' experience, and possibly catch certain remarks of frustration or appreciation of their digital colleague that can be valuable for the discussion of the relationship between the co-creators. 


%We collected the following data:

%\begin{itemize}
%    \item \textbf{Audio Recordings:} 
%\end{itemize}

Eight participants tested our tool with game design and level design experience. One participant was a professional game designer with eight years of professional experience (first participant), and seven participants were third-year Game Development students. They all had an individual digital session, where we shared our screen, and they took remote control to conduct the study. Participants accepted to participate, signed consent forms, and then received a short introduction describing the experiment and its steps. The participants were then asked to design two contiguous rooms in a dungeon, repeating this process for each of the AI variants and expressing their design decisions verbally whenever they felt like it. After using the tool, the participants were interviewed, focusing on and covering an overarching understanding of the user experience, particularly in terms of creativity and interaction with the AI.

% the relationship that occurs between the AI and human designer (See Appendix B). 

For all the sessions, human designers could place up to 12 tiles, and the AI could place as many tiles as the human placed. The AI could contribute only in a rectangular area surrounding the tiles the human designer recently contributed with, including a margin of 1 tile. This choice is made to support a responsive and collaborative behavior of the AI that builds on the human designer's contribution.


% The locations available for the AI to contribute in for each turn are limited to a rectangular area surrounding the tiles the human designer recently contributed with, including a margin of 1 tile. This choice is made to support a responsive and collaborative behavior of the AI that builds on the human designer's contribution.

% The margin for the contribution area is set to 1, as it was found during experimentation that any margin bigger than this is likely perceived as the AI contributing to other areas than the ones the human is focused on, because of the default size of the room being relatively small.

%  as this enables the designer to contribute with an adequate amount of tiles during their turn and create representable structures

% The rooms produced during the user study are displayed in Figure 6, 7 and 8. Rooms with red borders are infeasible, meaning there are unreachable tiles. The UI displays a warning when this happens, and the AI can repair this during its turn, however the resulting rooms that are infeasible are a result of the human designer creating unreachable areas, and then immediately selecting to go to the World Editing view, before pressing "End Turn". 
% Each participant created two rooms for each version of the AI. Participant 1 created Room 1 and Room 2 for all version, Participant 2 created Room 3 and Room 4 for all version, etc. All of the participants had the option to adjust the sizes of the rooms in the World Editing view before entering the Room Editing view, however none of them did, and therefore all of the resulting rooms are of the default size. The designer also has the option to change the location of the hero and the doors. The location of the hero was only moved twice in all of the session, and the doors where never moved. 






%, the participant will take part in an interview. The questions, and their order, are planned out and designed to cover an overarching understanding of the user experience, in particular in terms of creativity, and the relationship that occurs between the AI and human designer (See Appendix B). 




%The participants were asked to repeat this for each of the AI-initiatives. was asked to repeatEach pair of room 

% then asked to complete three tasks, each regarding

%The users were then asked to complete three tasks that covered the tool's functionality and the AI-initiatives, respectively for each task. The tasks were 


%and different approaches to creating quests. The tasks were to 1) manually create a quest, 2) automatically create a quest, and 3) create a quest through mixed-initiative. They were also asked to create a dungeon that suited their preferences and objectives before creating quests. The questionnaire consisted of 17 closed-ended questions, and the rest were open-ended. The interview began with a questionnaire with six questions about the users' background and experience within game development and finish with questions about their experience and opinions on the tool. Both the questionnaire and interview followed guidelines described by



%The participants used the tool


