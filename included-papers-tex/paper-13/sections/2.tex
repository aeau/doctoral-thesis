\subsection{Related Work}

% \subsubsection{Procedurally Generated Game Content}
% Procedural Content Generation (PCG) is a method of creating content through intelligent algorithms ~\cite{p13PCG}. By using computers to create content from manually created assets and rules, large amounts of game content can be produced efficiently, while also promoting replayability as it can produce endless variations of content~\cite{p13PCG}. As a result of the benefits to this method, there are many domains for PCG. One of the more well-established uses of PCG in games is terrain-generation, for example in the game Minecraft~\cite{p13minecraft}, or the Diablo-series~\cite{p13diablo}. Other uses of PCG include quest-generation, such as in the game Legends of Aethereus~\cite{p13Legends-of-Aethereus}, and procedurally generated narrative and story, such as in the game Dwarf Fortress~\cite{p13dwarf-fortress}.

% Autonomy \cite{p13negrete-yankelevich_co-creativity_2020}

% Reflection \cite{p13kreminski_reflective_2021}

% Interactiveness~\cite{p13deterding_mixed-initiative_2017}

% \item Also important to bring that paper about mixed-initiative tools from FDG 2021, and~\cite{p13partlan_design-driven_2021}
%     \item the paper of the three pillars for mixed-initiative.~\cite{p13lai_towards_2020}
    
    




% was described by Novick and Sulton as a multi-factor model that combines: choosing the task, choosing the agent in control and how the interaction is established, and choosing the expected outcome from the collaboration~\cite{p13novick_what_1997}. 

% Mixed-initiative Co-Creativity focuses on the joint effort between humans and AI with proactive initiative to tackle creative tasks not only with the assisstance of AI 

% Mixed-Initiative Co-Creativity poses that humans can not only be assisted through the support of AI, but possibly also advances the humans creativity~\cite{p13yannakakis_mixed-initiative_2014}.

% The proposed question is not new and has been approached by different disciplines, under the~\acrfull{mi} paradigm.~\acrshort{mi} refers to the collaboration between \emph{human} and \emph{computer} where both have some proactive initiative to solve some task.~\acrshort{mi} can be seen as a multi-agent collaboration scenario, where the interaction should be flexible, allowing for a continuous negotiation of initiative and leverage on each other's strengths to solve a task~\cite{p13allen_mixed-initiative_1999}. \emph{Initiative} was described by Novick and Sulton as a multi-factor model that combines: choosing the task, choosing the agent in control and how the interaction is established, and choosing the expected outcome from the collaboration~\cite{p13novick_what_1997}. 

% Moreover, Horvitz discussed such a question in terms of Intelligent User Interfaces~\cite{p13birnbaum_compelling_1997}, describing mixed-initiative systems and interfaces as a more natural collaboration in a user interface that emerges from intertwining human control and manipulation, and automation~\cite{p13horvitz_uncertainty_1999}. Horvitz presented several principles of mixed-initiative interaction and its challenges, many of which still exist~\cite{p13horvitz_principles_1999}, mainly describing this interaction as conversation systems between AI and humans~\cite{p13horvitz_computational_1999}. Moreover, Yannakakis et al. introduced the~\acrfull{micc} paradigm for the co-creation of creative content, where both AI and humans alternate in the initiative to co-design and solve tasks~\cite{p13yannakakis_mixed-initiative_2014}. Their work describes key findings and discussions for how MI-CC does not only help human designers solve tasks, but also fosters their creativity through an interactive feedback loop and lateral thinking~\cite{p13liapis_can_2016,liapis_computational_2014,alvarez_fostering_2018}.

% \subsubsection{Mixed-Initiative Co-Creativity}



MI-CC focuses on tackling tasks between humans and AI with proactive initiative, where AI does not only assist humans but could also collaborate with them, leveraging on both their strengths~\cite{p13yannakakis_mixed-initiative_2014,allen_mixed-initiative_1999}. \emph{Initiative} is a multi-factor model combining: choosing the task, the agent in control and how the interaction is established, and the expected outcome~\cite{p13novick_what_1997}. In this work, both humans and AI have the same task, and the interaction is established as turn-based, each taking discrete control. The outcome is expected to vary as AI agency increases since larger constraints are added for the human that might need to adapt towards those. 

Some MI-CC systems enable different collaborative approaches, which are considered in this paper. Tanagra~\cite{p13smith_tanagra_2011} is a design tool for platform levels where the system takes as input and constraints the current user's design and creates content fulfilling gaps around it. Morai Maker~\cite{p13guzdial_co-creative_2018} is an MI-CC tool where the human designer and AI take turns to design Super Mario Bros. levels. The AI adds content in its turn, which can be maintained or erased by the human designer, which the AI learns to adapt to through reinforcement learning. Furthermore, Lode Encoder~\cite{p13bhaumik_lode_2021} explores a creative collaboration where the human is constrained by only being able to use AI-generated content, which they need to choose to compose their design. This shows an unusual collaboration that users expressed as a playful, game-like creative process.

\begin{table*}[ht]
% \centering
\caption{General consensus on EDD's features} \label{p1tab:consensus}
\resizebox{0.8\textwidth}{!}{
\begin{tabularx}{\textwidth}{|p{0.2\textwidth}|p{0.99\textwidth}|}
\cline{1-2}

Description & Participants’ Consensus \\\cline{1-2}
World Grid  of the dungeon                   & Its purpose of establishing an illusion of a fully realized dungeon is somewhat achieved. However, limitations exist with how it defines feasibility, a dungeon’s starting point, and the entrances, which disrupts the designers’ decisions.                                                                                                                                                                                   \\\cline{1-2}
World View                                  & The world view’s usefulness for the most part could not be established, other than for the purpose of going to the suggestions view (which was already seldom during the user study) and having a closer look at the entire dungeon without any distractions. Some participants preferred features to be already in the room view’s minimap, and some wanted to see more specific functionalities within the world view itself. \\\cline{1-2}

Enabling and  \newline disabling rooms                & As the user study restricted participants to create 3x3 dungeons, this feature for the most part has been neglected. This is also in part because of its accessibility only being in the world view, which proved to be an inefficient view in general. However, its use for bigger dungeon sizes later on was appreciated, especially for more intricate design purposes.                                                      \\\cline{1-2}
Suggestions View                            & Similarly to enabling and disabling rooms, it was quite difficult to encourage the use of this functionality due to the world view’s inefficient usability. However, this could also be due to the dungeon’s small size, as some participants expressed high interest in using more suggestions with larger dungeon sizes.                                                                                                      \\\cline{1-2}
Minimap  \newline  navigation                      & The minimap proved to be a strong tool not only for navigation purposes, but also for supporting design decisions and choices. The directional buttons were rarely used, but their room previews were helpful in emphasizing the current room’s connection to adjacent rooms without looking at the minimap. On the other hand, this lowered the usability of the world view.                                                   \\\cline{1-2}
Parameters                                       & The parameters were, in general, lacking. They served to be important in decision-making when choosing a suggested map in room view, but there were still doubts on their accuracy and sufficiency when providing information about the generated suggestions.                                                                                                                                                                       \\\cline{1-2}
Generated maps for  \newline  suggestions in room view & Suggestions in the room view proved to be very helpful in supporting the whole design process as they primarily acted as inspirations for the users. The most prominent comment among the users is the preference of having more control on how suggestions should be generated depending on different types of parameters.                                                                                                     \\\cline{1-2}
Design \newline  patterns& The patterns’ visualization was, in general, lacking and not self-explanatory. Some participants have expressed interest in using patterns as a parameter in the generation of suggestions.                                                                                                                                                                                                                                     \\\cline{1-2}
Dark theme                                  & EDD’s dark theme for the user interface received a positive response as it makes working with the program easier.
	\\ \cline{1-2}
\end{tabularx}
}
\end{table*}

% \begin{table}
% \begin{center}
% {\caption{Best performing setups based on their internal validation and visualization of clustered data points.}\label{table:setups}}
% \resizebox{\textwidth}{!}{
% \begin{tabular}{ccccccc}
% \hline
% \rule{0pt}{12pt}
% Algorithm&Data&K&$\Diamond$&$\Box$&$\bigtriangleup$ 
% \\ 
% \hline
% \\[-6pt]
% K-Means & Tiles-PCA & 9 & 0.43 & 0.73 & 9438.233 \\ 
% K-Means & Tiles-PCA & 12 & 0.41 & 0.77 & 9436.928 \\
% K-Means & Dimensions-PCA & 12 & 0.43 & 0.73 & 7738.343 \\
% Agglomerative single & Combined-PCA & 6 & 0.51 & 0.43  & 38.833 \\ 
% Agglomerative avg. & Dimensions-PCA & 6 & 0.44 & 0.67 & 3463.567 \\ 
% \hline
% \\[-6pt]
% \multicolumn{6}{l}{$\Diamond$ Silhouette Score\ \
% $\Box$ Davies Bouldin Index\ \
% $\bigtriangleup$ Calinski-Harabasz Index}
% \end{tabular}
% }\end{center}
% \end{table}

This paper uses EDD as the tool to explore AI agency and control. EDD is an MI-CC system where designers can create interconnected rooms composing a dungeon~\cite{p13alvarez_empowering_2019}. As designers create their content, the AI constantly suggests content adapted to the designer's design using the Interactive Constrained MAP-Elites (IC MAP-Elites). We make extensive use of IC MAP-Elites to generate rooms that are adapted to the target room. In~\cite{p13alvarez_interactive_2020}, the authors show that IC MAP-Elites can generate high-performing and diverse rooms from different targets and using different dimension combinations. Its adaptiveness and stability, two necessary properties, were assessed with continuously edited rooms in~\cite{p13alvarez_assessing_2021}, showing that the designer has a positive effect and can steer the algorithm with their design.

%In EDD, designers can interact in different ways with the algorithm by locking tiles to preserve their changes and changing feature dimensions and their granularity to narrow the search in interesting generative spaces~\cite{p13alvarez_assessing_2018,alvarez_interactive_2020}.

%This paper makes use of EDD and the IC MAP-Elites to generate rooms that are adapted to the target room. In~\cite{p13alvarez_interactive_2020}, the authors show that IC MAP-Elites can generate high-performing and diverse rooms from different targets and using different dimension combinations. Its adaptiveness and stability, two necessary properties, were assessed with continuously edited rooms in~\cite{p13alvarez_assessing_2021}, showing that the designer has a positive effect and can steer the algorithm with their design. Given that we do not know what dimensions the designer might be interested on, we use all seven dimensions in the search and weight each of them equally. 





% which designers can interact with by changing dimensions,
% In EDD, the designer can create interconnected rooms composing a 

% InEDD is an MI-CC system for building adventure game dungeons. A human designer creates rooms interconnected in a graph structure that composes a dungeon. The computer constantly inputs the human designs to procedurally generate sets of suggested rooms on-the-fly, by means of an Interactive Constrained MAP-Elites (IC MAP-Elites) evolutionary algorithm driven by level design patterns. The designer can incorporate and further edit those suggestions at will, and also interacts with the evolutionary system by locking tiles and fine-tuning some hyper-parameters of the IC MAP-Elites algorithm.

% A detailed description of all EDD's features, including the use of design patterns, its implementation of the IC MAP-Elites algorithm, and its designer modeling module can be found in~\cite{p13eddy1, MAP-Elites-eddy, eddy2}.

% Mixed-Initiative Co-Creativity poses that human designers can not only be assisted through the support of AI, but possibly also advances the humans creativity~\cite{p13yannakakis_mixed-initiative_2014}. Both computer and human contribute to the creative tasks with their own perks, as they fluently inspire each other and inflict on each other's creative process. 

% The computer can present the human with creative choices made with the element of randomness, which presents the human with ideas that the human was possibly unlikely to make, and therefore provide unexpected mutations to the game content~\cite{p13MI-CC}. As described by Yannakakis et al., some of the main aspects of how we define human creativity will benefit greatly from what a computer can provide in a digital creative environment~\cite{p13MI-CC}. Yannakakis et. al. describes how one major part of creativity, lateral thinking, benefits greatly from computer-made randomness as follows:

% \textit{"The random stimulus principle of lateral thinking relies on the introduction of a foreign conceptual element with the purpose of disrupting preconceived notions and habitual patterns of thought, by forcing the user to integrate and/or exploit the foreign element in the creation of an idea or the production of a solution. Randomness within lateral thinking is the main guarantor of foreignness and hence of stimulation of creativity."}~\cite{p13MI-CC}

% Mixed-Initiative systems is a powerful tool to enable efficient and innovative content generation, but requires further research to explore how to optimize its full potential. 

% Notable research projects covering this subject include Tanagra~\cite{p13smith_tanagra_2011}, which is a MI-CC game design tool to create 2-dimensional platform style game levels. Tanagra's level generator can either autonomously create levels in absence of human input with a PCG-algorithm, or respond to a human designer's input, as well as display helpful information about the level such as patterns, and ensuring that the level is playable. When a human designer edits a generated level, the AI will keep those edits when suggesting the following generated content.

% Morai Maker is another example of a research project within MI-CC. Morai Maker is a game level design tool for the game Super Mario Bros., where human and AI create in a turn-based manner~\cite{p13guzdial_co-creative_2018}. The AI in the system uses PCGML to generate content.
% A study that was performed on Morai Maker in 2019 examined how human users reacted to the system having three different generation algorithms, and discussed the resulting creative experiences, as well as the resulting relationship they developed with the computer in their creative processes. The study concluded that, although users experienced varying satisfaction with the additions the AI contributed with, the users demonstrated a willingness to adapt their behaviour to the agent. 

% % This conclusion is of great relevance to this thesis project, as it supports the idea that MI-CC can possibly achieve a more equal collaboration and creativity with a human-AI team. The turn-based procedure of human-AI interaction used in Morai Maker~\cite{p13morai-maker} has also inspired this thesis project to use a similar tactic, to enable both the computer and the human to influence the end product, and striving towards an equal influence in the end result. 


% Lode Encoder is an AI-assisted tool for creating levels using procedural content generation via machine learning for the game Lode Runner~\cite{p13bhaumik_lode_2021}. This tool explores a creative collaboration where the human is constrained by only being able to use AI-generated content. The human user is presented with AI-generated levels, and has the choice to add elements from the generated rooms into their creation, refresh the generation, or use an AI-controlled "wand tool" to attempt to repair broken ladders or platforms to increase the amount of playable levels being created. The tool Lode Encoder started out as a completely autonomous level generator. This version of the tool presented a problem, the generated levels where of unsatisfactory quality, except for if they were overfit on the data. The results from the overfitted data resulted in generated levels of high quality, but would always result in a level almost identical to an existing level in the training data. This lead to the idea of the wand tool, by using the results of an overfitted network to repair the generated levels that where of poor quality. With this idea in mind, the tool was redeveloped as a MI-CC tool. The study shows an unusual creative collaboration to take place, that users expressed as a playful, game-like creative process. 

\subsubsection{AI Roles and Adaptability}

% It is important to highlight the role of the computer in this interaction. 

Lubart discusses four different roles a computer might take to promote creativity; \emph{computer as nanny}: management of creative work; \emph{computer as pen-pal}: communication service between collaborators; \emph{computer as coach}: Using creative enhancement techniques; and \emph{computer as colleague}: partnership between computer and humans~\cite{p13lubart_how_2005}. This is further explored by Guzdial et al. where designers perceived the AI collaborator with more or less value depending on their desired role for the AI, varying between: \emph{friend}, \emph{collaborator}, \emph{student}, or \emph{manager}~\cite{p13guzdial_friend_2019}.

% Paramount is the role of the computer agent in this interaction, as it would help establish the boundaries of the interaction, what is expected, and how creativity could be fostered. Lubart analyzed this interaction and examined the different ways computers could be involved in creative work to promote creativity. In his work, he proposed four roles: \emph{computer as nanny}: management of creative work; \emph{computer as pen-pal}: communication service between collaborators; \emph{computer as coach}: Using creative enhancement techniques; and \emph{computer as colleague}: partnership between computer and humans~\cite{p13lubart_how_2005}. Recently, this was explored by Guzdial et al., where designers perceived the AI collaborator with more or less value depending on their desired role for the AI, varying between: \emph{friend}, \emph{collaborator}, \emph{student}, or \emph{manager}~\cite{p13guzdial_friend_2019}.

%  The relationship that occurs between co-creators in a MI-CC tool is arguably one of the most important element within a system that aims to foster creativity.

Establishing different roles such as colleague and collaborator might require some user model within the system. Designer modeling, as defined by Liapis et al.~\cite{p13liapis_designer_2013}, is a way to classify and predict a designer's style, goals, preferences, and processes. Preference models~\cite{p13alvarez_learning_2020,liapis_adapting_2012} have been built based on designers' choices and used as surrogate models to evaluate further generated content. Similarly, using the designers' creation, the designers' processes and styles could be modeled to inform other systems and adapt the generated content~\cite{p13liapis_designer_2014,alvarez_designer_2022,halina_threshold_2022}.

% modeling the designers processes and styles~\cite{p13liapis_designer_2014,alvarez_designer_2022,halina_threshold_2022} using the designers creations 

% Designer models regarding preferences~\cite{p13alvarez_learning_2020,liapis_adapting_2012} or  have been explore

% Designer Model

% Alvarez et. al. has implemented a novel method of designer modeling within EDD, which successfully identified four frequent and unique designer styles, with related common sub-styles~\cite{p13designer-modelling-edd}. The study showed promise of this method being used to create an AI-colleague within EDD~\cite{p13designer-modelling-edd}. It is possible that an AI that can recognize a designers persona, style, preferences and goals, and adapting its behaviour to support these attributes, can result in a well-performing and interesting co-creator~\cite{p13designer-modelling, designer-modelling-edd}.

% Designer modeling, Lubart, allen, Novick, Guzdial, Liapis (aesthetic models), check the other paper 

% Lubart~\cite{p13lubart_how_2005} describes four roles for a computer system in a collaborative task. Nanny, pen-pal, coach, and colleague. 

% Guzdial~\cite{p13guzdial_friend_2019}

% Can be asymmetric, focus on completely different tasks, and can be dynamic? maybe that we leave for Dynamicity

% Designer Modeling~\cite{p13alvarez_designer_2022,liapis_designer_2013,liapis_designer_2014,halina_threshold_2022}

% Adaptive aesthetic models~\cite{p13liapis_adapting_2012}





% The relationship that occurs between co-creators in a MI-CC tool is arguably one of the most important element within a system that aims to foster creativity. Founding a valuable and co-creative relationship between AI and human is a major challenge, and further developing designs of AI is necessary to work towards perfecting this relationship. Designer modeling, as defined by Liapis et. al.~\cite{p13designer-modelling}, is a way to classify and predict a designer's style, goals, preferences, processes, and their definitions. As proposed by Alvarez et. al.~\cite{p13designer-modelling-edd}, this concept can be used to create an adaptive and collaborative colleague in a MI-CC environment. Alvarez et. al. has implemented a novel method of designer modeling within EDD, which successfully identified four frequent and unique designer styles, with related common sub-styles~\cite{p13designer-modelling-edd}. The study showed promise of this method being used to create an AI-colleague within EDD~\cite{p13designer-modelling-edd}. It is possible that an AI that can recognize a designers persona, style, preferences and goals, and adapting its behaviour to support these attributes, can result in a well-performing and interesting co-creator~\cite{p13designer-modelling, designer-modelling-edd}.

% With the high level of initiative of the AI in a collaborative relationship, similarly to the one in Lode Encoder, human creativity can be challenged and explored in a new setting, to further our understanding of the possibilities of human-AI collaboration.


% \subsubsection{Interactive Constrained MAP-Elites}
% The Multi-dimensional Archive of Phenotypic Elites (MAP-Elites) algorithm is a quality-diversity search algorithm which presents a comprehensive view of high-performing solutions according to specifically defined dimensions of the search space~\cite{p13MAP-Elites}. MAP-Elites produces a variety of high-performing solutions, with different attributes, allowing diverse results with the possibility to easily investigate specific attributes and combinations of attributes. The resulting elites are represented in a map, where each cell contains one solution~\cite{p13MAP-Elites-eddy}. One axis of the map corresponds to one dimension of the search space. When a new generation is started, an offspring is generated based on one or more existing solutions. The offspring is then evaluated according to its feature dimensions. If the new solution is placed in a cell with an existing solution, the one with the highest fitness value occupies the cell. The result is a map of solutions ordered in the best performing solution found in the particular feature dimensions.

% The program that will be the subject of the study, EDD, uses an evolutionary PCG-algorithm to generate content and Interactive Constrained MAP-Elites (IC MAP-Elites) to search for the best performing generated content~\cite{p13MAP-Elites-eddy}. The user of EDD can edit content, and the user's choices are then fed into the algorithm, making the MAP-Elites interactive. 
% The existing MAP-Elites system in EDD will be used in this thesis project, as a foundation for the decision making of the AI to determine what tile to place and where. By doing so, the thesis project can effectively produce an intelligent agent, and with that allow more resources to be put into the theoretical framework and case study.

% \subsubsection{The Evolutionary Dungeon Designer}

% The Evolutionary Dungeon Designer EDD is an MI-CC system for building adventure game dungeons. A human designer creates rooms interconnected in a graph structure that composes a dungeon. The computer constantly inputs the human designs to procedurally generate sets of suggested rooms on-the-fly, by means of an Interactive Constrained MAP-Elites (IC MAP-Elites) evolutionary algorithm driven by level design patterns. The designer can incorporate and further edit those suggestions at will, and also interacts with the evolutionary system by locking tiles and fine-tuning some hyper-parameters of the IC MAP-Elites algorithm.

% A detailed description of all EDD's features, including the use of design patterns, its implementation of the IC MAP-Elites algorithm, and its designer modeling module can be found in~\cite{p13eddy1, MAP-Elites-eddy, eddy2}.

% \begin{table}[]
% \begin{tabular}{p{0.25\linewidth}| p{0.65\linewidth}}
% Feature          & Definitions                                                                                                                  \\ \hline
% Similarity (Sim)       & Aesthetic (tile-by-tile) similarity between a generated level and the designer's design.                         \\ \hline
% Inner Similarity (IS) & Different tiles' sparsity and density similarity between a generated level and the designer's design.     \\ \hline
% Symmetry         & Room's aesthetic symmetry.                                                                                 \\ \hline
% Leniency (Len)        & Challenge based on enemies and treasures.\\ \hline
% Linearity (Lin)       & Paths that exist connecting entry points in a level.  \\ \hline
% \#Meso-Patterns (Meso) & Amount of meso-patterns that exist within a level. This is a discrete dimension rather than continuous.\\ \hline
% \#Spatial-Patterns (Spa) & Amount of spatial-patterns that exist within a level. \\ \hline
% \end{tabular}
% \caption{Level design based dimensions used in EDD with IC MAP-Elites.}
% \label{tab:dimensions}
% \end{table}

% The Evolutionary Dungeon Designer (EDD) ~\cite{p13eddy1, eddy2} is a MI-CC tool to co-create 2D dungeons in the style of the seminal game \emph{The Legend of Zelda}~\cite{p13tloz}. Designers manually edit the dungeon structure as well as the interior of every room in it. EDD constantly offers tailored room suggestions on the fly that designers may decide to incorporate to their designs at any moment. 

% In EDD, the system analyzes the level-design patterns (i.e., micro- and meso-patterns) that exist in each room, calculating and utilizing their quality to assess rooms. Micro-patterns are the building blocks in a design, which in EDD are categorized as \textit{spatial micro-patterns}: chamber, corridor, intersections, connector; and \textit{inventorial micro-patterns}: enemy, treasure, and door. On the other hand, Meso-patterns are defined as the relation between micro-patterns or other meso-patterns, and by the composition between inventorial micro-patterns and spatial micro-patterns. Meso-patterns are used to identify structures in the room that join together a set of micro-patterns and can be: \textit{ambush}, \textit{guard chamber}, \textit{treasure chamber}, and \textit{guarded treasure}. All patterns are shown in figure~\ref{fig:basecomponents}, and further information and discussion can be found in~\cite{p13Baldwin2017,Alvarez2018}.

% The Evolutionary Dungeon Designer (EDD) is an MI-CC game design tool for designing  dungeons~\cite{p13eddy1, eddy2}. The user is first met by the dungeon-editing view, where the dungeon is represented by a graph-structure, displaying rooms and their connections. In this view, the user can add rooms of any size between 3 x 3 and 20 x 20 tiles, and connectors between rooms in the form of door-tiles. The smallest valid dungeon consists of two rooms with one two-way connection between them. The designer marks one of the rooms as the initial room to enable the program to calculate the dungeons feasibility. If a room or door is unreachable from the initial room, they will be highlighted in red, to alert the designer. The user can also select the option "Start with our suggestions" which will present six procedurally generated suggestions the user can start with, and edit to their preferences.
% % \begin{figure}[!h]
% %     \includegraphics[width=\columnwidth]{images/Artifact/world_view new.png}
% %     \caption{The world editing view.}
% % \end{figure}


% To edit a room in the dungeon, the user double-clicks a room in the dungeon-editing view. In the room editing view, there is a pane of editing options on the left side of the screen. This pane displays the options available for manually editing the room by brush painting with one of the available tiles. The available tile-brushes are floor, wall, treasure, enemy or boss. There are three brush sizes, single tile, three-tile cross-shape, or five-tile cross shape. By control-clicking, the user can bucket paint all adjacent tiles of the same type. The middle of the view displays the currently edited room, also referenced as the target room. The right side of the room-editing view displays a matrix with suggested room designs. These suggestions have been procedurally generated through an evolutionary algorithm, and then selected by the IC MAP-Elites genetic algorithm~\cite{p13MAP-Elites-eddy}. The IC MAP-Elites runs continuously while the user is in the room-editing view, and updates the matrix with elites throughout. The evolutionary process is fed with the target room, and with that, all changes to the target room affect the generated suggestions. Additionally, the user has the option to paint locks on tiles in the target room to preserve specific tiles in all procedurally generated suggestions. The user has the option to apply one of the suggested room designs, restart the evolutionary generation, or editing the target room manually. The user can press "Show Patterns" to toggle the visibility of meso-patterns (ambush, guard chamber, treasure chamber and guarded treasure) in the target room, and can also toggle the visibility of locks drawn in the room by clicking "Show Locks". The button "Go To World Grid" takes the user back to the world editing view.


% % \begin{figure*}
% %     \includegraphics[width=\textwidth]{images/Artifact/empty room editing view.png}
% %     \caption{The modified room editing view in the tool.}
% % \end{figure*}

% The generating algorithm uses an interactive evolution, meaning the user input has influence in the generation of content. To supply the user with the most useful suggestions out of all the generated rooms, the program uses a MAP-Elites algorithm, which uses multidimensional discretization of the generated content to present the resulting highest rated candidates of each dimension to the user ~\cite{p13MAP-Elites, MAP-Elites-eddy}. 
% There are seven dimensions used for the MAP-Elites in EDD, representing a specific attribute of rooms~\cite{p13MAP-Elites-eddy}, described below.  


% \textbf{Symmetry}

% Symmetric structures tend to be more visually appealing for the user, and is often used by human designers to distribute content evenly over the canvas. Symmetry is assessed by non-passable tiles (i.e. walls), and evaluated along the X-axis, the Y-axis and the diagonals ~\cite{p13MAP-Elites-eddy}.


% \textbf{Similarity}

% Similarity is calculated by comparing tile by tile with the target room. Similarity is used to present the user with generated rooms that vary from their design, while also preserving aesthetic choices~\cite{p13MAP-Elites-eddy}.


% \textbf{Number of Meso-Patterns}

% The meso-patterns in a room displays the encounters the theoretical player could experience in that room . The considered patterns are treasure rooms, guard rooms, and ambushes . By identifying patterns in sets of tiles, such as a treasure surrounded by walls with an enemy guarding it being identified as a guard room, the number of meso-patterns can represent the utility of a room ~\cite{p13MAP-Elites-eddy}. 


% \textbf{Number of Spatial Patterns}

% The Spatial Patterns that can be identified are corridors, chambers, connectors and nothing . A high amount of spatial patterns correlates with a room that is subdivided well with walls, and has many elements other than floor tiles, which indicates a more unique and useful room ~\cite{p13MAP-Elites-eddy}.


% \textbf{Linearity}

% Linearity represents the number of traversable paths that exist between the doors in the room . This correlates to what kind of gameplay the designer is trying to accomplish, for example a more maze-like approach might be accomplished by having the player go through other rooms to reach the doors, while a more combat-based approach might be accomplished by high linearity with many guarding enemies~\cite{p13MAP-Elites-eddy}.


% \textbf{Inner Similarity}

% Inner similarity compares the micro-patterns of the generated room with the target room . The density and sparsity of enemies, treasures and walls is calculated to identify similarity in micro-patterns, to present the user with generated rooms that have similar design choices. This differs from the evaluation of Similarity which focuses on aesthetic similarity, as Inner Similarity focuses on the similarity in utility and functional design choices~\cite{p13MAP-Elites-eddy}.


% \textbf{Leniency}

% Leniency represents the difficulty in a room at any given point. The amount of treasures and enemies in a room, as well as their density and sparsity, together with how safe doors are, are considered in the equation. A room with low leniency has few enemies, or enemies placed sparsely, and a balanced amount of rewarding treasure. Leniency corresponds to the design of a room in terms of reward and risk~\cite{p13MAP-Elites-eddy}.


% \subsubsection{Designer Modeling}

% The relationship that occurs between co-creators in a MI-CC tool is arguably one of the most important element within a system that aims to foster creativity. Founding a valuable and co-creative relationship between AI and human is a major challenge, and further developing designs of AI is necessary to work towards perfecting this relationship. Designer modeling, as defined by Liapis et. al.~\cite{p13designer-modelling}, is a way to classify and predict a designer's style, goals, preferences, processes, and their definitions. As proposed by Alvarez et. al.~\cite{p13designer-modelling-edd}, this concept can be used to create an adaptive and collaborative colleague in a MI-CC environment. Alvarez et. al. has implemented a novel method of designer modeling within EDD, which successfully identified four frequent and unique designer styles, with related common sub-styles~\cite{p13designer-modelling-edd}. The study showed promise of this method being used to create an AI-colleague within EDD~\cite{p13designer-modelling-edd}. It is possible that an AI that can recognize a designers persona, style, preferences and goals, and adapting its behaviour to support these attributes, can result in a well-performing and interesting co-creator~\cite{p13designer-modelling, designer-modelling-edd}.


%\subsubsection{AI as colleague}
%text~\cite{p13roles} % Will maybe add later, but here I will describe the relationship between human and AI in co-creating. How does one create a colleague relationship? How is it used? Examples of Co-Creative colleagues in Game Design Tools. Examples of other uses.
