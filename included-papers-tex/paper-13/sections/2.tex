\subsection{Background}

% Machine Learning (ML) has gained an increased interest from game researchers, achieving remarkable success on training AI agents for very popular games, such as AlphaStar on Starcraft 2 \citepsixth{p6alphastarblog} and OpenAI Five on Dota 2 \citepsixth{p6berner2019dota}. Its combination with PCG has led to the raise of  Procedural Content Generation via Machine Learning (PCGML), defined as the generation of game content by models that have been trained on existing game content \citepsixth{p6summerville2018procedural}, with applications to autonomous content generation, content repair, content critique, data compression, and mixed-initiative design. 

Player modeling, the ability to recognize general socio-emotional
and cognitive/behavioral patterns in players \citepsixth{p6thawonmas2019artificial}, has been appointed by the game research community as an essential process in many aspects of game development, such as designing of new game features, driving marketing and profitability analyses, or as a means to improve PCG and game content adaptation. Player modeling frequently relies on data-driven and ML approaches to create such models out of several sorts of user-generated gameplay data \citepsixth{p6liapismodellingquality19,p6melhart2020feel,p6Drachen2009-playerModellingTombRaider,p6Holmgard2019-proceduralPersonas,p6Melhart2019-ModellingMotivation}.

Using player data from \textit{Iconoscope}, a freeform creation game for visually depicting semantic concepts, Liapis et al. trained and compared several ML algorithms by their ability to predict the appeal of an icon from its visual appearance~\citepsixth{p6liapismodellingquality19}. Furthermore, Alvarez and Vozaru explored personality-driven agents based on individuals' personalities using the \textit{cibernetic big five model}, evaluating how observers judged and perceived agents using data from their personality test when encountering multiple situations~\citepsixth{p6Alvoz2019-PersonalityDriven}. 

%  using Bartle's player archetypes~\citepsixth{p6bartle1996-taxonomy}

Moreover, training models on gameplay data from \textit{Tom Clancy's The Division} has also been used to model, and therefore find predictors of player motivation \citepsixth{p6Melhart2019-ModellingMotivation}, which renders a very valuable tool for understanding the psychological effects of gameplay. Former research followed a similar approach in \textit{Tomb Raider Underworld}, training player models on high-level playing behavior data, identifying four types of players as behavior clusters, which provide relevant information for game testing and mechanic design \citepsixth{p6Drachen2009-playerModellingTombRaider}. Melhart et al. take these approaches one step further by modeling a user's \textit{Theory of Mind} in a human-game agent scenario \citepsixth{p6melhart2020feel}, finding that players' perception of an agent's frustration is more a cognitive process than an affective response. %Alvarez and Vozaru did similar work, exploring personality-driven agents based on individuals' personality using the \textit{cibernetic big five model}, evaluating how observers judged and perceived agents using data from their personality test when encountering multiple situations~\citepsixth{p6Alvoz2019-PersonalityDriven}.

%Alvarez and Vozaru did similar work, exploring personality-driven agents based on individuals' personality using the \textit{cibernetic big five model}, which treats personality-driven agents as goal-based entitites, evaluating how observers judged and perceived agents using data from their personality test when encountering multiple situations~\citepsixth{p6Alvoz2019-PersonalityDriven}.
%modeling individual agents based 

\subsubsection{The Player is the Designer}

Mixed-initiative co-creativity (MI-CC)~\citepsixth{p6yannakakis2014micc}, is the subset of PCG algorithms where human users and AI systems engage in a constant mutual inspiration loop towards the creation of game content \citepsixth{p6charity2020baba,p6machado2019pitako,p6shaker2013ropossum,p6smith_tanagra:_2011,p6liapis_generating_2013}. Understanding player behavior and experience, as well as predicting the player's motivation and intention is key for mixed-initiative creative tools while aiming to offer in real-time user-tailored procedurally generated content. Nevertheless, the player is the designer in MI-CC, and gameplay data is replaced by a compilation of designer-user actions and AI model reactions over time while both user and model are engaged in a mutually inspired creative process. A fluent MI-CC loop should provide good human understanding and interpretation of the system, as well as accurate user behavior modelling by the system, capable of projecting the user's subsequent design decisions \citepsixth{p6ComptonPhD}. 

%Similar to user or player modeling, designer modeling for content creation tools (CAD and MI-CC tools) was suggested by Liapis et al~\citepsixth{p6Liapis2013-designerModel}, where it is proposed the use of designers models that capture their styles, preferences, goals, intentions, and interaction processes. In their work, they suggest methods, indications, and advice on how each part can be model to be integrated into a holistic designer model, and how each game facet can use and benefit from designer modeling. Moreover, in \citepsixth{p6Liapis2014-designerModelImpl} the same authors discuss their implementation of designer modeling and the challenges of integrating all together in their MI-CC tool, Sentient Sketchbook, which had a positive outcome on the adaptation of the tool towards individual “artificial” users.

Shifting towards a designer-centric perspective means that besides focusing on player modeling, it is necessary to focus on modeling the designers. Liapis et al.~\citepsixth{p6Liapis2013-designerModel,p6Liapis2014-designerModelImpl} introduced designer modeling for personalized experiences when using computer-aided design tools, with a focus on the integration of such in automatized and mixed-initiative content creation. The focus is on capturing the designer's style, preferences, goals, intentions, and iterative design process to create representative models of designers. Through these models, designer's and their design process could be understood in-depth, enabling adaptive experiences, further reducing their workload and fostering their creativity. 

%\citepsixth{p6charity2020baba,machado2019pitako,shaker2013ropossum,smith_tanagra:_2011,Machado2017,liapis_generating_2013}. 

% Moreover, goal 13 in the guidelines for Human-AI interaction \citepsixth{p6amershi2019guidelines} highlights the importance of learning from user behavior and personalize the user’s experience by learning from their actions over time. 


%Nourani et al.~\citepsixth{p6Nourani2019-meaningfulExplanations}, who discuss the effects of meaningful and meaningless explanations to users of an AI interactive systems, and their results demonstrates that when an explanation is not aligned with human-logic it significantly affect the user's perception of the system and it's usability is hindered.

Furthermore, lack of transparency is a key impediment for the advancement of human-AI systems, being eXplainable AI (XAI) an emergent research field that holds substantial promise for improving model explainability while maintaining high-performance levels~\citepsixth{p6adadi2018peeking,Doshi-Velez2018}. However, explanations should be aligned with the users' understanding to don't hinder the usability of systems, as demonstrated by Nourani et al.~\citepsixth{p6Nourani2019-meaningfulExplanations}, who discuss the effects of meaningful and meaningless explanations to users of an AI interactive systems.

Zhu et al.~\citepsixth{p6Zhu2018-XAIDesignersMICC} proposed the field of eXplainable AI for Designers (XAID) as a human-centered perspective on MI-CC tools. This work discusses three principles of mixed-initiative, \emph{explainability}, \emph{initiative}, and \emph{domain overlap}, where the latter focuses on the study of the overlapping creative tasks between game designers and black-box PCG systems in mixed-initiative contexts. This work deems of high relevance the inclusion of data-driven and trained artifacts to facilitate a fluent bi-directional communication of the internal mechanisms of such a complex co-creative process in which \textit{the designer provides the vision, the AI provides capabilities, and they merge that into the creation}. Mapping the designer's internal model to the AI's internal model is suggested as a meaningful way for creating a common ground that establishes a shared language that enables such communication. In the same line, Xie et al.~\citepsixth{p6xie2019interactive} explored visualization techniques through an interactive level designer tool called \textit{QUBE} to explain and introduce machine learning principles to game designers.

Moreover, Guzdial et al.~\citepsixth{p6guzdial-lvldsg-aiide-2018} discuss the insufficiency of current approaches to PCGML for MI-CC, as well as the need for training on specific datasets of co-creative level design. Guzdial et al. work on the mixed-initiative Morai Maker~\citepsixth{p6guzdial2019friend} shows the relevance of exploring the ways designers and AI interact towards co-creation, identifying four human-AI relationships (friend, collaborator, student, and manager), as well as the different ways they impact on the designer-user experience. Our study advocates for the importance of designer modeling through ML as the generation of surrogate models of designer styles by training on existing designer-generated data, aiming for an improvement in quality and diversity in computational creativity and, in particular, MI-CC tools. 

\subsubsection{The Designer Preference Model in EDD}

EDD is an MI-CC tool where designers can create dungeons and rooms; meanwhile, a PCG system analyzes their design and proposes generated suggestions to the designer~\citepsixth{p6Alvarez2018, Baldwin2017}. EDD uses the \emph{Interactive Constrained MAP-Elites} (IC-MAP-Elites)~\citepsixth{p6alvarez2019empowering}, an evolutionary algorithm that combines Constrained MAP-Elites~\citepsixth{p6Khalifa2018} with interactive and continuous evolution. 

The work presented in \citepsixth{p6Alvarez2020-DesignerPreference} introduced the Designer Preference Model, a data-driven solution that learns from user-generated data in the MI-CC Evolutionary Dungeon Designer. This preference model uses an Artificial Neural Network to model the designer based on the choices she makes while using EDD. Both systems constantly interact and depend on each other, so that the Designer Preference Model learns from the generated and selected elites, and IC-MAP-Elites uses the Designer Preference Model as a surrogate model of the designer to complement the fitness evaluation of new individuals. 

This approach's main goal is modeling the user's design style to better assess the tool's procedurally generated content, increasing the user's agency over the generated content without stalling the MI-CC loop \citepsixth{p6ComptonPhD} or increasing user fatigue with periodical suggestion handpicking \citepsixth{p6liapis2016mixed,p6Takagi2001-InteractiveEvo}. The results showed the need for stability and robustness in the data-driven model, to counterbalance the highly dynamic designer's creative process. 

