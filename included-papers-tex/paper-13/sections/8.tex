


\subsection{Conclusion}

This study explored the limitations and possibilities of an MI-CC-tool with an AI with a varied agency. We aimed at doing an initial exploratory study with static parameters, resulting in baselines to analyze what can be done and how designers experienced the system. This, in turn, opens up and continues the discussion towards AI collaborating as a colleague and enabling alternative ways to foster creativity (e.g., constraining the design space such as in~\cite{p13bhaumik_lode_2021}). Our study showed that AI gaining control over the design results in frustration and feeling constrained. Constraints are not bad per se, as they can be a way to foster creativity~\cite{p13boden_creative_2004,acar_creativity_2019}, but they need to be placed in a way that the human designer might feel inspired, motivated, or supported to continue the design. Human designers had to adapt towards those imposed goals instead of the other way around, which creates an unwanted dynamic when human designers perceive the AI's behavior as erratic, random, and without clear objectives.

 %One limitation of our study is that there are Our study is limited by static changes across versions with no adaptable parameter based on what the designer creates that could have influenced this unwanted dynamic.

%We have aimed at doing an initial explorative study with static changes and no adaptable parameters. With these parameters and naïve baselines, the extent of what can be done and how designers experienced the system could be initially explored and discussed. In turn, this opens up and continues the discussion towards AI collaborating as a colleague and enabling alternative ways to foster creativity (e.g., constraining the design space).

% One of the main challenges within MI-CC is to develop an AI co-creator which performs well, and makes valuable and satisfactory decisions that the human designer appreciate and wants to incorporate. This study explicitly aimed towards exploring the limitations and possibilities of an MI-CC-tool with an AI with varied initiative and definitive impact on the creations, which showed several challenges such as lose of control and perceived behavior by the AI. Our study showed that AI gaining control over the design results in frustration and feeling constrained. Constrains are not fundamentally bad as they can be a way to foster creativity~\cite{p13boden_creative_2004,acar_creativity_2019}, but constraints need to be placed in a way that the human designer might feel inspired, motivated, or supported to continue the design. As a result of the AI gaining more initiative, human designers had to adapt towards those imposed goals instead of the other way around, which creates an unwanted dynamic when human designers perceive the AI's behavior as erratic, random, and without clear objectives.

% Moreover, the study identified possible issues that may arise in MI-CC design tools. As the advantages of MI-CC and the interest from creators to use MI-CC systems has been further confirmed in this study, the importance of further research to improve these systems is evident. This study focused on adjusting the control to explore how the human designer reacted to another type of AI, one that has a more equal role to the human designer. 

%Improving an equally influential AI co-creator (i.e., collaborator) is important in MI-CC.

Many of the results pointed to a general preference for an AI with a more supportive role in collaborative tools. One approach could be to have a hybrid model between what is presented in this paper and other typical MI-CC systems that focus more on suggesting final designs. The AI could take parameters from the human designer, such as an area in a room, amount of tiles, or an attribute that the human designer would like to increase in the room, but still maintain their design, effectively constraining the AI to find creative ways to achieve its goals. In EDD, designers can lock tiles to not be changed by the AI, which is something to be experimented with. Although this would give the human designer a slightly higher degree of influence on the end product compared to the AI, the constraints of how many tiles can be locked, or possibly what types of tiles can be locked, can be experimented with to adjust the relationship between AI and human designer. Currently, the search is steered, to some extent, by the designers' design, but in future work, we could bias the search even more towards interesting areas based on the creation process and the trajectories they are taking in behavior dimension space. 

Additionally, using designer models is a feasible approach. By predicting design goals, adapting to phases of the design process, or identifying certain design styles and adapting to the human designer, a responsive and adaptive, intelligent, and human-like artificial co-creator could be developed. This could allow for an AI that adapts to the human designer and performs well enough that the frustrations and feelings of constraints are minimal or perceived as less prevalent as the designs turn out more similar to what the human desired.

% For example, the human designer could place down tiles freely, and then order the AI to increase the symmetry in the room, and the AI would then edit tiles to reach a certain level of symmetry. This would likely be easily implemented for all the current dimension in the MAP-elites algorithm, already present in EDD.

% \subsubsection{Improving an Equally Influential AI Co-Creator}


% The results in the study show that an AI co-creator, that has equal control as the human designer or more, is easily perceived as frustrating and constraining. Because the MI-CC shows great promise for efficiently creating game content, and furthering the human's creativity, it is still important to try to develop an AI that can co-create with the human in a more satisfactory way. One example of how this can be achieved is to give the human designer the option to lock a limited amount of tiles, disabling them from being overwritten by the AI. Although this would give the human designer a slightly higher degree of influence on the end product compared to the AI, the constraints of how many tiles can be locked, or possibly what types of tiles can be locked, can be experimented with to adjust the relationship between AI and human designer. Alternatively, this can be used to evaluate the overlap of roles in the AI as an assistant and an equal colleague, and how the human reacts to the differences in the resulting relationships. 


% An additional example of how an AI co-creator of high quality can be created is by incorporating designer modelling into the AI. By predicting design goals, adapting to phases of the design process, or identifying certain design styles and adapting to the human designer, a responsive and adaptive, intelligent and human-like artificial co-creator could be developed. This could allow for an AI that adapts to the human designer, and performs well enough that the frustrations and feelings of constraints are minimal, or perceived as less prevalent as the designs turn out more similar to what the human desired.


% \textbf{AI as an Assistant in Mixed-Initiative Co-Creative Systems}

    
% Many research projects have already explored the AI collaborator as an assistant to the human designer. This study focused on adjusting the control to explore how the human designer reacted to another type of AI, one that has a more equal role to the human designer. Many of the results pointed to a general preference of an AI that has a more supportive role in collaborative tools.
% One example of what can be explored in to cover this area of further research is implementing the AI co-creator with another purpose, namely to be a creative assistant to the human designer. 
% Within EDD for example, this new version of the tool could work by implementing AI that takes parameters from the human designer, such as an area in a room, amount of tiles, or an attribute that the human designer would like to increase in the room. For example, the human designer could place down tiles freely, and then order the AI to increase the symmetry in the room, and the AI would then edit tiles to reach a certain level of symmetry. This would likely be easily implemented for all the current dimension in the MAP-elites algorithm, already present in EDD.

    
% \textbf{Improving an Equally Influential AI Co-Creator}


% The results in the study show that an AI co-creator, that has equal control as the human designer or more, is easily perceived as frustrating and constraining. Because the MI-CC shows great promise for efficiently creating game content, and furthering the human's creativity, it is still important to try to develop an AI that can co-create with the human in a more satisfactory way. One example of how this can be achieved is to give the human designer the option to lock a limited amount of tiles, disabling them from being overwritten by the AI. Although this would give the human designer a slightly higher degree of influence on the end product compared to the AI, the constraints of how many tiles can be locked, or possibly what types of tiles can be locked, can be experimented with to adjust the relationship between AI and human designer. Alternatively, this can be used to evaluate the overlap of roles in the AI as an assistant and an equal colleague, and how the human reacts to the differences in the resulting relationships. 


% An additional example of how an AI co-creator of high quality can be created is by incorporating designer modelling into the AI. By predicting design goals, adapting to phases of the design process, or identifying certain design styles and adapting to the human designer, a responsive and adaptive, intelligent and human-like artificial co-creator could be developed. This could allow for an AI that adapts to the human designer, and performs well enough that the frustrations and feelings of constraints are minimal, or perceived as less prevalent as the designs turn out more similar to what the human desired.



% it is of importance to attempt to identify possible solutions to these challenges. 

% This section concludes the paper by summarizing the identified answers to the research questions, as well as introducing future work that may improve related studies, based on the results, analysis and discussion in previous sections. 


% \textbf{RQ1:  How does adjusting the control of the design decisions in mixed-initiative systems affect the human user's creativity?}


% The results show that humans feel constrained and frustrated as the AI co-creator gains control over the design process. Mixed-Initiative systems are suitable to provide new ideas for humans in creative tools, and this was further confirmed by the descriptions of the participants experiences with the tool. However, as the AI gain control over the design, the human sometimes looses interest in being creative and lets the AI take over the creative process.


% \textbf{What are the effects on the human designer's design goal during the process?}

    
% Almost all of the human designers felt forced to adapt their design goals as a result of the AI's decisions, but only when the AI had equal to or more control over the design than the human. 

    
% \textbf{What are the effects on the human designer's perception of limitations or frustration?}

    
% Almost all participants felt constrained and limited in their creativity. This contributed to frustration when using the tool, and had negative effects on the perception of the creative process. Some reacted to the constraints with a decreased interest to design thee room ,and let the AI design alone instead. Some adapted their design style to more iterative style, where the design goal was no longer important, but focus was steered towards creating something satisfying only during the current turn instead.

    



% \textbf{RQ2: How can the three degrees of initiative be explored to asses the support of the AI in terms of the human's creative process?}


% The degrees of control used in this study showed that in this tool, humans generally prefer an AI with low initiative, over one with equal to or higher than the human co-creator. As the AI gained more influence on the design process, the human designers felt increasingly stumped in their creativity. 
% For a human designer to feel satisfied with an AI co-creator of higher initiative, one hypothesis may be that the AI has to have an advanced intelligent behaviour, which is adaptive to the human designer. The results from this study supports the speculation that it is more likely that humans are more willing to collaborate with AI that does exceed the human in terms of control.


% \textbf{How do they support lateral thinking and the introduction of new ideas?}

    
% The low and medium level of initiative show potential as supporters of lateral thinking. The Highest level of control does introduce new ideas, however it also proposes limitations and constraints to the human designer that often makes the human feel creatively constrained.

    
    
% \textbf{How does the designer respond to the AI's differing degrees of initiative?}

    
% The results suggest a willingness to adapt to the different levels of control of the AI. The responses to the AI with a low degree of initiative where generally positive, and most participants preferred this version as it provided interesting suggestions and ideas without having a definitive influence on the design. 
% Responses to the AI with the medium degree of initiative where generally frustrating, specifically with the behaviour of the AI and not the concept of an AI that contributes with editable content. The AI with the highest degree of initiative made the designers feel constrained and limited in their design.






% \subsubsection{Future Work}

% % \emph{Suggestions of Improvements (Q11)}


% % All participants suggested removing the constrain of turns and amount of tiles per turn. Three participants suggested that the AI could be used more as an assistant to the human designer, by giving the AI parameters such as area or type of tiles to place. Three participants suggested using a more intelligent or human-like AI-agent. Examples of how the AI could be improved included valuing the types of tiles similarly to how the human designer does, learning from what the human contributes with and adapting its behaviour, and possibly attempting to predict the design goal that the human designer has. One participant answered that they would like to have the AI suggest complete generated rooms, that the human designer can then polish or edit freely. One participant had a related suggestion, which is a final step of the design process when the room is complete, where the human can edit an additional set amount of tiles, giving the human a chance to overwrite some AI-placed tiles.

% % \vspace{3mm}

% The study has identified possible issues that may arise in MI-CC game level design tools. As the advantages of MI-CC and the interest from creators to use MI-CC systems has been further confirmed in this study, the importance of further research to improve these systems is evident. 


% \textbf{AI as an Assistant in Mixed-Initiative Co-Creative Systems}

    
% Many research projects have already explored the AI collaborator as an assistant to the human designer. This study focused on adjusting the control to explore how the human designer reacted to another type of AI, one that has a more equal role to the human designer. Many of the results pointed to a general preference of an AI that has a more supportive role in collaborative tools.
% One example of what can be explored in to cover this area of further research is implementing the AI co-creator with another purpose, namely to be a creative assistant to the human designer. 
% Within EDD for example, this new version of the tool could work by implementing AI that takes parameters from the human designer, such as an area in a room, amount of tiles, or an attribute that the human designer would like to increase in the room. For example, the human designer could place down tiles freely, and then order the AI to increase the symmetry in the room, and the AI would then edit tiles to reach a certain level of symmetry. This would likely be easily implemented for all the current dimension in the MAP-elites algorithm, already present in EDD.

    
% \textbf{Improving an Equally Influential AI Co-Creator}


% The results in the study show that an AI co-creator, that has equal control as the human designer or more, is easily perceived as frustrating and constraining. Because the MI-CC shows great promise for efficiently creating game content, and furthering the human's creativity, it is still important to try to develop an AI that can co-create with the human in a more satisfactory way. One example of how this can be achieved is to give the human designer the option to lock a limited amount of tiles, disabling them from being overwritten by the AI. Although this would give the human designer a slightly higher degree of influence on the end product compared to the AI, the constraints of how many tiles can be locked, or possibly what types of tiles can be locked, can be experimented with to adjust the relationship between AI and human designer. Alternatively, this can be used to evaluate the overlap of roles in the AI as an assistant and an equal colleague, and how the human reacts to the differences in the resulting relationships. 


% An additional example of how an AI co-creator of high quality can be created is by incorporating designer modelling into the AI. By predicting design goals, adapting to phases of the design process, or identifying certain design styles and adapting to the human designer, a responsive and adaptive, intelligent and human-like artificial co-creator could be developed. This could allow for an AI that adapts to the human designer, and performs well enough that the frustrations and feelings of constraints are minimal, or perceived as less prevalent as the designs turn out more similar to what the human desired.







