\subsection{Conclusions and Future Work}

%In this paper, we have evaluated in-depth the expressive range of IC MAP-Elites, how various pairs of dimensions yield a better search with more diverse and high-quality individuals  

% \begin{enumerate}
%     \item Regarding what pair of dimensions enable IC MAP-Elites to explore greater ranges...
%     \item Regarding the comparative analysis we can observe...
%     \item Regarding bias in the search space, it is shown that IC MAP-Elites is able to explore past the area of the target design, exploring in avg. around 60\% of the delimited search space by the granularity and dimensions. Moreover, Similarity has a high correlation with fitness, and depending on the objective of the designer, this might affect positively or negatively since there will be a strong bias towards highly similar individuals. 
% \end{enumerate}

% \begin{table}[]
% \centering
% \caption{Comparison of the avg. explored space and avg. fitness of the generated individuals in different evaluations.}
% \label{tab:evaluationTable}
% % \resizebox{0.25\textwidth}{!}{%
% \begin{tabular}{|l|ccc|}
% \hline
% Dimension & \multicolumn{1}{c|}{$\bigtriangleup$} & \multicolumn{1}{c|}{$\bigcirc$} \\ \hline
% NMP & 36.93\% & 53.62\% \\ \hline
% NSP & 38.17\% & 60.77\%      \\ \hline
% Leniency & 36.42\% & 56.55\%      \\ \hline
% Symmetry & 38.42\% & 57\%      \\ \hline

% \multicolumn{3}{l}{$\bigtriangleup$ Avg. coverage across dimensions} \\ 
% \multicolumn{3}{l}{$\bigcirc$ Avg. coverage in respective dimension pair}
% \end{tabular}%
% % }
% \end{table}



In this paper, we have done an in-depth evaluation of IC MAP-Elites by analyzing the expressive range of all the possible pairs of dimensions, their relation to other dimensions and the fitness. %While one of the properties of MAP-Elites is to foster the exploration of a diverse and high-quality population, 
Our results indicate specific dimensions in level generation, such as when using NMP, NSP, \textsc{leniency}, or Symmetry, that foster greater exploration. The exploration is not only fostered in their respective dimensions (in avg. 54\%, 61\%, 57\%, 57\%, respectively, when used) but also in all the others due to the diverse individuals generated within the respective dimensions as shown in \Cref{figs:dimensions-example}. As observed on~\Cref{figs:dimensions-related-fitness}, aesthetic feature dimensions such as~\emph{Similarity} and~\emph{Symmetry} have an impact when used or not in the search. When not using~\emph{Symmetry}, the search does not explore high levels of symmetry, disregarding to some extent that aesthetic feature in favor of exploring the other feature dimensions. Moreover, \emph{Similarity} has a high correlation with fitness as observed in~\Cref{figs:dimensions-related-fitness} and depending on the designer's objective, this might affect positively or negatively since there will be a strong bias towards highly similar individuals. In contrast, \textsc{is} seems to be more robust in the fitness landscape and the exploration of other dimensions because it captures the properties of the target room rather than its aesthetics. 

Regarding the bias in the exploration when using IC MAP-Elites together with continuous and adaptive evolution, our results show that the generated content is highly diverse with dense areas along most of the searched space, which is shown as well in \Cref{tab:evaluationTable}. This means that due to the diversity pressure imposed by IC MAP-Elites, the search is unlikely to be biased towards creating content that is similar in the feature dimensions' scores of the target room. Yet IC MAP-Elites adapts to the target room and generates high-performing individuals along the rest of the space in the other dimensions, especially in those explicitly used in the search.

To further assess the algorithm, we ran an experiment using all possible dimensions (\Cref{figs:all-dimensions-earun}a) rather than specific pairs to observe the exploration and exploitation of the algorithm when not using specific pair of dimensions. As expected, it explores a substantial area of the search space in all dimensions (in avg. $51.7$\%) but surprisingly, the search results in the exploitation of sub-optimal individuals in all dimensions with an avg. fitness of $0.78$ as shown in \Cref{figs:dimensions-related-fitness}d. While using pair of dimensions results on $15.24$\% reduced exploration, the exploitation seems to be fairly spread among the explored space, visible in \Cref{figs:all-dimensions-earun}b and c, and the density related to fitness is focused on high-performing individuals~\Cref{figs:dimensions-related-fitness}a-c. This points towards that there are difficulties in (1) \textbf{fully exploring} the space when using a pair of dimensions even when the exploitation is distributed and focused on high-performing individuals, and in (2) \textbf{exploiting the promising areas} of the search when using a higher range of dimensions even when the space is vastly explored. 

In addition, based on our experiments, such difficulties are exacerbated since the exploration stagnates and keeps exploiting the same areas after \~{}1000 generations, yet finding novel individuals. This happens regardless of the number of dimensions, which dimensions, or the target room. Our findings point to challenges in the \emph{selection step} of IC MAP-Elites, which selects cells uniformly random. Exploring different methods for the selection of cells and individuals is a promising future step. For instance, Gravina et al. \cite{p6Gravina2019-blendingNotionsDiversity} explored how four divergent search algorithms guide cells' selection.
%, Novelty Search (NS), Surprise Search (SS), Linear combination of NS and SS, and a combination  of NS and SS via NSGA-II, 
benefit MAP-Elites standard selection method.

% In addition, based on our experiments, such difficulties are exacerbated since the exploration stagnates and keeps exploiting the same areas after \~{}1000 generations, yet finding novel individuals. This happens regardless of the number of dimensions, which dimensions, or the target room. Our findings point to challenges in the \emph{selection step} of IC MAP-Elites, which selects cells uniformly random. Exploring different methods for the selection of cells and individuals deserves more attention, and it is a promising future step. For instance, Gravina et al. \cite{p6Gravina2019-blendingNotionsDiversity} explored how the selection of cells is guided by four divergent search algorithms.

Furthermore, preliminary experiments seem to indicate that some dimensions, which are more explorative (e.g., NMP, NSP, Leniency, Symmetry), are more robust to changes in the target room, exploring similar areas of the search space regardless of the target, which can be observed in \Cref{figs:all-dimensions-earun} (b) and (c). This points towards dimensions that would make the algorithm more robust to changes, reinforcing its adaptability feature. Further experiments are needed to analyze how different dimensions are better at adapting to continuous changes in the target room, which would also indicate better stability in the search. Along these lines, further evaluation is needed with human designers to assess and explore whether IC MAP-Elites is beneficial for the MI-CC workflow and interaction. 

% At the very least, based on our results and comparing it to the previous EA, IC MAP-Elites will generate more diverse levels while retaining high-quality among them, giving more options and suggestions to designers. All of this, while also giving the designer a meaningful choice, as what dimensions are chosen impacts the search of IC MAP-Elites as demonstrated.% but our results indicate that spending resources on analyzing how dimensions behave and intertwine can arise better results in the explorative-exploitative tradeoff than just adding dimensions together blindly expecting more quality-diverse in the solutions.
%Further experiments are needed to analyze how different pair of dimensions are better at adapting 

Currently, IC MAP-Elites is only used in EDD on a per room basis. Future work should consider the whole dungeon for the evolution procedure and further develop for the generation of complete dungeons using holistic metrics to evaluate the dungeon.

%, generating suggestions independently from the whole dungeon.

In conclusion, based on our results, IC MAP-Elites generate more diverse levels while retaining high-quality among them, which would result in richer and more options and suggestions to designers. Our experiments show that which dimensions are used significantly impact the search space, fostering the search of high-quality individuals within the selected dimensions while not discouraging exploration in other dimensions. This means that by editing their levels and choosing which dimension IC MAP-Elites should focus on, the designers will be given more meaningful choices and interactions.

% In conclusion, our evaluation is aligned with our expectations when enabling designers to explore different pairs of dimensions. Individuals found in the search space are, in general, more diverse and high-performing, which results in a richer population to be suggested to the designer. Our experiments show that which dimensions are used have a big impact in the search space, fostering the search of high-quality individuals within the selected dimensions while not discouraging exploration in other dimensions. In other words, enabling the designers to proactively decide which dimensions should be used in the search, gives them a high level of controllability with minimal loss in the expressive range.