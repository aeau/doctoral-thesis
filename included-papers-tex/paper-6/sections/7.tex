\subsection{Designer Personas Usage}

The archetypical paths presented in this paper are steps 

Furthermore, clustering both the room style and the paths taken by the designers, showed two emergent partitions and distributions. When clustering the room style, we observed that the clusters naturally divided into three supersets in they Y axis, (a) architectural patterns complexity, relating to clusters composed of rooms with clearer or complex architectural shapes done with walls. (b) Goal creation, enemy\/treasure balance, with clusters comprehending the strategic addition of enemies and treasures to establish some objective for the player within the room. In terms of EDD, these rooms are composed of more meso patterns. And (c), over-population, which relate to clusters filled with less organized and dense rooms, where probably designers tried more random designs such as in cluster 9 (''Dense, disorganized micro-patterns"). Identifying the designer in such superset, and the path they have taken to get there, could show meaningful information in the design process. For instance, the intentions of the designer, in what phase of the design process she is at the moment i.e. trying the tool or observing how the tool reacts, or scraping her current goal towards a new goal within the room. 

Of course, it is important to also observe the nature of the previous and future rooms that are created by a designer; thus, observing the dungeon as a whole, to understand the designers' intentions and goals when they proceed to create a new room. However, identifying the paths that designers are taking, to understand their intentions and goals, and either help them achieve this through suggestions with the IC MAP-Elites, showing them other ways of achieving their goals, or leading them through a different path.

Furthermore, when analyzing how the different design sequences were clustered and forming the designer personas, we observed an interesting dual tendency of the designers. This dual tendency is to either focus on the aesthetic configuration of the room based on what is perceived in the editor through the personas: \textsc{Architectural-focus} and \textsc{Split central-focus}, and to focus on the player experience through the personas: \textsc{Goal-oriendted} and \textsc{Complex-balance}. This exemplified quite good 
dualistic role 
When forming the designer personas, and analyzing how different design sequences

The archtypical paths a dual tendency of the designers to either go for a strategy that reflects their perception of the level from the editor - like the aesthetic configurations of it, instead of the experiential ones. for example the ones that had a split central focus and a structural focus (which btw maybe i would change to architectural focus). and then there's the ones that have a focus on the player experience like the goal oriented and complex behavior ones. i think this split reflex a very nice dualistic role that the designer has in front of the editor - that of creating an aesthetically pleasing object, as they see it in the editor, and that of creating an experience.


Finally, in our work we did not observe any type of cross-path i.e. a design which deviating from one path to another. We believe that this is due to the level at which we are observing the archetypical paths i.e. room-wise, as the designer normally would design the room with a set of goals. However, preliminary analysis on the dataset used in this paper and as expected, the design process of designed rooms within the same dungeon does follow different paths, and sometimes even crossing each other. This opens an interesting and exciting area to explore as while our focus have been room-wise, the extension to a wider layer observing rooms as a set of archetypical paths taken by designers might help to identify the designer's goals and intentions, and model a better system.