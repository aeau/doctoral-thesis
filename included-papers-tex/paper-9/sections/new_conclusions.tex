\subsection{Conclusions and Future Work}

% \begin{itemize}
%     \item ADD ABOUT FUTURE WORK ON EVALUATING THIS WITH A NON-INTERACTIVE VERSION AND USING OTHER ALGORITHMS. 
%     \item important to point out that static evaluation was done in~\cite{p9Alvarez2020-ICMAPE}
% \end{itemize}

This paper analyzes and evaluates the benefits of dynamically interacting with quality-diversity algorithms, specifically, the IC MAP-Elites. We have examined the adaptability and stability of MAP-Elites in relation to 21 dimension pairs highlighting different characteristics and properties through different simulated design scenarios. We examined key metrics when exploring the generative space, as depicted in table~\ref{tab:resultsTable}, and conducted three different case studies that highlighted different dynamics with the algorithm.
% IC MAP-Elites shows a high degree of adaptability to dynamic environments, adapting the generated content to the design process and design goals while stably generating high-performing and diverse solutions. This is made possible due to the inherent properties of QD-algorithms. For MI-CC systems and interactive approaches as in EDD, this is especially relevant and important. The fitness function adapts to the current design; thus, adaptability and stability go hand in hand. Furthermore, the deployment of an MI-CC approach in a scenario such as the ones presented would benefit both Map-Elites and the human designer. On the one hand, it enables MAP-Elites to explore more of the generative space while producing quality solutions; on the other hand, users would have more control over the suggestions as they influence and guide the search and generation.

% While our results show several MAP-Elites’ properties and promising ways to improve the MI-CC workflow, further evaluation is needed with human users to assess these properties in-the-wild and evaluate the interactive dynamic between humans and algorithms. To further highlight the importance of interaction, it would be interesting to analyze and compare with MAP-Elites disabling adaptive mechanisms (i.e., rendering the algorithm static and agnostic to changes) and with other non QD algorithms. Likewise, another interesting project for future work, would be to evaluate and compare IC MAP-Elites using TERAs with other co-creative systems using different algorithms such as reinforcement learning~\cite{p9delarosa2020-RLbrushMixedinit,guzdial-lvldsg-aiide-2018} or constraint solving algorithms~\cite{p9Karth2019-pcgmlDiscriminativeLearning}.
 
While our results show several MAP-Elites’ properties and promising ways to improve the MI-CC workflow, further evaluation is needed with human users to assess these properties in-the-wild and evaluate the interactive dynamic between humans and algorithms. To further highlight the importance of interaction, it would be interesting to analyze and compare with MAP-Elites disabling adaptive mechanisms (i.e., rendering the algorithm static and agnostic to changes) and with other non QD algorithms. Likewise, another interesting project for future work, would be to use TERAs to compare IC MAP-Elites with other co-creative systems using other algorithms such as reinforcement learning~\cite{p9delarosa2020-RLbrushMixedinit,guzdial-lvldsg-aiide-2018} or constraint solving algorithms~\cite{p9Karth2019-pcgmlDiscriminativeLearning}.
 
% Using a similar methodology as the one presented in this paper, an interesting project for future work would be to evaluate and compare MAP-Elites and its properties to other co-creative systems using different algorithms such as reinforcement learning~\cite{p9delarosa2020-RLbrushMixedinit,guzdial-lvldsg-aiide-2018} or constraint solving algorithms~\cite{p9Karth2019-pcgmlDiscriminativeLearning}.


%In this paper, we also present a methodology that extends, to some extent, the work by Smith and Whitehead~\cite{p9Smith:2010:Expressive-range} to continuously evaluate generative systems using simulated design sessions and aggregated and non-aggregated ERAs, which highlights properties and dynamics of the algorithms. Thus, an interesting future work would be to use such an approach to evaluate and compare MAP-Elites and its properties to other co-creative systems using different algorithms such as reinforcement learning~\cite{p9delarosa2020-RLbrushMixedinit,guzdial-lvldsg-aiide-2018} or constraint solving algorithms~\cite{p9Karth2019-pcgmlDiscriminativeLearning}.

%Similarity and Inner Similarity were excluded from the results given that these two dimensions change dynamically as the design changes. Preliminary observations from table~\ref{tab:resultsTable}, indicates that when using IS the generative space is very much covered while retaining an overall high fitness. This points towards a robust dimension, highly adaptable, and with stable growth and performance, but at the expense of general stability and reliability in a mixed-initiative setup as it . Analysis of these dynamic dimensions is left for future work. 

Finally, a promising step is to analyze MAP-Elites together with surrogate designer models that capture preference, style, and design processes~\cite{p9Liapis2013-designerModel,Alvarez2020-DesignerPreference,alvarez2020-designerpersonas}, and how these influence the properties discussed in this paper. For instance, Designer Personas~\cite{p9alvarez2020-designerpersonas} could be used to explore how the user's design moves through the space, identifying possible paths, and analyzing if changes, i.e., moving between style clusters, connect to key moments in the search.