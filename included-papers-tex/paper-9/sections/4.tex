\subsection{Approach}

% \begin{table*}[t]
% \centering
% \caption{Developed game based features used as dimensions in the IC MAP-Elites}\label{tab:dimensions}
% % \resizebox{\textwidth}
% % \resizebox{\textwidth}
% \begin{tabularx}{\textwidth}{|c|X|}
% \hline
% \rule{0pt}{12pt}
% Feature&Definition\\ \hline
% % \\[-6pt]
% Similarity & Refers to the aesthetic (tile-by-tile) similarity between a room and the current designer's design.\\ \hline
% Inner Similarity & Refers to the similarity of the sparsity and density of the different tile types of a room designer's current design.\\ \hline
% Symmetry & Refers to the aesthetic symmetry of a room.\\ \hline
% Leniency & Refers to how challenging rooms are; calculated based on the position of enemies and balance between enemies and treasures.\\ \hline
% Linearity & Refers to the amount of paths connecting doors within a room; calculated based on how many spatial patterns are traversed.\\ \hline
% \#Meso-Patterns & Refers to the number of meso-patterns that exist within a room, normalized by an estimated maximum number based on the room's size and the minimum chamber size.\\ \hline
% \#Spatial-Patterns & Refers to the number of spatial-patterns that exist within a room, which can be chambers, corridor, turns, junctions, and intersections.\\ \hline
% \end{tabularx}
% \end{table*}

\begin{table}[]
\begin{tabular}{p{0.25\linewidth}| p{0.65\linewidth}}
Feature          & Definitions                                                                                                                  \\ \hline
Similarity (Sim)       & Aesthetic (tile-by-tile) similarity between a generated level and the designer's design.                         \\ \hline
Inner Similarity (IS) & Different tiles' sparsity and density similarity between a generated level and the designer's design.     \\ \hline
Symmetry         & Room's aesthetic symmetry.                                                                                 \\ \hline
Leniency (Len)        & Challenge based on enemies and treasures.\\ \hline
Linearity (Lin)       & Paths that exist connecting entry points in a level.  \\ \hline
\#Meso-Patterns (Meso) & Amount of meso-patterns that exist within a level. This is a discrete dimension rather than continuous.\\ \hline
\#Spatial-Patterns (Spa) & Amount of spatial-patterns that exist within a level. \\ \hline
\end{tabular}
\caption{Level design based dimensions used in EDD with IC MAP-Elites.}
\label{tab:dimensions}
\end{table}

IC MAP-Elites is a variation of Constrained MAP-Elites that incorporates adaptive mechanisms and implements continuous and interactive evolution~\cite{p9Alvarez2020-ICMAPE}. Our evaluation described in the following section is applied to EDD, which implements IC MAP-Elites, allowing us to evaluate the effects and dynamics of interacting with MAP-Elites. 

% IC MAP-Elites is a variation of Constrained MAP-Elites that incorporates adaptive mechanisms and implements continuous and interactive evolution. IC MAP-Elites uses an adaptive fitness function continuously adapting to the user's design, and enables users to flexibly change dimensions and cells' granularity at runtime~\cite{p9Alvarez2020-ICMAPE}. Our evaluation described in the following section is applied to EDD, which implements IC MAP-Elites, allowing us to evaluate the effects and dynamics of interacting with MAP-Elites. 

% in EDD are defined as enemy, treasure, chamber, corridor, connector, entrance, and door. These micro-patterns are further divided into two categories, \textit{spatial micro-patterns}:  



%EDD is a mixed-initiative tool to create levels in a dungeon game where the designer edits rooms, and the system generates a set of suggestions using IC MAP-Elites that are suggested to the designer in a grid form. EDD's nature and mixed-initiative approach allow us to test the effects of interacting with MAP-Elites, the benefits for the interactive system, and the effects for users and MAP-Elites.

%During the time of evaluation, the agent is also determiningthe level’s population placement. A chromosome is placed in thefeasible population if it satisfies the following two constraints:•Number of spawners doesn’t exceed a fixed maximum value.•There are at least 10 bullets present for more than 50% offrames.

EDD's IC MAP-Elites implementation uses a single-objective weighted fitness function with a FI2Pop genetic algorithm~\cite{p9Kimbrough2008}. Individuals are deemed infeasible when they contain unreachable areas from any of the room's entry points and are evaluated based on how many unreachable tiles they have. Feasible individuals are evaluated as the following equally divided weighted sum:  

\begin{equation} 
\label{eq:fitness_func}
f_{fitness}(r) = \frac{1}{2}f_{inventorial}(r) \,+ \, \frac{1}{2}f_{spatial}(r)
\end{equation}

This evaluation is adaptive, meaning that the tile's ratios, patterns, and balance between chambers and corridors are related to the target and collected by MAP-Elites after every modification to the target. $f_{inventorial}$ calculates the quality of all inventorial micro-patterns in relation to the current edited room, and $f_{spatial}$ calculates both the quality of spatial micro-patterns and the distribution and composition of tiles in the room described by the meso-patterns.

% is the inventorial quality of the room, which calculates the quality of inventorial micro-patterns

% Where $f_{inventorial}$ is the inventorial quality of the room measure by tiles and $f_{spatial}$ is the spatial distribution of the design related to the distribution of the tiles as patterns in the room~\cite{p9Baldwin2017}. This evaluation is adaptive, meaning that the tile's ratios, patterns, and balance between open areas and corridors are related to the target and collected by MAP-Elites after every modification to the target. 

% EDD uses a single-objective fitness function (shown in~\Cref{eq:fitness_func}) with a FI2Pop genetic algorithm, where fitness is a weighted sum divided equally between (1) the inventorial aspect of the rooms and (2) the spatial distribution of the design. $f_{inventorial}$ is the evaluation of the aggregated and normalized quality of treasures, enemies, and doors (inventorial patterns). $f_{spatial}$ refers to the quality and distribution of chambers, i.e. open areas in the room and corridors (both categorized as spatial patterns), and the meso-patterns that are created within chambers and their quality. Quality refers to the positioning, safety, composition, and relation between patterns. The fitness adapts to the user's current design, automatically informing target ratios and distributions to be used as targets. In-depth evaluation of EDD's fitness function, as well as discussion and explanation of the quality of each inventorial, spatial, and meso pattern, can be found in~\cite{p9Alvarez2018a,Baldwin2017,Baldwin2017a}.

Seven level-design related feature dimensions are implemented in EDD. The designer can pick dimension pairs at a time and change the dimensions' granularity. When the designer changes dimensions, IC MAP-Elites seamlessly reshape the cells and move around the current elites, allowing the designer to switch between features. The seven features are briefly described in table~\ref{tab:dimensions}, although an extensive discussion can be found in~\cite{p9Alvarez2020-ICMAPE}. 

% In this implementation, each chromosome is an $N \times M$ integer array with values between $0$ and $5$ representing the possible tiles, and where $N$ and $M$ are the target level's rows and columns, respectively.

%Each individual also has a list of valid zones that contain the chromosome areas where the algorithm can work on, i.e., what parts of the chromosome can be evolved and used for crossover. The user can explicitly lock tiles of their design that they want to retain in the generated levels, which would then subdivide the level using a Binary Space Partition algorithm and produce the valid and invalid zones~\cite{p9Alvarez2018a}.

%The valid zones are created  based on the Binary Space Partition algorithm-- In this implementation, each chromosome is represented as a list of valid zones  for the evolutionary algorithm,--- In this implementation, each chromosome is an $N \times M$ array where $N$ and $M$ are the rows and columns, respectively, of the target level. ---is a list of valid zones for the evolutionary algorithm


% While MAP-Elites have shown excellent results for the generation of high-performing and diverse individuals in games~\cite{p9fontaine2019covariance}, robotics~\cite{p9Cully2015-qdRobotsAnimals}, and other areas, and have highlighted the benefits from its generated repertoire for more than creating diverse individuals, but also to allow rapid adaptation~\cite{p9Cully2015-qdRobotsAnimals,Gonzalez-Duque2020-DifficultyTrialError}

% We will take the GOLEM project by Pollack and Lipson~\cite{p9Pollack2000-GOLEMProject}, and reformulate its design as a mixed-initiative system to exemplify a use case. The GOLEM project 

% If we would take the GOLEM project by Pollack and Lipson, and would reformulate it as an interactive scenario we can 

% For instance, imagine the scenario where an user is designing a robot piece by piece or some 3d objective?= in a CAD software that uses some type of recommendation, suggestions, assistance, or some type of mixed-initiative system. If we use an evolutionary algorithm for this assistance system, we will be able to explore the space vastly based on some type of objective, which if using \textit{interactive evolution} can be adaptive and provided by the user. Now, if we use MAP-Elites, based on previous results on \textit{"static"} evaluations, it can be estimated that both the range of individuals generated and their performance will be higher than other algorithms~\cite{p9cvt-mape2016,cully2020-multiemitter,Gravina2019-blendingNotionsDiversity}. However, the effects of this interactive loop and "dialogue" between user and algorithm, the continuous development, and ... is unknown. MAP-Elites have been used in interactive situations before~\cite{p9alvarez2019empowering,charity2020baba} and even used for modeling preferences and adapting its evaluation based on that~\cite{p9Alvarez2020-DesignerPreference} with good performance and positive results. Yet, it has not been evaluated on the interactive system rather it has always been evaluated in a static scenario and on its own. Thus, we seek to answer is there any benefit for MAP-Elites and the user for using MAP-Elites in an interactive system?

% uses MAP-Elites 

% MAP-Elites requires minimum change to be adapted to its constrained and interactive variant, where 


% Whenm using IC MAP-Elites, 


% This is done through an adaptive fitness function (based on the designer's design) that adapts the content generation, and by enabling the designer to flexibly change the feature dimensions and the granularity of the cells. It also adapts the usability of MAP-Elites to generate dungeon and adventure levels in an MI-CC system, which gives more control to the designers over non-intuitive parameters and aspects of MAP-Elites, while providing a richer set of high-performing and diverse suggestions.

% When using IC MAP-Elites, the user focu  

% The overarching goal of MI-CC is to collaborate with the user to produce content, either to optimize (i.e., exploit) their current design towards some goal or to foster (i.e., explore) their creativity by surprising them with diverse proposals. By implementing MAP-Elites~\cite{p9Mouret2015} and continuous evolution into EDD, our algorithm can (1) account for the multiple dimensions that a user can be interested in, (2) explore multiple areas of the search space and produce a diverse amount of high-quality suggestions to the user, and (3) still evaluate how interesting and useful the tile distribution is within a specific room. Henceforth, we name the presented approach~\textbf{Interactive Constrained MAP-Elites} (IC MAP-Elites).

% EDD uses a single-objective fitness function (shown in~\Cref{eq:fitness_func}) with a FI2Pop genetic algorithm, where fitness is a weighted sum divided equally between (1) the inventorial aspect of the rooms and (2) the spatial distribution of the design. $f_{inventorial}$ is the evaluation of the aggregated and normalized quality of treasures, enemies, and doors (inventorial patterns). $f_{spatial}$ refers to the quality and distribution of chambers, i.e. open areas in the room and corridors (both categorized as spatial patterns), and the meso-patterns that are created within chambers and their quality. Quality refers to the positioning, safety, composition, and relation between patterns. The fitness adapts to the user's current design, automatically informing target ratios and distributions to be used as targets. In-depth evaluation of EDD's fitness function, as well as discussion and explanation of the quality of each inventorial, spatial, and meso pattern, can be found in~\cite{p9Alvarez2018a,Baldwin2017,Baldwin2017a}.


% % . An in-depth explanation and formulas of EDD's fitness function can be found in~\cite{p9Alvarez2018a, Baldwin2017}Where $f_{inventorial}$ is the evaluation of the aggregated and normalized quality of treasures, enemies, and doors (inventorial patterns). Quality refers to positioning, safety, and the relation between inventorial patterns. $f_{spacial}$ refers to the quality and distribution of chambers i.e. open areas in the room, and corridors, and the meso-patterns that are created within chambers and their quality. 

% % , which relates to the placement of enemies and treasures in relation to doors and target ratios, and (2) the spatial distribution of the design patterns, which refers to the distribution between corridors and rooms, and the meso-patterns that those encompass. The fitness is shown in~\Cref{eq:fitness_func}, where 



% Furthermore, IC MAP-Elites adds interactive and continuous evolution to the Constrained MAP-Elites presented by Khalifa et al.~\cite{p9Khalifa2018}. This is done through an adaptive fitness function (based on the designer's design) that adapts the content generation, and by enabling the designer to flexibly change the feature dimensions and the granularity of the cells. It also adapts the usability of MAP-Elites to generate dungeon and adventure levels in an MI-CC system, which gives more control to the designers over non-intuitive parameters and aspects of MAP-Elites, while providing a richer set of high-performing and diverse suggestions.

% IC MAP-Elites is

% Describe IC MAP-Elites and dimensions. Also fitness (very brief!)

