\section{Discussion}

One good point is that we can filter out the search space! 

RAw results....
Results are that there are unexplored areas of the search space that are hard for the algorithm to find [but exist] and once the user introduces to the algorithm the space with the user’s design, the algorithms is able then to explore the area greatly. Another result is that the changes in the design that does not need to introduce the algorithm to a new area of the search but rather, are able to give the needed building blocks for the crossover and mutation steps to explore a new area. This particularly happens when the designer creates an explicit chamber pattern or dead end, which then the EA can use as next step. Another result is that the designer is not only able to make the search explore new areas instead she can push the search (exploit areas of the search space?) towards areas of the space that are more related for the designer, exploiting individuals around the area of the design. With this knowledge, we could then identify where in the search space the designer is and then dynamically adjust the weight on how cells should be picked to either foster the selection of cells away from the designer to exploit 

Another result is that while the EA is adapting the generated content to the design process of the user, it stably produce and encounter high-performing individuals to be suggested to the designer. This is specially important, given that the fitness function adapts to the current design of the user, adapting to distributions and ratios of patterns and tiles in the room. Thus, as the design changes, the fitness function changes as well, and what was previously evaluated as high-performing could, after several edition steps, be low-performing. The property of QD-algorithms, and specially of MAP-Elites [and variants], on finding quality in the space is therefore, preserved even when the fitness function changes dynamically, stably producing high-performing individuals. 

%%FUTURE WORK

I am thinking if we should or not show/use inner similarity (IS) and similarity (SIM) because (1) the edited room does not "move" in the solution space - since it will always be the reference point (IS and SIM = 1). And (2), those 2 dimensions are dynamic, meaning that the evaluation for low till high scores in the cells, varies as the edited room is changed (since 1 in similarity will be exactly the same as the edited room and 10 steps later as the room change, what was before 1 could be now 0). However, at the same time, this is very interesting to analyze and discuss because it is another property of the interactivity of MAP-Elites, that dimension scores changes dynamically as the target artifact changes? which is uncommon in how MAP-Elites has being evaluated in other works so far.

Finally, what happens when the dimension evaluation is also dynamic? To date, all evaluations done to MAP-Elites [and variants], have been done using non-dynamical dimensions or non-dynamical fitness functions. In EDD, there are 2 dimensions, Inner similarity and similarity, which calculation is directly affected by the current design of the user. Analyzing the data of using these 2 dimensions in a dynamic environment together with the other static dimensions, show that the search space is substantially searched regardless of the other dimensions used, showing a more robust exploration of the space. However, this comes at the expense of reliability and stability of the dimensions, because as the dimension changes dynamically, it also changes the population within cells (as individuals might not be within the cell scores). In the same line, when inspecting the expressive ranges using as second dimension the design goal to create the room e.g. using IS and Leniency when the design goal was to create a high lenient room, the IC-MAP-Elites finds rooms within all the scores, even if for other dimensions which are not IS and Similarity, finding low lenient individuals is close to impossible. This is because of the nature of both dimensions, since if we are aiming to create a highly lenient room, the complete opposite (Similarity = 0), would likely be a room that is with a very low leniency score.
