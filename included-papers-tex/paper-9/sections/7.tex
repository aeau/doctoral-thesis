\section{Conclusion and Future Work}

%This paper analyzes and evaluates the benefits of when dynamically interacting with quality-diversity algorithms,
This paper analyzes and evaluates the benefits of dynamically interacting with quality-diversity algorithms, specifically, the IC MAP-Elites. We have examined the adaptability and stability of MAP-Elites in relation to 21 dimension pairs highlighting different characteristics and properties through different simulated design scenarios. We examined key metrics when exploring the generative space, as depicted in table~\ref{tab:resultsTable} and table~\ref{tab:perDimension}, and conducted four different case studies that highlighted different effects and dynamics with the algorithm.

% of all viable dimension pairs for our MAP-Elites implementation, resorting to four different design scenarios.

%We have examined the adaptability and stability of all viable dimension pairs for MAP-Elites in four different design scenarios, by examining key metrics when exploring the generative space, as depicted in table~\ref{tab:resultsTable}, and through four different case studies highlighting different properties and dynamics of this interaction through ERAs. 

Results from Case 1 indicate that formerly unreached areas in the generative space, due to the seemingly mutually exclusive dimension scores, can be explored after the designer inputs an example in the unreached area into the IC MAP-Elites. This helps the algorithm to generate new individuals and fosters the search for novel individuals. However, the opposite result can also occur, as exemplified in Case 3, where the design moves away from certain areas in the generative space, and the IC MAP-Elites quickly stops producing individuals in those spaces.

% Results from Case 1 indicate that formerly unreached areas in the generative space, due to the seemingly mutually exclusive dimension scores, can be explored after the designer inputs an example in the unreached area into the IC MAP-Elites. 
% % Case 1 highlights this finding and the benefit from the interaction with QD-algorithms, which allows previously thought "impossible areas" to be explored once demonstrated by the human. 
% This helps the algorithm to generate new individuals and fosters the search for novel individuals. However, the opposite result can also occur as exemplified in Case 3, where the design moves away from certain areas in the generative space and the IC MAP-Elites quickly stops producing individuals in those spaces.

%This enables the search and generation of high-performing individuals in those unexplored areas.

%Subtle guidance and filtering is possible without the design moving into unexplored areas within the specific pair of dimensions. 
In Case 2 and 3, the exploration of new areas in the generative space and the generation of novel individuals in previously explored spaces are fostered once that the room is structured and some dead-ends appear. Case 3 also shows how the algorithm can drive the search and generation by moving across the generative space by simple editions. In turn, the algorithm focuses on exploiting areas closely related to the design. In practice, by understanding this dynamic, we could identify where the designer is in the generative space and then either foster the selection of cells farther away to exploit a larger area or focus around the design to provide perhaps more relevant suggestions. Nevertheless, adapting to the design path in the generative space can carry negative consequences. 

Alvarez et al.~\cite{Alvarez2020-ICMAPE} presented an analysis of IC MAP-Elites in a static scenario, wherein average the covered space of MAP-Elites after $5000$ generation was ~52.4\% using pair of dimensions and ~51.7\% using all seven dimensions. Our results show a clear advantage for MAP-Elites when used continuously and interactively with an avg. covered space of ~70.9\%. Further, as seen in Case 3, and discussed by them, when the design remains still in the space, exploration is hindered in MAP-Elites; albeit this is less alarming as the interactive nature of tools such as EDD prevents this issue from emerging during extended periods.

%  as seen in Case 3, and discussed by Alvarez et al.~\shortcite{Alvarez2020-ICMAPE}, when the design remains still in the space, exploration is hindered in MAP-Elites; albeit this is less alarming as the interactive nature of tools such as EDD prevents this issue from emerging during extended periods.

% Alvarez et al.~\cite{Alvarez2020-ICMAPE} presented an analysis of IC MAP-Elites in an static scenario, where in average the covered space of MAP-Elites after $5000$ generation was of ~52.4\% using pair of dimensions and ~51.7\% using all seven dimensions. Our results show a clear advantage for MAP-Elites when used continuously and interactively with an avg. covered space of ~70.9\%. 

%Thus, through adaptability, IC MAP-Elites is able to focus on augmenting its 

In Case 4, we presented an exemption to the previously discussed trade-off. The design remains motionless in the space, but this does not affect to the same extent the algorithm when using Linearity as one of the dimensions. Linearity performed quite stably in all our scenarios regardless of the design traversing around the generative space or not, which indicates that Linearity is more robust and stable and more agnostic and independent from the design. These characteristics are beneficial in certain cases, but based on the results presented in table~\ref{tab:resultsTable} and Case 4, this stability comes at the expense of adaptability and higher fitness scores on average.


%it influences and guides the human designer in the search for adequate solutions.

%In conclusion, IC MAP-Elites has a high degree of adaptability to dynamic environments, adapting the generated content to the design process and design goals, while stably producing and encountering high-performing individuals to suggest to the designer thanks to the inherent properties of QD-algorithm. In our case, this is especially relevant and important given that the fitness function adapts to the current design of the user; thus, adaptability and stability go hand in hand. Furthermore, interacting with MAP-Elites in an MI-CC approach, benefits MAP-Elites by enabling it to explore more of the generative space while producing high-performing individuals; and the human by influencing and guiding the search and generation. 

%Furthermore, by using MAP-elites in a Mixed-Initiative approach, not only will the computer aid the generative part of the two sided process explore more of the search-space, but the human will also be able to seed and guide the search in areas previously not explored by search-algorithm. In this manor, the problems of user-fatigue and inexplicable AI will be alleviated.

%Finally, we purposely excluded examining and analyzing Inner Similarity and Similarity from the results given that these two dimensions change dynamically as the design of the user changes. 
Furthermore, Similarity and Inner Similarity were excluded from the results given that these two dimensions change dynamically as the user's design changes. Preliminary observations from table~\ref{tab:resultsTable}, seem to indicate a vast exploration of the generative space while retaining an overall high fitness ($\bigcirc$) and growth per step ($\bigtriangleup$), especially for Inner Similarity. This points towards a robust dimension, highly adaptable, and with stable growth and performance, but at the expense of general stability and reliability in a mixed-initiative setup. Analysis of these dynamic dimensions is left for future work.

MAP-Elites was evaluated using four design scenarios through simulating design sessions that a designer would do. However, further evaluation is needed with human users to assess the MAP-Elites' properties discussed and analyzed in this paper in-the-wild and evaluate more in-depth the interactive dynamic between humans and algorithms. In this paper, we also present a methodology that extends the work by Smith and Whitehead~\cite{Smith:2010:Expressive-range} to continuously evaluate generative systems using simulated design sessions and aggregated and non-aggregated ERAs, which highlights properties and dynamics of the algorithms. Thus, an interesting future work would be to use such an approach to evaluate and compare MAP-Elites and its properties to co-creative systems that use other approaches and algorithms such as reinforcement learning~\cite{delarosa2020-RLbrushMixedinit,guzdial-lvldsg-aiide-2018} or constraint solving algorithms~\cite{Karth2019-pcgmlDiscriminativeLearning}.

% Several co-creative systems exist that uses other approaches and algorithms such as reinforcement learning~\cite{delarosa2020-RLbrushMixedinit,guzdial-lvldsg-aiide-2018} or constraint solving algorithms~\cite{Karth2019-pcgmlDiscriminativeLearning}; thus, using the methodology presented in this paper to continuously evaluate a generative system
% Along these lines, further evaluation is needed with
% human designers to assess and explore whether IC MAP-Elites
% is beneficial for the MI-CC workflow and interaction

% But point out in the future work to do USER STUDY and Compare with other co-creative systems or at least algorithm for this..

A promising step is to analyze MAP-Elites together with surrogate models that capture the preference, style, and process of designers~\cite{Liapis2013-designerModel,Alvarez2020-DesignerPreference,alvarez2020-designerpersonas}, and how these influence the properties discussed in this paper. For instance, Designer Personas~\cite{alvarez2020-designerpersonas} could be used to explore how the user's design moves through the space, identifying possible paths, and analyzing if key changes, i.e., moving between style clusters, connect to key moments in the MAP-Elites generation.

In conclusion, IC MAP-Elites has a high degree of adaptability to dynamic environments, adapting the generated content to the design process and design goals while stably producing and suggesting high-performing and diverse solutions to the designer. This is made possible due to the inherent properties of QD-algorithms. For MI-CC systems and interactive approaches as in EDD, this is especially relevant and important. The fitness function adapts to the user's current design; thus, adaptability and stability go hand in hand. Furthermore, the deployment of an MI-CC approach in a scenario such as the one presented would benefit both Map-Elites and the human designer. On the one hand, it enables MAP-Elites to explore more of the generative space while producing quality solutions; on the other hand, users would have more control over the suggestions as they influence and guide the search and generation.

% Likewise, our evaluation confirms? asserts the validity? asserts not only the benefits of using MAP-Elites in games and interactive systems, but also validates the different feature dimensions used as usable for level design evaluation and suitability for generating diverse content.