\subsection{Discussion}

% Results from Case 1 indicates that areas that 

% Results from Case 1 indicate that formerly unreached areas in the generative space, due to seemingly mutually exclusive dimension scores, are explored after the design enters such area and we add the example in the population. This helps the algorithm to generate new individuals and fosters the search for novel individuals. Case 3 also shows how the algorithm can drive the search and generation by moving across the generative space with simple edits. In turn, the algorithm focuses on exploiting areas closely related to the design. In practice, by understanding this dynamic, we could identify where the designer is in the generative space and then either foster the selection of cells farther away to exploit a larger area or focus around the design to provide perhaps more relevant suggestions.

% Results from Case 1 indicate that formerly unreached areas in the generative space, due to seemingly mutually exclusive dimension scores, can be explored after manually inputting an example in the unreached area. This helps the algorithm to generate new individuals and fosters the search for novel individuals. However, the opposite result can also occur, as exemplified in Case 3, where the design moves away from certain areas in the generative space, and the IC MAP-Elites quickly stops producing individuals in those spaces.

% In Case 2 and 3, the exploration of new areas and the generation of novel individuals are fostered once the room is more structured. In Case 3,

% In Case 2 and 3, the exploration of new areas in the generative space and the generation of novel individuals in previously explored spaces are fostered once that the room is structured and some dead-ends appear. Case 3 also shows how the algorithm can drive the search and generation by moving across the generative space with simple edits. In turn, the algorithm focuses on exploiting areas closely related to the design. In practice, by understanding this dynamic, we could identify where the designer is in the generative space and then either foster the selection of cells farther away to exploit a larger area or focus around the design to provide perhaps more relevant suggestions. Nevertheless, adapting to the design path in the generative space can carry negative consequences. 

Our evaluation shows that IC MAP-Elites has a high degree of adaptability to dynamic environments, adapting the generated content to the design process and design goals while stably generating high-performing and diverse solutions. For MI-CC systems and interactive approaches as in EDD, this is especially relevant and important. The fitness function adapts to the current design; thus, adaptability and stability go hand in hand. Furthermore, the deployment of an MI-CC approach in a scenario as the ones presented would benefit both Map-Elites and the human designer. On the one hand, it enables MAP-Elites to explore more of the generative space while producing quality solutions. On the other hand, users would have more control over the suggestions as they seamlessly influence and guide the search and generation with their design, in a similar approach to Anderson et al., who explored explicit human guidance in the automated solution search~\citepninth{p9anderson2000-humanguidedSearch}. 

% explores human-AI collaboration by introducing explicit human guidance in the automated solution search~\citepninth{p9anderson2000-humanguidedSearch}, which was further expanded and elaborated as Interactive Evolution by Takagi~\citepninth{p9Takagi2001-InteractiveEvo}.~\citepninth{p9anderson2000-humanguidedSearch}.

We also observe that when using Linearity as a dimension, IC MAP-Elites performed quite stably in all our scenarios regardless of the design traversing around the generative space or not, which indicates that Linearity is more robust and stable and more agnostic and independent from the design. These characteristics are beneficial in certain cases, but based on the results presented in table~\ref{tab:resultsTable}, this stability comes at the expense of adaptability and higher fitness scores.

Furthermore, Alvarez et al.~\citepninth{p9Alvarez2020-ICMAPE} presented an analysis of IC MAP-Elites in a static scenario, where on average the covered space of MAP-Elites after $5000$ generation was 52.4\% using pair of dimensions and 51.7\% using all seven dimensions. Our results show a clear advantage for MAP-Elites when used continuously and interactively with an avg. coverage of 70.9\%. However, when the design remains still in the space defined by the dimensions, exploration is hindered; thus, what dimensions are used and how the design maps to them is crucial.

% However, Alvarez et al.~\citepninth{p9Alvarez2020-ICMAPE} use a similar experiment set up, executing MAP-Elites for 5000 generations in a static environment resulting in limited coverage, which we expect to achieve 

% Disabling adaptation mechanisms would effectively render the algorithm static and in this case make the algorithm agnostic to any edition, which could further support the findings in this paper. In [1] (discussed in the paper), the authors’ experiments use a similar set up, executing IC MAP-Elites for 5000 generations in a static environments. Their results showed constant unexplored areas with 52.4of coverage in average, which as we discuss in our paper, when adapting to changes we increase the average to 70.9. We expect to encounter similar results if disabling adaptive mechanism as the approach would be even more static and agnostic than [1]. Likewise, in [1] they also conduct the experiment using an objective-based GA, which we also expect to encounter similar results. Since we are given 1 extra page, we will add a discussion over this and add it as an interesting future work avenue, as we think that this information is beneficial and could be used to further evidence the benefits of interaction.

Finally, we used and introduced TERAs to analyze the dynamic behaviors in generative systems and algorithms, and observe the effects of changes over time in the expressive range based on the edition steps. We used two variations, non-aggregated TERA, which shows the delta maps of the search, and aggregated TERA, showing the search density and aggregated results. TERAs are generic and could be used with other generative system to evaluate their dynamics by simply defining a pair of features and a step period such as design editions, amount of generations, or whenever a suggestion is applied. TERAs can also be used to spot key and non-trivial steps or changes that have an effect in the search, which can help to understand more in-depth the sensibility of the algorithm and the system. 

% Our evaluation shows that IC MAP-Elites has a high degree of adaptability to dynamic environments, adapting the generated content to the design process and design goals while stably generating high-performing and diverse solutions. This is made possible due to the inherent properties of QD-algorithms. For MI-CC systems and interactive approaches as in EDD, this is especially relevant and important. The fitness function adapts to the current design; thus, adaptability and stability go hand in hand. Furthermore, the deployment of an MI-CC approach in a scenario such as the ones presented would benefit both Map-Elites and the human designer. On the one hand, it enables MAP-Elites to explore more of the generative space while producing quality solutions; on the other hand, users would have more control over the suggestions as they influence and guide the search and generation.


% \subsubsection{How to use the findings in this paper?}

% Here we should describe how other interactive tools can use the findings in this paper, the impact of the presented dynamics (borrow what was the final conclusion)

% IC MAP-Elites shows a high degree of adaptability to dynamic environments, adapting the generated content to the design process and design goals while stably generating high-performing and diverse solutions. This is made possible due to the inherent properties of QD-algorithms. For MI-CC systems and interactive approaches as in EDD, this is especially relevant and important. The fitness function adapts to the current design; thus, adaptability and stability go hand in hand. Furthermore, the deployment of an MI-CC approach in a scenario such as the ones presented would benefit both Map-Elites and the human designer. On the one hand, it enables MAP-Elites to explore more of the generative space while producing quality solutions; on the other hand, users would have more control over the suggestions as they influence and guide the search and generation.



% An interesting future project would be to evaluate and compare 

% Using a similar methodology as the one presented in this paper, an interesting project for future work would be to evaluate and compare MAP-Elites and its properties to other co-creative systems using different algorithms such as reinforcement learning~\citepninth{p9delarosa2020-RLbrushMixedinit,guzdial-lvldsg-aiide-2018} or constraint solving algorithms~\citepninth{p9Karth2019-pcgmlDiscriminativeLearning}.