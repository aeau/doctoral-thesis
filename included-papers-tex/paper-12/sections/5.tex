\subsection{Conclusions and Future Work}

% In this paper, we have proposed \emph{TropeTwist}, a system that uses tropes and trope patterns to describe and construct narrative structures. We demonstrated through three proof-of-concept structures and a limited set of nodes and connections, the expresiveness of the system to define and describe games with diverse genres, mechanics, and game phases. Further, we illustrated how we could generate variations using MAP-Elites to generate a set of quality-diverse structures from the three proof-of-concept structures, outperforming them on our metrics.

In this paper, we have presented \emph{TropeTwist}, a system that interconnects tropes and trope patterns to describe narrative structures. We demonstrated through three proof-of-concept structures the system's expressiveness to describe games with diverse genres and mechanics, and different game phases. Further, we illustrated how we could generate novel structures from the three proof-of-concept structures using MAP-Elites, improving them on our metrics. 

Tropes could be seem as something to avoid when exploring creativity, mainly due to the possibility of showing unoriginal views by definition. However, a set of combined tropes, patterns, and structures could give rise to novel combinations that express the wanted structure. Similarly, identifying, visualizing, and defining the tropes and patterns and doing ``twists'' with them; thus, transforming something typical into atypical is the goal with TropeTwist.

The narrative structures show essential aspects of how the story will develop and lead, and important components such as events, conflicts, or roles. However, to further operationalize these structures, it is necessary other systems that make use of them, such as quest~\citeptwelvth{p12Alvarez2021-questgram,p12ammanabrolu2019-towardQuestGeneration} or plot~\citeptwelvth{p12Ammanabrolu2020-PlotEventsSentences} generators. Another interesting future work would be to explore the multi-faceted nature of games~\citeptwelvth{p12Liapis2019-OrchestratingGames} and combine this type of system with generators that focus on other facets such as level design~\citeptwelvth{p12sarkar2021-dungeonPlatformer,p12alvarez2019empowering} or game mechanics~\citeptwelvth{p12green2021-gamemechanicsAlignment,p12charity2020mech}.

Generating novel narrative structures resulted in interesting variations, but the system could not exploit all the advantages of MAP-Elites. Our results point towards difficulties exploring the space, possibly because \emph{coherence} and \emph{interestingness} are to some extent competing objectives. Therefore, we aim at extending TropeTwist towards a mixed-initiative co-creative system~\citeptwelvth{p12yannakakis2014micc}, and with that, evaluate with human participants. Given that our metrics are dependant on the designed graph; then, we could constantly adapt the content generation and have adaptive models, for instance, of interestingness, based on the user's creation similar to~\citeptwelvth{p12alvarez2019empowering,p12Panagiotis2021-susketch}. 

% The MAP-Elites experiments resulted in interesting variations, but it could not exploit all the advantages of MAP-Elites. 

% The MAP-Elites experiments resulted in interesting variations, but it could not exploit all the advantages of MAP-Elites. Our results point towards difficulties exploring the space, possibly because \textit{coherence} and \textit{interestingness} are to some extent competing objectives. Therefore, we aim at extending TropeTwist towards an interactive system such as a mixed-initiative co-creative system~\citeptwelvth{p12yannakakis2014micc}, and with that, evaluate with human participants. Then, we could constantly adapt the content generation and have adaptive models, for instance, of interestingness, based on the user's creation similar to~\citeptwelvth{p12alvarez2019empowering,Panagiotis2021-susketch}. 

% similar to~\citeptwelvth{p12Cook2014-ARogueDream,hoover2015-audioinspace,ashmore2007-questGeneratedWorld,dormans2011generating}.

% For instance, space and narrative have a special relation and have been intertwined and linked~\citeptwelvth{p12} such as level design~\citeptwelvth{p12sarkar2020-sequentialVAELVLGen,alvarez2019empowering} since both space

% we aim at both, test our system with human participants to analyze the usabality and expresiveness, and

% Our experiments evolving narratives resulted in interesting variations, but we were unable to exploit all the advantages of MAP-Elites. Our results points towards difficulties exploring the space, possibly because \textit{Coherence} and \textit{Interestingness} are to some extent competing objectives. Therefore, we aim at both, test our system with human participants to analyze the usabality and expresiveness, and extend TropeTwist towards an interactive system focusing on moving towards a mixed-initiative co-creative system~\citeptwelvth{p12yannakakis2014micc} similar to~\citeptwelvth{p12kreminski2020-Germinate}. Then, we could constantly adapt the content generation and have adaptive models, for instance, of interestingness, based on the user's creation similar to~\citeptwelvth{p12alvarez2019empowering,Panagiotis2021-susketch}.

% possible difficulties exploring the space here it goes things like~\citeptwelvth{p12kreminski2020-Germinate}

% This paper proposes the use of T

% In this paper, we demonstrated POWER! UNLIMITED POWER!

% Given that \emph{int} combines these three qualities designers might not be interested in highly interesting generated narrative graphs as they could degrade their narrative objective.  Then we want to use Mixed-initiative bra!

% In the introduction we discussed the multi-faceted nature of Games, and it is our goal to continue.

% Really need to point out that while the narrative structures are telling us important aspects of how the story will develop and lead, we need other systems such as a quest generator or plot generator that put the narrative in context and operationalize it.

% Even less than this!

% Given that \emph{int} combines these three qualities designers might not be interested in highly\footnote{Interesting-boring qualities are subjective measurements, thus what is highly interesting in our system might not necessarily be for a designer, which is why (in part) we leverage on the narrative graph created by the designer to measure and evaluate patterns and their quality.} interesting generated narrative graphs as they could degrade their narrative objective.  Then we want to use Mixed-initiative bra!

% Really need to point out that while the narrative structures are telling us important aspects of how the story will develop and lead, we need other systems such as a quest generator or plot generator that put the narrative in context and operationalize it.

%We present a very simple yet interesting proof-of-concept mixed-initiative quest generation tool implemented in and combined with a mixed-initiative level generation tool called EDD. 

%We envision the system as an abstract narrative layer where designers are given the freedom to construct the narrative structure they envision. These structures could be the main structure of the game or media, be a side part of it, or directly aim at how several encounters should be defined.

%in a mixed-iniative approach that enables   a mixed-initiative quest generation in

%Through using Doran and Parberry´s quest structure and production rules, a mixed initiative quest tool was able to be created, to further develop EDD. The experiment related to the expressive range of the quest generator displayed the dominance of certain quests actions, although some actions have a higher dominance rate it was not noticed by the participating testers in the user study. 

%The tool’s mixed initiative approach was positively met by the testing participants, along with the manual creation. However automatic creation and the automatic suggestions received mixed response, mainly because of a random placement of position on the actions. However the overall response of the tool was positive and and a majority of the participants experienced increased creativity while using the tool, many participants expressed the tools usability as a way to gain inspiration, as a solution to inspiration blockages and as a resource-efficient tool for game developers to use.  

%Future work on the quest tool could include improving the automatic generation as well as the user interface and  functionality  for the quest sequence structures for an increased usability. In addition, further work on EDD could be conducted to increase its usability and maintainability. 
