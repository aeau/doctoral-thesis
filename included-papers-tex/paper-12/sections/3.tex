\subsection{Building narrative structures with tropes}

In storytelling, a trope~\citeptwelvth{p12tropesSimpsons} is a convention or figure of speech that the storyteller assumes to be recognizable by the audience. TvTropes is an online wiki that compiles and describes several thousand tropes in many sorts of media~\citeptwelvth{p12tvtropes}. As exemplified by \citeptwelvth{p12periodicTable}, tropes could be interconnected in graph-like structures, called story molecules, to succinctly depict the structure behind a narrative. 

\subsubsection{TropeTwist}

TropeTwist elaborates on the concept of story molecule to represent narratives using graph-like structures of interconnected tropes, called narrative graphs (NG). NGs encode narrative structures in an abstract level that show and define the game's narrative structure and certain abstract properties such as key items, roles, relations, or main events. Table \ref{tab:tropes} shows all the included tropes to be used as nodes. Nodes are depicted (fig~\ref{fig:teaserfig}) with shapes specific to their trope base type: heroes (rectangle), conflicts (diamond), enemies (hexagon), and plot devices (circle). HERO is the base pattern of 5MA, NEO, and SH. ENEMY is the base pattern of EMP, BAD, and DRAKE. PLD is the base pattern of CHK, MCG, and MHQ.

Nodes in a narrative graph are necessarily interconnected by either unidirectional or bidirectional edges (with one or both arrowheads) or by entailment edges (with a single diamond head). Given nodes A and B, A $\diamondsuit$--- B, reads as ``A entails B,'' whereas A $\rightarrow$ B denotes a relationship from A to B, and B $\rightarrow$ A the opposite. A $\leftrightarrow$ B denotes a reflexive relationship between A and B. As an example, HERO $\rightarrow$ CONFLICT $\rightarrow$ EMP denotes a hero who is in conflict against an empire-type enemy, whereas HERO $\leftrightarrow$ CONFLICT denotes a hero who is in conflict with themselves. EMP $\diamondsuit$--- DRAKE $\diamondsuit$--- NEO, denotes an empire that entails a dragon enemy that, once beaten, will lead to the appearance of a chosen one hero, creating some causal links. The system is ambiguous by design. We take advantage of the ambiguity for 1) the generation of new structures (fewer constraints), 2) removing the focus on details by designers to let them focus on the overarching picture, and 3) for other systems to define and interpret these abstract properties. 

Furthermore, interconnecting tropes can give rise to other tropes and patterns, described in the following section. The nodes and their respective trope and pattern were chosen from a subset of tropes in generic categories such as heroes or plot devices. These categories were inspired and chosen based on tropes from TVTropes, the division by James Harris~\citeptwelvth{p12periodicTable}, and previous research such as Propp's morphology~\citeptwelvth{p12propp1975-morphology} or Greimas' actantial model~\citeptwelvth{p12Greimas84-structuralSemantics}. 
% Given the nature of tropes, interconnecting them can arise other tropes, which we describe in the following section.

\subsubsection{Trope Patterns}

%These patterns were identified based on tropes, such as RevP, or as a functional combination of tropes and patterns such as DerP. 

Tropes and interconnected tropes (i.e., subgraphs) give rise to different types of patterns. These patterns can be \textbf{micro-patterns}, encapsulating a single trope node, \textbf{meso-patterns}, often composed by more than one micro-pattern with special meaning, and \textbf{auxiliary patterns}, denoting graph problems. We calculate the relative tropes and patterns' quality within an NG and use this to assess the general quality of the graph. These qualities are proxies for certain characteristics among the defined patterns that are used to evaluate the graphs, but they do not capture any story quality; especially, since we are only defining structures. When generating narrative graphs from a root (explained in section~\nameref{sec:evolvingNarratives}), the quality of a narrative graph becomes relative to the root, henceforth, the ``root graph'' (RG). In the following descriptions, we will use \textbf{EG} referring to the ``evaluated graph'' we are calculating the pattern's quality (the generated individual), and \textbf{RG} to refer to the relative and root graph. When using subscript ``pat,'' we refer to the current pattern that is evaluated. 

%buHowever, these qualities do not capture any story quality; rather they are used as proxys for certain characteristics among patterns 

For most patterns, we calculate three general qualities (indicated when used) that add to the quality of the pattern. $G_{q}(pattern)$ relates to the \textit{Generic} quality of patterns in EG, which calculates the general occurrence of a pattern within EG compared to its occurrence in RG, calculated in eq.~\ref{eq:generic_qual}. $R_{q}(pattern)$ relates to the \textit{Repetition} quality of patterns, which calculates if a trope is unique in EG ($R_{q}(pattern) = 1$) or its ratio among the same base pattern. Lastly, $I_{q}(pattern)$ relates to the \textit{Involvement} quality of patterns in EG, which calculates the amount of associations a pattern has with \textbf{structure patterns}. Involvement means that the pattern is either \textit{source} or \textit{target} in a structure and is calculated as the ratio of structure pattern involvement by the structure pattern count in EG. These three metrics incentivize graphs with similar amount and type of nodes than RG, minimal repetitions, and more involvement. 

\begin{equation}
\label{eq:generic_qual}
    G_{q}(pattern) = 1.0 - |RG_{pat} - EG_{pat}|/\max(RG_{pat}, EG_{pat})
\end{equation}



% RG compared to its ocurreas $G_{q}(pattern) = |RG_{pat} - EG_{pat}|/\max(RG_{pat}, RG_{pat})$. For instance, if a pattern appears as many times in EG as it appears in RG, then $G_{q}(pattern) = 1$, else we calculate the difference using normal distribution centered on the RG's pattern occurrences value.

% $R_{q}(pattern)$ relates to the \textit{Repetition} quality of patterns, which calculates how many of the same trope exists in EG, calculated as $R_{q}(pattern) = pattern_type$. If the trope appears once in EG, then $R_{q}(pattern) = 1$, else we simply set $R_{q}(pattern)$ as the number of repetitions normalized by the number of tropes of the same generic type. 

% Lastly, $I_{q}(pattern)$ relates to the \textit{Involvement} quality of patterns in EG, which calculates the amount of \textbf{structure patterns} that a specific pattern is involved in. Involvement means that the pattern is either \textit{source} or \textit{target} in a structure and is calculated as the amount of structures patterns the trope is involved normalized by the total amount of structure patterns in EG.

\paragraph{Micro-Patterns}

Micro-patterns are the fundamental unit in the system, which aims at categorizing different sets of the individual patterns that are shown in table~\ref{tab:tropes}. Micro-patterns are single nodes and the basic building block that, when interconnected, allows the detection of meso-patterns.

\emph{Structure Pattern (SP)} is any type of trope that would give some structural definition to a narrative, whether this being a conflict, specific act, or a part in a dramatic arc (e.g., climax). Currently, the only type of structure trope is the \textsc{conflict} (CONF) trope, which represents the most basic structural interaction. The quality $SP_{q}$ is calculated as the equally weighted linear combination of:

\begin{equation}
    SP_{q} = G_{q}(SP) + I_{q}(SP)
\end{equation}

\emph{Character Pattern (CP):} are identified as nodes within the narrative that could be either the player, possible ally or enemy NPCs, or simple enemies. In TropeTwist, it is distinguished between heroes and villain patterns, and these are commonly used as \textbf{sources} or \textbf{targets} (or both) of other patterns, and on a few special occasions to denote a relation to another character. The quality $CP_{q}$ is calculated per group (heroes and villains), and it is the equally weighted linear combination of:

\begin{equation}
    CP_{q} = G_{q}(CP) + R_{q}(CP) + I_{q}(CP)
\end{equation}

\emph{Plot Device Pattern (PDP)} is described as the element within the narrative that moves it forward, as a goal, object, or dramatic element to be used or encountered by any of the characters. The quality $PDP_{q}$ is calculated as the equally weighted linear combination of:

\begin{equation}
    PDP_{q} = G_{q}(PDP) + R_{q}(PDP)
\end{equation}


\paragraph{Meso-Patterns}
%~\citeptwelvth{p12dahlskog2014-multimultilevel} 
Meso-patterns are the features that emerge in the narrative from dynamically combining micro-patterns and, on some occasions, these with other meso-patterns. They are always composed of more than one pattern denoting some spatial, semantic, or usability relationship within the narrative graph. We identified a subset of Tropes (extracted from TVTropes~\citeptwelvth{p12tvtropes}) that requires or works as the combination between more fundamental units. For instance, the \textit{reveal pattern} relates to the ``Good all along'' or ``evil all along.''

\emph{Conflict Pattern (ConfP)} is a type of structure pattern composed by a conflict node (Con), a source $s$ node, and a target $t$ node, which are both CPs and usually a hero and a villain or the same character as $s$ and $t$. For instance, the subgraph HERO $\rightarrow$ CONFLICT $\rightarrow$ EMP, indicates that a hero CP has a conflict with an enemy CP. A conflict node can be used indefinitely to define several ConfP. A ConfP is also either~\textsc{explicit} or~\textsc{implicit}. \textsc{Explicit} conflicts are explicitly encoded in the graph and directed from $s$ to $t$ passing through the conflict trope. On the other hand, \textsc{Implicit} conflicts relates to the conflicts from $t$ (or derivatives) to $s$ (or derivatives) that are not encoded in the graph. For instance, the previous example is an \textsc{explicit} conflict from HERO to EMP, and at the same, the EMP has an \textsc{implicit} conflict with the HERO. The quality $ConfP_{q}$ is calculated as the equally weighted linear combination of:

\begin{equation}
    ConfP_{q} = G_{q}(ConfP) + R_{q}(ConfP)% + I_{q}(ConfP)
\end{equation}

%\begin{equation}
%    ConfP_{q} = G_{q}(ConfP) + R_{q}(ConfP) + I_{q}(ConfP)
%\end{equation}

\emph{Derivative Pattern (DerP)} defines a relationship between tropes connected by ``entails'' connections ($\diamondsuit$---). Therefore, a DerP contains a list of patterns connected by entails, named derivatives. DerP starts from a root micro-pattern and continue until no more ``entail'' connections are encountered, effectively establishing a hierarchy from the root derivative to the rest. By design, the patterns within a DerP have a local and temporal order and a causal relationship. For instance, in the subgraph EMP $\diamondsuit$--- DRAKE $\diamondsuit$--- NEO, engaging with the \emph{EMP}, entails both the conflict with \emph{DRAKE} and the appearance of \emph{NEO}. This means that only by overcoming the \emph{DRAKE}, \emph{NEO} will appear - as a new hero or the evolution of another. The quality $DerP_{q}$ is calculated (eq.~\ref{eq:derp}) based on its $G_{q}(DerP)$, the ratio of derivatives within the DerP among the total amount of derivatives across all DerPs in EG ($ratio\theta_{q}$), and the derivatives' diversity. 

%within the pattern to the total amount of derivatives among all DerPs in EG ($bal\theta_{q}$), and the derivatives' diversity.

%\begin{equation}
%\label{eq:derp}
%    DerP_{q} = G_{q}(DerP) + 
%    bal\theta_{q} + 
%    \frac{\sum_{i=0}^{len(DerP_{der})}i_{basepattern}}{len(DerP_{der})}
%\end{equation}

\begin{equation}
\label{eq:derp}
    DerP_{q} = G_{q}(DerP) + 
    %\frac{len(DerP_{der})}{len(EG_{DerP_der})} +
    ratio\theta_{q} + 
    \frac{\sum_{i=0}^{len(DerP_{der})}DerP_{der_ibasepat}}{len(DerP_{der})}
\end{equation}

\emph{Reveal Pattern (RevP)} connects two independent CPs as one, meaning that character A was, in fact, always character B, and vice-versa. This pattern identifies confusion and surprise within an EG, as, for instance, a villain could have been, in fact, ``Good All Along''\footnote{https://tvtropes.org/pmwiki/pmwiki.php/Main/GoodAllAlong}. In practice, a RevP is identified as a villain or hero connected with a unidirectional connection ($\rightarrow$) to another hero or villain. As a consequence, all existing conflicts between them would become \emph{fake}. $RevP_{q}$ is calculated based on its $G_{q}(RevP)$, the number of reveals in EG in relation to characters, and the number of fake conflicts given the specific reveal.

\begin{multline}
    RevP_{q} = G_{q}(RevP) + \frac{len(EG_{RevP})}{len(EG_{CP})} +  \\ \Bigg( 1.0 - \frac{\sum_{i=0}^{len(EG_{conf})}    \begin{cases}
        1,& \text{if } RevP \in x_{i}\\
        0,              & \text{otherwise}
    \end{cases}}{len(EG_{conf})} \Biggl)
\end{multline}

\emph{Active Plot Device Pattern (APD)} operationalize and integrate PDPs within a narrative since PDPs only describe an abstract goal or target. In practice, an \textit{APD} is identified as PDPs that have at least one incoming connection, and optionally, one single outgoing connection. These limitations are added to limit the effect of a PDP within a narrative. $APD_{q}$ is measured based on its $G_{q}(APD)$, and the APD's usability, calculated based on the sum of incoming and outgoing connections divided by half of the nodes in EG depicted as $bal\gamma_{q}$, penalizing APDs for not using all their connections. % that do not use . $bal\gamma_{q}$ penalizes APDs that do not use as many connections as possible.

\begin{equation}
    APD_{q} = G_{q}(APD) + bal\gamma_{q}
\end{equation}

\emph{Plot Points (PP)} are key events within the EG, identified as discrete moments given some pattern. The derivatives within a \textit{DerP}, RevP's source, and PDPs that are \textit{APD} are considered as plot points. $PP_{q}$ is measured based on the number of PPs within RG ($ G_{q}(PP)$), and the number of PPs within EG in relation to the number nodes within it ($Balance_{q}(PP)$).

\begin{equation}
    PP_{q} = G_{q}(PP) + Balance_{q}(PP)
\end{equation}

\emph{Plot Twist (PT)} takes advantage of plot points to identify those that could have a bigger impact on the narrative. In practice, \emph{PTs} consider the source of \textit{RevP}, derivatives from \textit{DerP} that are a different micro-pattern than the root of the DerP (except PDPs), and \textit{APDs} that are connected to other \textit{APDs}. For instance, in the subgraph: EMP $\diamondsuit$--- DRAKE $\diamondsuit$--- NEO, given that NEO is a different micro-pattern than root EMP (Hero and Villain, respectively), NEO will be identified as a \textit{Plot Twist} as it alters the ``natural'' order in the DerP. $PT_{q}$ is based on the number of PTs within RG ($ G_{q}(PT)$), the PT's involvement in EG, and the balance of PTs based on the PPs in EG. Involvement varies depending on the associated pattern to PT. When a PT is associated with a \textit{RevP}, involvement is calculated as how much the structure changes based on that (i.e., how many fake conflicts are created). When it is related to \textit{DerP}, involvement is calculated as how different the pattern is and its order within the derivatives. Finally, when it is related to \textit{APD}, involvement is based on how usable the \textit{APD} is within the narrative based on incoming and outgoing connections.

\begin{equation}
    PT_{q} = G_{q}(PT) + I_{q}(PT_{assoc_pat}) + \frac{len(EG_{pt})}{len(EG_{pp})}
\end{equation}

\paragraph{Auxiliary Patterns}

Auxiliary patterns denote problems in the graph and sub-optimal or impractical nodes and connections within a graph. They are classified into \textit{Nothing}, which are nodes that are not identified as part of a meso-pattern; and \textit{Broken Link}, which are outgoing connections from a node that are not used or do not lead to any pattern.

\subsubsection{Proof-of-Concept}
\label{sec:PoC}

TropeTwist can be used to represent different narrative structures and parts of games. To test and show TropeTwist's expressiveness, we chose to form three different narrative graphs representing different games shown in figure~\ref{fig:teaserfig}, top row: \emph{Zelda: Ocarina of Time} (Zelda:OoT)~\citeptwelvth{p12tloz:oot}, \emph{Zelda: A Link to the Past} (Zelda:LttP)~\citeptwelvth{p12tloz:lttp} - eastern palace, and \emph{Super Mario Bros} (SMB)~\citeptwelvth{p12mario}. They represent different games from different genres (fig.~\ref{fig:teaserfig}.a and \ref{fig:teaserfig}.b are adventure-dungeon games, and \ref{fig:teaserfig}.c is a platformer), and represent different game's phases; in the case of fig.~\ref{fig:teaserfig}.a and \ref{fig:teaserfig}.c, both represent the main structure of the game, while \ref{fig:teaserfig}.b, represents a specific area and sequence of the game.

Figure~\ref{fig:teaserfig}.a represents a simplified overarching narrative structure from Zelda: OoT. The ocarina of time, given by Zelda to Link, is defined as a McGuffin (MCG) that, when collected by ``young link,'' allows him to go forward in time to ``adult link,'' the chosen one (NEO). This, in turn, enables explicit conflicts between hero and enemy characters, which represents the main loop of the game. The structure shows two factions, a set of heroes and the BAD. %Pattern-wise, $HERO \rightarrow SH$ represents an \textbf{RevP}, which in turn, represents an \textbf{PT}; $HERO \rightarrow MCG \rightarrow NEO$ represents an \textbf{APD}, and the rest of elements are interconnected by \textbf{ConfP}.

%The tri-force, which is the main item to collect in the game, is defined as a McGuffin (MCG) 

Figure~\ref{fig:teaserfig}.b represents the structure and plot points from the eastern palace in Zelda: LttP. All palaces in \textit{A Link to the Past} follow a very similar structure and sequence. The HERO's goal is to get the ``Pendant of Courage'' (MCG). However, the MCG derives from ENEMY and BAD, so the HERO must overcome them to achieve his goal. The structure shows a causal and linear narrative that could be used to identify elements that need to appear before others, similar to the work by Dormans and Bakkes~\citeptwelvth{p12dormans2011generating}. %Pattern-wise, ENEMY $\diamondsuit$-- MHQ $\diamondsuit$-- BAD $\diamondsuit$-- MCG represent an \textbf{DerP}, and ENEMY $\diamondsuit$--- MHQ $\diamondsuit$-- BAD, BAD $\diamondsuit$-- MCG $\rightarrow$ HERO, and HERO $\diamondsuit$-- MHQ $\rightarrow$ HERO represent \textbf{APDs}.

Figure~\ref{fig:teaserfig}.c represents the overarching narrative structure of SMB. In SMB, the objective of Mario (HERO) is to rescue Peach (HERO) from Bowser (BAD). To do this, the player goes through a series of platform worlds that always end in a ``Fake Bowser'' (DRAKE). The player must continue until encountering the ``Real Bowser'' (BAD), which then would enable the player to get to their objective (MCG). %Pattern-wise, the game represents a simple narrative structure, where EMP $\diamondsuit$-- DRAKE $\diamondsuit$-- BAD $\diamondsuit$-- MCG, denotes an \textbf{DerP} as a linear increase in difficulty, and HERO $\rightarrow$ MCG $\rightarrow$ HERO denotes an \textbf{APD}.