\graphicspath{{included-papers-tex/paper-12/}}

%\includedPaper{\textsc{paper vi - designer modeling through design style clustering}}{\textsc{paper vi - designer modeling through design style clustering}}{Alberto Alvarez, Jose Font, and Julian Togelius}

\includedPaper{\textsc{paper x - TropeTwist: \\ Trope-based Narrative Structure Generation}}{Paper X - TropeTwist: Trope-based \\ Narrative Structure Generation}{Alberto Alvarez and Jose Font}

\normalfont
\textbf{\textsc{ABSTRACT}}

Games are complex, multi-faceted systems that share common elements and underlying narratives, such as the conflict between a hero and a big bad enemy or pursuing a goal that requires overcoming challenges. However, identifying and describing these elements together is non-trivial as they might differ in certain properties and how players might encounter the narratives. Likewise, generating narratives also pose difficulties when encoding, interpreting, and evaluating them. To address this, we present TropeTwist, a trope-based system that can describe narrative structures in games in a more abstract and generic level, allowing the definition of games' narrative structures and their generation using interconnected tropes, called narrative graphs. To demonstrate the system, we represent the narrative structure of three different games. We use MAP-Elites to generate and evaluate novel quality-diverse narrative graphs encoded as graph grammars, using these three hand-made narrative structures as targets. Both hand-made and generated narrative graphs are evaluated based on their coherence and interestingness, which are improved through evolution.

\textbf{\textsc{PUBLISHED IN}}

Proceedings of the 18th Procedural Content Generation Workshop at the Foundations of Digital Games (FDG), ACM, 2022.

\section*{TROPETWIST: TROPE-BASED NARRATIVE STRUCTURE GENERATION}

\subsection{Introduction}

There exists a plethora of games\footnote{For instance, currently there are more than 68k games in steam \url{https://store.steampowered.com/search/?category1=998}.}, with diverse genres and each containing a different set of gameplay mechanics, audio, level, graphic, and narrative facets. The creation and combination of these facets make game development a hard task, commonly involving a diverse group of developers~\citeptwelvth{p12Blow2004-gamesHard}. Likewise, the generation of these facets in conjunction has been categorized as one of the biggest and most challenging tasks within computational creativity~\citeptwelvth{p12Liapis2014-gameCreativity,p12Liapis2019-OrchestratingGames}. However, games share common elements and underlying narratives, but it is non-trivial how to identify these, how to define and analyze these games structurally, or what type of common underlying structures exist; pointed out as well by~\citeptwelvth{p12aarseth2018-ontologicalMeta,p12vozaru_game_2022}.

Among the different facets, narrative stands out in games as it helps to create meaning, make sense of situations, and make games [stories] recognizable~\citeptwelvth{p12mateas2003-facade,p12Aarseth2012-Narrativetheory,p12kybartas2016survey,p12flodtol2020-WIPMakeSenseDungs}. Narrative structures can be used to describe how an experience or story is to be developed as argued by Barthes~\citeptwelvth{p12Barthes75-introStructNarr}, and to create an abstract representation based on the narrative structure instead of a temporal and partially-ordered sequence of events~\citeptwelvth{p12Szilas2003-structuralModelsIDtension}. Common narrative structures used in many domains are Aristotle's drama structure, which subdivides a story into \textit{exposition}, \textit{climax}, and \textit{resolution} or Propp's analysis on the morphology of russian folktale, which revealed a common structure among them, denoted as Propp's 31 ``narremes''~\citeptwelvth{p12propp1975-morphology}.

This paper presents \emph{TropeTwist}, a preliminar system that uses Tropes~\citeptwelvth{p12Thompson2018-usingTropesNarrativeEvents,p12tropesSimpsons} extracted from TvTropes~\citeptwelvth{p12tvtropes,p12periodicTable} as patterns and fundamental units, which when combined can compose structures further representing other composed tropes. Common narrative structures can be identified and defined using \emph{TropeTwist}. TropeTwist can define generic aspects of a story, leading to the identification of events, roles, and narrative elements, as well as a novel way to form narratives. As a proof-of-concept, we built, analyzed, and described structurally three game examples shown in figure~\ref{fig:teaserfig}, top row.

%We propose graph grammars as indirect encoding of narrative graphs and the use of the Multi-dimensional Archive of Phenotypic Elites (MAP-Elites)~\citeptwelvth{p12Mouret2015-MAPElites} to generate novel variations (shown in figure~\ref{fig:teaserfig}, bottom row) using the proof-of-concept examples as roots. Simultaneously, we propose metrics to evaluate the coherence, cohesion, and interestingness of the resulting narrative graphs, which we use as comparison between hand-made and generated narrative graphs. Our preliminary results show that by using MAP-Elites, it is possible to vary the structure in such a way that more interesting structures appear while retaining coherence. 

We propose graph grammars as indirect encoding of narrative graphs and the use of the Multi-dimensional Archive of Phenotypic Elites (MAP-Elites)~\citeptwelvth{p12Mouret2015-MAPElites} to generate novel variations (shown in figure~\ref{fig:teaserfig}, bottom row) using the proof-of-concept examples as roots. Simultaneously, we propose metrics to evaluate the resulting narrative graphs' coherence, cohesion, and interestingness. Our preliminary results show that we can produce more interesting structures retaining coherence based on our metrics. 

% We propose indirectly encoding these as graph grammars and using the Multi-dimensional Archive of Phenotypic Elites (MAP-Elites)~\citeptwelvth{p12Mouret2015-MAPElites} to generate 

% We Further, using graph grammars and evolutionary algorithms (EA) and these graphs as targets and roots, we show preliminary results on how to generate and evaluate novel variations using the Multi-dimensional Archive of Phenotypic Elites (MAP-Elites)~\citeptwelvth{p12Mouret2015-MAPElites}. We evaluate the proof-of-concept narrative graphs and the generated ones based on their inter



% In our system, narrative structures serve as an abstraction layer, which can be used to create more generic aspects of a story, and at the same time, we can leverage its ambiguity as the structure is not necessarily bounded to specific partially-ordered events. As a proof-of-concept, we built, analyzed, and described structurally three game examples shown in figure~\ref{fig:teaserfig}, top row. Further, using graph grammars and evolutionary algorithms (EA) and these graphs as targets and roots, we show preliminary results on how to generate and evaluate novel variations using the Multi-dimensional Archive of Phenotypic Elites (MAP-Elites)~\citeptwelvth{p12Mouret2015-MAPElites}. We evaluate the proof-of-concept narrative graphs and the generated ones based on their inter

% This could allow a designer to create the structure of what they intend the narrative to be rather than focusing on how it will occur, how to communicate it to the player, or the specific plot points. As a proof-of-concept, we built, analyzed, and described structurally three game examples shown in figure~\ref{fig:teaserfig}, top row. Further, using graph grammars and evolutionary algorithms (EA) and these graphs as targets and roots, we show preliminary results on how to generate and evaluate novel variations.

% In our system, narrative structures serve as an abstraction layer, which can be used to create more generic aspects of a story, and at the same time, we can leverage its ambiguity as the structure is not necessarily bounded to specific partially-ordered events. This could allow a designer to create the structure of what they intend the narrative to be rather than focusing on how it will occur (i.e., syuzhet), how to communicate it to the player (i.e., discourse), or the specific plot points (i.e., fabula). As a proof-of-concept, we built, analyzed, and described structurally three game examples shown in figure~\ref{fig:teaserfig}, top row. Further, using graph grammars and evolutionary algorithms (EA) and these graphs as targets and roots, we show preliminary results on how to generate and evaluate novel variations.

\begin{figure*}
    \centering
    \includegraphics[width=\textwidth]{figures/figure-1_extra8.png}
       \caption{Proof-of-concept narrative structures of existing games (top row) created using TropeTwist with the available nodes (table~\ref{tab:tropes}). Bottom row shows exemplar elites generated with MAP-Elites using the respective top row narrative structure as root. Color matching squares, lines, and triangles denote different meso-patterns in the structures. Squares and triangles are the start and end of a meso-pattern, respectively.}
       \label{fig:teaserfig}
\end{figure*}
\subsection{Related Work}



% Propp~\citeptwelvth{p12propp1975-morphology} analyzed Russian folktales identifying their fundamental structure in 31 steps. His work contributed to the identification of core elements, the proposal of actions and events as \emph{functions} and narrative atoms, and roles that are recurrent within the folktales. Propp emphasized that these 31~\emph{functions} and their arrangement were the structure and what gave meaning to the story discourse. Bremond~\citeptwelvth{p12Bremond80-NarrativePossibilities} took Propp's~\emph{functions} and concepts, and developed them into~\emph{sequences} as temporal and causal structures that consider the narrative possibilities and alternative and dynamic events that result from choice, either from characters or narrators. 

% \subsubsection{Narrative Structures}

Propp~\citeptwelvth{p12propp1975-morphology} analyzed Russian folktales identifying their fundamental structure in 31 steps. His work contributed to the identification of core elements, the proposal of actions and events as \emph{functions} and narrative atoms, and roles that are recurrent within the folktales. Propp emphasized that these 31~\emph{functions} and their arrangement were the structure and what gave meaning to the story discourse. Barthes~\citeptwelvth{p12Barthes75-introStructNarr} proposed three intertwined and progressively integrated levels in narrative work: \emph{functions}, \emph{actions}, and \emph{narration}. His work is characterized by the proposal of fundamental narrative units in the~\emph{function} level to better assess and identify structures in a narrative. Furthermore, Baikadi and Cardona-Rivera~\citeptwelvth{p12Baikadi2012-Narreme} further discuss these fundamental units as~\emph{narremes} encoding narrative state and how they could be combined to narrative structures. Their work, similar to TropeTwist, proposes a graph structure of interconnected~\emph{narremes}. However, they defined narrative axes like Barthes, where each connection between~\emph{narremes} means a change along a narrative axis. In games, the narrative is usually directed by quests, which Aarseth~\citeptwelvth{p12aarseth2005hunt} discusses as a central element in games to make sense of other elements, and which are defined by Yu et al. as a form of structure, dividing the story into achievable rewards and partially ordered set of tasks~\citeptwelvth{p12yu2020quest}.



%Propp~\citeptwelvth{p12propp1975-morphology} analyzed Russian folktales identifying their fundamental structure in 31 steps. His work contributed to the identification of core elements, the proposal of actions and events as \emph{functions} and narrative atoms, and roles that are recurrent within the folktales. Propp emphasized that these 31~\emph{functions} and their arrangement were the structure and what gave meaning to the story discourse. Bremond~\citeptwelvth{p12Bremond80-NarrativePossibilities} took Propp's~\emph{functions} and concepts, and developed them into~\emph{sequences} as temporal and causal structures that consider the narrative possibilities and alternative and dynamic events that result from choice.

%Barthes~\citeptwelvth{p12Barthes75-introStructNarr} proposed three intertwined and progressively integrated levels in narrative work based on the work by Propp and Bremond, \emph{functions}, \emph{actions}, and \emph{narration}. His work is characterized by the proposal of fundamental narrative units in the~\emph{function} level to better assess and identify structures in a narrative, which then combine into hierarchical \emph{sequence} patterns. These fundamental units are:~\emph{functions}, which create the base for the narrative, and~\emph{indices} that expand~\emph{functions} with descriptions and classifiers. Baikadi and Cardona-Rivera~\citeptwelvth{p12Baikadi2012-Narreme} further discuss these fundamental units as~\emph{narremes} encoding narrative state and how they could be combined to narrative structures. Their work, similar to TropeTwist, proposes a graph structure of interconnected~\emph{narremes}. However, they defined narrative axes like Barthes, where each connection between~\emph{narremes} means a change along a narrative axis, emphasizing the temporal and causal relationships in the narrative structure.

% \subsubsection{Narrative and quest generation in Games}

%In games, the narrative is conducted by quests, which Aarseth~\citeptwelvth{p12aarseth2005hunt} discusses as a central element in games to make sense of other elements, and which are defined by Yu et al.~\citeptwelvth{p12yu2020quest} as a form of structure, dividing the story into achievable rewards and partially ordered set of tasks~\citeptwelvth{p12yu2020quest}. Aarseth~\citeptwelvth{p12Aarseth2012-Narrativetheory} discusses the similarities and distinctions between games and other types of narrative media, pointing towards games being composed predominantly of \textit{world}, \textit{objects}, \textit{characters}, and \textit{events}, which, when put together in a variable model, can be used to describe games on these dimensions and find relations and describe them.

%Furthermore, the generation of narratives, stories, and quests using a variety of techniques such as planning algorithms~\citeptwelvth{p12Riedl2006-StoryPlanningCreativity,young2013-plansNarrGen,Horswill2020-Generativetext}, grammars~\citeptwelvth{p12hartsook2011-storyWorlds,Alvarez2021-questgram}, or machine learning~\citeptwelvth{p12tambwekar2019-controllableNeuralStory,vanstegeren2021-gpt2quests}, is a growing and important field within games research and narrative research in general~\citeptwelvth{p12Gervas2009-ComputationalStoryCreativity,kybartas2016survey,yu2020quest,Eladhari2014-storymakinggames}. One typical approach for the generation of content and stories is the use of patterns representing different elements such as level design patterns~\citeptwelvth{p12Alvarez2020-ICMAPE,flodtol2020-WIPMakeSenseDungs}, quest patterns and common quests in games~\citeptwelvth{p12Trenton2010-questpatterns,Doran2011-questsMMORPGs}, or identifying fundamental units and assembling them based on various pre-conditions~\citeptwelvth{p12Kreminski2018-SketchingStorylets,Garbe2019-StoryletsAssembler}. A particular type of pattern is tropes, which are concepts that are recurrently used in transmedia storytelling~\citeptwelvth{p12tropesSimpsons,tvtropes}. Horswill~\citeptwelvth{p12Horswill2016-DearLeaderTrope} focused on constructing an expressive language that could encode plot tropes as story fragments, composing a database of fragments combined sequentially with a planner. Similarly, Thompson et al.~\citeptwelvth{p12Thompson2018-usingTropesNarrativeEvents} used the idea of tropes as story bits where a system would construct valid stories from users' defined story bits with pre-and post-conditions.

Furthermore, the generation of narratives, stories, and quests using a variety of techniques such as planning algorithms~\citeptwelvth{p12Riedl2006-StoryPlanningCreativity,p12young2013-plansNarrGen}, grammars~\citeptwelvth{p12hartsook2011-storyWorlds,p12Alvarez2021-questgram}, or machine learning~\citeptwelvth{p12tambwekar2019-controllableNeuralStory,p12vanstegeren2021-gpt2quests}, is a growing and important field within games research and narrative research in general~\citeptwelvth{p12Gervas2009-ComputationalStoryCreativity,p12kybartas2016survey,p12yu2020quest,p12Eladhari2014-storymakinggames}. One typical approach for the generation of content and stories is the use of patterns representing different elements such as level design patterns~\citeptwelvth{p12alvarez2019empowering,p12flodtol2020-WIPMakeSenseDungs}, quest patterns and common quests in games~\citeptwelvth{p12Trenton2010-questpatterns,p12Doran2011-questsMMORPGs}, or identifying fundamental units and assembling them based on various pre-conditions~\citeptwelvth{p12Kreminski2018-SketchingStorylets,p12Garbe2019-StoryletsAssembler}. A particular type of pattern is tropes, which are concepts that are recurrently used in transmedia storytelling~\citeptwelvth{p12tropesSimpsons,p12tvtropes}. Horswill~\citeptwelvth{p12Horswill2016-DearLeaderTrope} focused on constructing an expressive language that could encode plot tropes as story fragments, composing a database of fragments combined sequentially with a planner. Similarly, Thompson et al.~\citeptwelvth{p12Thompson2018-usingTropesNarrativeEvents} used the idea of tropes as story bits where a system would construct valid stories from users' defined story bits with pre-and post-conditions. TropeTwist uses the idea of tropes for nodes and patterns in structures and encodes and represents these as a graph. \emph{Scheherazade} is a system that can capture narrative structures by encoding and annotating narrative texts, which introduced the Story Intention Graph model, a formal and expressive representation of narratives~\citeptwelvth{p12elson-2012-dramabank}.

%instrel and Scheherazade are two earlier story graph systems. The former, 

%and which introduced the Story Intention Graph model, a formal and expressive representation of narratives~\citeptwelvth{p12elson-2012-dramabank}.

%that is able to capture narrative structures by encoding and annoting narrative texts

%whi story graph system, where 



%Taking this system and discussion a step forward, TropeTwist could be put in-place in a holistic system, where the generic nature of the structures could be useful for other systems such as quest or level design tools~\citeptwelvth{p12Liapis2019-OrchestratingGames}. For instance, the work by Dormans and Bakkes~\citeptwelvth{p12dormans2011generating} generate missions and space using a ``key and lock'' structural idea, which is similar to how we encode TropeTwist. However, their work creates levels after missions rather than orchestrating the facets and intertwining them.

%Ludoscope? 

%Moreover, to encode and generate narrative structures, we use graph grammars and grammar recipes. This approach is similar to how Dormans and Bakkes \citeptwelvth{p12dormans2011generating} generate missions and space using a ``key and lock'' structural idea, which was later employed by Juliani et al.~\citeptwelvth{p12Juliani2019-obstacleTower} to create levels targeted to reinforcement learning agents.

% where users could define their own set with pre-and post-conditions, and through Answer Set Programming, the system constructed valid stories. 



% MAP-Elites has been used... 

% % \textbf{Add the following papers}: The paper from Gozalez-duque et al. where they used intelligent trial and error algorithm to find levels suitable to agents~\citeptwelvth{p12Gonzalez-Duque2020-DifficultyTrialError}. The paper from Schrum et al., where they use interactive evolution to evolve GANs (perhaps fits better as an approach to interactive evolution)~\citeptwelvth{p12Schrum2020-IE_GAN}. The work by Cully and Demiris where they present a framework for QD, but specifically because of the curiosity score they introduced to select cells~\citeptwelvth{p12Cully2018-QDFramework}. The work by Cully where he takes the work by Fontaine et al.~\citeptwelvth{p12fontaine2019covariance} that introduces emitters, and create a set of multi-emitters to generate individuals (or evaluate individuals differently) that outperfom all other map-elites.~\citeptwelvth{p12cully2020-multiemitter}. The work by Justesen et al., where they extended MAP-Elites with \emph{Adaptive Sampling} and \emph{Drifting-Elites}, discussing about domains where fitness functions and behavior evaluations are stochastic i.e., noisy domains~\citeptwelvth{p12Justesen2019-MAPElitesNoisyDomains}. The work by Steckel and Schrum where they use different amounts of data to train GANs and used MAP-Elites to explore the diversity of generated levels (this paper is quite new, I guess that they are submitting to GECCO)~\citeptwelvth{p12steckel2021-MAPElitesGANLodeRunner}. Finally, the work by Gaier et al. where they use Variational Autoencoders to model the highest performing individuals in MAP-Elites and is able to create multiple encodings of the solutions, reducing the dimensionality and direct encoding, as well as reaching high-performance faster~\citeptwelvth{p12Gaier2020-AutomatingRDMAP-Elites}.
% MAP-Elites, a quality-diversity (QD) algorithm, seeks to \emph{illuminate} a \emph{behavior space}
% %Unlike traditional optimization algorithms that aim at finding a single best solution, QD algorithms try 
% by trying to find the best solutions across a feature-dimension grid~\citeptwelvth{p12Mouret2015}.
% % many diverse solutions. The standard version of MAP-Elites creates a grid of $n$ dimensions, where $n$ is the number of different behavioral characteristics (BC). It then tries to find the best solution in each grid cell~\citeptwelvth{p12Mouret2015}.
% Some versions skip the grid in favour of voronoi tesselation to decide which elite individuals to keep in the map~\citeptwelvth{p12cvt-mape2016}. Other works combine the effective adaptive search of Covariance Matrix Adaptation Evolution Strategies with a map of elites, yielding large improvements for real-valued representations in terms of both objective value and number of elites discovered~\citeptwelvth{p12fontaine2019covariance}. 
% %This work was extended by Cully, introducing Multi-Emitter MAP-Elites (ME-MAP-Elites)~\citeptwelvth{p12cully2020-multiemitter}.
% ME-MAP-Elites~\citeptwelvth{p12cully2020-multiemitter} creates a set of emitters to focus on different optimization processes that are active at different generations, generating higher performing and diverse individuals.% than tested baselines.
% %has addressed the core optimization process of MAP-Elites, which in its original form consists of rank selection and non-adaptive mutation and crossover. In particular, the CMA-ME algorithm 
% %MAP-Elites do away with the grid structure, such as CVT-MAP-Elites, which instead uses 

% Constrained MAP-Elites~\citeptwelvth{p12Khalifa2018}
% %was introduced by Khalifa et al.~\citeptwelvth{p12Khalifa2018} in the context of generating bosses for bullet hell games. The key innovation here is to 
% combines divergent search with a two-population approach to constraint satisfaction, taken from the FI-2Pop algorithm~\citeptwelvth{p12Kimbrough2008}. Constrained MAP-Elites has been used as the basis for subsequent experiments, e.g., to find sets of levels implementing diverse game mechanics~\citeptwelvth{p12charity2020mech}. This algorithm was later combined with interactive evolution to yield the aforementioned Interactive Constrained MAP-Elites~\citeptwelvth{p12Alvarez2020-ICMAPE}. %\citeptwelvth{p12alvarez2019empowering} 
% Moreover, MAP-Elites has been shown to be robust at adapting to changing conditions after running the algorithm thanks to its generated behavioral repertoire. This was proposed and tested in the intelligent trial-and-error algorithm~\citeptwelvth{p12Cully2015-qdRobotsAnimals,Gonzalez-Duque2020-DifficultyTrialError}. 
% %In games, this algorithm was used by Gonzalez-Duque et al. to evolve a repertoire of levels suited to different agents and finding levels difficult enough for a different set of agents in a few trials~\citeptwelvth{p12Gonzalez-Duque2020-DifficultyTrialError}.
% Related work extended MAP-Elites with \emph{Adaptive Sampling} and \emph{Drifting-Elites} to be more robust in noisy environments and domains where the fitness and behavior evaluation might be stochastic such as games~\citeptwelvth{p12Justesen2019-MAPElitesNoisyDomains}.

% \subsubsection{Narrative Evaluation}

Moreover, we use graph grammars and grammar recipes to generate structures. This approach is similar to how Dormans and Bakkes \citeptwelvth{p12dormans2011generating} generate missions and space using a ``key and lock'' structural idea. Our approach uses MAP-Elites, a quality-diversity algorithm that uses behavioral dimensions that are orthogonal to the objective function to store diverse individuals in a grid~\citeptwelvth{p12Mouret2015-MAPElites}. Evolutionary algorithms are a popular approach in PCG to generate diverse type of content~\citeptwelvth{p12Togelius2011}, but not as much for narrative content. MAP-Elites have been used to generate content in different game facets such as levels~\citeptwelvth{p12charity2020baba,p12Alvarez2020-ICMAPE}, mechanics~\citeptwelvth{p12charity2020mech}, or enemy behavior~\citeptwelvth{p12Khalifa2018}.

%, which was later employed by Juliani et al.~\citeptwelvth{p12Juliani2019-obstacleTower} to create levels targeted to reinforcement learning agents. 

% The latter, introduced Constrained MAP-Elites combining MAP-Elites with a FI-2Pop algorithm to find \textbf{think on another facet!}. Evolutionary algorithms are a popular  Search based-approaches within procedural content generation mainly

Assessing narratives is a complex and non-trivial task. The goal is to create a narrative that is both syntactically correct (e.g., coherent and consistent) and semantically rich (e.g., novel and interesting)~\citeptwelvth{p12Rowe2009-STORYEVAL,p12Hargood2011-NarrativeCohesion,p12Castricato2021-FabulaStoryCoherenceMeasure}. Perez y Perez and Ortiz~\citeptwelvth{p12Perez2013-AutomaticModelInterestingness} proposed a model to evaluate interestingness based on novelty and correct story recount, with emphasis on the story's opening, closure, and dramatic tensions. Szilas et al.~\citeptwelvth{p12Szilas2016-QualQuantInterestingness} discuss interestingness as a paradox dramatic situation with obstacles and conflicts, albeit applicable to stories as successive events. Yet, to approach subjective measurements such as interestingness, most research turns towards having human evaluation~\citeptwelvth{p12Kreminski2019-StorySifter,p12Lankoski2013-StoryConsistencyInteresting} or using such to form human models to be used as surrogate models~\citeptwelvth{p12Sharna2007-PreferenceModelingStories,p12Li2010-plannerPlotAdapt}.
%, which is the future work for TropeTwist, implementing it in a mixed-initiative system~\citeptwelvth{p12yannakakis2014micc}. 

% MAP-Elites has been used...

% In TropeTwist, we use trope patterns to evaluate interestingness and coherence.

%\begin{table}[ht]
% \centering
\caption{Participants’ most requested features} \label{p1tab:demands}
% \resizebox{\paperwidth}{!}{\begin{minipage}{\paperwidth}
\resizebox{0.8\textwidth}{!}{
% \begin{tabularx}{\textwidth}{|p{0.2\textwidth}|p{0.99\textwidth}|}
\begin{tabularx}{\textwidth}{|p{0.2\textwidth}|p{0.99\textwidth}|}
\cline{1-2}
Feature                                 & Description                                                                                                                                                                                                                                                                                                                                                                                                         \\\cline{1-2}
Design patterns                   & Their visualization and accuracy should be improved. Other than acting as visual guide for map information, they should be used to help generate rooms as well. They should also be available for the entire dungeon.                                                                                                                                                                                 \\\cline{1-2}
Parameters                                  & They need to have more information about the specific room, and have better visualization in order to make the designer trust their accuracy more. The parameters should also consider the entire dungeon as a whole in different terms such as difficulty and balance.
\\\cline{1-2}
Generated suggestions               & In general, the participants want more variety and control in the generation of suggestions using different types of parameters e.g. their degree of similarity and fitness functions.                                                       \\\cline{1-2}
Redefined feasibility                           & Eddy 3.0’s definition of feasibility should be revised which considers the whole dungeon and its connected rooms.                                                                                                      \\\cline{1-2}
World View                      & The World View should be revised and enhanced with more special features which would encourage users to visit it more.                                                    \\\cline{1-2}
World grid                                       & The computation of the whole dungeon should be improved. It should have an option to define a starting point. Its definition of entrance doors should be improved, as well as the calculation of distances of tile types.                                                                                                                                                                       \\\cline{1-2}
Version control & Some participants want to preview suggestions within the Room View to help their judgment and the ability to save suggestions for later use. They also want to revert to old designs in case they have second thoughts.                                                                                                     \\\cline{1-2}
Templates                             & Some participants want the ability to save their own manual designs to be carried over to other grids.                                                                                                                                                                                                                                     \\\cline{1-2}
Automated assistance                                  & The participants in general welcome a bit more automated assistance when doing manual designs, which can reduce clicking around the program. It should also not be too invasive for the designer. 
	\\ \cline{1-2}
\end{tabularx}
}
% \end{minipage}}
\end{table}

% \begin{table}[h]
% \centering
% \caption{Developed game based features used as dimensions in the~\acrlong{icmape}}\label{table:mape-dimensions}
% % \resizebox{\textwidth}
% % \resizebox{\textwidth}
% \begin{tabularx}{\textwidth}{|c|X|}
% \cline{1-2}
% \rule{0pt}{12pt}
% Feature&Definition\\ \cline{1-2}
% % \\[-6pt]
% Similarity & Refers to the aesthetic (tile-by-tile) similarity between a room and the current designer's design.\\ \cline{1-2}
% Inner Similarity & Refers to the similarity of the sparsity and density of the different tile types of a room designer's current design.\\ \cline{1-2}
% Symmetry & Refers to the aesthetic symmetry of a room.\\ \cline{1-2}
% Leniency & Refers to how challenging rooms are; calculated based on the position of enemies and balance between enemies and treasures.\\ \cline{1-2}
% Linearity & Refers to the amount of paths connecting doors within a room; calculated based on how many spatial patterns are traversed.\\ \cline{1-2}
% \#Meso-Patterns & Refers to the number of meso-patterns that exist within a room, normalized by an estimated maximum number based on the room's size and the minimum chamber size.\\ \cline{1-2}
% \#Spatial-Patterns & Refers to the number of spatial-patterns that exist within a room, which can be chambers, corridor, turns, junctions, and intersections.\\ \cline{1-2}
% \end{tabularx}
% \end{table}
% Please add the following required packages to your document preamble:
% \usepackage{graphicx}
\begin{table}[]
\caption{Tropes included and used in TropeTwist, extracted from~\cite{tvtropes}.}
\resizebox{\linewidth}{!}{%
\begin{tabular}{l|l|l}
Name                       & \textit{Symbol} & \textit{Definition}                                                                  \\ \hline
Hero                       & HERO            & A protagonist character.                                                             \\
Five-man band              & 5MA             & \begin{tabular}[t]{@{}l@{}}Group composed by up-to-five \\ archetypical characters.\end{tabular}                            \\
The chosen one             & NEO             & Specific hero chosen as the one.                                                     \\
Superhero                  & SH              & Specific hero with unique abilities.                                                 \\
Conflict                   & CONF             & \begin{tabular}[t]{@{}l@{}}Non-specific problem to overcome \\ between characters.\end{tabular}                                   \\
Enemy                      & ENEMY           & A nemesis to the hero.                                                               \\
Empire                     & EMP             & \begin{tabular}[t]{@{}l@{}}Collective enemy with the \\ambition of conquering the world.\end{tabular}                               \\
Big bad                    & BAD             & \begin{tabular}[t]{@{}l@{}}Specific enemy, which is the \\ ultimate cause for all the bad.\end{tabular}                               \\
Dragon                     & DRAKE             & \begin{tabular}[t]{@{}l@{}}Specific enemy, which is the right \\ hand of BAD.  \end{tabular}                                          \\
Plot device                & PLD             & \begin{tabular}[t]{@{}l@{}}A feature or element that drives \\ the plot forward. \end{tabular}                                        \\
Chekhov's gun              & CHK             & PLD relevant to the story \\
MacGuffin                  & MCG             & \begin{tabular}[t]{@{}l@{}}PLD with irrelevant nature to \\ drive the story.\end{tabular}          \\
May help in quest & MHQ             & PLD important to resolve a conflict.    
\end{tabular}%
}
\label{tab:tropes}
\end{table}

%  thus far in Story Designer depicted by their name, symbol, and definition, extracted from \cite{tvtropes}.
\subsection{Building narrative structures with tropes}

In storytelling, a trope~\cite{p12tropesSimpsons} is a convention or figure of speech that the storyteller assumes to be recognizable by the audience. TvTropes is an online wiki that compiles and describes several thousand tropes in many sorts of media~\cite{p12tvtropes}. As exemplified by \cite{p12periodicTable}, tropes could be interconnected in graph-like structures, called story molecules, to succinctly depict the structure behind a narrative. 

\subsubsection{TropeTwist}

TropeTwist elaborates on the concept of story molecule to represent narratives using graph-like structures of interconnected tropes, called narrative graphs (NG). NGs encode narrative structures in an abstract level that show and define the game's narrative structure and certain abstract properties such as key items, roles, relations, or main events. Table \ref{tab:tropes} shows all the included tropes to be used as nodes. Nodes are depicted (fig~\ref{fig:teaserfig}) with shapes specific to their trope base type: heroes (rectangle), conflicts (diamond), enemies (hexagon), and plot devices (circle). HERO is the base pattern of 5MA, NEO, and SH. ENEMY is the base pattern of EMP, BAD, and DRAKE. PLD is the base pattern of CHK, MCG, and MHQ.

Nodes in a narrative graph are necessarily interconnected by either unidirectional or bidirectional edges (with one or both arrowheads) or by entailment edges (with a single diamond head). Given nodes A and B, A $\diamondsuit$--- B, reads as ``A entails B,'' whereas A $\rightarrow$ B denotes a relationship from A to B, and B $\rightarrow$ A the opposite. A $\leftrightarrow$ B denotes a reflexive relationship between A and B. As an example, HERO $\rightarrow$ CONFLICT $\rightarrow$ EMP denotes a hero who is in conflict against an empire-type enemy, whereas HERO $\leftrightarrow$ CONFLICT denotes a hero who is in conflict with themselves. EMP $\diamondsuit$--- DRAKE $\diamondsuit$--- NEO, denotes an empire that entails a dragon enemy that, once beaten, will lead to the appearance of a chosen one hero, creating some causal links. The system is ambiguous by design. We take advantage of the ambiguity for 1) the generation of new structures (fewer constraints), 2) removing the focus on details by designers to let them focus on the overarching picture, and 3) for other systems to define and interpret these abstract properties. 

Furthermore, interconnecting tropes can give rise to other tropes and patterns, described in the following section. The nodes and their respective trope and pattern were chosen from a subset of tropes in generic categories such as heroes or plot devices. These categories were inspired and chosen based on tropes from TVTropes, the division by James Harris~\cite{p12periodicTable}, and previous research such as Propp's morphology~\cite{p12propp1975-morphology} or Greimas' actantial model~\cite{p12Greimas84-structuralSemantics}. 
% Given the nature of tropes, interconnecting them can arise other tropes, which we describe in the following section.

\subsubsection{Trope Patterns}

%These patterns were identified based on tropes, such as RevP, or as a functional combination of tropes and patterns such as DerP. 

Tropes and interconnected tropes (i.e., subgraphs) give rise to different types of patterns. These patterns can be \textbf{micro-patterns}, encapsulating a single trope node, \textbf{meso-patterns}, often composed by more than one micro-pattern with special meaning, and \textbf{auxiliary patterns}, denoting graph problems. We calculate the relative tropes and patterns' quality within an NG and use this to assess the general quality of the graph. These qualities are proxies for certain characteristics among the defined patterns that are used to evaluate the graphs, but they do not capture any story quality; especially, since we are only defining structures. When generating narrative graphs from a root (explained in section~\ref{sec:evolvingNarratives}), the quality of a narrative graph becomes relative to the root, henceforth, the ``root graph'' (RG). In the following descriptions, we will use \textbf{EG} referring to the ``evaluated graph'' we are calculating the pattern's quality (the generated individual), and \textbf{RG} to refer to the relative and root graph. When using subscript ``pat,'' we refer to the current pattern that is evaluated. 

%buHowever, these qualities do not capture any story quality; rather they are used as proxys for certain characteristics among patterns 

For most patterns, we calculate three general qualities (indicated when used) that add to the quality of the pattern. $G_{q}(pattern)$ relates to the \textit{Generic} quality of patterns in EG, which calculates the general occurrence of a pattern within EG compared to its occurrence in RG, calculated in eq.~\ref{eq:generic_qual}. $R_{q}(pattern)$ relates to the \textit{Repetition} quality of patterns, which calculates if a trope is unique in EG ($R_{q}(pattern) = 1$) or its ratio among the same base pattern. Lastly, $I_{q}(pattern)$ relates to the \textit{Involvement} quality of patterns in EG, which calculates the amount of associations a pattern has with \textbf{structure patterns}. Involvement means that the pattern is either \textit{source} or \textit{target} in a structure and is calculated as the ratio of structure pattern involvement by the structure pattern count in EG. These three metrics incentivize graphs with similar amount and type of nodes than RG, minimal repetitions, and more involvement. 

\begin{equation}
\label{eq:generic_qual}
    G_{q}(pattern) = 1.0 - |RG_{pat} - EG_{pat}|/\max(RG_{pat}, EG_{pat})
\end{equation}



% RG compared to its ocurreas $G_{q}(pattern) = |RG_{pat} - EG_{pat}|/\max(RG_{pat}, RG_{pat})$. For instance, if a pattern appears as many times in EG as it appears in RG, then $G_{q}(pattern) = 1$, else we calculate the difference using normal distribution centered on the RG's pattern occurrences value.

% $R_{q}(pattern)$ relates to the \textit{Repetition} quality of patterns, which calculates how many of the same trope exists in EG, calculated as $R_{q}(pattern) = pattern_type$. If the trope appears once in EG, then $R_{q}(pattern) = 1$, else we simply set $R_{q}(pattern)$ as the number of repetitions normalized by the number of tropes of the same generic type. 

% Lastly, $I_{q}(pattern)$ relates to the \textit{Involvement} quality of patterns in EG, which calculates the amount of \textbf{structure patterns} that a specific pattern is involved in. Involvement means that the pattern is either \textit{source} or \textit{target} in a structure and is calculated as the amount of structures patterns the trope is involved normalized by the total amount of structure patterns in EG.

\paragraph{Micro-Patterns}

Micro-patterns are the fundamental unit in the system, which aims at categorizing different sets of the individual patterns that are shown in table~\ref{tab:tropes}. Micro-patterns are single nodes and the basic building block that, when interconnected, allows the detection of meso-patterns.

\emph{Structure Pattern (SP)} is any type of trope that would give some structural definition to a narrative, whether this being a conflict, specific act, or a part in a dramatic arc (e.g., climax). Currently, the only type of structure trope is the \textsc{conflict} (CONF) trope, which represents the most basic structural interaction. The quality $SP_{q}$ is calculated as the equally weighted linear combination of:

\begin{equation}
    SP_{q} = G_{q}(SP) + I_{q}(SP)
\end{equation}

\emph{Character Pattern (CP):} are identified as nodes within the narrative that could be either the player, possible ally or enemy NPCs, or simple enemies. In TropeTwist, it is distinguished between heroes and villain patterns, and these are commonly used as \textbf{sources} or \textbf{targets} (or both) of other patterns, and on a few special occasions to denote a relation to another character. The quality $CP_{q}$ is calculated per group (heroes and villains), and it is the equally weighted linear combination of:

\begin{equation}
    CP_{q} = G_{q}(CP) + R_{q}(CP) + I_{q}(CP)
\end{equation}

\emph{Plot Device Pattern (PDP)} is described as the element within the narrative that moves it forward, as a goal, object, or dramatic element to be used or encountered by any of the characters. The quality $PDP_{q}$ is calculated as the equally weighted linear combination of:

\begin{equation}
    PDP_{q} = G_{q}(PDP) + R_{q}(PDP)
\end{equation}


\paragraph{Meso-Patterns}
%~\cite{p12dahlskog2014-multimultilevel} 
Meso-patterns are the features that emerge in the narrative from dynamically combining micro-patterns and, on some occasions, these with other meso-patterns. They are always composed of more than one pattern denoting some spatial, semantic, or usability relationship within the narrative graph. We identified a subset of Tropes (extracted from TVTropes~\cite{p12tvtropes}) that requires or works as the combination between more fundamental units. For instance, the \textit{reveal pattern} relates to the ``Good all along'' or ``evil all along.''

\emph{Conflict Pattern (ConfP)} is a type of structure pattern composed by a conflict node (Con), a source $s$ node, and a target $t$ node, which are both CPs and usually a hero and a villain or the same character as $s$ and $t$. For instance, the subgraph HERO $\rightarrow$ CONFLICT $\rightarrow$ EMP, indicates that a hero CP has a conflict with an enemy CP. A conflict node can be used indefinitely to define several ConfP. A ConfP is also either~\textsc{explicit} or~\textsc{implicit}. \textsc{Explicit} conflicts are explicitly encoded in the graph and directed from $s$ to $t$ passing through the conflict trope. On the other hand, \textsc{Implicit} conflicts relates to the conflicts from $t$ (or derivatives) to $s$ (or derivatives) that are not encoded in the graph. For instance, the previous example is an \textsc{explicit} conflict from HERO to EMP, and at the same, the EMP has an \textsc{implicit} conflict with the HERO. The quality $ConfP_{q}$ is calculated as the equally weighted linear combination of:

\begin{equation}
    ConfP_{q} = G_{q}(ConfP) + R_{q}(ConfP)% + I_{q}(ConfP)
\end{equation}

%\begin{equation}
%    ConfP_{q} = G_{q}(ConfP) + R_{q}(ConfP) + I_{q}(ConfP)
%\end{equation}

\emph{Derivative Pattern (DerP)} defines a relationship between tropes connected by ``entails'' connections ($\diamondsuit$---). Therefore, a DerP contains a list of patterns connected by entails, named derivatives. DerP starts from a root micro-pattern and continue until no more ``entail'' connections are encountered, effectively establishing a hierarchy from the root derivative to the rest. By design, the patterns within a DerP have a local and temporal order and a causal relationship. For instance, in the subgraph EMP $\diamondsuit$--- DRAKE $\diamondsuit$--- NEO, engaging with the \emph{EMP}, entails both the conflict with \emph{DRAKE} and the appearance of \emph{NEO}. This means that only by overcoming the \emph{DRAKE}, \emph{NEO} will appear - as a new hero or the evolution of another. The quality $DerP_{q}$ is calculated (eq.~\ref{eq:derp}) based on its $G_{q}(DerP)$, the ratio of derivatives within the DerP among the total amount of derivatives across all DerPs in EG ($ratio\theta_{q}$), and the derivatives' diversity. 

%within the pattern to the total amount of derivatives among all DerPs in EG ($bal\theta_{q}$), and the derivatives' diversity.

%\begin{equation}
%\label{eq:derp}
%    DerP_{q} = G_{q}(DerP) + 
%    bal\theta_{q} + 
%    \frac{\sum_{i=0}^{len(DerP_{der})}i_{basepattern}}{len(DerP_{der})}
%\end{equation}

\begin{equation}
\label{eq:derp}
    DerP_{q} = G_{q}(DerP) + 
    %\frac{len(DerP_{der})}{len(EG_{DerP_der})} +
    ratio\theta_{q} + 
    \frac{\sum_{i=0}^{len(DerP_{der})}DerP_{der_ibasepat}}{len(DerP_{der})}
\end{equation}

\emph{Reveal Pattern (RevP)} connects two independent CPs as one, meaning that character A was, in fact, always character B, and vice-versa. This pattern identifies confusion and surprise within an EG, as, for instance, a villain could have been, in fact, ``Good All Along''\footnote{https://tvtropes.org/pmwiki/pmwiki.php/Main/GoodAllAlong}. In practice, a RevP is identified as a villain or hero connected with a unidirectional connection ($\rightarrow$) to another hero or villain. As a consequence, all existing conflicts between them would become \emph{fake}. $RevP_{q}$ is calculated based on its $G_{q}(RevP)$, the number of reveals in EG in relation to characters, and the number of fake conflicts given the specific reveal.

\begin{multline}
    RevP_{q} = G_{q}(RevP) + \frac{len(EG_{RevP})}{len(EG_{CP})} +  \\ \Bigg( 1.0 - \frac{\sum_{i=0}^{len(EG_{conf})}    \begin{cases}
        1,& \text{if } RevP \in x_{i}\\
        0,              & \text{otherwise}
    \end{cases}}{len(EG_{conf})} \Biggl)
\end{multline}

\emph{Active Plot Device Pattern (APD)} operationalize and integrate PDPs within a narrative since PDPs only describe an abstract goal or target. In practice, an \textit{APD} is identified as PDPs that have at least one incoming connection, and optionally, one single outgoing connection. These limitations are added to limit the effect of a PDP within a narrative. $APD_{q}$ is measured based on its $G_{q}(APD)$, and the APD's usability, calculated based on the sum of incoming and outgoing connections divided by half of the nodes in EG depicted as $bal\gamma_{q}$, penalizing APDs for not using all their connections. % that do not use . $bal\gamma_{q}$ penalizes APDs that do not use as many connections as possible.

\begin{equation}
    APD_{q} = G_{q}(APD) + bal\gamma_{q}
\end{equation}

\emph{Plot Points (PP)} are key events within the EG, identified as discrete moments given some pattern. The derivatives within a \textit{DerP}, RevP's source, and PDPs that are \textit{APD} are considered as plot points. $PP_{q}$ is measured based on the number of PPs within RG ($ G_{q}(PP)$), and the number of PPs within EG in relation to the number nodes within it ($Balance_{q}(PP)$).

\begin{equation}
    PP_{q} = G_{q}(PP) + Balance_{q}(PP)
\end{equation}

\emph{Plot Twist (PT)} takes advantage of plot points to identify those that could have a bigger impact on the narrative. In practice, \emph{PTs} consider the source of \textit{RevP}, derivatives from \textit{DerP} that are a different micro-pattern than the root of the DerP (except PDPs), and \textit{APDs} that are connected to other \textit{APDs}. For instance, in the subgraph: EMP $\diamondsuit$--- DRAKE $\diamondsuit$--- NEO, given that NEO is a different micro-pattern than root EMP (Hero and Villain, respectively), NEO will be identified as a \textit{Plot Twist} as it alters the ``natural'' order in the DerP. $PT_{q}$ is based on the number of PTs within RG ($ G_{q}(PT)$), the PT's involvement in EG, and the balance of PTs based on the PPs in EG. Involvement varies depending on the associated pattern to PT. When a PT is associated with a \textit{RevP}, involvement is calculated as how much the structure changes based on that (i.e., how many fake conflicts are created). When it is related to \textit{DerP}, involvement is calculated as how different the pattern is and its order within the derivatives. Finally, when it is related to \textit{APD}, involvement is based on how usable the \textit{APD} is within the narrative based on incoming and outgoing connections.

\begin{equation}
    PT_{q} = G_{q}(PT) + I_{q}(PT_{assoc_pat}) + \frac{len(EG_{pt})}{len(EG_{pp})}
\end{equation}

\paragraph{Auxiliary Patterns}

Auxiliary patterns denote problems in the graph and sub-optimal or impractical nodes and connections within a graph. They are classified into \textit{Nothing}, which are nodes that are not identified as part of a meso-pattern; and \textit{Broken Link}, which are outgoing connections from a node that are not used or do not lead to any pattern.

\subsubsection{Proof-of-Concept}
\label{sec:PoC}

TropeTwist can be used to represent different narrative structures and parts of games. To test and show TropeTwist's expressiveness, we chose to form three different narrative graphs representing different games shown in figure~\ref{fig:teaserfig}, top row: \emph{Zelda: Ocarina of Time} (Zelda:OoT)~\cite{p12tloz:oot}, \emph{Zelda: A Link to the Past} (Zelda:LttP)~\cite{p12tloz:lttp} - eastern palace, and \emph{Super Mario Bros} (SMB)~\cite{p12mario}. They represent different games from different genres (fig.~\ref{fig:teaserfig}.a and \ref{fig:teaserfig}.b are adventure-dungeon games, and \ref{fig:teaserfig}.c is a platformer), and represent different game's phases; in the case of fig.~\ref{fig:teaserfig}.a and \ref{fig:teaserfig}.c, both represent the main structure of the game, while \ref{fig:teaserfig}.b, represents a specific area and sequence of the game.

Figure~\ref{fig:teaserfig}.a represents a simplified overarching narrative structure from Zelda: OoT. The ocarina of time, given by Zelda to Link, is defined as a McGuffin (MCG) that, when collected by ``young link,'' allows him to go forward in time to ``adult link,'' the chosen one (NEO). This, in turn, enables explicit conflicts between hero and enemy characters, which represents the main loop of the game. The structure shows two factions, a set of heroes and the BAD. %Pattern-wise, $HERO \rightarrow SH$ represents an \textbf{RevP}, which in turn, represents an \textbf{PT}; $HERO \rightarrow MCG \rightarrow NEO$ represents an \textbf{APD}, and the rest of elements are interconnected by \textbf{ConfP}.

%The tri-force, which is the main item to collect in the game, is defined as a McGuffin (MCG) 

Figure~\ref{fig:teaserfig}.b represents the structure and plot points from the eastern palace in Zelda: LttP. All palaces in \textit{A Link to the Past} follow a very similar structure and sequence. The HERO's goal is to get the ``Pendant of Courage'' (MCG). However, the MCG derives from ENEMY and BAD, so the HERO must overcome them to achieve his goal. The structure shows a causal and linear narrative that could be used to identify elements that need to appear before others, similar to the work by Dormans and Bakkes~\cite{p12dormans2011generating}. %Pattern-wise, ENEMY $\diamondsuit$-- MHQ $\diamondsuit$-- BAD $\diamondsuit$-- MCG represent an \textbf{DerP}, and ENEMY $\diamondsuit$--- MHQ $\diamondsuit$-- BAD, BAD $\diamondsuit$-- MCG $\rightarrow$ HERO, and HERO $\diamondsuit$-- MHQ $\rightarrow$ HERO represent \textbf{APDs}.

Figure~\ref{fig:teaserfig}.c represents the overarching narrative structure of SMB. In SMB, the objective of Mario (HERO) is to rescue Peach (HERO) from Bowser (BAD). To do this, the player goes through a series of platform worlds that always end in a ``Fake Bowser'' (DRAKE). The player must continue until encountering the ``Real Bowser'' (BAD), which then would enable the player to get to their objective (MCG). %Pattern-wise, the game represents a simple narrative structure, where EMP $\diamondsuit$-- DRAKE $\diamondsuit$-- BAD $\diamondsuit$-- MCG, denotes an \textbf{DerP} as a linear increase in difficulty, and HERO $\rightarrow$ MCG $\rightarrow$ HERO denotes an \textbf{APD}.
% % Please add the following required packages to your document preamble:
% \usepackage{graphicx}
\begin{table}[]
\centering
\begin{tabular}{|l|lll|}
\hline
        & AIv1       & AIv2        & AIv3        \\ \hline
Leniency        & 0.56±0.07  & 0.62±0.09   & 0.57±0.08   \\
Linearity        & 0.91±0.02  & 0.92±0.02   & 0.91±0.01   \\
MesoPat       & 0.15±0.05  & 0.13±0.07   & 0.12±0.05   \\
SpatialPat    & 0.35±0.1   & 0.41±0.11   & 0.34±0.09   \\
Symmetry   & 0.43±0.11  & 0.35±0.18   & 0.35±0.12   \\
$W_{dens}$ & 0.27±0.09  & 0.26±0.08   & 0.21±0.05   \\
$W_{spar}$ & 0.21±0.05  & 0.19±0.03   & 0.15±0.01   \\
$E_{dens}$ & 0.24±0.07  & 0.27±0.06   & 0.3±0.06    \\
$E_{spar}$ & 0.22±0.05  & 0.32±0.05   & 0.35±0.06   \\
$T_{dens}$ & 0.37±0.13  & 0.28±0.09   & 0.34±0.07   \\
$T_{spar}$ & 0.36±0.11  & 0.3±0.1     & 0.37±0.05   \\ \hline
Steps      & 39.25±6.38 & 84.31±14.85 & 76.75±17.02 \\ \hline
\end{tabular}
\caption{Summary of the created rooms filtered by the AI version used. All values are the average of all the created rooms using the specific AI version. The first five values relates to the MAP-Elites dimensions, then the fitness of the rooms, the density and sparsity values for wall (W), enemies (E), and treasures (T), and finally the avg. steps taken to design a room.}
\label{tab:AIavgValues}
\end{table}

\subsection{Experiment Setup}

We conducted a user study to explore the user experience of using different levels of AI agency, the different design characteristics, and the relationship between the human designer and the AI. We collected both quantitative data on the AI's impact on the co-designed end product and qualitative data through think-a-loud and semi-structured interviews regarding the users' experience when interacting with the AI. The interview structure is inspired by the pyramid model, meaning the interviews will begin with specific questions, and gradually have more open questions, which naturally allows for a discussion towards the end. This model is chosen to support the variation of subjects the interview is desired to cover, as well as support natural transitions between the questions and their openness. The questions and user study procedure can be found in Appendix A. 

%The interviews are semi-structured, meaning it includes both closed and more open questions, and depending on the discussion and answers, some questions might be omitted.

% We collected quantitative data regarding what impact the AI had on the co-designed end product, and how the human designer interacted with the AI's contributions. Likewise, we collected qualitative data through recorded think-aloud observations and semi-structured interviews regarding the users' experience, and possibly catch certain remarks of frustration or appreciation of their digital colleague that can be valuable for the discussion of the relationship between the co-creators. 


%We collected the following data:

%\begin{itemize}
%    \item \textbf{Audio Recordings:} 
%\end{itemize}

Eight participants tested our tool with game design and level design experience. One participant was a professional game designer with eight years of professional experience (first participant), and seven participants were third-year Game Development students. They all had an individual digital session, where we shared our screen, and they took remote control to conduct the study. Participants accepted to participate, signed consent forms, and then received a short introduction describing the experiment and its steps. The participants were then asked to design two contiguous rooms in a dungeon, repeating this process for each of the AI variants and expressing their design decisions verbally whenever they felt like it. After using the tool, the participants were interviewed, focusing on and covering an overarching understanding of the user experience, particularly in terms of creativity and interaction with the AI.

% the relationship that occurs between the AI and human designer (See Appendix B). 

For all the sessions, human designers could place up to 12 tiles, and the AI could place as many tiles as the human placed. The AI could contribute only in a rectangular area surrounding the tiles the human designer recently contributed with, including a margin of 1 tile. This choice is made to support a responsive and collaborative behavior of the AI that builds on the human designer's contribution.


% The locations available for the AI to contribute in for each turn are limited to a rectangular area surrounding the tiles the human designer recently contributed with, including a margin of 1 tile. This choice is made to support a responsive and collaborative behavior of the AI that builds on the human designer's contribution.

% The margin for the contribution area is set to 1, as it was found during experimentation that any margin bigger than this is likely perceived as the AI contributing to other areas than the ones the human is focused on, because of the default size of the room being relatively small.

%  as this enables the designer to contribute with an adequate amount of tiles during their turn and create representable structures

% The rooms produced during the user study are displayed in Figure 6, 7 and 8. Rooms with red borders are infeasible, meaning there are unreachable tiles. The UI displays a warning when this happens, and the AI can repair this during its turn, however the resulting rooms that are infeasible are a result of the human designer creating unreachable areas, and then immediately selecting to go to the World Editing view, before pressing "End Turn". 
% Each participant created two rooms for each version of the AI. Participant 1 created Room 1 and Room 2 for all version, Participant 2 created Room 3 and Room 4 for all version, etc. All of the participants had the option to adjust the sizes of the rooms in the World Editing view before entering the Room Editing view, however none of them did, and therefore all of the resulting rooms are of the default size. The designer also has the option to change the location of the hero and the doors. The location of the hero was only moved twice in all of the session, and the doors where never moved. 






%, the participant will take part in an interview. The questions, and their order, are planned out and designed to cover an overarching understanding of the user experience, in particular in terms of creativity, and the relationship that occurs between the AI and human designer (See Appendix B). 




%The participants were asked to repeat this for each of the AI-initiatives. was asked to repeatEach pair of room 

% then asked to complete three tasks, each regarding

%The users were then asked to complete three tasks that covered the tool's functionality and the AI-initiatives, respectively for each task. The tasks were 


%and different approaches to creating quests. The tasks were to 1) manually create a quest, 2) automatically create a quest, and 3) create a quest through mixed-initiative. They were also asked to create a dungeon that suited their preferences and objectives before creating quests. The questionnaire consisted of 17 closed-ended questions, and the rest were open-ended. The interview began with a questionnaire with six questions about the users' background and experience within game development and finish with questions about their experience and opinions on the tool. Both the questionnaire and interview followed guidelines described by



%The participants used the tool



\begin{figure*}[t]
    \centering
    \includegraphics[width=\textwidth]{figures/gen2seq.jpg}
    \caption{sample complete process from an individual's genotype to the phenotype.}
    \label{fig:gen2phen}
\end{figure*}


% Actually, all of this could easily be a new chapter.
\subsection{Evolving Narratives with Graph Grammars} \label{sec:evolvingNarratives}

We use the Constrained MAP-Elites~\cite{p12Khalifa2018}, and adapt it to work with graph grammars, evolve production rules, and adapt the evolution towards a target similar to~\cite{p12Alvarez2020-ICMAPE}. Constrained MAP-Elites adds feasible-infeasible two populations to each cell, effectively evolving sub-populations per cell. An individual's phenotype is a narrative graph, and its encoding genotype is the production rules of a graph grammar. A graph grammar is a context-free grammar whose productions add, remove, and modify nodes and edges of a graph. Our implementation uses the tropes listed in Table \ref{tab:tropes} as nodes, and the three available connection types as edges ($\rightarrow$, $\leftrightarrow$, $\diamondsuit$--). Graph grammars do not apply rules sequentially; instead, every individual does a random sampling of the rules in their genotype to produce \emph{recipes} to generate graphs. \emph{Recipes} describe the rules' order and repetition, and their size is limited by the amount of production rules as minimum and the minimum plus five as maximum. \emph{Recipes} do not have repetitions within them, i.e., if rule 1 is added at step 2, subsequent addition would simply add to the number of times that rule will be applied at step 2. Their size is limited by the number of production rules as minimum and up to five more samples as maximum. Figure~\ref{fig:gen2phen} shows a sample complete process from an individual's genotype (i.e., rules) to the phenotype (i.e., narrative graph).

%The EA manages an infeasible and feasible population within each cell~\cite{p12Kimbrough2008}. 

Individuals move between the feasible and infeasible population depending on the feasibility constraint. NGs are deemed infeasible if the nodes are not fully connected or if there exists a conflict pattern with more than one self-conflict. Infeasible individuals are evaluated based on how close they are to be fully connected and not having any inadequate self-conflict. The fitness function assesses NGs that are deemed feasible based on their coherence (equation~\ref{eq:coherence_fitness}), which we use to assess how correct, coherent, and in general, syntactically correct the narrative graphs are. Coherence aims at maximizing an equally weighted sum between cohesion and consistency. Cohesion refers to the link between elements that hold together to form some group. In our implementation, it focuses on minimizing the number of auxiliary patterns by calculating the proportion of \emph{Nothing} and \emph{Broken Link} among all patterns in NG. A consistent NG should be regular and free of contradictions. Thus, we calculate \emph{consistency} (eq.~\ref{eq:consistency_fitness}) as the collective quality of micro-patterns since they are the building blocks, and conflicts' goodness based on the number of fake conflicts. Thus, we aim at maximizing the quality of micro-patterns and minimizing contradictions created by meso-patterns.

%In effect, with consistency, we aim to maximize the quality of micro-patterns and minimize contradictions created by meso-patterns . %It is important, as well, to highlight that these and subsequent metrics are heuristics developed to evaluate the graphs, but they do not stand in or replace human judgement. 

%These and subsequent metrics are not evaluated with humans; thus, they do not stand in or replace human judgement, and they are just heuristics . 


\begin{equation}
\label{eq:consistency_fitness}
f_{consistency} = \frac{\sum_{i=0}^{len(ng_{micro})} i_{qual}}{len(ng_{micropat})} -  \\ 
\frac{len(ng_{fakeConfP})}{len(ng_{confP})} 
\end{equation}

\begin{equation}
\label{eq:coherence_fitness}
f_{coherence} = f_{consistency} + (1.0 - f_{cohesion})
\end{equation}

Furthermore, MAP-Elites uses behavioral dimensions in a grid shape to retain and foster diversity throughout generations. We use the following two dimensions to evaluate the diversity: 

\textbf{Step.} Step (eq.~\ref{eq:StepDim}) calculates the Levenshtein distance~\cite{p12Levenshtein96-editDistance} between two narrative graphs, taking into consideration the number and type of nodes and connections. Step is normalized using step threshold $\theta = 11$ determined through a process of experimentation, which does not consider steps farther than $\theta$, avoiding the generation of too dissimilar graphs.

\begin{equation}
\label{eq:StepDim}
D_{step} =  \min (lev_{EG,RG} (|EG|, |RG|), \theta)
\end{equation}

\textbf{Interestingness (int).} We aim at measuring the semantic quality of a narrative graph. A narrative graph can be syntactically correct and coherent yet lack a good semantic quality and do not evoke interest for designers or players. Therefore, we leverage \textbf{plot point}, \textbf{plot twist}, and \textbf{active plot device} patterns to measure the \emph{interestingness} of the NGs. The nature of \emph{interestingness} creates pressure on the fitness function since the incidence of the three meso-patterns could (if overused) ``degenerate'' the narrative; thus, decreasing its coherence. $D_{int}$ is calculated as the weighted sum ($w_{0}=0.4, w_{1}=0.2, w_{2}=0.4$) of the normalized cumulative quality of \textbf{APDs}, \textbf{PPs}, and \textbf{PTs} within an NG (eq.~\ref{eq:interesting_fitness}).

\begin{equation}
\label{eq:interesting_fitness}
D_{int} = w_{0} \times \frac{APD_{q}}{\#APD} + w_{1} \times \frac{\#PP_{q}}{\#PP} +  w_{2} \times \frac{PT_{q}}{\#PT}
\end{equation}

%These metrics are not evaluated with humans; thus, they do not stand in or replace human judgement. 


%which we have to set the system 

%In future work, we aim at validating them with human designers

%, and they just serve as proxies. They arWe aim at validating them in future work and combine 

%the quality is an estimated heuristic based on their functionality and relation to other similar patterns. We agree with the description done by R1, but these qualities are relative to the edited graph. We envision these as parameterized qualities, as one of our goals with the system is to create a mixed-initiative system where qualities will be adapted to what the designer creates.

%during
%system development we need to be able to tune the graph outputs without humans in the loop, so we are using X/Y/Z arbitrary metrics that we've found to work and which we plan to validate against human
%judgements later

\subsubsection{Experiments}

% \begin{table}[t]
% \label{tab:algorithm_info}
% \caption{Table captions should be placed above the
% tables.}\label{tab1}
% % \renewcommand{\arraystretch}{1.2}
% \setlength{\tabcolsep}{4pt}
% \resizebox{\textwidth}{!}{%
% \begin{tabular}{|l|l|l|l|l|}
% \hline
% Target Graph & Coverage (\%) & Avg. Fitness & QD-Score & Interestingness\\
% \hline
% Zelda: OoT (fig.\ref{fig:teaserfig}.a)          & 20.9$\pm$0.02 & 0.65$\pm$0.00 & 131.96$\pm$5.86 & 0.39$\pm$0.02\\
% Zelda: LttP (fig.\ref{fig:teaserfig}.b)         & 21.1$\pm$0.00 & 0.70$\pm$0.00 & 170.22$\pm$10.65 & 0.38$\pm$0.01 \\
% Super Mario Bros (fig.\ref{fig:teaserfig}.c)    & 21.1$\pm$0.01 & 0.69$\pm$0.00 & 137.82$\pm$36.04 & 0.36$\pm$0.02\\
% \hline
% \end{tabular}
% }%
% \end{table}

% Please add the following required packages to your document preamble:
% \usepackage{graphicx}

% \caption{Comparative results between target graphs (RG - fig~\ref{fig:teaserfig}, top row) and generated elites (fig~\ref{fig:teaserfig}, bottom row) using MAP-Elites}

% Please add the following required packages to your document preamble:
% \usepackage{graphicx}
\begin{table}[t]
\caption{Comparative results between root graphs and generated elites (shown in fig. \ref{fig:teaserfig})}
\resizebox{\columnwidth}{!}{%
\begin{tabular}{|l|c|c|c|c|}
\hline
Graph                        & \multicolumn{1}{l|}{Cohesion} & \multicolumn{1}{l|}{Consistency} & \multicolumn{1}{l|}{Coherence (fitness)} & \multicolumn{1}{l|}{Interestingness} \\ \hline
RG (fig \ref{fig:teaserfig}.a1)    & 1.0                           & 0.66                             & 0.825                                     & 0.61                                 \\
Elite (fig \ref{fig:teaserfig}.a2) & 1.0                           & 0.76                             & 0.875                                     & 0.73                                 \\ \hline
RG (fig \ref{fig:teaserfig}.b1)    & 1.0                           & 0.75                             & 0.87                                     & 0.38                                 \\
Elite (fig \ref{fig:teaserfig}.b2) & 1.0                           & 0.91                             & 0.95                                    & 0.55                                 \\ \hline
RG (fig \ref{fig:teaserfig}.c1)    & 1.0                           & 0.77                             & 0.88                                     & 0.4                                  \\
Elite (fig \ref{fig:teaserfig}.c2) & 1.0                           & 0.85                             & 0.92                                     & 0.52                                 \\ \hline
\end{tabular}
}
\label{tab:best-generated}
\end{table}



We conducted a series of experiments to evaluate and analyze how the system could evolve NGs into quality-diverse and valid narrative structures. We evolved the three manually constructed narrative graphs shown in figure~\ref{fig:teaserfig}, top row. They were used as root graphs and axioms in the EA, and we used \textit{interestingness} and \textit{step} as behavioral dimensions. We did $5$ MAP-Elites runs per narrative graph, ran each for $500$ generations, and set the initial population to $1000$ randomly created individuals. The initial population is generated by randomly creating between two and five production rules. Each feasible and infeasible population per cell has $25$ individuals. Each individual is limited to test $10$ recipes regardless of the chromosome size. Offspring were produced either by selecting either the left-side or right-side of a random production rule and exchanging them or with a $50$\% mutation chance. If an offspring was generated by mutation, there was a $10$\% chance to add or remove a production rule and a $90$\% to modify in various ways existing production rules.

%We calculated the \textit{coverage}: how much of the constrained search space is explored (i.e., constrained by the behavioral dimensions); the avg. fitness and the avg. interestingness of the population. All runs, regardless of the Root Graph, performed similarly with an avg. of 21\% coverage, 0.68 avg. fitness, and 0.38 avg. interestingness. These results exemplify both the hard task of generating narrative graphs and exploring the possibility space, and the seemingly competing qualities of coherence (i.e., fitness) and interestingness. 

We calculated the \textit{coverage}: how much of the constrained search space is explored (i.e., constrained by the behavioral dimensions); the avg. fitness and the avg. interestingness. All experiments had little variation regarding these metrics, and got in avg. 23.5\% coverage (24.9\%, 21.4\%, and 24.2\%, respectively), 0.79 fitness (0.76, 0.8, 0.8, respectively), and 0.37 interestingness (0.39, 0.37, 0.36, respectively). These results exemplify both the hard task of generating narrative graphs and exploring the possibility space, and the seemingly competing qualities of coherence (i.e., fitness) and interestingness. 

%The experiments got in avg. 23.5\% coverage, 0.79 fitness, and 0.37 interestingness. There was little variation in 

%All runs, regardless of the Root Graph, performed similarly with an avg. of 21\% coverage, 0.68 avg. fitness, and 0.38 avg. interestingness. These results exemplify both the hard task of generating narrative graphs and exploring the possibility space, and the seemingly competing qualities of coherence (i.e., fitness) and interestingness. 

Furthermore, in figure~\ref{fig:teaserfig}, bottom row, it is shown three different example elite narrative graphs, generated from their respective root graphs on the top row and with each individual evaluation shown in table~\ref{tab:best-generated}. The root graphs have a cohesion of 1.0 since none of them have unused nodes or connections and have similar mid-high consistency values because of using generic nodes (e.g., HERO or ENEMY), repeating them, and low involvement in structures by characters. In the case of fig \ref{fig:teaserfig}.a1, the \textbf{RevP} from HERO to SH creates some fake conflicts, which affect the consistency but also boost the interestingness value of the narrative graph. Both fig \ref{fig:teaserfig}.b1 and \ref{fig:teaserfig}.c1, are evaluated similarly with low interestingness; c1 involves a simplistic and linear structure, and b1, while in principle more complex, is also a relatively linear structure with no \textbf{PTs}.

%We calculated the \textit{coverage}: how much of the constrained search space is explored (i.e., constrained by the behavioral dimensions); the avg. fitness and the avg. interestingness of the population. All runs, regardless of the Root Graph, performed similarly with an avg. of 21\% coverage, 0.68 avg. fitness, and 0.38 avg. interestingness. These results exemplify both the hard task of generating narrative graphs and exploring the possibility space, and the seemingly competing qualities of coherence (i.e., fitness) and interestingness. 

%Furthermore, in figure~\ref{fig:teaserfig}, bottom row, it is shown three different example elite narrative graphs, generated from their respective root graphs on the top row and with each individual evaluation shown in table~\ref{tab:best-generated}. The root graphs have a cohesion of 1.0 since none of them have unused nodes or connections and have similar mid-high consistency values because of using generic nodes (e.g., HERO or ENEMY), repeating them, and low involvement in structures by characters. In the case of fig \ref{fig:teaserfig}.a1, the \textbf{RevP} from HERO to SH creates some fake conflicts, which affect the consistency but also boost the interestingness value of the narrative graph. Both fig \ref{fig:teaserfig}.b1 and \ref{fig:teaserfig}.c1, are evaluated similarly with low interestingness; c1 involves a simplistic and linear structure, and b1, while in principle more complex, is also a relatively linear structure with no \textbf{PTs}.

Furthermore, all the exemplar elites have better \textit{consistency}, \textit{coherence}, and \textit{interestingness} than the respective root graph. In figure~\ref{fig:teaserfig}.a2, the graph has been reduced towards a bottleneck, \textbf{RevP} (HERO $\rightarrow$ SH) is removed, and MCG is added as the objective for SH, which could point towards competition or cooperation to enable NEO. Such a change gives more \textit{consistency} to the graph while seemingly reducing its \textit{interestingness}, but this relation and the $\diamondsuit$-- connection between MCG and NEO increase its \textit{interestingness}. In figure~\ref{fig:teaserfig}.b2, the narrative has more interaction between \textbf{Plot Devices}, and the BAD has a more active role. Particularly, the fact that now HERO $\rightarrow$ MCG $\diamondsuit$-- BAD and MHQ $\diamondsuit$-- CHK $\diamondsuit$-- BAD could enable and force the HERO towards two main objectives before overcoming the boss, which is reflected in the higher \textit{Interestingness}. Finally, in figure~\ref{fig:teaserfig}.c2, the narrative did not change much (only four steps away), yet the graph is seemingly better, and the narrative could be very different. The graph has broken the loop which connected DRAKE $\diamondsuit$-- BAD, and could could point towards a side objective. Further, the connection between BAD and MCG has been reversed; thus, the HERO does not need to face the BAD to get the MCG, rather reaching the MCG will have as a consequence the emergence of the BAD. Finally, BAD is no longer connected to EMP and DRAKE; thus, BAD could be its own enemy faction, in this case, complexifying the narrative and creating more challenge.
% % Please add the following required packages to your document preamble:
% \usepackage{graphicx}
\begin{table}[]
\centering
\begin{tabular}{|l|lll|}
\hline
        & AIv1       & AIv2        & AIv3        \\ \hline
Leniency        & 0.56±0.07  & 0.62±0.09   & 0.57±0.08   \\
Linearity        & 0.91±0.02  & 0.92±0.02   & 0.91±0.01   \\
MesoPat       & 0.15±0.05  & 0.13±0.07   & 0.12±0.05   \\
SpatialPat    & 0.35±0.1   & 0.41±0.11   & 0.34±0.09   \\
Symmetry   & 0.43±0.11  & 0.35±0.18   & 0.35±0.12   \\
$W_{dens}$ & 0.27±0.09  & 0.26±0.08   & 0.21±0.05   \\
$W_{spar}$ & 0.21±0.05  & 0.19±0.03   & 0.15±0.01   \\
$E_{dens}$ & 0.24±0.07  & 0.27±0.06   & 0.3±0.06    \\
$E_{spar}$ & 0.22±0.05  & 0.32±0.05   & 0.35±0.06   \\
$T_{dens}$ & 0.37±0.13  & 0.28±0.09   & 0.34±0.07   \\
$T_{spar}$ & 0.36±0.11  & 0.3±0.1     & 0.37±0.05   \\ \hline
Steps      & 39.25±6.38 & 84.31±14.85 & 76.75±17.02 \\ \hline
\end{tabular}
\caption{Summary of the created rooms filtered by the AI version used. All values are the average of all the created rooms using the specific AI version. The first five values relates to the MAP-Elites dimensions, then the fitness of the rooms, the density and sparsity values for wall (W), enemies (E), and treasures (T), and finally the avg. steps taken to design a room.}
\label{tab:AIavgValues}
\end{table}

\subsection{Experiment Setup}

We conducted a user study to explore the user experience of using different levels of AI agency, the different design characteristics, and the relationship between the human designer and the AI. We collected both quantitative data on the AI's impact on the co-designed end product and qualitative data through think-a-loud and semi-structured interviews regarding the users' experience when interacting with the AI. The interview structure is inspired by the pyramid model, meaning the interviews will begin with specific questions, and gradually have more open questions, which naturally allows for a discussion towards the end. This model is chosen to support the variation of subjects the interview is desired to cover, as well as support natural transitions between the questions and their openness. The questions and user study procedure can be found in Appendix A. 

%The interviews are semi-structured, meaning it includes both closed and more open questions, and depending on the discussion and answers, some questions might be omitted.

% We collected quantitative data regarding what impact the AI had on the co-designed end product, and how the human designer interacted with the AI's contributions. Likewise, we collected qualitative data through recorded think-aloud observations and semi-structured interviews regarding the users' experience, and possibly catch certain remarks of frustration or appreciation of their digital colleague that can be valuable for the discussion of the relationship between the co-creators. 


%We collected the following data:

%\begin{itemize}
%    \item \textbf{Audio Recordings:} 
%\end{itemize}

Eight participants tested our tool with game design and level design experience. One participant was a professional game designer with eight years of professional experience (first participant), and seven participants were third-year Game Development students. They all had an individual digital session, where we shared our screen, and they took remote control to conduct the study. Participants accepted to participate, signed consent forms, and then received a short introduction describing the experiment and its steps. The participants were then asked to design two contiguous rooms in a dungeon, repeating this process for each of the AI variants and expressing their design decisions verbally whenever they felt like it. After using the tool, the participants were interviewed, focusing on and covering an overarching understanding of the user experience, particularly in terms of creativity and interaction with the AI.

% the relationship that occurs between the AI and human designer (See Appendix B). 

For all the sessions, human designers could place up to 12 tiles, and the AI could place as many tiles as the human placed. The AI could contribute only in a rectangular area surrounding the tiles the human designer recently contributed with, including a margin of 1 tile. This choice is made to support a responsive and collaborative behavior of the AI that builds on the human designer's contribution.


% The locations available for the AI to contribute in for each turn are limited to a rectangular area surrounding the tiles the human designer recently contributed with, including a margin of 1 tile. This choice is made to support a responsive and collaborative behavior of the AI that builds on the human designer's contribution.

% The margin for the contribution area is set to 1, as it was found during experimentation that any margin bigger than this is likely perceived as the AI contributing to other areas than the ones the human is focused on, because of the default size of the room being relatively small.

%  as this enables the designer to contribute with an adequate amount of tiles during their turn and create representable structures

% The rooms produced during the user study are displayed in Figure 6, 7 and 8. Rooms with red borders are infeasible, meaning there are unreachable tiles. The UI displays a warning when this happens, and the AI can repair this during its turn, however the resulting rooms that are infeasible are a result of the human designer creating unreachable areas, and then immediately selecting to go to the World Editing view, before pressing "End Turn". 
% Each participant created two rooms for each version of the AI. Participant 1 created Room 1 and Room 2 for all version, Participant 2 created Room 3 and Room 4 for all version, etc. All of the participants had the option to adjust the sizes of the rooms in the World Editing view before entering the Room Editing view, however none of them did, and therefore all of the resulting rooms are of the default size. The designer also has the option to change the location of the hero and the doors. The location of the hero was only moved twice in all of the session, and the doors where never moved. 






%, the participant will take part in an interview. The questions, and their order, are planned out and designed to cover an overarching understanding of the user experience, in particular in terms of creativity, and the relationship that occurs between the AI and human designer (See Appendix B). 




%The participants were asked to repeat this for each of the AI-initiatives. was asked to repeatEach pair of room 

% then asked to complete three tasks, each regarding

%The users were then asked to complete three tasks that covered the tool's functionality and the AI-initiatives, respectively for each task. The tasks were 


%and different approaches to creating quests. The tasks were to 1) manually create a quest, 2) automatically create a quest, and 3) create a quest through mixed-initiative. They were also asked to create a dungeon that suited their preferences and objectives before creating quests. The questionnaire consisted of 17 closed-ended questions, and the rest were open-ended. The interview began with a questionnaire with six questions about the users' background and experience within game development and finish with questions about their experience and opinions on the tool. Both the questionnaire and interview followed guidelines described by



%The participants used the tool



\subsection{Discussion and Limitations}

%Limitations with the trope-graph representation, the use of ambiguity (or unambiguity), Focalization, Metrics not based on human evaluation and judgement (moving towards a mixed-initiative system will help although, dont solve the problems). Also, the need for an intermediary system that can make sense of the graphs, or at least an ''interpreter'' that can interpret many stories from the structure, and can learn/be guided by the user.

%Limitations with the trope-graph representation, the use of ambiguity (or unambiguity), Focalization, Metrics not based on human evaluation and judgement (moving towards a mixed-initiative system will help although, dont solve the problems). Also, the need for an intermediary system that can make sense of the graphs, or at least an ''interpreter'' that can interpret many stories from the structure, and can learn/be guided by the user.

%TropeTwist allows the generation 

The trope-graph representation in TropeTwist allows for a quick definition of narrative structures. They are, by design, ambiguous, do not encode temporal information besides causal chains, and are, to some extent, generic, which makes structures relatively simple to develop but more complex to interpret. These design decisions make the system encode less rich information than others, such as Scheherazade~\citeptwelvth{p12elson-2012-dramabank}, but allow the structure to be interpreted in multiple ways. For instance, the generated graphs could equally describe different stories, and the interpretation given in this paper is just one of many. Thus, the system effectively shifts the complexity from the structure to the ``interpreter.'' While the generated structures could already serve as inspiration for users, an interpreter could provide alternative interpretations that could be guided by or learned from users, which is part of our future work.

%The trope-graph representation in TropeTwist allows for quick definition of narrative structures. They are, by design, ambiguous, do not encode temporal information besides causal chains, and are to some extent, generic which makes structures relatively simple to develop, but more complex to interpret. These design decisions make the system encode less rich information in comparison to others such as Scheherazade, but allows the structure to be interpreted in multiple ways. For instance, the generated graphs could equally describe different stories, and the interpretation given in this paper is just one of many. Thus, the system is effectively shifting the complexity from the structure to the ``interpreter.'' The development of an interpreter system. to interpret stories form the structures is an exciting future work. While the generated structures could already serve as inspiration for users; an interpreter could provide alternative interpretations that could be guided by or learn from users.

%The need and development of an ``interpreter'' system to interpret stories from the structures is  

%I think that the graphs you give as examples, while they are much too ambiguous to be interpreted by themselves and have serious coherence issues under many interpretations, could already serve as excellent inspirations for human creativity. But this is not an angle you emphasize throughout the paper. 


%For instance, the interpretation given about the generated graphs could have easily been  The system is effectively 

%The 



%The narrative structure endogenous properties such as gener

%The complexity in interpretation 

%They encode, however, less information 

Furthermore, the metrics proposed and developed here were used to tune and evaluate the graph outputs without humans in the loop. However, they do not stand in or replace human judgment. The metrics are estimated heuristics mainly based on the graph functionality and relation among patterns. Most of them are related to a ``root graph,'' which is a preliminary step for making TropeTwist interactive and have humans-in-the-loop. We aim to develop a mixed-initiative version of TropeTwist, where metrics depend on the designer's creation. This would, in turn, allow the designer to steer the MAP-Elites search, generating content adapted to them~\citeptwelvth{p12alvarez_assessing_2021}, and for MAP-Elites to assist designers with ideation proposing varied structures.

%Furtheremore, These metrics (pattern quality, fitness functions, and behavior dimensions) were developed and used to tune and evaluate the graph outputs without humans in the loop. However, they do not stand in or replace human judgement. The metrics are estimated heuristics mainly based on the graph functionality and relation among patterns. Most of them are related to a ``root graph,'' which is a preliminary step for making TropeTwist interactive and have humans-in-the-loop. In future work, we aim at validating these metrics with human designers as well as developing a mixed-initiative version of TropeTwist, where metrics are dependant on the designer's creation. This would, in turn, allow the designer to steer the generation~\citeptwelvth{p12alvarez_assessing_2021}, and for MAP-Elites to assist designers with ideation proposing related but different structures.

%You: the quality is an estimated heuristic based on their functionality and relation to other similar patterns. We agree with the description done by R1, but these qualities are relative to the edited graph. We envision these as parameterized qualities, as one of our goals with the system is to create a mixed-initiative system where qualities will be adapted to what the designer creates.
 
\subsection{Conclusions and Future Work}

% In this paper, we have proposed \emph{TropeTwist}, a system that uses tropes and trope patterns to describe and construct narrative structures. We demonstrated through three proof-of-concept structures and a limited set of nodes and connections, the expresiveness of the system to define and describe games with diverse genres, mechanics, and game phases. Further, we illustrated how we could generate variations using MAP-Elites to generate a set of quality-diverse structures from the three proof-of-concept structures, outperforming them on our metrics.

In this paper, we have presented \emph{TropeTwist}, a system that interconnects tropes and trope patterns to describe narrative structures. We demonstrated through three proof-of-concept structures the system's expressiveness to describe games with diverse genres and mechanics, and different game phases. Further, we illustrated how we could generate novel structures from the three proof-of-concept structures using MAP-Elites, improving them on our metrics. 

Tropes could be seem as something to avoid when exploring creativity, mainly due to the possibility of showing unoriginal views by definition. However, a set of combined tropes, patterns, and structures could give rise to novel combinations that express the wanted structure. Similarly, identifying, visualizing, and defining the tropes and patterns and doing ``twists'' with them; thus, transforming something typical into atypical is the goal with TropeTwist.

The narrative structures show essential aspects of how the story will develop and lead, and important components such as events, conflicts, or roles. However, to further operationalize these structures, it is necessary other systems that make use of them, such as quest~\citeptwelvth{p12Alvarez2021-questgram,p12ammanabrolu2019-towardQuestGeneration} or plot~\citeptwelvth{p12Ammanabrolu2020-PlotEventsSentences} generators. Another interesting future work would be to explore the multi-faceted nature of games~\citeptwelvth{p12Liapis2019-OrchestratingGames} and combine this type of system with generators that focus on other facets such as level design~\citeptwelvth{p12sarkar2021-dungeonPlatformer,p12alvarez2019empowering} or game mechanics~\citeptwelvth{p12green2021-gamemechanicsAlignment,p12charity2020mech}.

Generating novel narrative structures resulted in interesting variations, but the system could not exploit all the advantages of MAP-Elites. Our results point towards difficulties exploring the space, possibly because \emph{coherence} and \emph{interestingness} are to some extent competing objectives. Therefore, we aim at extending TropeTwist towards a mixed-initiative co-creative system~\citeptwelvth{p12yannakakis2014micc}, and with that, evaluate with human participants. Given that our metrics are dependant on the designed graph; then, we could constantly adapt the content generation and have adaptive models, for instance, of interestingness, based on the user's creation similar to~\citeptwelvth{p12alvarez2019empowering,p12Panagiotis2021-susketch}. 

% The MAP-Elites experiments resulted in interesting variations, but it could not exploit all the advantages of MAP-Elites. 

% The MAP-Elites experiments resulted in interesting variations, but it could not exploit all the advantages of MAP-Elites. Our results point towards difficulties exploring the space, possibly because \textit{coherence} and \textit{interestingness} are to some extent competing objectives. Therefore, we aim at extending TropeTwist towards an interactive system such as a mixed-initiative co-creative system~\citeptwelvth{p12yannakakis2014micc}, and with that, evaluate with human participants. Then, we could constantly adapt the content generation and have adaptive models, for instance, of interestingness, based on the user's creation similar to~\citeptwelvth{p12alvarez2019empowering,Panagiotis2021-susketch}. 

% similar to~\citeptwelvth{p12Cook2014-ARogueDream,hoover2015-audioinspace,ashmore2007-questGeneratedWorld,dormans2011generating}.

% For instance, space and narrative have a special relation and have been intertwined and linked~\citeptwelvth{p12} such as level design~\citeptwelvth{p12sarkar2020-sequentialVAELVLGen,alvarez2019empowering} since both space

% we aim at both, test our system with human participants to analyze the usabality and expresiveness, and

% Our experiments evolving narratives resulted in interesting variations, but we were unable to exploit all the advantages of MAP-Elites. Our results points towards difficulties exploring the space, possibly because \textit{Coherence} and \textit{Interestingness} are to some extent competing objectives. Therefore, we aim at both, test our system with human participants to analyze the usabality and expresiveness, and extend TropeTwist towards an interactive system focusing on moving towards a mixed-initiative co-creative system~\citeptwelvth{p12yannakakis2014micc} similar to~\citeptwelvth{p12kreminski2020-Germinate}. Then, we could constantly adapt the content generation and have adaptive models, for instance, of interestingness, based on the user's creation similar to~\citeptwelvth{p12alvarez2019empowering,Panagiotis2021-susketch}.

% possible difficulties exploring the space here it goes things like~\citeptwelvth{p12kreminski2020-Germinate}

% This paper proposes the use of T

% In this paper, we demonstrated POWER! UNLIMITED POWER!

% Given that \emph{int} combines these three qualities designers might not be interested in highly interesting generated narrative graphs as they could degrade their narrative objective.  Then we want to use Mixed-initiative bra!

% In the introduction we discussed the multi-faceted nature of Games, and it is our goal to continue.

% Really need to point out that while the narrative structures are telling us important aspects of how the story will develop and lead, we need other systems such as a quest generator or plot generator that put the narrative in context and operationalize it.

% Even less than this!

% Given that \emph{int} combines these three qualities designers might not be interested in highly\footnote{Interesting-boring qualities are subjective measurements, thus what is highly interesting in our system might not necessarily be for a designer, which is why (in part) we leverage on the narrative graph created by the designer to measure and evaluate patterns and their quality.} interesting generated narrative graphs as they could degrade their narrative objective.  Then we want to use Mixed-initiative bra!

% Really need to point out that while the narrative structures are telling us important aspects of how the story will develop and lead, we need other systems such as a quest generator or plot generator that put the narrative in context and operationalize it.

%We present a very simple yet interesting proof-of-concept mixed-initiative quest generation tool implemented in and combined with a mixed-initiative level generation tool called EDD. 

%We envision the system as an abstract narrative layer where designers are given the freedom to construct the narrative structure they envision. These structures could be the main structure of the game or media, be a side part of it, or directly aim at how several encounters should be defined.

%in a mixed-iniative approach that enables   a mixed-initiative quest generation in

%Through using Doran and Parberry´s quest structure and production rules, a mixed initiative quest tool was able to be created, to further develop EDD. The experiment related to the expressive range of the quest generator displayed the dominance of certain quests actions, although some actions have a higher dominance rate it was not noticed by the participating testers in the user study. 

%The tool’s mixed initiative approach was positively met by the testing participants, along with the manual creation. However automatic creation and the automatic suggestions received mixed response, mainly because of a random placement of position on the actions. However the overall response of the tool was positive and and a majority of the participants experienced increased creativity while using the tool, many participants expressed the tools usability as a way to gain inspiration, as a solution to inspiration blockages and as a resource-efficient tool for game developers to use.  

%Future work on the quest tool could include improving the automatic generation as well as the user interface and  functionality  for the quest sequence structures for an increased usability. In addition, further work on EDD could be conducted to increase its usability and maintainability. 


\bibliographystyleptwelvth{ieeetr}
\bibliographyptwelvth{included-papers-tex/paper-12/references.bib,included-papers-tex/paper-12/games.bib}