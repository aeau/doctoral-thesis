\subsection{Introduction}

% Procedural Content Generation (PCG) is defined as the use of algorithms to generate game content (levels, narrative, visuals, or even game rules) with limited human input \citepsixth{p6shaker_procedural_2016}. Recent research works have shown PCG's ability to enable new game genres, as well as to create better environments and benchmarks for learning algorithms \citepsixth{p6risi2019procedural}.

% In parallel, 

% Julian's text

How can we best build a system that lets a human designer collaborate with procedural content generation (PCG) algorithms in order to create useful and novel game content? Various systems have been proposed that allow for humans and algorithms to share authorship by both editing and critiquing the content being created, in what is called the mixed-initiative paradigm~\citepsixth{p6yannakakis2014micc,p6liapis2016mixed}. However, for such collaboration to reach its true potential, there needs to be an understanding between the human designer and the software system about what needs to be designed; ideally even a shared goal.

Reaching such a shared understanding is a hard task even when both collaborators share significant cultural and professional background. When one of the collaborators is a computer program, this task is perhaps AI-complete. But we can take steps towards the goal of shared understanding. One idea is to train a supervised learning model on traces of other collaborative creation session and try to predict the next step the human would take in the design process. The main problem with this is that people are different, and different creators will want to take different design actions in the same state; another problem is what to do in design states that have not been encountered in the training data. To remedy this, it has been proposed to train multiple different models, predicting the next step for different designer ``personas'' (akin to procedural personas in game-playing~\citepsixth{p6Holmgard2019-proceduralPersonas}). However, for such a procedure to be effective, we need to have a sufficient amount of training data. The more different designer personas there are, the more training data is necessary.

One way of overcoming this problem could be to change the level of abstraction at which design actions are modeled and predicted. Instead of predicting individual edits, one could identify different styles or phases of the artifact being created, and model how a designer moves from one to another. To put this concretely in the context of designing rooms for a Zelda-like dungeon crawler~\citepsixth{p6tloz}, one could classify room styles depending on whether they were enemy onslaughts, complex wall mazes, treasure puzzles, and so on. One could then train models to recognize which types of rooms a user creates in which order. By clustering sequences of styles, we could formulate designer personas as archetypical trajectories through style space, rather than as sequences of individual edits. For example, in the context of creating a dungeon crawler, some designers might start with the outer walls of the rooms and then populate it with NPCs, whereas another type of designer might first sketch the path they would like the player to take from the entrance to the exit and then add parts of the room outside the main path. These designer models could then be combined with search-based or other procedural generation methods to suggest ways of getting to the next design style from the current one. 

In this paper we provide a prototype implementation of designer personas as archetypical paths through style space. Through this, we take a step further into modeling designers  For this we use the Evolutionary Dungeon Designer (EDD), a research platform for exploring mixed-initiative creation of dungeon crawler content~\citepsixth{p6alvarez2019empowering,Baldwin2017}. Data from several dozen users designing game levels with the tool have been used to train the models. Based on this data, we clustered room styles to identify a dozen distinct types of rooms. To understand the typical progress of designers and validate the clustering, we visualize how typical design sessions traverse the various clusters. We also perform frequent sequence mining on the design sessions to find a small handful of designer personas.

% end of Julian's text






