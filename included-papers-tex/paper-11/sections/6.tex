\subsection{Conclusions and Future Work}

% \begin{itemize}
%     \item Main conclusion about the system and what it allows!
%     \item future work on using the system intertwined with other facets.
% \end{itemize}

In this paper, we have presented the first iteration of \emph{Story Designer} as a step towards a mixed-initiative co-creative system implementation of TropeTwist~\citepeleventh{p11alvarez_tropetwist_2022} to design narrative structures in EDD. The system allows the creation of narrative structures as narrative graphs that defines the overarching narrative, identifying characters, their roles and involvement, objectives, and core events. Story Designer also presents a step towards creating a holistic system, intertwining level design and narrative through simple level design constraints, effectively delimiting the search space of MAP-Elites with promising results. We analyzed and evaluated Story Designer and the impact of these level constraints through four simulated experiments; experiments 1-3 approach Story Designer in a more static scenario, while experiment 4 focuses on the step-by-step creation process.

Experiment 4 and, in general, the design process in Story Designer shows how the tropes, nodes, and connections, can be used to design a narrative structure step by step, changing the components of the narrative and how different elements in the game can be used and interpreted with simple changes. Defining conflicts among characters (thus, creating factions), defining primary and side objectives, as well as important elements in the narrative (e.g., plot devices), is a simple process. Changing these to adapt to the designer's goal is possible with minimal input. For instance, see the change from experiment 4.4 to 4.5 (fig.~\ref{fig:examples}.d4,d5), where DRA passes from a side objective to a main part of the structure by creating an "entails" connection and forming a DerP meso-pattern (increasing the graph's interestingness score to 0.33). Equally important, the system preserves its properties and adapts to the narrative graph created, which could create a better experience for the designer. However, our evaluation was through simulated design sessions (especially, experiment 4) highlighting properties and tradeoffs of the system. We aim at evaluating Story Designer with a user study to assess its usability, the expressiveness designers have when creating structures, and the experience intertwining and creating level design constraints. 

%we only evaluated the system through

%However, we aim at evaluating Story Designer with a user study to assess its usability, the expressiveness designers have when creating structures, and the experience intertwining and creating level design constraints. 

%Our next steps would be to continue the development of Story Designer in EDD to reincorporate narrative constraints into both the level design facet~\citepeleventh{p11alvarez_empowering_2019} and the quest system to adapt main and side objectives~\citepeleventh{p11alvarez_questgram_2021}.

Our next steps would be to continue the development of Story Designer to reincorporate the narrative structure into other facets and systems. Following a similar approach with constraints, narrative structures could constrain the search space for other facets, creating a feedback loop across facets and systems for a holistic approach. For instance, within EDD, narrative constraints could be reincorporated into both the level design facet~\citepeleventh{p11alvarez_empowering_2019} and the quest system to adapt main and side objectives~\citepeleventh{p11alvarez_questgram_2021}.

%Yeah, we definitely lose the possibility to search more complex narrative structures when using constraints. But that does not seem to hinder the capability of themodel, showing good results for the adoption and intertwined of level design and narrative. Next step would be to reincorporate those constraints into the level design part to observe how that would perform and how that would limit the level design aspect.. Finally, incorporate all together with other quest narrative systems such as questgram is another goal

%Another interesting step, would be Following a similar ap with 

%in EDD to reincorporate narrative constraints into both the level design facet~\citepeleventh{p11alvarez_empowering_2019} and the quest system to adapt main and side objectives~\citepeleventh{p11alvarez_questgram_2021}.

%The narrative structures show essential aspects of how the story will develop and lead, and important components such as events, conflicts, or roles. However, to further operationalize these structures, it is necessary other systems that make use of them, such as quest~\citepeleventh{p11Alvarez2021-questgram,ammanabrolu2019-towardQuestGeneration} or plot~\citepeleventh{p11Ammanabrolu2020-PlotEventsSentences} generators. Another interesting future work would be to explore the multi-faceted nature of games~\citepeleventh{p11Liapis2019-OrchestratingGames} and combine this type of system with generators that focus on other facets such as level design~\citepeleventh{p11sarkar2021-dungeonPlatformer,alvarez2019empowering} or game mechanics~\citepeleventh{p11green2021-gamemechanicsAlignment,charity2020mech}.

%Our results point towards difficulties exploring the space, possibly because \textit{coherence} and \textit{interestingness} are to some extent competing objectives. Therefore, we aim at extending TropeTwist towards a mixed-initiative co-creative system~\citepeleventh{p11yannakakis2014micc}, and with that, evaluate with human participants. Then, we could constantly adapt the content generation and have adaptive models, for instance, of interestingness, based on the user's creation similar to~\citepeleventh{p11alvarez2019empowering,Panagiotis2021-susketch}. 

%Next step would be to reincorporate those constraints into the level design part to observe how that would perform and how that would limit the level design aspect.. Finally, incorporate all together with other quest narrative systems such as questgram is another goal!

%Defining primary and side objectives, as well as conflicts among characters 

%Experiment 4 shows how the tropes, nodes, and connections, can be used to design a structure step by step, changing the components of the narrative and how different elements in the game can be used and interpreted with simply changes. 


%Definitely present now the thing of defining step by step the narrative graph, and how changes in the connections and the nodes can change ther sequence of events and primary and side objectives. 

%Yeah, we definitely lose the possibility to search more complex narrative structures when using constraints. But that does not seem to hinder the capability of themodel, showing good results for the adoption and intertwined of level design and narrative. Next step would be to reincorporate those constraints into the level design part to observe how that would perform and how that would limit the level design aspect.. Finally, incorporate all together with other quest narrative systems such as questgram is another goal!

%In this paper, we have presented \emph{TropeTwist}, a system that interconnects tropes and trope patterns to describe narrative structures. We demonstrated through three proof-of-concept structures the system's expressiveness to describe games with diverse genres and mechanics, and different game phases. Further, we illustrated how we could generate quality-diverse structures from the three proof-of-concept structures using MAP-Elites, improving them on our metrics. 

%Tropes could be seem as something to avoid when exploring creativity, mainly due to the possibility of showing unoriginal views by definition. However, a set of combined tropes, patterns, and structures could arise novel combinations that express the wanted structure. Similarly, identifying, visualizing, and defining the tropes and patterns and doing "twists" with them; thus, transforming something typical into atypical is the goal with TropeTwist.

%The narrative structures show essential aspects of how the story will develop and lead, and important components such as events, conflicts, or roles. However, to further operationalize these structures, it is necessary other systems that make use of them, such as quest~\citepeleventh{p11Alvarez2021-questgram,ammanabrolu2019-towardQuestGeneration} or plot~\citepeleventh{p11Ammanabrolu2020-PlotEventsSentences} generators. Another interesting future work would be to explore the multi-faceted nature of games~\citepeleventh{p11Liapis2019-OrchestratingGames} and combine this type of system with generators that focus on other facets such as level design~\citepeleventh{p11sarkar2021-dungeonPlatformer,alvarez2019empowering} or game mechanics~\citepeleventh{p11green2021-gamemechanicsAlignment,charity2020mech}.

%Generating novel narrative structures resulted in interesting variations, but the system could not exploit all the advantages of MAP-Elites. Our results point towards difficulties exploring the space, possibly because \textit{coherence} and \textit{interestingness} are to some extent competing objectives. Therefore, we aim at extending TropeTwist towards a mixed-initiative co-creative system~\citepeleventh{p11yannakakis2014micc}, and with that, evaluate with human participants. Then, we could constantly adapt the content generation and have adaptive models, for instance, of interestingness, based on the user's creation similar to~\citepeleventh{p11alvarez2019empowering,Panagiotis2021-susketch}. 


%Definitely present now the thing of defining step by step the narrative graph, and how changes in the connections and the nodes can change ther sequence of events and primary and side objectives. 

%This paper presents Questgram [Qg], a quest generation tool with a mixed-initiative approach integrated into the Evolutionary Dungeon Designer. Questgram lets a human designer co-create an overarching quest that fits in a dungeon level, as the dungeon is being developed in the designer. Both map and quest are designed in parallel and with the suggestions provided by EDD. Quests make use of Doran and Parberry's quest structure as production rules in a grammar so that all quests are well-formed with respect to the grammar and the level landscape. We show results from a two-fold evaluation, an expressive range analysis, and a user study. 

%The expressive range analysis shows several dominant quest actions and structures, though all types of actions could be generated at a wide range of quest lengths. The mixed-initiative approach was positively met by the user study participants, along with the manual creation. However, automatic creation and automatic suggestions received a mixed response, mainly because of a random placement of position on the actions and the use of abstract quest actions. The tool's overall response was positive, and a majority of the participants reported increased creativity while using the tool. Many participants expressed its usability to gain inspiration, as a solution to inspiration blockages, and as a resource-efficient tool for game developers to use. None of the testers noticed the dominance of some actions detected in the expressive range analysis.

%This inspirational use points towards the need to explore other fundamental and more useful ways to establish effective MI-CC workflows where systems can adapt and be effectively employed and used. For instance, some interesting future paths would be to explore the creation of more adaptive collaboration that considers the designer's style or to give more autonomy to the AI to have more participation in the creative process and its effects. Within this, one interesting area is the one of eXplainable AI for Designers~\citepeleventh{p11Zhu2018-XAIDesignersMICC} where the goal is to achieve system explainability to improve the collaboration and interaction between human and AIs.

%Yeah, we definitely lose the possibility to search more complex narrative structures when using constraints. But that does not seem to hinder the capability of themodel, showing good results for the adoption and intertwined of level design and narrative. Next step would be to reincorporate those constraints into the level design part to observe how that would perform and how that would limit the level design aspect.. Finally, incorporate all together with other quest narrative systems such as questgram is another goal! 

%In this paper, we have presented \emph{TropeTwist}, a system that interconnects tropes and trope patterns to describe narrative structures. We demonstrated through three proof-of-concept structures the system's expressiveness to describe games with diverse genres and mechanics, and different game phases. Further, we illustrated how we could generate quality-diverse structures from the three proof-of-concept structures using MAP-Elites, improving them on our metrics. 

%Tropes could be seem as something to avoid when exploring creativity, mainly due to the possibility of showing unoriginal views by definition. However, a set of combined tropes, patterns, and structures could arise novel combinations that express the wanted structure. Similarly, identifying, visualizing, and defining the tropes and patterns and doing "twists" with them; thus, transforming something typical into atypical is the goal with TropeTwist.

%The narrative structures show essential aspects of how the story will develop and lead, and important components such as events, conflicts, or roles. However, to further operationalize these structures, it is necessary other systems that make use of them, such as quest~\citepeleventh{p11Alvarez2021-questgram,ammanabrolu2019-towardQuestGeneration} or plot~\citepeleventh{p11Ammanabrolu2020-PlotEventsSentences} generators. Another interesting future work would be to explore the multi-faceted nature of games~\citepeleventh{p11Liapis2019-OrchestratingGames} and combine this type of system with generators that focus on other facets such as level design~\citepeleventh{p11sarkar2021-dungeonPlatformer,alvarez2019empowering} or game mechanics~\citepeleventh{p11green2021-gamemechanicsAlignment,charity2020mech}.

%Generating novel narrative structures resulted in interesting variations, but the system could not exploit all the advantages of MAP-Elites. Our results point towards difficulties exploring the space, possibly because \textit{coherence} and \textit{interestingness} are to some extent competing objectives. Therefore, we aim at extending TropeTwist towards a mixed-initiative co-creative system~\citepeleventh{p11yannakakis2014micc}, and with that, evaluate with human participants. Then, we could constantly adapt the content generation and have adaptive models, for instance, of interestingness, based on the user's creation similar to~\citepeleventh{p11alvarez2019empowering,Panagiotis2021-susketch}. 


%We present a very simple yet interesting proof-of-concept mixed-initiative quest generation tool implemented in and combined with a mixed-initiative level generation tool called EDD. 

%We envision the system as an abstract narrative layer where designers are given the freedom to construct the narrative structure they envision. These structures could be the main structure of the game or media, be a side part of it, or directly aim at how several encounters should be defined.

%in a mixed-iniative approach that enables   a mixed-initiative quest generation in

%Through using Doran and Parberry´s quest structure and production rules, a mixed initiative quest tool was able to be created, to further develop EDD. The experiment related to the expressive range of the quest generator displayed the dominance of certain quests actions, although some actions have a higher dominance rate it was not noticed by the participating testers in the user study. 

%The tool’s mixed initiative approach was positively met by the testing participants, along with the manual creation. However automatic creation and the automatic suggestions received mixed response, mainly because of a random placement of position on the actions. However the overall response of the tool was positive and and a majority of the participants experienced increased creativity while using the tool, many participants expressed the tools usability as a way to gain inspiration, as a solution to inspiration blockages and as a resource-efficient tool for game developers to use.  

%Future work on the quest tool could include improving the automatic generation as well as the user interface and  functionality  for the quest sequence structures for an increased usability. In addition, further work on EDD could be conducted to increase its usability and maintainability. 
