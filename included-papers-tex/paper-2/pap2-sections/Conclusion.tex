\section{Conclusions and Future Work} \label{p2conclusion}

In this paper, we have presented the advancements done on EDD in relation to the evolutionary system with different evaluations, encoding, genotype representation and strategies that aims on preserving and consider the designer's aesthetic criteria.

By introducing the capability of locking sections of a room, we changed the individual's encoding from direct to semi-direct, and in turn, offered new and easier possibilities to perform different operations to the individuals, as well as, allowing the designer to preserve individual tiles, shapes, routes and even design patterns.

%By changing the encoding of the evolutionary algorithm from direct to semi-direct encoding we opened the possibilities to perform different operations to the individuals, as well as a fair way of preserving the sections of the map which were considered important, significant and unchangeable by the designer. As result, the generator has increased on controllability at expenses of expressiveness. Moreover, this approach allows the designer not only to lock and preserve interesting aesthetical changes done in the map but also indirectly, is able to preserve routes and design patterns.

Moreover, we successfully integrated and produced rooms evaluated  on symmetry and similarity that held the overlying structure of the micro-patterns. The added evaluations establishes the path to preserve and consider more in-depth the designers criteria and produce personalized work that accurately transmit the ideas and intentions of the designer.

%In the end, we successfully integrated and produced dungeon levels which, held the overlying structure of the micro-patterns and symmetry together \textbf{(here is missing the part of the similarity)}. Moreover, the added evaluations to the fitness of each individual allows us to establish the path to preserve and consider more in-depth the designers criteria and produce a more personalized work that accurately transmit the ideas and intentions of the designer.

We aim to more throughly evaluate the system by incorporate the three techniques into a user study, similar to the one done by~\citet{Baldwin2017TowardsGeneration} to validate the tool's capacity on assessing the designer's criteria. It would be interesting to add more aesthetic concepts to evaluate the produced content, for instance, density, simplicity, sparseness and individuality.

%We aim to further evaluate the system with different configurations and observe how the different fitness functions can interact and cooperate with each other to create more interesting content, as well as, joining both approaches for a case study, similar to the one done by Baldwin et al \cite{Baldwin2017TowardsGeneration}. It would be interesting to continue using aesthetic concepts, for instance, density, simplicity, sparseness and individuality, to evaluate the content 

The subdivision of the map could be extended to perform a parallel evolution on the custom aesthetic structures locked by the designers and propose interesting variations. Moreover, a zone analysis could be introduced to increase the dungeon's knowledge for the designer by suggesting changes to fulfill different player models, similar to Holmg\r{a}rd's approach~\cite{Holmgard2014EvolvingModeling}, or paths and statistics. Finally, we would like to explore different types of representations towards more generative encodings to test, compare and measure the differences and advantages of the resulting maps.

%Further use the division of the map by performing zone analysis, which could result on suggesting changes to the designers in order to fulfill different player models, similar to Holmg\r{a}rd's approach \cite{Holmgard2014EvolvingModeling} or do a separated evolution on the manually locked tiles providing the designers with interesting shapes and patterns. Finally, we would like to go down the road towards more indirect encodings and test different approaches and, compare and measure the differences and advantages of the resulting maps.