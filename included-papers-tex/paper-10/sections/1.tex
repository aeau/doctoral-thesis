\subsection{Introduction}

% Procedural Content Generation (PCG) is defined as the use of algorithms to generate game content (levels, narrative, visuals, or even game rules) with limited human input \cite{p10shaker_procedural_2016}. Recent research works have shown PCG's ability to enable new game genres, as well as to create better environments and benchmarks for learning algorithms \cite{p10risi2019procedural}.

% In parallel, 

% Julian's text

How can we best build a system that lets a human designer collaborate with procedural content generation (PCG) algorithms to create useful and novel game content? Collaboration between AI and humans to co-design and co-create content is a major challenge in AI, and the main focus of Mixed-Initiative Co-Creativity~\cite{p10yannakakis2014micc,liapis2016mixed}. These systems' objectives are to foster creativity and provide seamless proactive collaboration; ultimately enabling a colleague relation and collaboration as described by Lubart~\cite{p10LUBART2005-computerPartners}. However, there needs to be an understanding between the human designer and the AI system about what needs to be designed, ideally even a shared goal.

%Collaboration between AI and humans to design and create content is a major challenge in AI, and the main focus of Mixed-Initiative Co-Creative PCG~\cite{p10yannakakis2014micc}. Multiple approaches have been proposed as alternatives for creating systems that model the interaction between both AI and humans to create game content, and that use different techniques to study such interactions and its implications~\cite{p10Alvarez2020-ICMAPE,smith_tanagra:_2011,Liapis2014,charity2020baba}. 

% However, for this interaction to be fully fleshed, the human needs to understand the behavior of the AI through interpretable and explainable models and systems, and the AI needs to recognize and interpret the intentions of the humans seamlessly as they create their designs. The former is the focus of~\emph{Interpretable and Explainable AI}, which seek to create or adapt models and systems for a better workflow between humans and AI, where humans can understand the AI's decision process to enable trust relationships and reach deeper interactions~\cite{p10Zhu2018-XAIDesignersMICC,Doshi-Velez2018,adadi2018peeking}. The latter, which is the focus of this paper, would mean that the AI could adapt its behavior and functionality to the needs, expertise, and workflow of individual designers or specific group of designers. To do so, the AI is required to do an analysis on several design processes such as the designer's preferences, styles, and goals, which holistically is called \emph{Designer Modeling}~\cite{p10Liapis2013-designerModel,Liapis2014-designerModelImpl}.

Reaching such a shared understanding is a hard task, even when both collaborators share significant cultural and professional backgrounds.
%When one of the collaborators is a computer program, this task is perhaps AI-complete. 
However, we can take steps towards the goal of shared understanding. One idea is to train a supervised learning model on traces of other collaborative creation sessions and try to predict the next step the human would take in the design process. The main problem with this is that people are different, and creators will want to take different design actions in the same state. Another problem is what to do in design states that have not been encountered in the training data. To remedy this, it has been proposed to train multiple models, predicting the next step for different designer ``personas'' (akin to procedural personas in game-playing~\cite{p10Holmgard2019-proceduralPersonas}). However, for such a procedure to be effective, we need to have sufficient training data. The more different designer personas there are, the more training data is necessary.

One way of overcoming this problem could be to change the level of abstraction at which design actions are modeled and predicted. Instead of predicting individual edits, one could identify different styles or phases of the artifact being created and model how a designer moves from one to another. To put this concretely in the context of designing rooms for a Zelda-like dungeon crawler~\cite{p10tloz}, one could classify room styles depending on whether they were enemy onslaughts, complex wall mazes, treasure puzzles, and so on. One could then train models to recognize which types of rooms a user creates in which order. By clustering sequences of styles, we could formulate designer personas as archetypical trajectories through style space rather than as sequences of individual edits. For example, in the context of creating a dungeon crawler, some designers might start with the outer walls of the rooms and then populate it with NPCs, whereas another type of designer might first sketch the path they would like the player to take from the entrance to the exit and then add parts of the room outside the main path. These designer models could then be combined with search-based~\cite{p10Togelius2011} or other procedural generation methods~\cite{p10khalifa2020-pcgrl} to suggest ways of getting to the next design style from the current one. 

In this paper, we present a method to create and identify designer personas as archetypical paths through style space and provide a prototype implementation of it. For this, we use the Evolutionary Dungeon Designer (EDD), a research platform for exploring mixed-initiative creation of adventure and dungeon crawler content~\cite{p10alvarez2019empowering,Baldwin2017}. Data from $48$ users designing game levels with the tool following different goals and styles have been used to fit the models. Based on this data, we clustered room styles to identify a dozen distinct types of rooms. To understand the typical progress of designers and validate the clustering, we visualize how typical design sessions traverse the various clusters. We also perform frequent sequence mining on the design sessions to find a small handful of designer personas.

% In this paper, we propose such an alternative approach, designer modeling through clustering the style space, and designer personas as archetypical paths through the clustered style space. We used the Evolutionary Dungeon Designer (EDD), a research platform for investigating mixed-initiative generation of adventure and dungeon crawler game content~\cite{p10alvarez2019empowering,Baldwin2017}. We analyzed and used data from 190 human-made designs with diverse properties such as multiple goals and style, which in turn, allowed us to identify a set of 12 style clusters as distinct types of rooms. To validate our approach, we used internal validations of multiple clustering approaches and visualized how typical design sessions traverse the clusters. Analyzing this data, and using the steps between clusters of all the 190 designs, we identified a handful of designer personas.

% end of Julian's text




%  designer personas as archetypical paths through style space.

