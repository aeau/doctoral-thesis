\subsection{Conclusions}



% \begin{itemize}
%     \item Discussion on what does this archetypical design trajectories mean?
%     \item how to use them? next steps into integrating this into a system. To use this in a search-based approach as objectives for the generation to move towards the directions where (according to our archetypical design trajectories) the designers will move towards in their design process. Perhaps I could also bring the discussion from the workshop-paper for HC-AI.
%     \item discussion on creativity? is the output or the process where the actual creativity is outputted? Compare using end-design clustering to using sequences to cluster.
%     \item Discussion on how PCGRL relates to this type of work? --> Perhaps this is something for the background instead.
% \end{itemize}{}


% This paper presents a step towards designer modelling in a MI-CC environment by providing an implementation of designer personas as archetypical trajectories through style space, as a means to characterize several representative and frequent design styles together. 

%This paper presents a novel approach and meaningful steps towards designer modeling in an MI-CC environment. By providing an implementation of designer personas as archetypical trajectories through style space, we show that 

This paper presents a novel method and meaningful steps towards designer modeling through an experiment on archetypical design trajectories analysis in an MI-CC environment. Through this, we characterize several representative design styles as designer personas. We have first run and compared several clustering setups to find the best partitioning of the room style using the design step sequence of the collected $180$ unique rooms, ending in $8196$ data points, and resulting in a set of twelve cohesive and representative room style clusters. We have then mapped these $180$ design sequences in terms of these clusters, applying frequent sequence mining to find four frequent and unique designer styles, with related common sub-styles. As a result, we have presented a roadmap of design styles over a map of data-driven room style clusters. 

%This paper presents a step towards designer modeling through an experiment on archetypical design trajectories analysis in an MI-CC environment, as a means to characterize several representative design styles as designer personas. We have first run and compared several clustering setups to find the best partitioning using the edition sequences of the collected $180$ unique rooms, ending in $8196$ data points, and resulting in a set of twelve cohesive, coherent, and meaningful clusters. We have then mapped these $180$ design sequences in terms of these clusters, applying frequent sequence mining to find four frequent unique designer styles, with related common sub-styles. As a result, we have presented a roadmap of design styles over a map of data-driven design clusters. %The examples in Figure \ref{fig:archetypical-examples}, help us to clarify 

%  namely the \textsc{Designer Personas}

% Our work draws on the ideas, concepts, and goals and concepts proposed by Liapis et al. when introducing the Designer Modeling as a model to capture multiple designer's processes. A prototype of such was implemented in the sentient sketchbook~\citeptenth{p10Liapis2014-designerModelImpl}, where it is proposed the use of interactive evolution by biasing the search space in favor of hand-crafted features of the design. we propose an alternative and novel route to designer modeling through clustering the design space and the room style based on the collected data. Moreover, we differ as well on the type of level design, being the sentient sketchbook a tool for strategy games~\citeptenth{p10liapis_generating_2013}, while EDD a tool for adventure and rogue-like games~\citeptenth{p10Alvarez2020-ICMAPE}. These differences strengthen the importance and usefulness of designer modeling, and highlight the holistic and generic properties of this designer-centric perspective.

% Designer modeling was proposed as an approach to capture multiple designer's processes to create a better workflow by Liapis et al.~\citeptenth{p10Liapis2013-designerModel}, and our work draws on many of their ideas, concepts, and goals. We propose an alternative route to designer modeling through clustering the design space and the room style based on the collected data. Moreover, we differ in the type of level design, being the sentient sketchbook a tool for strategy games~\citeptenth{p10liapis_generating_2013}, while EDD is a tool for adventure and dungeon crawler games~\citeptenth{p10Alvarez2020-ICMAPE}. These differences strengthen the importance and usefulness of designer modeling and highlight the holistic and generic properties of this designer-centric perspective.% and its possibilities.

% Furthermore, a prototype of such was implemented in the sentient sketchbook~\citeptenth{p10Liapis2014-designerModelImpl}, where it is proposed different approaches to model style, process, and goals based on choice-based evolution and the designer's current design to adapt the provided suggestions accordingly. 

% Designer modeling in computer-aided design tools was proposed by Liapis et al.~\citeptenth{p10Liapis2013-designerModel} as an approach to capture multiple designer's processes to create a better workflow, and a prototype of such was implemented in the sentient sketchbook~\citeptenth{p10Liapis2014-designerModelImpl}. While our work drags on many of the concepts, ideas, and goals described by Liapis et al., we propose an alternative route to designer modeling through clustering the design space

% In their work, they propose the use of hand-crafted

% Our work drags on many of the concepts, ideas, and goals described in~\citeptenth{p10Liapis2013-designerModel}, but we propose an alternative route to designer modeling through clustering the design space and the room style based on the collected data. In contrast 

% Their work propose the use of interactive evolution by biasing the search space in favor of hand-crafted features of the design akin to~\citeptenth{p10Alvarez2020-DesignerPreference}. However, we propose an alternative and novel route to designer modeling through clustering the design space and the room style based on the collected data. Moreover, we differ as well on the type of level design, being the sentient sketchbook a tool for strategy games~\citeptenth{p10liapis_generating_2013}, while EDD a tool for adventure and rogue-like games~\citeptenth{p10Alvarez2020-ICMAPE}. Applying the idea of designer modelling to both genres, not only shows the importance and usefulness of designer modeling but also the holistic and generic view 

% These differences strengthen the importance and usefulness of designer modeling, and highlight the holistic and generic properties of this designer-centric perspective.

% % might be interesting to discuss this.
% While the approach described in this paper is applied in a tool for creating zelda-like dungeon games~\citeptenth{p10tloz}, the approach can be reused and extended to other domains 

% These contributions allow us to better understand, cluster, categorize and isolate designer behavior. This is very valuable for mixed-initiative approaches, where a clear virtual model of the designer's style allows us to better drive the search process for procedurally generating content that is valuable for the designer. Designer personas have the potential to be used in many different scenarios. For instance, as objectives for a search-based approach to enable a more style-sensitive system, to evaluate the fitness of evolutionary generated content or to train PCG agents via Reinforcement Learning~\citeptenth{p10khalifa2020-pcgrl}. We aim at implementing our approach in a functional system within EDD to assess and validate the benefits and usability of these adaptable systems with human designers.

%and validate Further experiments are needed in a functional and operational system  with human designers to assess and validate our approach as well as exploring the benefits of these adaptable systems.

% Further experiments are needed to analyze how different dimensions are better at adapting to continuous changes in the target room, which would also indicate better stability in the search. Along these lines, further evaluation is needed with human designers to assess and explore whether IC MAP-Elites is beneficial for the MI-CC workflow and interaction. 

Recognizing the designers' current style and the path taken so far, which would indicate a possible designer persona, would open the possibility for recognizing their intentions, preferences, and goals. This traced roadmap of designer personas could let a content generator anticipate a designer's next moves without heavy computational cost, just by identifying their current location on the map and offering content suggestions that lie in the most promising clusters to be visited next. Conversely, it would also identify designers who do not follow a certain path, i.e., deviating from the pattern, trying to understand their objective through their design style. Finally, we aim at implementing our approach in a functional system within EDD to assess and validate the benefits and usability of these adaptable systems with human designers.


% \textbf{We aim at testing these archetypical paths with users to understand }

% Finally, in our work, we did not observe any type of cross-path i.e. a design going from one path to another. We believe that this is due to the level at which we observe the archetypical paths. However, preliminary analysis on the dataset used in this paper and as expected, the design process of designed rooms within the same dungeon does follow different paths, and sometimes even crossing each other. This opens an interesting and exciting area to explore a wider layer, taking rooms as a set of archetypical paths taken by designers. Observing the paths taken in previous and future rooms, and the dungeon as a whole, as briefly introduced in section~\ref{sec:designStyle}, to understand the designers' intentions and goals when they proceed to create a new room is a promising future step to take with the current system. 

%  i.e. room-wise, as the designer typically would design the room with a set of goals

% Finally, it is also important to observe the nature of the previous and future rooms created by a designer. Observing the dungeon as a whole, as briefly introduced in section~\ref{sec:designStyle}, to understand the designers' intentions and goals when they proceed to create a new room is a promising future step.

%by explore several of the proposed ideas in the article and in the future work, and validate and evaluate the benefits and usability of these adaptable systems. 

% Furthermore, the designer personas addresses the dynamic-dynamic system vs. dynamic-static system open question raised by Alvarez and Font~\citeptenth{p10Alvarez2020-DesignerPreference}, which relates to the challenge of adapting a system to the a ever-changing designer. With the use of the archetypical paths, the model is not anymore adapting and moving through the solution space with the designer, rather the designer traverse through an already clustered space. 

% With the use of the archetypical paths, we can not only identify the current designer persona the designer is following but we can also adapt and anticipate to what they might end up doing. 

% Furthermore, the designer personas addresses an open question raised by Alvarez and Font \citeptenth{p10Alvarez2020-DesignerPreference}, related to the challenges  using a dynamic-dynamic system vs. a dynamic-static system. The authors describe the dynamic-dynamic system as a system where both designer and AI-system move through the solution space, with the AI-system constantly trying to adapt to the designer. They concluded that the main challenge correspond to designers constantly concept drifting resulting in them continuously changing their decisions. Instead, the authors proposed the use of a dynamic-static system, where the model is not anymore adapting and moving through the solution space with the designer, rather the designer traverse through an already clustered space. With the use of the archetypical paths, we can not only identify the current designer persona the designer is following but we can also adapt and anticipate to what they might end up doing. 


%and conclude that the main challenges in

% Moreover, this traced roadmap of designer personas could let a content generator anticipate a designer's next moves without heavy computational cost, just by identifying her current location on the map and offering content suggestions that lie in the most promising clusters to be visited next. %Further, one could also be able to identify designers that do not follow a certain path i.e. deviating from the pattern, and try to understand through their design style their objective.





% From the $180$ unique rooms, we extracted and used the edition sequence of each of the rooms, from their initial design to the more elaborated end-design, to compose a richer dataset that could capture the design process of a designer rather than focusing on the end-point. Through this, we ended up using $8196$ data points in our dataset.

% We have first run and compared s


% through experimenting with 

% This paper presents a step towards designer modelling by providing a prototype implementation of designer personas as archetypical trajectories through style space. These archetypical paths

% This paper presents an experiment on archetypical design trajectories analysis in a MI-CC environment, as a means to characterize several representative design styles as designer personas. We have first run and compared several clustering setups to find the best partitioning, resulting into a set of twelve cohesive, coherent, and meaningful clusters. We have then mapped almost 200 complete design sequences in terms of these clusters, applying sequence mining to find four frequent unique designer styles, with related common sub-styles. As a result, we have presented a roadmap of design styles over a map of data-driven design clusters. 



% be used as goal for other systems were anticipating a design or creating a synthetic objective might be more complicated. We envision that these designer personas can be used within 