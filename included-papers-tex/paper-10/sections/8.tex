\subsection{Discussion}
% \subsection{Discussion - General Takeaways}

% In this paper, we present and thoroughly examine an approach towards

% \begin{itemize}
%     \item Paragraph discussing that we are presenting an alternative approach to designer modeling, which encompass a thorough system and prototype implementation to exemplify the process to create this designer personas and the type of information that can be understood with it, as well as what could it be used for.
%     \item What could this system be used for (extend a bit what appears in conclusions); and what does it show.
%     \item We expect that by presenting an account of the process and procedures (check the difference between those words) to create Designer Personas and model archetypical paths, others can build on top of this; implementing their own approach, identifying designer personas applicable to their domain, genre, and facet.
%     \item We believe that our work allows 
%     \item discuss the combination with other designer models such as designer preferences in EDD, or something from Sentient Sketchbook.
%     \item how the results here can be used to create style-sensitive suggestions
% \end{itemize}

% In this paper, we thoroughly discuss the procedure to create designer personas focused on level design generation. We used a prototype implementation in EDD to exemplify the process and the type of information that can be understood and produced from using the archetypical paths. Our work contributes to an alternative route to designer modeling through clustering the design space and the room style based on level design step sequences. These contributions allow us to better understand, cluster, categorize and isolate designer behavior. This is very valuable for mixed-initiative approaches, where a clear virtual model of the designer's style allows us to better drive the search process for procedurally generating content that is valuable for and aligned with the designers intentions.

% Our work contributes

% % than the one discussed by Liapis et al.~\citeptenth{p10Liapis2014-designerModelImpl},
%  We propose an alternative route to designer modeling through clustering the design space and the room style based on level design step sequences. These contributions allow us to better understand, cluster, categorize and isolate designer behavior. This is very valuable for mixed-initiative approaches, where a clear virtual model of the designer's style allows us to better drive the search process for procedurally generating content that is valuable for and aligned with the designers intentions.

% However, while important for the development of EDD, this is used as a demonstration of what can be done and as an exemplification on the method to model designers' design style based on the content they create and collective data from a group of designers. 

Our work draws on many of Liapis et al.'s~\citeptenth{p10Liapis2013-designerModel,p10Liapis2014-designerModelImpl} ideas, concepts, and goals but differs on the tool and type of game being created and the methods to create designer models. These differences strengthen the importance and usefulness of designer modeling and highlight this designer-centric perspective's holistic and generic properties.

The archetypical paths presented in sec.~\nameref{sec:archetypical-paths} relate to EDD and the options that exist within it. However, while important for EDD's development, this is used to demonstrate what can be done and as an exemplification of the method to model designers. Our method focuses on designer modeling through clustering the design space and the room style based on level design step sequences, resulting in archetypical paths. We leverage a collective dataset to build a collective cluster model of a set of designers and utilize this to mitigate, to some extent, problems regarding lack of individual designer data and continuous model adaptation~\citeptenth{p10Alvarez2020-DesignerPreference}. Our method allows us to better understand, cluster, categorize and isolate designer behavior. A virtual model of the designer's style could allow to better drive the search process for procedurally generating content that is valuable for designers and aligned with the designers' intentions, observed as well by Liapis et al.~\citeptenth{p10Liapis2013-designerModel}.

%that is valuable for and aligned with the designer's intentions as discussed by Liapis et al..


%This could be very valuable for mixed-initiative approaches, where a clear virtual model of the designer's style allows us to better drive the search process for procedurally generating content that is valuable for and aligned with the designer's intentions.


We expect that work on similar domains and tools working with level design could discuss these archetypical paths in relation to what designers create. They can build on and adapt the proposed method to model designers to their respective domains, identifying valuable and applicable designer personas. Then, these could be used to understand what designers create and to design adaptive systems and better Human-AI collaborations and interactions. The resulting Designer Personas have the potential to be used in many different scenarios. For instance, as objectives for a search-based approach to enable a more style-sensitive system, to evaluate the fitness of evolutionary generated content, or to train PCG agents via Reinforcement Learning~\citeptenth{p10khalifa2020-pcgrl}.

%We expect that work on similar domains and tools working with level design could discuss these archetypical paths in relation to what designers create, and create their particular archetypical paths by following the proposed method. Others can also build on and adapt this method to designer modeling to their respective domains, identifying valuable and applicable designer personas. Then, these could be used to understand what designers create and to design adaptive systems and better Human-AI collaborations and interactions. The resulting Designer Personas have the potential to be used in many different scenarios. For instance, as objectives for a search-based approach to enable a more style-sensitive system, to evaluate the fitness of evolutionary generated content, or to train PCG agents via Reinforcement Learning~\citeptenth{p10khalifa2020-pcgrl}.

%I'd watch out for that in the discussion section as well, particularly with the sentence starting with "This is very valuable for mixed-initiative approaches,". Second, the third and final paragraph of the discussion starts with two lengthy and somewhat confusing sentences that may be run-ons. I'd recommend breaking these two sentences into four smaller sentences for readability. Otherwise, I think everything looks good.

% Yet, the work here serves as a demonstration of what can be done and as an exemplification on the method to model designers' design style based on the content they create and collective data from a group of designers.

%  to form designer personas based on level design style; representing design style as archetypical paths traversed through room style clusters.

% Similar domains could take the findings from this work, and use it as a base of the archetypical paths that designers could align with. Yet, the work here serves as a demonstration of what can be done and as an exemplification on the procedure to model designers' design style based on the content they create and collective data from a group of designers. The resulting Designer Personas have the potential to be used in many different scenarios. For instance, as objectives for a search-based approach to enable a more style-sensitive system, to evaluate the fitness of evolutionary generated content or to train PCG agents via Reinforcement Learning~\citeptenth{p10khalifa2020-pcgrl}.

% We thoroughly discuss the method to create designer personas focused on level design generation, using a prototype implementation in EDD to exemplify the process and the type of information that can be understood and produced from using the archetypical paths. We expect that using this as example, others can build and adapt this approach to designer modeling to their respective domains, identifying valuable and applicable designer personas that can then be used to create adaptive systems and better Human-AI collaboration and interaction.


% We expect that using this as example, others can build and adapt this approach to designer modeling to their respective domains, identifying valuable and applicable designer personas that can then be used to create adaptive systems and better Human-AI collaboration and interaction.

% We expect that work on similar domains and tools that aim at working with level design could discuss these archetypical paths in relation to what designers, and by following the method proposed here, they could create their particular designer personas. 





% Not only the method (how to do it)! but also the \textbf{IDEA} to form designer personas based on level design style; representing design style as archetypical paths traversed through room style clusters.

% represent designer style as archetypical paths using their design steps as a trail 

% Also the collective model! One can think on modelling designers exclusively; problems with the amount of data!~\citeptenth{p10Alvarez2020-DesignerPreference} However, in this paper we mitigate this by leveraging on a collective dataset and building a collective cluster model. This, in turn, shows 

% Furthermore, the archetypical paths presented in~\ref{sec:archetypical-paths} relate to the domain of EDD and the alternatives that exist within it. 

% These archetypical paths 





% The work by Liapis et al. differs from ours in 

% Our work draws on many of Liapis et al.'s~\citeptenth{p10Liapis2013-designerModel} ideas, concepts, and goals. We differ in the type of level design, being the sentient sketchbook a tool for strategy games~\citeptenth{p10liapis_generating_2013}, while EDD is a tool for adventure and dungeon crawler games~\citeptenth{p10Alvarez2020-ICMAPE}, and the way. These differences strengthen the importance and usefulness of designer modeling and highlight the holistic and generic properties of this designer-centric perspective.% and its possibilities.


% that with this, others 

% others can build on top of this; implementing their own approach, identifying designer personas applicable to their domain, genre, and facet.


% However, \textbf{and now we add info about other domains}



% They are very useful 



% While our results and designer personas can be used directly in EDD, we expect that 



% By accounting the 


% In this paper, we present the procedure to create such a system


% Our work draws on many of Liapis et al.'s~\citeptenth{p10Liapis2013-designerModel} ideas, concepts, and goals. Our work contributes to an alternative route to designer modeling through clustering the design space and the room style based on the collected data. 


% These contributions allow us to better understand, cluster, categorize and isolate designer behavior. This is very valuable for mixed-initiative approaches, where a clear virtual model of the designer's style allows us to better drive the search process for procedurally generating content that is valuable for the designer. Designer personas have the potential to be used in many different scenarios. For instance, as objectives for a search-based approach to enable a more style-sensitive system, to evaluate the fitness of evolutionary generated content or to train PCG agents via Reinforcement Learning~\citeptenth{p10khalifa2020-pcgrl}.However, \textbf{and now we add info about other domains}

% Moreover, we differ in the type of level design, being the sentient sketchbook a tool for strategy games~\citeptenth{p10liapis_generating_2013}, while EDD is a tool for adventure and dungeon crawler games~\citeptenth{p10Alvarez2020-ICMAPE}. These differences strengthen the importance and usefulness of designer modeling and highlight the holistic and generic properties of this designer-centric perspective.% and its possibilities.




% We aim at implementing our approach in a functional system within EDD to assess and validate the benefits and usability of these adaptable systems with human designers. 


% The reason we extend on describing everything, not only the selected approach is to have a thorough documentation on how and why we did this, so that others can build on top of and extend our approach and system.

% The reason we extend on describing everything, not only the selected approach is to have a thorough documentation on how and why we did this, so that others can build on top of and extend our approach and system.