\subsection{Conclusion}

In this paper, we have presented the Designer Preference Model, which is a data-driven system that learns an individual designer's preference through the designer’s proactive choosing of generated suggestions without disrupting the continuous reciprocal workflow in MI-CC. We implemented our approach in the Evolutionary Dungeon Designer, a Quality-Diversity MI-CC tool, where designers can create dungeons and rooms while the underlying evolutionary system provides suggestions adapted to their current design. 

We used the model as a complementary evaluation system to the fitness function of the suggestions in a weighted sum, where the model gained more weight as it became more confident and performed better. Therefore, we aimed at better assessing these provided suggestions with the use of the Designer Preference Model, for them to be interesting and preferable but still usable for designers.

Through our experiments and preliminary studies on using the model to adapt to different designers, we identified a set of challenges and open areas for active research that integrates MI-CC with PCG through Machine Learning. Those challenges relate to the amount of user data needed to accurately learn from the user's preferences, what type of data is needed from the process, the cold start problem, the seldom collection of data to train, the quality of the dataset, and the designer-model setup. Moreover, we wanted to come closer to machine teaching~\citepfifth{p5simard2017machineTeaching} approaches where the human provides fewer data points but with higher quality (i.e. the necessary data to correctly learn) rather than classic approaches to ML (i.e. offline training with a substantial amount of data). In our approach, while the designer has the decision on when to train the algorithm and to a certain extent, with what data to train, we are still missing certain granularity to empower designers to give the right information to the algorithm.

The combination of MI-CC tools with PCG through Machine Learning is a promising area of research that has the potential to enhance content creation. Specifically, designer modeling and our approach to model the designer's preference can have a great impact on the creative process of designers by considering their preferences, intentions, and objectives into the loop, by adapting the workflow to their requirements, or by smoothing the communication among various designers.

Finally, by adding the preference model as a complementary evaluation to the generated suggestions of the evolutionary algorithm, we can give more control, to a certain extent, to the designers over the evaluation of the individuals. In consequence, we can generate higher quality suggestions that better fit a specific designer.% and through using the learned preferences we are  by means of and through this, we 

%we are able to copethe usability of the EA, since the generated suggestion and the usability of through adapting solutions and  

%However, there remain two challenges (1) the long-time it takes for it to converge into an effective reference model, and (2) how sensible the model is to be disrupted by very different training input. 