\subsection{Introduction}

As game production grows, so does the usage of computer-aided design (CAD) tools to develop various facets of games. CAD tools enable users to create new content or refine previously created content with the assistance of some type of technology that focuses on reducing the workload of the developer. Procedural Content Generation (PCG) denotes the use of algorithms to generate different types of game content, such as levels, narrative, visuals, or even game rules, with limited human input \citepfifth{p5shaker_procedural_2016}. Search-based PCG is the subset of techniques whose approach generates content by using a search algorithm, a content representation mechanism, and a set of evaluation functions to drive the content creation process towards near-optimal solutions \citepfifth{p5Yannakakis2018}. 

Mixed-initiative co-creativity (MI-CC)~\citepfifth{p5yannakakis2014micc} is a branch of PCG through which a computer and a human user create content by engaging into an iterative reciprocal stimuli loop~\citepfifth{p5shaker2013ropossum,p5smith_tanagra:_2011,p5machado2019pitako,p5liapis_generating_2013,p5guzdial-lvldsg-aiide-2018,p5lucas-3buddy-iccc2017}. This approach addresses the design process with insight and understanding of the affordances and constraints of the human process for creating and designing games \citepfifth{p5Liapis2016}. MI-CC helps designers to either optimize their current design towards a specific goal (thus exploiting the search space) or foster their creativity by proposing unexpected suggestions (exploring the search space). To these ends, diversity has been an important feature for the research community to focus on during the past decade, including novelty search~\citepfifth{p5Novelty-Lehman2011}, surprise~\citepfifth{p5Surprise-Gravina2016}, curiosity~\citepfifth{p5CuriositySearch-Stanton} and, more recently, quality-diversity approaches \citepfifth{p5Khalifa2018}. 

PCG through Quality-Diversity (PCG-QD) \citepfifth{p5gravina2019procedural} is a subset of search-based PCG, which uses quality-diversity algorithms~\citepfifth{p5Pugh2016} to explore the search space and produce high quality and diverse suggestions. MAP-Elites \citepfifth{p5Mouret2015} is a successful quality-diversity algorithm that maintains a map of good suggestions distributed along several feature dimensions. A constrained MAP-Elites implementation was presented by Khalifa et al.~\citepfifth{p5Khalifa2018}, combining MAP-Elites with a feasible-infeasible (FI2Pop) genetic algorithm~\citepfifth{p5Kimbrough2008} for the procedural generation of levels for bullet hell games. The first implementation of a PCG-QD algorithm for MI-CC was presented by Alvarez et al. \citepfifth{p5alvarez2019empowering}, elaborating on the combined MAP-Elites and FI2Pop approach by introducing a continuous evolution process that benefits from the multidimensional discretization of the search space performed in MAP-Elites.

In all the above MI-CC approaches, the designers play an active role in the procedurally generated content while struggling between the expressiveness of the automatic generation and the control that they want to exert over it \citepfifth{p5Alvarez2018}. Having this as motivation, this paper takes the work in \citepfifth{p5alvarez2019empowering} one step forward by adding an underlying interactive PCG via machine learning algorithm \citepfifth{p5summerville2018procedural}, the Designer Preference Model, that models the user's design style, to be able to predict future designer's choices and thus, driving the content generation with a combination of the designer's subjectivity and the search for quality-diverse content.

% , which have been concluded in several studies [ref]. 

% [ref to picbreeder, novelty search picking paper, spaceship generation]. 

