\begin{table*}[ht]
% \centering
\caption{General consensus on EDD's features} \label{p1tab:consensus}
\resizebox{0.8\textwidth}{!}{
\begin{tabularx}{\textwidth}{|p{0.2\textwidth}|p{0.99\textwidth}|}
\cline{1-2}

Description & Participants’ Consensus \\\cline{1-2}
World Grid  of the dungeon                   & Its purpose of establishing an illusion of a fully realized dungeon is somewhat achieved. However, limitations exist with how it defines feasibility, a dungeon’s starting point, and the entrances, which disrupts the designers’ decisions.                                                                                                                                                                                   \\\cline{1-2}
World View                                  & The world view’s usefulness for the most part could not be established, other than for the purpose of going to the suggestions view (which was already seldom during the user study) and having a closer look at the entire dungeon without any distractions. Some participants preferred features to be already in the room view’s minimap, and some wanted to see more specific functionalities within the world view itself. \\\cline{1-2}

Enabling and  \newline disabling rooms                & As the user study restricted participants to create 3x3 dungeons, this feature for the most part has been neglected. This is also in part because of its accessibility only being in the world view, which proved to be an inefficient view in general. However, its use for bigger dungeon sizes later on was appreciated, especially for more intricate design purposes.                                                      \\\cline{1-2}
Suggestions View                            & Similarly to enabling and disabling rooms, it was quite difficult to encourage the use of this functionality due to the world view’s inefficient usability. However, this could also be due to the dungeon’s small size, as some participants expressed high interest in using more suggestions with larger dungeon sizes.                                                                                                      \\\cline{1-2}
Minimap  \newline  navigation                      & The minimap proved to be a strong tool not only for navigation purposes, but also for supporting design decisions and choices. The directional buttons were rarely used, but their room previews were helpful in emphasizing the current room’s connection to adjacent rooms without looking at the minimap. On the other hand, this lowered the usability of the world view.                                                   \\\cline{1-2}
Parameters                                       & The parameters were, in general, lacking. They served to be important in decision-making when choosing a suggested map in room view, but there were still doubts on their accuracy and sufficiency when providing information about the generated suggestions.                                                                                                                                                                       \\\cline{1-2}
Generated maps for  \newline  suggestions in room view & Suggestions in the room view proved to be very helpful in supporting the whole design process as they primarily acted as inspirations for the users. The most prominent comment among the users is the preference of having more control on how suggestions should be generated depending on different types of parameters.                                                                                                     \\\cline{1-2}
Design \newline  patterns& The patterns’ visualization was, in general, lacking and not self-explanatory. Some participants have expressed interest in using patterns as a parameter in the generation of suggestions.                                                                                                                                                                                                                                     \\\cline{1-2}
Dark theme                                  & EDD’s dark theme for the user interface received a positive response as it makes working with the program easier.
	\\ \cline{1-2}
\end{tabularx}
}
\end{table*}

% \begin{table}
% \begin{center}
% {\caption{Best performing setups based on their internal validation and visualization of clustered data points.}\label{table:setups}}
% \resizebox{\textwidth}{!}{
% \begin{tabular}{ccccccc}
% \hline
% \rule{0pt}{12pt}
% Algorithm&Data&K&$\Diamond$&$\Box$&$\bigtriangleup$ 
% \\ 
% \hline
% \\[-6pt]
% K-Means & Tiles-PCA & 9 & 0.43 & 0.73 & 9438.233 \\ 
% K-Means & Tiles-PCA & 12 & 0.41 & 0.77 & 9436.928 \\
% K-Means & Dimensions-PCA & 12 & 0.43 & 0.73 & 7738.343 \\
% Agglomerative single & Combined-PCA & 6 & 0.51 & 0.43  & 38.833 \\ 
% Agglomerative avg. & Dimensions-PCA & 6 & 0.44 & 0.67 & 3463.567 \\ 
% \hline
% \\[-6pt]
% \multicolumn{6}{l}{$\Diamond$ Silhouette Score\ \
% $\Box$ Davies Bouldin Index\ \
% $\bigtriangleup$ Calinski-Harabasz Index}
% \end{tabular}
% }\end{center}
% \end{table}