\section{INTRODUCTION} \normalfont 
\label{sec:intro}
% \addcontentsline{toc}{section}{\nameref{sec:intro}}

%\epigraph{\textit{During the first three millennia, the Earthmen complained a lot.}}{John McCarthy}

% \setlength{\epigraphwidth}{3.5in} 
% \epigraph{\textit{There's a difference between knowing the path and walking the path.}}{Morpheus, The Matrix}


% \setlength{\epigraphwidth}{3in} 
% \epigraph{\textit{We can only see a short distance ahead, but we can see plenty there
% that needs to be done.}}{Alan M. Turing, Computing Machinery and Intelligence}

\setlength{\epigraphwidth}{4in} 
\epigraph{\textit{We all test the rules, and consider bending them; even a saint can appreciate science fiction. We add constraints [...] to see what happens then. We seek the imposed constraints [...] and try to overcome them by changing the rules. We follow up hunches [...], and - sometimes - break out of dead-ends. Some people even make a living out of pushing the existing rules to their limits, finding all the computational 'cans' that exist: creative tax-lawyers call them loopholes (and creative tax-legislators close them).}}{Margaret Boden \\ The Creative Mind: Myths and Mechanisms, pp. 58}

% \begin{dissQuote}{\~{}John McCarthy}
%     During the first three millennia, the Earthmen complained a lot.
% \end{dissQuote}

% \subsection{Tools}
% John McCarthy was spot on, as we humans, are a main source of complain. Nevertheless, such complains makes us strive and search for solutions and better approaches to cope with our needs and objectives. Ironically, we would end up complaining about it, restarting the loop.

\setlength{\parindent}{0.0em}

%John McCarthy was spot on, as we humans are a major source of complaint. Nevertheless, such complaints make us strive and search for solutions and better approaches to cope with our needs and objectives. Ironically, we would end up complaining about it, restarting the loop.

%Indeed Morpheus, there is a big difference between both! 

%We are all creative creatures 



%Look, we all want to feel special but the truth is, everyone is creative. Creativity is a human quality that exists in every single one of us. The degree of your or anyone else’s creativity doesn’t depend on an innate quality but rather on how hard you work.

%You could be as talented as Pablo Picasso, but if you don’t pump out the product, who cares? Your creativity is worthless

%While this might make people who have failed out creative fields (read: didn’t work hard enough) feel better, creativity is not something that a person is born with.

%Jory MacKay points out in his piece on Crew that not only is creativity taught in colleges and universities all over the world, but you can actually train your own brain to be more creative simply by doing creative work over a long period of time.

%If creativity was something we were just innately born with, that just wouldn’t be true

%Creativity

% Creativity is ``... the ability to produce work that is both novel (i.e., original, unexpected) and appropriate (i.e., useful, adaptive concerning task constraints)~\cite{sternberg_concept_1999}''. But what does that really means?

% \setlength{\parindent}{0.9em}

%and this happens all the time.

% Creativity is an outstanding ability that we possess. Some believe that they are not creative creatures, but in reality, and being candid about it, we are all creative and creative actions are taken all the time. We are all confronted with challenges and situations that require us to be creative on how we approach them. Whether this means being creative for artistic purposes such as painting or developing games, or to question your field and write a dissertation in political science, or to create variations on how to write your name; these all require creativity. Creativity varies on the task, process, outcome, and personal perception, which might be why people feel as non-creative~\cite{kaufman_beyond_2009}. Nevertheless, creativity as our other abilities, can be refined, developed, and fostered. 

% On the other hand, \textit{Computational Creativity} is the study of computational systems that demonstrate some human-like creative behaviors~\cite{colton_computational_2012}. We can imagine these as systems that can write stories~\cite{perez_y_perez_mexica_2001}, design games~\cite{cook_angelina_2016}, or SOMETHING MORE BASIC!. 

% Creativity is instrumental for the everyday life, exploration of ideas, and problem solving. Game design is a creative task that requires multi

% \begin{retQuestion}{}
%  How can we create tools that no longer behave just as aid to support our work but can collaborate with us, to some extent, in the same way as human collaboration functions? 
% \end{retQuestion}

% This thesis focuses on how game design related tasks such as the creation of game facets or designers' design process could be explored through Human-AI collaborative tools. While Procedural Content Generation algorithms could be employed by themselves for automated game design~\cite{nelson_towards_2007,cook_software_2020,cook_getting_2020}, this thesis takes a different approach and explores game design; and the creative, organizational, and collaborative tasks that encompass that; through Human-AI collaboration. Through this, we focus on fostering, enhancing, and augmenting human's capabilities such as creativity and adaptability 

% this thesis focuses on exploring multiple approaches for Human-AI collaboration. The goal is to develop systems and algorithms representing a computational designer to collaborate in the creation of content with a human designer. Tasks that could not only be assisted by AI but rather AI could be a \emph{colleague} in the design process. To tackle these shared tasks, a mutual feedback loop could be established, whereby AI and humans could inspire each other to explore unknown areas in the design landscape and reach better and more creative solutions.

% Using these tools 

% This thesis focuses on exploring how, through Human-AI collaborative tools, it would be possible to foster, enhance, and augment human's capabilities such as creativity. We explore this 

% Creativity is ``the ability to produce work that is both novel (i.e., original, unexpected) and appropriate (i.e., useful, adaptive concerning task constraints)~\cite{Sternberg1999-CreativityConcept}''. How creative processes occur, how an individual might come up with novel ideas, or how to assess creativity is very much an open research area~\cite{sternberg1999-handbookCreativity,boden2004-creative,Sternberg2005-creativityCreativities,Csikszentmihalyi97-Creativity}. Moreover,~\acrlong{cc} is a multidisciplinary field that studies computational systems that demonstrate human-like creative behaviors~\cite{Colton2012-CC}. As a multidisciplinary field,~\acrshort{cc} is not only interested in the algorithms or the outcome; it also aims to study the creative process and psychological causes of creative behaviors. Thus, through~\acrshort{cc}, some core concepts and research areas in creativity can be addressed. For instance, in \textit{the Creative Mind: Myths and Mechanism}, Boden studies and analyzes \emph{Creativity} and \emph{creative behaviors} with the use and help of~\acrshort{ai} through the lenses of~\acrlong{cc}. Boden discusses three forms of creativity: \textit{combinatorial}: combining existing knowledge in unfamiliar ways to produce new artifacts; \textit{exploratory}: exploring the conceptual space to encounter possible ideas;~\textit{transformational}: transforming the conceptual space, the imposed constraints, and the encountered ideas~\cite{boden2004-creative}.

% Within~\acrshort{cc}, games have been proposed as the optimal artifact to create to test the creative-like abilities of a~\acrshort{cc} system, since games are \emph{content-intensive}, \emph{multi-faceted content}, and should be~\emph{interacted with and experienced}~\cite{Liapis2014-gameCreativity}. As described above, game content relates to the main facets that represent any game: audio, visuals, narrative, levels, rules, and gameplay~\cite{Liapis2019-OrchestratingGames}. Thus, creating systems that develop, to some extent, games poses an interesting application and challenge for~\acrshort{cc}, which can address some of the core questions in~\acrshort{cc}. For instance, investigating the creative process not only to create one type of content but the arrangement of such in a harmonious way as a team of humans creatively does, or the assessment of such content.



%, and we areall the time. 

% \setlength{\parindent}{0.9em}

Since the dawn of time, we humans have been searching [and in need] for tools to develop our ideas or execute mundane objectives. As time and technology advanced, more sophisticated types of assistance emerged to cope with humans' needs, such as vehicles to traverse longer paths or ways to facilitate writing. With the invention of hardware and software, its ubiquity, and the raise of~\acrfull{ai}, a new path for human assistance opened up. Tools that were used to facilitate our work or assist us into doing repetitive work, could now provide advance assistance with smarter tools that allows us to work more efficiently. However, tools that assist us in our tasks are not the only key factor; the collaboration between humans has remained virtually unchanged as an essential way to move forward and to develop new experiences. Not only to achieve greater objectives as a group but also to develop as individuals. While the current tools to support humans' work and creative output are valuable in many ways; this raises an essential question that holistically motivates this thesis:

\setlength{\parindent}{0.9em}

\begin{retQuestion}{}
 How can we create tools that no longer behave just as aid to support our work but can collaborate with us, to some extent, in the same way as human collaboration functions? 
\end{retQuestion}

This thesis focuses on how game design related tasks such as the creation of game facets or designers' design process could be explored through Human-AI collaborative tools. While Procedural Content Generation algorithms could be employed for automated game design~\cite{nelson_towards_2007,cook_software_2020,cook_getting_2020}, this thesis takes a different approach and explores game design; and the creative, organizational, and collaborative tasks that encompass that; through Human-AI collaboration. Through this, we focus on fostering, enhancing, and augmenting human abilities such as creativity. Multiple approaches, systems, and algorithms representing a computational designer to collaborate in the creation of content with a human designer are developed. These tasks could not only be assisted by AI, but rather AI could be a \emph{colleague} in the design process. To tackle these shared tasks, a mutual feedback loop could be established, whereby AI and humans could inspire each other to explore unknown areas in the design landscape and reach better and more creative solutions.

% Since the dawn of time, we humans have been searching [and in need] for tools to develop our ideas or execute mundane objectives. As time and technology advanced, more sophisticated types of assistance emerged to cope with humans' needs, such as vehicles to traverse longer paths or ways to facilitate writing. With the invention of hardware and software, its ubiquity, and the raise of~\acrfull{ai}, a new path for human assistance opened up. Tools that were used to facilitate our work by doing impossible things for us, e.g., move 500 km in a day, or assist us into doing repetitive work, could now provide advance assistance with smart tools that allows us to work and explore differently and more efficiently (Unity, 2005; Photoshop, 1990). However, tools that assist us in our tasks are not the only key factor; the collaboration between humans has remained virtually unchanged as an essential way to move forward and to develop new experiences. Not only to achieve greater objectives as a group but also to develop as individuals. While the current tools to support humans' work and creative output are valuable and helpful in many ways; this raises an essential question that holistically motivates and drives this thesis:


% Rather than having tools that facilitate our work by doing impossible things for us, e.g., move 500 km in a day, or assist us into doing repetitive work, they can now provide advance assistance with smart tools that allows us to work and explore differently and more efficiently (Unity, 2005; Photoshop, 1990).

% \begin{retQuestion}{}
%  How can we create tools that no longer behave just as aid to support our work but can collaborate with us, to some extent, in the same way as human collaboration functions? 
% \end{retQuestion}

% This thesis focuses on exploring multiple approaches for Human-AI collaboration. The goal is to develop systems and algorithms representing a computational designer to collaborate in the creation of content with a human designer. Tasks that could not only be assisted by AI but rather AI could be a \emph{colleague} in the design process. To tackle these shared tasks, a mutual feedback loop could be established, whereby AI and humans could inspire each other to explore unknown areas in the design landscape and reach better and more creative solutions.

\subsection{Problem Statement} \label{sec:problemst}

The proposed question is not new and has been approached by different disciplines, under the~\acrfull{mi} paradigm.~\acrshort{mi} refers to the collaboration between \emph{human} and \emph{computer} where both have some proactive initiative to solve some task.~\acrshort{mi} can be seen as a multi-agent collaboration scenario, where the interaction should be flexible, allowing for a continuous negotiation of initiative and leverage on each other's strengths to solve a task~\cite{allen_mixed-initiative_1999}. \emph{Initiative} was described by Novick and Sulton as a multi-factor model that combines: choosing the task, choosing the agent in control and how the interaction is established, and choosing the expected outcome from the collaboration~\cite{novick_what_1997}. 

Moreover, Horvitz discussed such a question in terms of Intelligent User Interfaces~\cite{birnbaum_compelling_1997}, describing mixed-initiative systems and interfaces as a more natural collaboration in a user interface that emerges from intertwining human control and manipulation, and automation~\cite{horvitz_uncertainty_1999}. Horvitz presented several principles of mixed-initiative interaction and its challenges, many of which still exist~\cite{horvitz_principles_1999}, mainly describing this interaction as conversation systems between AI and humans~\cite{horvitz_computational_1999}. Moreover, Yannakakis et al. introduced the~\acrfull{micc} paradigm for the co-creation of creative content, where both AI and humans alternate in the initiative to co-design and solve tasks~\cite{yannakakis_mixed-initiative_2014}. Their work describes key findings and discussions for how MI-CC does not only help human designers solve tasks, but also fosters their creativity through an interactive feedback loop and lateral thinking~\cite{liapis_can_2016,liapis_computational_2014,alvarez_fostering_2018}.


% ~\cite{Liapis2016-CanComputersFosterCreativity,Liapis2014-gameCreativity,Alvarez2018}. 

Paramount is the role of the computer agent in this interaction, as it would help establish the boundaries of the interaction, what is expected, and how creativity could be fostered. Lubart analyzed this interaction and examined the different ways computers could be involved in creative work to promote creativity. In his work, he proposed four roles: \emph{computer as nanny}: management of creative work; \emph{computer as pen-pal}: communication service between collaborators; \emph{computer as coach}: Using creative enhancement techniques; and \emph{computer as colleague}: partnership between computer and humans~\cite{lubart_how_2005}. Recently, this was explored by Guzdial et al., where designers perceived the AI collaborator with more or less value depending on their desired role for the AI, varying between: \emph{friend}, \emph{collaborator}, \emph{student}, or \emph{manager}~\cite{guzdial_friend_2019}.

Nevertheless, this collaborative approach raises an \emph{initiative} challenge for either agent: Which agent should have the initiative at different stages of the development and over the goal? The question reflects the diffuseness of the challenge and situation, as many factors need to be considered before appropriately indicating this. At the very least, some could say that depending on the task to be performed and the expertise of both, either would clearly be the one taking the development initiative. Whereas others would position the human as the one always in control. Yet, even with a clear answer, what happens in creative tasks to the expressivity of one of the sides due to the other taking the initiative? 

Furthermore, one context where the~\acrshort{mi} paradigm would be very beneficial is games. Games, either digital or tabletop, are created through a complex creative process that couple together many different creative facets in different ways. Games contain a large amount of creative content carefully combined and intertwined to craft specific experiences, with the addition of rules that dictate how a player is to interact with it. In contrast with other creative content, games are multifaceted, content-intensive, and should be interacted, experienced, and enjoyed by others, which also creates a complex subjective task~\cite{liapis_computational_2014}. Usually, games are developed by more than a person (although many exceptions exist~\cite{minecraft,undertale,stardewvalley}), reaching to hundreds and thousands of developers, with each developer specialized in different areas such as gameplay, AI, animation, concept art, etc. Each creates a specific part of the game and the content through collaboration and following a road map~\cite[Chapter~14]{fullerton_game_2004}. However, no matter the team's size and talent, the fact remains that developing games is a hard challenge~\cite{blow_game_2004}. As technology advances, the requirements increase substantially for any game facet, coupled with the users' increase demand, the higher competitiveness in the market, and the launch of many more platforms~\cite{washburn_jr_what_2016}.

\acrfull{pcg} is a field within computational intelligence in games, that focuses on the use of algorithms to create game content~\cite{yannakakis_artificial_2018}.~\acrshort{pcg} algorithms have been used to aid in the creation of a plethora of games such as No Man Sky~\cite{nomansky}, Spelunky~\cite{spelunky}, or Minecraft~\cite{minecraft}, to the extent that~\acrshort{pcg} and~\acrshort{ai} have enabled experiences and interactions that were not possible before~\cite{aidungeon,rogue,elite}. Moreover, as one of the properties of~\acrshort{pcg} is to increase replayability by creating an abundance of well-made content~\cite{shaker_procedural_2016}, games are not the only beneficiaries of~\acrshort{pcg} methods. For instance, they have the opportunity to be used to increase the generality of~\acrfull{ml} approaches~\cite{risi_increasing_2020}, or a step towards open-endedness~\acrfull{ec}~\cite{clune_ai-gas_2019}. 

% In this thesis, the focus is on exploring multiple approaches for the collaboration between AI and humans to co-create game content in a~\acrlong{micc} paradigm. The goal is to develop systems, techniques, and algorithms representing a [creative] computational designer to tackle the different tasks in game design and development. Tasks that could not only be assisted by AI but rather AI could be a \emph{colleague} in the creative process. To tackle these shared tasks, a mutual feedback loop could be established, whereby AI and humans could inspire each other to explore unknown areas in the design landscape and reach better and more creative solutions.

% \subsection{Problem Statement} \label{sec:problemst}

% Furthermore, there has been a significant effort to analyze the possibilities of such a collaboration, resulting in multiple Ph.D. theses exploring and focusing on different aspects of~\acrshort{mi}~\cite{SmithPhD,LiapisPhD,ComptonPhD,GuzdialPhD,MachadoPhD}. Thus, there is a growing interest in how to establish mixed-initiative interactions and new ways of humans and AI to collaborate. Earlier work focused on tools that supported the work of humans, combining the strengths of both. Later on, research demonstrated that such interaction could enable the completion of tasks not previously able to be done by either alone, and enable creative work to be stimulated. 

Moreover, in design and creative tasks such as games, the designer usually has intentions in what they are creating and goals that they want to achieve with their design. Thus, to enable deeper~\acrshort{mi} levels to co-create content, some control mechanisms with a varying degree of control over the algorithms might be necessary for the designer. Through this, the designer could direct or constraint the generated content by the computational designer and oversee that it is within their intentions and goals. In this case, each agent's control and expressive properties are at the expense of the other agents, as it constrains the space of possibilities~\cite{baldwin_mixed-initiative_2017}. This is especially relevant when the aim is a creative work such as games, where the creative expression needs to be fostered~\cite{alvarez_fostering_2018}. Yet, it becomes particularly challenging when using mixed-initiative methods, where smart approaches need to be in place for a natural conversation and successful collaboration. The more control is given, the more constraint it exists, but is this a problem? Is it inevitable? Boden explains it conspicuously ``...  We [humans] seek the imposed constraints [...], and try to overcome them by changing the rules.~\cite{boden_creative_2004}''. Constraints limit the space, and as a consequence, they are overcome by encountering creative solutions.

For this interaction to be complete, the human needs to understand the AI's behavior through interpretable and explainable models and systems, and the AI needs to recognize and interpret the intentions of the humans seamlessly as they create their content. The former is the focus of~\emph{Interpretable} and \emph{Explainable AI}, which seeks to create or adapt models and systems for a better workflow between humans and AI, where humans could understand the AI's decision process to enable trust relationships and reach deeper interactions~\cite{zhu_explainable_2018,doshi-velez_considerations_2018,adadi_peeking_2018}. The latter would mean that the AI could adapt its behavior and functionality to the needs, expertise, and workflow of individual designers or a specific group of designers. To do so, the AI must analyze several design processes, such as the designer's preferences, styles, and goals, which holistically is called \emph{Designer Modeling}~\cite{liapis_designer_2013,liapis_designer_2014}. How to create these models and use them to develop adapted experiences is a complex challenge, and understanding the implications of its usability in the control-expressive properties, as well as other consequences, is not trivial.

To explore this, the main body of work presented in this dissertation is applied and evaluated through the~\acrfull{edd}, a~\acrlong{micc} system, where designers can create levels for rogue-like and adventure games such as Zelda~\cite{tloz} or The Binding of Isaac~\cite{bindingISAAC}. In~\acrshort{edd}, the human designer can quickly create interconnected rooms forming a dungeon to be experienced by players. Meanwhile, the computational designer collaborates by providing suggestions using different algorithms and following multiple heuristics. The human designer can interact in several ways with the computational designer so that this adapts its output to whatever goal the human designer has, while still providing a diverse amount of alternatives and different experiences to the human designer.

% Interactive systems can help humans complete creative tasks that are tedious or require substantial human workload. Such creative tasks could be as simple as drafting a grammar corrected text (e.g., MS Word), to more complex scenarios such as creating visual artifacts (e.g., Adobe Photoshop) or designing entire video games (e.g., Unity Engine).~\acrlong{micc} is a paradigm within~\acrshort{pcg}, where both human and AI proactively collaborate to co-create and co-design games or creative content. Through~\acrshort{micc}, the potential of interactive tools increases substantially.~\acrshort{micc} enables a new way of tackling creative tasks engaging in-depth humans and AI rather than just helping humans to complete tasks and reduce their workload. Furthermore, enabling a mutual feedback loop could also foster both participants' creativity, create adaptive experiences for the users, or focus on achieving tasks with better and interesting results in a hybrid format. Multiple approaches have been proposed as alternatives for creating systems that model the interaction between AI and humans to create game content, and that use different techniques to study such interactions and its implications~\cite{Alvarez2020-ICMAPE,smith_tanagra:_2011,Liapis2013-sentientsketchbook,charity2020baba}. 



% In this thesis, the focus is on exploring multiple approaches for the collaboration between AI and humans to co-create game content in an~\acrlong{micc} paradigm. The goal is to develop systems, techniques, and algorithms representing a [creative] computational designer to tackle the different tasks in game design and development. Tasks that could not only be assisted by AI but rather AI could be a \emph{colleague} in the creative process. To tackle these shared tasks, a mutual feedback loop could be established, whereby AI and humans could inspire each other to explore unknown areas in the design landscape and reach better and more creative solutions.



% Nevertheless, this collaborative approach divided into human control and AI automation as presented by Horvitz~\cite{Horvitz99-mixedInit} with multiple initiatives at different points of the development as discussed by Yannakakis et al.~\cite{yannakakis2014micc}, raises a controllability challenge for either actor: Which of the two should have the control at different steps of the development and over the goal? There is no real answer since it is very diffuse, and many factors need to be consider before appropriately indicating this. At the very least, some could say that depending on the task to be performed and the expertise of both, one or the other would clearly be the one in control of the development, whereas others would clearly position the human as the one in control. Yet, even if the answer would be clear, what happens to the expressivity of one of the sides as a consequence of the other controlling? The more control is given, the more constraint it exist, but is this a problem? Is it inevitable? At last, Boden explains it quite clear ´´...  We seek the imposed constraints [...], and try to overcome them by changing the rules.''~\cite{boden2004-creative}

% Point out that this collaborative search for human control and AI automation with multiple initiatives at different points of the development, give raise to a controllability challenge for either two actors. Moreover, when we discuss content generation 

% If the aim of this research area is to push harder for mixed-initiative tools, where more autonomy is given to the AI, and for humans to consider the AI as a collaborator as described by Lubart~\cite{LUBART2005-computerPartners} and Guzdial et al.~\cite{guzdial2019friend} that can be taken serious and used its input as a key factor in the development of any type of content. Then we are required to develop AIs and tools that not only provides interesting and valuable input to the human, but also adaptive experiences that 


% This collaboration may take different forms as studied by Lubbart~\cite{LUBART2005-computerPartners}, and recently explored by Guzdial et al.~\cite{guzdial2019friend}.


% \subsection{Computational Creativity and Games}

% Games, either digital or tabletop, are created through a complex creative process that couple together many different creative facets in different ways. Games contain a large amount of creative content carefully combined and intertwined to craft specific experiences, with the addition of rules that dictate how a player is to interact with it. In contrast with other creative content, games are multifaceted, content-intensive, and should be interacted, experienced, and enjoyed by others, which also creates a complex subjective task~\cite{Liapis2014-gameCreativity}. Usually, games are developed by more than a person (although many exceptions exist~\cite{minecraft,undertale,stardewvalley}), reaching to hundreds and thousands of developers, with each developer specialized in different areas such as gameplay, AI, animation, concept art, etc. Each creates a specific part of the game and the content through collaboration and following a road map~\cite[Chapter~14]{fullerton2004-gamedesign}. However, no matter the team's size and talent, the fact remains that developing games is a hard challenge~\cite{Blow2004-gamesHard}. As technology advances, the requirements increase substantially for any game facet, coupled with the users' increase demand, the higher competitiveness in the market, and the launch of many more platforms~\cite{Washburn2016-gamesPostmorten}.

% Moreover,~\acrfull{cc} is one of the grand challenges of~\acrshort{ai}, where the quest is to study, develop, and build computational systems that demonstrate creative behaviors and can create multiple types of artifacts such as games or stories~\cite{Colton2012-CC}. Within the various content that can be created, games offer a unique property that distances them from other creative outputs, making them more interesting to be analyzed and experimented on than others~\cite{Liapis2014-gameCreativity}. They offer a set of intertwined facets that represent any game: audio, visuals, narrative, levels, rules, and gameplay~\cite{Liapis2019-OrchestratingGames}, whereas other creative content focuses almost exclusively on a single-aspect, e.g., music or dance. However, single-aspect creative content has its own set of challenges. For instance, creating a system that creates music could mean creating each separated instrument, music sheet, arrangements, etc~\cite{HooverPhD}. Furthermore, Games must be interacted with and enjoyed by others than the developers, and the fact that these facets must naturally fit each other poses them as a very exciting application to be researched and developed within the~\acrshort{cc} field.

% \acrfull{pcg} is a field within computational intelligence in games, that focuses on the use of algorithms to create game content~\cite{Yannakakis2018}.~\acrshort{pcg} algorithms have been used to aid in the creation of a plethora of games such as No Man Sky~\cite{nomansky}, Spelunky~\cite{spelunky}, or Minecraft~\cite{minecraft}, to the extent that~\acrshort{pcg} and~\acrshort{ai} have enabled experiences and interactions that were not possible before~\cite{aidungeon,rogue,elite}. Moreover, as one of the properties of~\acrshort{pcg} is to increase replayability by creating an abundance of well-made content~\cite{shaker_procedural_2016}, games are not the only beneficiaries of~\acrshort{pcg} methods. For instance, they have the opportunity to be used to increase the generality of~\acrfull{ml} approaches~\cite{Risi2020-pcgGeneralityML}, or a step towards open-endedness~\acrfull{ec}~\cite{clune2019-aigas}. 

% Furthermore, several computational designers and systems have been developed to create complete games such as Angelina~\cite{Cook2016-Angelina1}, Ludi~\cite{Browne2010-ludii}, a system to create card games~\cite{font2013-GenCardGames}, or a system that uses data from Wikipedia to create mystery games~\cite{barros2018-DATAeinstein}. Automated game design, as developed in those systems, show interesting and important advancements towards~\acrshort{cc} systems. However, not including the human designer creates constraints, challenges, and limitations in these systems, such as modeling fun, enjoyment, and interaction. Another interesting and promising path to explore is the~\acrlong{micc} paradigm within~\acrshort{pcg}, combining both~\acrshort{ai} and humans to co-create the game content, which is the focus of this thesis.

% \subsection{Problem Statement} \label{sec:problemst}

% Interactive systems can help humans complete creative tasks that are tedious or require substantial human workload. Such creative tasks could be as simple as drafting a grammar corrected text (e.g., MS Word), to more complex scenarios such as creating visual artifacts (e.g., Adobe Photoshop) or designing entire video games (e.g., Unity Engine).~\acrlong{micc} is a paradigm within~\acrshort{pcg}, where both human and AI proactively collaborate to co-create and co-design games or creative content. Through~\acrshort{micc}, the potential of interactive tools increases substantially.~\acrshort{micc} enables a new way of tackling creative tasks engaging in-depth humans and AI rather than just helping humans to complete tasks and reduce their workload. Furthermore, enabling a mutual feedback loop could also foster both participants' creativity, create adaptive experiences for the users, or focus on achieving tasks with better and interesting results in a hybrid format. Multiple approaches have been proposed as alternatives for creating systems that model the interaction between AI and humans to create game content, and that use different techniques to study such interactions and its implications~\cite{Alvarez2020-ICMAPE,smith_tanagra:_2011,Liapis2013-sentientsketchbook,charity2020baba}. 

% Nevertheless, this collaborative approach divided into human control and automation~\cite{Horvitz99-mixedInit} with multiple initiatives~\cite{Allen99-MIinteraction,novick97-mixedInit}, and whereas it could foster humans' creativity~\cite{yannakakis2014micc}, raises an \emph{initiative} challenge for either agent: Which agent should have the initiative at different stages of the development and over the goal? The question reflects the diffuseness of the challenge and situation, as many factors need to be considered before appropriately indicating this. At the very least, some could say that depending on the task to be performed and the expertise of both, either would clearly be the one taking the development initiative. Whereas others would position the human as the one always in control. Yet, even with a clear answer, what happens in creative tasks to the expressivity of one of the sides due to the other taking the initiative? 

% Moreover, in design and creative tasks, the designer usually has intentions in what they are creating and goals that they want to achieve with their design. Thus, to enable deeper~\acrshort{mi} levels to co-create content, some control mechanisms with a varying degree of control over the algorithms might be necessary for the designer. Through this, the designer could direct or constraint the generated content by the computational designer and oversee that it is within their intentions and goals. In this case, each agent's control and expressive properties are at the expense of the other agents, as it constrains the space of possibilities~\cite{Baldwin2017}. This is especially relevant when the aim is a creative work such as games, where the creative expression needs to be fostered~\cite{Alvarez2018}. Yet, it becomes particularly challenging when using mixed-initiative methods, where smart approaches need to be in place for a natural conversation and successful collaboration. The more control is given, the more constraint it exists, but is this a problem? Is it inevitable? Boden explains it conspicuously ``...  We [humans] seek the imposed constraints [...], and try to overcome them by changing the rules.~\cite{boden2004-creative}''. Constraints limit the space, and as a consequence, they are overcome by encountering creative solutions.

% Furthermore, for this interaction to be fully fleshed, the human needs to understand the AI's behavior through interpretable and explainable models and systems, and the AI needs to recognize and interpret the intentions of the humans seamlessly as they create their content. The former is the focus of~\emph{Interpretable} and \emph{Explainable AI}, which seeks to create or adapt models and systems for a better workflow between humans and AI, where humans could understand the AI's decision process to enable trust relationships and reach deeper interactions~\cite{Zhu2018-XAIDesignersMICC,Doshi-Velez2018,adadi2018peeking}. The latter would mean that the AI could adapt its behavior and functionality to the needs, expertise, and workflow of individual designers or a specific group of designers. To do so, the AI must analyze several design processes, such as the designer's preferences, styles, and goals, which holistically is called \emph{Designer Modeling}~\cite{Liapis2013-designerModel,Liapis2014-designerModelImpl}. How to create these models and use them to develop adapted experiences is a complex challenge, and understanding the implications of its usability in the control-expressive properties, as well as other consequences, is not trivial.

% In this thesis, the focus is on exploring multiple approaches for the collaboration between AI and humans to co-create game content in an~\acrlong{micc} paradigm. The goal is to develop systems, techniques, and algorithms representing a [creative] computational designer to tackle the different tasks in game design and development. Tasks that could not only be assisted by AI but rather AI could be a \emph{colleague} in the creative process. To tackle these shared tasks, a mutual feedback loop could be established, whereby AI and humans could inspire each other to explore unknown areas in the design landscape and reach better and more creative solutions.

% To explore this, the main body of work presented in this dissertation is applied and evaluated through the~\acrfull{edd}, a~\acrlong{micc} system, where designers can create levels for rogue-like and adventure type of game such as Zelda~\cite{tloz} or The Binding of Isaac~\cite{bindingISAAC}. In~\acrshort{edd}, the human designer can quickly create interconnected rooms forming a dungeon to be experienced by players. Meanwhile, the computational designer collaborates by providing suggestions using different algorithms and following multiple heuristics. The human designer can interact in several ways with the computational designer so that this adapts its output to whatever goal the human designer has, while still providing a diverse amount of alternatives and different experiences to the human designer.

\subsection{Research Questions} \label{sec:RQS}

As motivated thus far, this thesis focuses on exploring different approaches for procedurally generating content for games or other creative content, specifically through the~\acrshort{micc} paradigm, where a human designer collaborates with an underlying AI to create creative content. Exploring the role of \emph{computers as colleagues} as defined by Lubart~\cite{lubart_how_2005}, this thesis delves into the use of~\acrshort{micc} tools and the multiple properties that emerge from the dynamic interaction between AI and Humans. The aim is to understand how we can enable a rich, fruitful, and better feedback loop in these types of tools using and developing novel AI techniques in the field of~\acrlong{ec} and~\acrlong{ml} to improve the interaction and create adapted experiences. The thesis also analyzes and studies the requirements, challenges, and benefits of enabling in-depth collaboration, tailored experiences, the properties that emerge (some seemingly competing properties), and their dynamics. Therefore, this thesis aims at addressing, discussing, and exploring the following research questions: 

% \begin{retQuestion}{}
% %   \textbf{RQ1.} How can we use and implement different algorithms into a mixed-initiative approach to help designers produce high-quality content and foster their creativity? 
   
% %   \textbf{RQ1.} How can we use, implement, and combine quality-diversity algorithms and  different algorithms into a mixed-initiative approach to help designers produce high-quality content and foster their creativity? 
   
%   \textbf{RQ1.} How can we generate content in tandem with designers in a mixed-initiative system to help them produce high-quality content and foster their creativity?
   
%       \begin{retQuestion}{}
%         \textbf{RQ1.1} How can we use and integrate quality-diversity algorithms into the mixed-initiative system? 
%         % into a mixed-initiative approach in order to generate high-performing and diverse content for designers?
%     \end{retQuestion}
   
%   \begin{retQuestion}{}
%         \textbf{RQ1.2} How can we use and integrate grammars into the mixed-initiative system?
        
% %         How can we integrate patterns for the generation of different type of content  
%     \end{retQuestion}
    
%     \begin{retQuestion}{}
%         \textbf{RQ1.3} To what extent can designers control the algorithms to steer and adapt the generated content?
%     \end{retQuestion}
%     %   How can we use and integrate quality-diversity algorithms into a mixed-initiative approach to help designers produce high-quality content and foster their creativity while allowing them to control, to a certain extent, the generated content?
% \end{retQuestion}

\begin{retQuestion}{}
   
   \textbf{RQ1.} How can we use and integrate multiple algorithms such as quality-diversity algorithms and grammars into a mixed-initiative approach to help designers produce high-quality content and foster their creativity while allowing them to control, to a certain extent, the generated content?
\end{retQuestion}


There are plenty of approaches and algorithms such as Evolutionary Computation, Machine Learning, or Wave Function Collapse that have been used to generate content as it will be discussed in the background chapter (sec.~\ref{background}). In this thesis, we are interested in exploring these algorithms and approaches, particularly quality-diversity algorithms, pattern-based systems, grammar systems and their use as encoding representation in MI-CC systems. In these MI-CC systems, the aim is to have more controllable yet expressive generators, and at the same time explore how they can be used to foster the designers' creativity and establish better human-AI collaboration and interactions.

\acrfull{qd} algorithms are a relatively new family of algorithms, specifically aimed at tasks and environments that require the strengths of convergence and divergence search~\cite{pugh_quality_2016}. Leveraging on~\acrshort{qd} algorithms to search for a surfeit of heterogeneous content while not losing sight of the content's quality could enable~\acrshort{micc} systems to explore a big area of the generative space producing more diverse and high-quality solutions. Through this, the system could propose a higher range of diverse solutions to the user, aiming at fostering the creativity of the human designer~\cite{liapis_can_2016}. Thus, how to integrate~\acrshort{qd} algorithms in~\acrshort{micc} systems that need to take into account the human work to provide valuable input is a promising open research area and one that this thesis explores. However, it is paramount to understand how to effectively use~\acrshort{qd} algorithms in these systems to fully leverage their expressive power while providing control to human designers.

\begin{retQuestion}{}\sloppy
%   \textbf{RQ2.} How can we use gameplay, player, and designer data to understand better players' and designers' actions and behaviors to enhance their experiences?
   \textbf{RQ2.} How can we use player and designer data to better understand their behaviors and procedures to enhance and adapt~\acrlong{micc} systems?
   
%   understand better players' and designers' actions and behaviors to enhance their experiences?,
   
   
%   \textbf{RQ2.} What type of data is representative of player and designer 
   
%   How can we use gameplay, player, and designer data to understand better players' and designers' actions and behaviors to enhance their experiences?
\end{retQuestion}

Games and creative contexts are spaces where both players and designers can express themselves, producing data on how they both interact. Research areas such as Experience-driven PCG~\cite{yannakakis_experience-driven_2011}, player modeling~\cite{pedersen_modeling_2010,holmgard_automated_2019} or designer modeling~\cite{liapis_designer_2013}, explore the use of such data to understand particular users~\cite{liapis_designer_2013,drachen_player_2009,melhart_your_2019} and to improve and enhance the experiences of players and designer. Especially focusing on enabling adaptive experiences~\cite{hastings_evolving_2009} and more accurate heuristics~\cite{marino_empirical_2015,canossa_towards_2015,summerville_understanding_2017}. However, how to use the data (and even what to collect) is still an open research area, especially when applied to adaptive experiences for~\acrshort{micc} tools with only a few relevant examples~\cite{liapis_designer_2014,liapis_adapting_2012,halina_threshold_2022}. Furthermore, the importance of enhancing the experience of~\acrshort{micc} tools' users lies in the search for deeper understanding and collaboration between humans and~\acrshort{ai}, which could enable a better experience for both.

\begin{retQuestion}{}
   \textbf{RQ3.} How can we model different designers' procedures and use them as surrogate models to anticipate the designers' actions, produce content that better fits their requirements, and enhance the dynamic workflow of mixed-initiative tools?
   
    \begin{retQuestion}{}
        \textbf{RQ3.1} What trade-offs arise from modeling and using designer's procedures to steer the generation of content towards personalized content?
    \end{retQuestion}
   
   \begin{retQuestion}{}
        \textbf{RQ3.2} What constraints are created over the generative process when using designer models?
    \end{retQuestion}
   
\end{retQuestion}

The advantage of having the human and AI collaborating is analogous to humans collaborating, each one with their own set of strengths and weaknesses to reach greater objectives and develop each other. However, mixed-initiative collaboration requires both human and AI to understand each other and the goals that the human aim to reach~\cite{horvitz_principles_1999,novick_what_1997}. This creates a particular problem where the AI needs to identify certain processes and characteristics of the human. When employing~\acrshort{micc} to co-create games and creative artifacts; this translates to design processes, style, preferences, intentions, and goals. This thesis aims to explore how to model different designer procedures such as preference or style, using several~\acrlong{ml} methods, and how to best use these as surrogate models to produce better content and enhance designers' experience using~\acrshort{micc} systems.

RQ2 and RQ3 drive the research on how to gather and use different types of data, i.e., player and designer data, and whereas designer modeling could be used in the~\acrshort{micc} feedback loop to create adaptive experiences. Through RQ3.1 and RQ3.2, this thesis focuses on exploring the trade-offs of using designer modeling. Specifically, the interest lies in the challenges and benefits that designer modeling creates for the algorithms and designers, and the overall experience that the designer wants to create, i.e., the game. 

Moreover, the constraints that emerge from using these models as surrogate models to steer the content generation are not trivial to address and are essential to study to understand and analyze their effect and extent. Using these models will inevitably create constraints over the generation process as we aim to adapt the experience to each designer or group of designers. Therefore, RQ3.2 specifically aims at understanding: what are these constraints? What is constrained? And whether these constraints are positive or negative? 

\begin{retQuestion}{}
   \textbf{RQ4.} How can level design and narrative interact, act as constraints, be intertwined, and in general, have an active role affecting each other to produce a holistic system? 
   
    % \begin{retQuestion}{}
    %     \textbf{RQ4.1} What are the requirements and main factors needed to establish a relation between the level design and the narrative, and what are the criteria to evaluate the respective generated content? 
    % \end{retQuestion}
   
   \begin{retQuestion}{}
        \textbf{RQ4.1} What are the factors to be considered when implementing such a paradigm and system in a mixed-initiative application, where a designer will be able to interact with the content?
    \end{retQuestion}
    
    \begin{retQuestion}{}
        \textbf{RQ4.2} What are the effects of producing and using a holistic system for the creative process of a designer, and what challenges are imposed on computational creativity? 
    \end{retQuestion}
   
\end{retQuestion}

The intertwined, multi-faceted, and collaborative nature of games invites the exploration of how to generate different facets and how to intertwine them. Facets in a game are seldom produced in a vacuum, and the coherence of a game and cohesion of game content rely on these facets to be relevant and aligned with each other. The automatic generation of game content could follow a similar process, so the resulting content feels coherent, which has been categorized as Holistic PCG~\cite{liapis_orchestrating_2019,salge_generative_2018}. Usually, systems that generate several facets focus on a hierarchical and step-by-step process where each facet is generated successively and, at times, not relying on the other generated facets but rather on an overarching design goal. The generation of game facets could then have some feedback loop, where content generated in one facet has an effect on another, and vice-versa, generating content in unison, i.e., orchestrating game content.

Moreover, the importance of space or game worlds and narrative has been pointed out by previous research in different disciplines. For instance, Aarseth links the space to the quest in games, both dependant on each other~\cite{kishino_hunt_2005}. Ashmore and Nitsche present a similar point in relation to the games' interactivity as they discuss that a generated level without depth and context lacks interest for the final user~\cite{ashmore_quest_2007}, further discussed and related by Kybartas and Bidarra with a focus on story automation~\cite{kybartas_quinn_survey_2017}. Similarly, Dehn~\cite{dehn_story_1981} defines space (i.e., the world) as a post-hoc development and justification for authored events, while Lebowitz~\cite{lebowitz_creating_1983} argues for the opposite view, the story gives meaning to a created world. Looking at the narrative-space discussion from whichever angle and perspective, it is noticeable that one requires the other to develop fully and fruitfully. Therefore, the intertwined generation of both facets is a suitable first approach to holistic PCG, which this thesis explores. 

Nevertheless, there are several challenges when posing content generation as a multi-faceted and intertwined task, such as how to represent the content, what elements to use as constraints, or how to evaluate the generated content. Yet one of the objectives explored in this thesis is to add and combine this into an MI-CC system. This exacerbates the challenge, as human input and control become central, and we consider their input and initiative and the role of the AI in the system. Likewise, MI-CC challenges and considerations are extended, whereas the generated content must not only adapt and be relevant for the human (considering their input as well) but also maintain and adapt to other constraints from different facets. 
%Likewise, how the system would adapt and generate relevant content considering human input while maintaining other constraints is a 

%  as we consider human input and initiative, and the role the AI agent takes in the system.

% The goal in this thesis is then, to combine this into an MI-CC system. This exacerbates the challenge as we need to consider humn input in the system and how the system would be able to adaptand generate relevant content that is in a feedback loop based on what the designer is creating.

% Adding this into an MI-CC system exacerbates the challenge as we need to consider human input in the system and how the system would we abvle to adapt and generate relevant content that is in a feedback loop based on what the designer inputs and what the content in other facets is generated. 

% the multi-faceted generation of content brings a set of challenges 

% Nevertheless, this multi-faceted generation of content have 

% This is especially relevant in MI-CC as 


% , where Human and AI collaborate to create content and both

% There are several examples that focus on a tandem generation of content



%Games are



% \begin{retQuestion}{}
%   \textbf{RQ5.} How does the agency of the designer and the agency of the underlying AI-technology component in mixed-initiative approaches, and the interaction between both actors, affects the overall design process and the usefulness of mixed-initiative tools?
% \end{retQuestion}

% \blindtext

% Esto es lo que pone el study handbook sobre lo que debe contener la tesis para el final seminar: “Similar to the when the licentiate thesis is reviewed, all parts (cover paper and papers) of the manuscript shall be in place for the final seminar. These parts may not be fully processed; however, the cover paper must at least include research questions, contributions, methodology and a mapping between the research questions and the papers.”

\clearpage
\subsection{Pronouns, Style, and Clarification}

Throughout the thesis, the pronoun ``we'' will be used in favor of ``I'', since the work and research achieved and presented in this thesis would not have been possible without my co-authors' collaboration. 

\sloppy
When referring to a player or designer, this thesis chooses the pronouns ``they'' and ``their'' to respect a gender-inclusive language. Moreover, throughout the thesis, it is referred to as user and designer alike, as a designer is the target user group within the possible user base of the systems and tools developed in this thesis. The player is referred to as the end-user: the user who could experience the creations in the~\acrlong{micc} system.

% When referring to end-user, we refer to the user that could and would experience and play the creative designers create collaboratively in the~\acrlong{micc} system, i.e., the player. 

When discussing the participants in a mixed-initiative system, i.e., AI and Human, this thesis uses the word ``agent'' when needed to refer to either, unless specifically discussing one in particular, as mixed-initiative systems have been described as multi-agent systems~\cite{allen_mixed-initiative_1999}.

Finally, this thesis will refer to as ``computational designer'' to the overall AI system that interacts and collaborates with the human designer to create content through the~\acrshort{micc} system.

% Finally, in the~\acrshort{micc} system presented, used, and developed in this thesis  there are several  AI systems fueling the AI that collaborates with designers in the~\acrshort{micc} system presented, used, and developed throughout the multiple publications. Thus, when not explicitly discussing individual algorithms, this thesis will refer to the overall AI system that collaborates with the designer as artificial designer.

% characteristics, properties
% The AI developed throughout the multiple publications and that 


% Finally, the content generated and suggested by the artificial designer is based on~\acrshort{ea}s; thus, when referring to this content it is discussed as individuals.
