%\section{ETHICAL CONSIDERATION} \normalfont
%\section[ETHICAL CONSIDERATION AND REFLECTIONS]{ETHICAL CONSIDERATION AND \\ REFLECTIONS} \normalfont

%Since I can remember, games have always been part of my life. It was a few days before the start of the new millennium when my brother and I got a Nintendo 64 with Zelda: Ocarina of Time. I did not go to sleep until 06:00am, which was probably the first time I stayed awake so long [and perhaps the root of my sleeping pattern nowadays]. Likewise, I remember playing Metal Gear Solid 1 on my PS1, the emotion when passing each boss and scenario, sneaking around hoping to don't get caught and the exclamation mark appearing was just incredible. There was a caveat though, I did not have a Memory Card; thus, I had to finish the game without turning off my PS1. I hope that you, as a reader, can imagine the challenge that meant for an 8 years old. Passed 3 days of non-stop playing, I finished the game. And with just the press of a button, the game restarted such as nothing happened. Yet, something happened; these experiences and the ones I leave out shaped my path. 

%I studied game design for 6 years, and now I am in the brink of finishing my PhD on game design and the use of artificial intelligence to generate game content. Many experiences through my education have shaped the interest on this research topic. From creating a small text adventure to spending long hours to build a game engine just to prototype a garbage truck simulator, to doing a small god game like black and white (although those villagers never followed me). You see, I had no doubt I wanted to do games, but I had no particular creative skill that I could hone to make something out of this, or so I thought. I was in awe playing these games for their fun and enjoyable moments, design, art, stories, and in general, the experience they created. Yet, I saw myself as completely out of touch with these elements, and I think I can [nowadays] narrow it down to what I felt was a lack of creativity and creative thinking. How do creativity works? What elements constitute the creative process? and how does it arise on us? This shaped, to a large extent, my interest on the exploration of artificial intelligence within creative domains and for creative tasks.

%I studied game design for 6 years, and now I am in the brink of finishing my PhD on game design and the use of artificial intelligence to generate game content. The reasons why games might be less diffuse now, but why did I decided to be a programmer or research artificial intelligence in-depth? You see, I had no doubt I wanted to do games, but I had no particular creative skill that I could hone to make something out of this. I was in awe playing these games for their fun and enjoyable moments, design, art, stories, and in general, the experience they created. Yet, I saw myself as completely out of touch with these elements, and I think I can [nowadays] narrow it down to what I felt was a lack of creativity and creative thinking. This shaped, to a large extent, my interest on the exploration of artificial intelligence within creative domains and for creative tasks.

%It was the start of the millennium (


%Since I can remember, games have always been part of my life. I remember fondly the Christmas when my parents gave me my Nintendo 64 with Zelda: Ocarina of Time; I did not go to sleep until 06:00am, which was probably the first time I stayed awake so long [and perhaps the root of my sleeping pattern nowadays]. Likewise, I remember playing Metal Gear Solid 1 on my PSX, the emotion when passing each boss and scenario, sneaking around hoping to don't get the exclamation mark sound was just incredible. There was a caveat though, I did not have a Memory Card; thus, I had to finish the game without turning off my PSX. I hope that you, as a reader, can imagine the challenge that meant for an 8 years old (and for the 2000). Passed 3 days of non-stop playing, I finished the game. And with just the press of a button, the game restarted such as nothing happened. Yet, something happened; these experiences and the ones I leave out shaped my path. I studied game design for 6 years, and now I am in the brink of finishing my PhD on game design and the use of artificial intelligence to generate game content. The reasons why games might be less diffuse now, but why did I decided to be a programmer or research artificial intelligence in-depth? You see, I had no doubt I wanted to do games, but I had no particular creative skill that I could hone to make something out of this. I was in awe playing these games for their fun and enjoyable moments, design, art, stories, and in general, the experience they created. Yet, I saw myself as completely out of touch with these elements, and I think I can [nowadays] narrow it down to what I felt was a lack of creativity and creative thinking. This shaped, to a large extent, my interest on the exploration of artificial intelligence within creative domains and for creative tasks.

%Creativity has always baffled me. Nowadays I believe creativity is an outstanding ability that we [all] possess. Some believe that they are not creative creatures, but in reality, and being candid about it, we are all creative and creative actions are taken all the time. We are all confronted with challenges and situations that require us to be creative on how we approach them. Whether this means being creative for artistic purposes such as painting or developing games, or to question your field and write a dissertation in political science, or to create variations on how to write your name; these all require creativity. Creativity varies on the task, process, outcome, and personal perception, which might be why people feel as non-creative~\cite{kaufman_beyond_2009}. Nevertheless, creativity as our other abilities, can be refined, developed, and fostered. The thesis that you are about to read, challenges the misconception that there are non creative people, yet it is not about creativity. This thesis explores a computational creative system that collaborates with humans in the exciting and creative area of game design to enhance, augment, and support these human's capabilities. However, in order to do so, this system needs to show creative output and to some extent, recognize the humans' creative process. Now, I agree that the embody part of creativity is essential to ``judge'' and ``assess'' creative content and process. But putting that aside, I present a computational creative system that I have used to study game design and, in that endeavour, analyze and research [computational] creativity, the very thing that baffles me.

Since I can remember, games have always been part of my life. It was a few days before the start of the new millennium when my brother and I got a Nintendo 64 with Zelda: Ocarina of Time. I played tirelessly until 06:00 am, which was probably the first time I stayed awake so long. Likewise, I remember playing Metal Gear Solid 1 on my PS1, the thrill when passing each boss and scenario, sneaking around hoping to don't get caught and the exclamation mark appearing was just incredible. There was a caveat though, I did not have a memory card; thus, I had to finish the game without turning off my PS1. I hope that you, as a reader, can imagine the challenge that meant for a 10 years old. Passed 3 days of non-stop playing, and I finished the game. And with just the press of a button, the game restarted as if nothing had happened. Yet, something happened; these experiences and the ones I left out shaped my path. 

I studied game design for 6 years, and now I am on the brink of finishing my PhD in game design and the use of artificial intelligence to generate game content. Many experiences through my education have shaped my interest in this research topic; from creating a small text adventure to spending long hours to build a game engine just to prototype a garbage truck simulator, to doing a small god game like black and white (although those villagers never followed me). You see, I had no doubt I wanted to do games, but I had no particular creative skill that I could hone to make something out of this, or so I thought. I was in awe playing these games for their fun and enjoyable moments, design, art, stories, and the experience they created. Yet, I saw myself as completely out of touch with these elements, and I think I can [nowadays] narrow it down to what I felt was a lack of creativity and creative thinking. How does creativity works? What elements constitute the creative process? How does it arise within us? This shaped, to a large extent, my interest in the exploration of artificial intelligence within creative domains and for creative tasks.

Creativity has always baffled me. Nowadays, I believe creativity is an outstanding ability that we [all] possess. Some believe that they are not creative creatures, but in reality, and being candid about it, we are all creative and creative actions are taken all the time. We are all confronted with challenges and situations that require us to be creative in how we approach them. Whether this means being creative for artistic purposes such as painting or developing games, questioning your field and writing a dissertation in political science, or creating variations on how to write your name, these all require creativity. Creativity varies on the task, process, outcome, and personal perception, which might be why people feel non-creative~\cite{kaufman_beyond_2009}. Nevertheless, creativity, like our other abilities, can be refined, developed, and fostered. The thesis that you are about to read challenges the misconception that there are non-creative people, yet it is not about creativity. This thesis explores a computational creativity system that collaborates with humans in the exciting and creative area of game design to enhance, augment, and support these human capabilities. However, in order to do so, this system needs to show creative output and, to some extent, recognize the human creative process. I present a computational creativity system that I have used to study game design and, in that endeavor, analyze and research [computational] creativity, the very thing that baffles me.

% Now, I agree that the embodied part of creativity is essential to ``judge'' and ``assess'' creative content and process. But putting that aside, 

\subsection*{This thesis and its implications}

%we\footnote{I use the pronoun ``we'' since the work and research in this thesis would not be possible without my collaborators}

%  if you may

Throughout this thesis, I will explore Human-AI collaboration as an approach to co-create game content. As a result, it is explored how to approach game design facets, particularly level and narrative design, with AI algorithms in tandem with human designers. A \emph{Computational Designer}. A core part of the research concerns how to establish an \emph{effective} collaboration and collaborative environment. This is addressed by studying the \emph{Computational Designer} in a collaborative system, the Evolutionary Dungeon Designer (EDD). Studying these collaborative systems made two elements clear and worth their exploration; the different properties that arise as we establish this collaboration\footnote{My licentiate thesis has a stronger focus on this~\url{http://www.diva-portal.org/smash/record.jsf?pid=diva2\%3A1471182&dswid=-2772}} and the need for user models adjusted to the design task, process, and creator~\cite{liapis_designer_2013}. The former is, to some extent, a consequence of these Procedural Content Generation systems as discussed in~\cite{shaker_procedural_2016}, and in general, the tradeoffs that exist among the different techniques (e.g., expressive generators, controllable output, explainable processes, or adaptive systems). These are exacerbated and complexified when moved toward collaborative tasks since the human's current work and design need to be considered. One way to address these is to create more adaptive experiences through modeling designers and their design process (i.e., \emph{Designer Modeling}). 


%Moreover, \textbf{collaboration} not \textbf{automation} is the thesis focus. 
% ; modeling human designers, and applying and using those models within the
Now, you may feel inclined to say that by studying this \emph{Computational Designer}, one might be able to effectively replace human designers with the caveat that we need their design traces and examples. This is an important ethical point to be made on these types of systems, particularly AI-assisted design tools. A parallel discussion goes on on Twitter every now and then. For instance, Open AI's DALL-E 2 and GPT-3, Google's Imagen, or Github's CoPilot are just some of the systems that are in the eye of the hurricane regarding the use of human creative output to create these models, and the consideration and impact to human creativity and creative works. While you do not necessarily need to engage or agree in these discussions, they are relevant, important, and necessary. There is, of course, a possibility that these algorithms, such as the \emph{Computational Designer} studied here, end up working autonomously. Yet, in this thesis, I make the argument for AI-assisted game design and explore the advantages and benefits of having such a system in a human-AI collaborative scenario for both the human designer and the algorithms. The findings support and embrace collaboration, not because it couldn't be used autonomously since we have demonstrated that ``high-performing'' content can be generated, but because its use is useless without the human designer. This is especially true when we consider the generation of subjective content where metrics must be reductionist and lose nuance and the adequate assembly or orchestration of content. However, \emph{automated game design} is an important research area where the aim is to study how multiple game design elements, facets, and aspects could be modeled, assessed, and generated; to explore computational creativity systems.\footnote{For the interested reader I refer to the awesome tutorial by Mike Cook: \url{https://www.youtube.com/watch?v=dZv-vRrnHDA}.}

%and to explore multiple metrics to evaluate the content

%In many cases, systems are developed first as auto

%automated systems are tested first and their generation capabilities 

%while exploring metrics that could act as proxies for that do not necessarily need to replacements for human judgment

%and 3) the adaptiveness .. the better strength distrib .. the strengths both human and...  AI the human-centered 


%However, \emph{automated game design} is an important research area, where the aim is to study how multiple game design aspects\footnote{For the interested reader, I refer to the awesome tutorial by Mike Cook: \url{https://www.youtube.com/watch?v=dZv-vRrnHDA}.}. Yet, our findings support and embrace collaboration, and refute this automated use; not because it couldn't be used since we have demonstrated ``high-performing'' content can be generated, but because its use is, forgive the redundancy, useless without the human designer. This is especially true when we consider 1) the generation of subjective content where metrics must be reductionist and lose nuance, 2) the adequate assembly or orchestration of content, and 3) 

%Now that the elephant in the room has been addressed (i.e., autonomous use of these techniques), we can move forward to some other ethical considerations and implications I have come across these years. 

% and challenge the status quo and 

Moving beyond the autonomous use of these techniques, there are other ethical considerations and implications I have come across throughout my PhD. Some of these points are factual discussions over what we can already see and experience. Others are speculative, to reflect, be thought-provoking, and raise awareness and discussion~\cite{fiesler_innovating_2021,klassen_run_2022}. For instance, what are the implications of implementing these human-AI collaborative systems in the workplace? These systems can completely change the workplace, how we work, and the interactions we have. This is visible when using automated-decision making systems in workplaces highlighting large problems with AI systems such as bias, privacy, transparency, and fairness~\cite{lepri_ethical_2021}. The introduction of \emph{AI as a colleague} in the workplace would have a higher impact. Society and human colleagues will need to understand how these systems affect their work and how to collaborate with them. The benefits, as discussed in this thesis, are plenty, but its development needs to be gradual and elaborated together with the target group~\cite{lai_towards_2020,partlan_design-driven_2021}.

Another contentious point is that it could also redefine the ``perfect'' colleague. As it will be shown and discussed throughout the thesis, within games, AI can be used to search unknown spaces for relevant content, simulate gameplay, gather statistics, or produce content that is both on the designers' style and aligned to players' requirements. This could redefine the landscape of what is expected from designers, creating false expectations. However, in my view, AI-assisted game design is part of the future in game development, and many of these systems could be used to improve the processes now in place. Popular systems like DALL-E, Imagen, or GPT, while not made with collaboration in mind, could be used for new interactions. Nevertheless, we need to be careful on how we implement these systems, need to have a wider reach to society, and understand better the needs of the target group. Furthermore, among the several points discussed by Mike Cook, one of them is the impact Procedural Content Generation (PCG) algorithms have on the developers' workload. Rather than reducing their workload (one of the core ideas with PCG), more is expected from the developers as PCG takes care of other work~\cite{cook_social_2021}. However, PCG takes care of asymmetrical workload; reducing repetitive and tedious tasks might mean that more laborious tasks await designers. Granted that these might be more interesting, they still require larger labor and different intensity from developers. Rather than working on a ``sinusoidal'' shape, where low (e.g., repetitive and tedious tasks) and high (e.g., creative output) intensity tasks are alternated, designers might end up working on a high-intensity plateau.

%but again, AI-assisted game design is, in my view, part of the future in game development. such as when using GPT to create AI Dungeons~\cite{aidungeon} or explored recently in a thesis I supervised

%It is important to raise up this discussion as it has a high impact in society the impact in society  as we, ultimately, work towards this collaboration; i.e., the impact in


% or \emph{algorithmic colleagues} in the workplace 
%Human-AI collaboration can potentially have the same challenges in addition to 

%The first relevant point relates 

%At first instance, 




%Furthermore, among the several points discussed by Mike Cook, one of them is the impact PCG algorithms have on the developers workload. Rather than reducing their workload (one of the core ideas with PCG), more is expected from the developers as PCG takes care of other work~\cite{cook_social_2021}. However, PCG takes care of asymmetrical workload; reducing repetitive and tedious tasks might mean that more laborious tasks awaits designers. Granted that these might be more interesting, they still require a larger labor and different intensity from developers. Rather than working on a ``sinusoidal'' shape, where low (e.g., repetitive and tedious tasks) and high (e.g., creative output) intensity tasks are alternated, designers might end up working on a high intensity plateau.

%However, with the introduction of ``AI as a colleague'' or ``algorithmic colleagues'' in the workplace, it could also redefine the ``perfect'' employee by employers. It is important to raise up this discussion as it has a high impact in society the impact in society  as we, ultimately, work towards this collaboration; i.e., the impact in
% I could speculate that 

On a tangential point, not much discussion has taken place regarding these repetitive tasks. On the one hand, we aim at reducing them, thus, reducing the human workload to focus on what matters to them. On the other hand, these tasks might be necessary for the creative process, which is still very unknown to us~\cite{boden_creative_2004}. They might not be necessary for all, but perhaps novice designers require them in order to try something different, workaround them, or simply because their grokloop is longer than those more experienced designers~\cite{compton_casual_2015}. One outcome could be that these could end up being replaced with another repetitive task, such as adjusting these algorithms to get the expected output\footnote{systems such as \emph{Danesh} works towards this~\cite{cook_danesh_2016}}. A similar idea is seen within EDD, where designers take their time exploring how the \emph{computational designer} adapts to them to understand how and when to use its suggestions. Given EDD's nature and goal, a similar creativity support process might be in place.

%Based on how technology has developed, we could speculate that these could end being replaced by either adjusting these algorithms to get the expected output (systems such as \emph{Danesh} are works towards this~\cite{cook_danesh_2016}) or work might be reduced for humans to develop other aspects of life. 

%Partlan et al. did a participatory design project to investigate how mixed-initiative tools and Computer Support Tool (CST) could be implemented and be in place, consulting with expert AI designers~\cite{partlan_design-driven_2021}. In their work, there is no discussion on these tasks, although, designers explicitly ask and discuss ``full control,'' ``anticipate requirements,'' or ``tool as a partner.''

%\footnote{My Licentiate thesis is more directed and explicit in these properties~\cite{alvarez_exploring_2020}.}

%Moreover, there are two more points that I find relevant to discuss with human-AI collaborative tools and their implementation. The role humans and AI have, and the system's purpose. Both are related to the role humans and AI have, as well as the use of human-AI collaboration in the workplace. Throughout this thesis, the discussion mainly surrounds the role of the computer as a colleague, but what are the alternatives and how does the role affects the collaboration and the workplace. Partlan et al. work is relevant in this discussion, but under the same category ``Designers direct implementation as editors,'' there seems to be a representative duality; designers want full control, but want a tool as a partner~\cite{partlan_design-driven_2021}. This duality is explored in the thesis, where the aim is at establishing a colleague relationship, and it explores ways for designers to have control over the algorithm without hindering expressivity. It also explores the use of designer modeling, where we try to model the designers creative and design process in multiple ways to personalize and adapt the \emph{Computational Designer}~\cite{liapis_designer_2013}. Yet, this thesis do not discuss the implications in the workplace for such a system as the work by Partlan et al.~\cite{partlan_design-driven_2021} or Lai et al.~\cite{lai_towards_2020,lai_mixed-initiative_2022} do. 

% Moreover, there are two more points that I find relevant to discuss

Moreover, I find one more point relevant to discuss with human-AI collaborative tools and their implementation, namely the role humans and AI have. Throughout this thesis, the discussion mainly surrounds the role of the computer as a colleague, but what are the alternatives, and how could that affect the collaboration and the workplace? Partlan et al. did a participatory design project to investigate how mixed-initiative tools and Computer Support Tool could be implemented and be in place, consulting with expert Game AI designers~\cite{partlan_design-driven_2021}. In their work, there seems to be a representative duality under the same category, ``Designer direct implementation as editors,'' where designers want full control but want a tool as a partner as well. This duality is explored in the thesis, where the aim is to establish a colleague relationship, exploring ways for designers to have control over the algorithm without hindering expressivity. However, \emph{colleague} is just one of the many roles the computer might have. How would the interaction be if either the human or the computer is the \emph{manager}, \emph{student}, or \emph{teacher}? As we assign different roles we expect different actions from the system, are able to grant different responsibilities and allow certain interactions such as constraining the design. For instance, as a \emph{manager}, the system could organize multiple designers, arrange the produced content, or require and ask for certain goals\footnote{The work by M in \emph{Baba is Y'all} has the AI request for specific goals that it is missing, which the human can of course ignore~\cite{charity_baba_2022}}, which could then mean that the system has more agency or initiative. On the other hand, as a \emph{student}, the system could take a background role, learning from the designer (e.g., a designer model) and using different strategies such as Machine Teaching or Active Learning~\cite{tegen_taxonomy_2021}. However, I believe these roles will be more fluid and dynamic throughout the design process. Instead of remaining static, they would change according to the needs of the project and designers.

%All things considered, I am, throughout this thesis, \emph{exploring game design through human-AI collaboration}. Firstly, to understand the 


%It also explores the use of designer modeling, where we try to model the designers creative and design process in multiple ways to personalize and adapt the \emph{Computational Designer}~\cite{liapis_designer_2013}. Yet, this thesis do not discuss the implications in the workplace for such a system as the work by Partlan et al.~\cite{partlan_design-driven_2021} or Lai et al.~\cite{lai_towards_2020,lai_mixed-initiative_2022} do. 

%Nevertheless, we need to ask the question, how does the role affect the relationship and collaboration; and  what is the purpose of the mixed-initiative tool?

%After discussing the role and relationship, we can finally ask, what is the purpose of the human-AI system? This question differs from the previous as the role does not need be relevant here; rather the goal of the system might drastically change. Perhaps we are not interested in using mixed-intiiative systems as systems to produce work and help designers reach those goals as initially conceived~\cite{yannakakis_mixed-initiative_2014,liapis_mixed-initiative_2016}, but they might be autotelic and help explore creativity~\cite{compton_casual_2015} or reflect on the process~\cite{kreminski_reflective_2021}.

%Now that I discussed the role and relationship, then we can ask what is the purpose of the mixed-initiative tool? This question differs from the previous as the role does not need be relevant here; rather the goal of the system might drastically change. Perhaps we are not interested in using mixed-intiiative systems as systems to produce work and help designers reach those goals as initially conceived~\cite{yannakakis_mixed-initiative_2014,liapis_mixed-initiative_2016}, but they might be autotelic and help explore creativity~\cite{compton_casual_2015} or reflect on the process~\cite{kreminski_reflective_2021}.


% , and explores the idea of designer 

% Moreover, what is the purpose of the mixed-initiative tool? This question differs from the 

% For long, AI in games researchers have avoided ethical considerations and bias in the systems they develop (me included). While these systems are, of course, not exempt from these. For instance, Lepri et al. discusses three main points: Privacy, accountability and transparency, and fairness~\cite{lepri_ethical_2021}. These are all relevant regardless of the system, area, and application environment, and of course, many other challenges that I do not have the space here to cover~\cite{bender_dangers_2021}. Puck, a brand new automated game design system, was recently proposed 

% Partlan et al. discuses 

% First, the role these tools (and in consequence, the \emph{Computational Designer}), and the role human designers have.

% I will discuss roles throughout the thesis with a focus on the role of the computer as a colleague. 

% This might be good for developers that 

% This might be something good for some developers that want the focus where it is needed

% \begin{itemize}
%     \item First, discuss my work considering its ethical implications. I mean, from work place point of view, understanding of these systems, their inclusion in different places and systems.
%     \item Second, I would like to perturb the discussion and bring a hypothetical case, and discuss its implications? 
%     \item Finally, I would like to bring interesting perspectives such as the responsibility paper from cook~\cite{cook_social_2021} to discuss games, but also AI in general. Maybe something from Gary Marcus? 
%     \item Actually, it is also interesting to bring up the idea that employers and people will start adjusting to the ''perfect colleague''. Expecting humans to behave equally. That is an important point. Cook also bring this up as PCG should reduce the workload, but rather due to this ``reduction,'' more is expected from these developers.
%     \item Also, what if doing those repetitive tasks, are parts where creativity or other abilities are developed/fostered? I  have not reflected on this.
%     \item Adaptation, what is that?
%     to have and create 
% \end{itemize}

%Moreover, and moving on from the elephant in the room (i.e., autonomous use of these techniques); as I wrote this thesis and through my journey, certain more relevant implications and considerations with  the Human-AI collaboration area have arose. What are the challenges of adding 

%Not everyone engages in these discussions.. regarding Google's newest model DALL-E 2 and . Perhaps not everyone engages into the discussion, particularly researchers within areas such as human-ai collaboration and AI in general. Nevertheless, these discussions are relevant, important, and necessary. 

%there are three points we have made clear 




%---

% Throughout this thesis, we have explored Human-AI collaboration as an approach to co-create game content, and as a result, we explored how to approach game design facets, particularly level and narrative design, with AI algorithms in tandem with human designers. A \emph{Computational Designer} if you may. A core part of the research concerned how to establish an ``effective'' collaboration and collaborative environment. This was addressed by studying the \emph{Computational Designer}

% Now, you may feel inclined to say that by studying this, modeling human designers and applying and using those models within the computational designer, one might be able to effectively replace human designers, with the caveat that we need their design traces. A parallel discussion goes on in Twitter regarding Google's newest model DALL-E 2 and . Perhaps not everyone engages into the discussion, particularly researchers within areas such as human-ai collaboration and AI in general. Nevertheless, these discussions are relevant, important, and necessary. There is, of course, a possibility that these algorithms such as the \emph{Computational Designer} studied here, ends up working autonomously and does automated game design\footnote{For the interested reader, I refer to the awesome tutorial by Mike Cook: \url{https://www.youtube.com/watch?v=dZv-vRrnHDA}.}. Yet, our findings support and embrace collaboration, and refute this automated use; not because it couldn't be used since we have demonstrated ``high-performing'' content can be generated, but because its use is useless without the human designer. This is especially true when we consider 1) the generation of subjective content where metrics must be reductionist and lose nuance, 2) 

% %there are three points we have made clear 

% Moreover, and moving on from the elephant in the room (i.e., autonomous use of these techniques); through this thesis and journey, certain more relevant implications and considerations with the Human-AI collaboration area have arose. While


% Throughout this thesis,
