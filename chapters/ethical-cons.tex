%\section{ETHICAL CONSIDERATION} \normalfont
%\section[ETHICAL CONSIDERATION AND REFLECTIONS]{ETHICAL CONSIDERATION AND \\ REFLECTIONS} \normalfont

%I want to tell you a story, one that won't change you at all

It was a few days before the start of the new millennium when me and my brother got a Nintendo 64 with Zelda: Ocarina of Time. I did not go to sleep until 06:00am, which was probably the first time I stayed awake so long [and perhaps the root of my sleeping pattern nowadays]. Likewise, I remember playing Metal Gear Solid 1 on my PSX, the emotion when passing each boss and scenario, sneaking around hoping to don't get the exclamation mark sound was just incredible. There was a caveat though, I did not have a Memory Card; thus, I had to finish the game without turning off my PSX. I hope that you, as a reader, can imagine the challenge that meant for an 8 years old. Passed 3 days of non-stop playing, I finished the game. And with just the press of a button, the game restarted such as nothing happened. Yet, something happened; these experiences and the ones I leave out shaped my path. 

I studied game design for 6 years, and now I am in the brink of finishing my PhD on game design and the use of artificial intelligence to generate game content. Many experiences through my education have shaped the interest on this research topic. From creating a small text adventure in C to spending long hours to build a game engine in C++ just to prototype a garbage truck simulator, to doing a small god game like black and white (those villagers never followed me). You see, I had no doubt I wanted to do games, but I had no particular creative skill that I could hone to make something out of this. I was in awe playing these games for their fun and enjoyable moments, design, art, stories, and in general, the experience they created. Yet, I saw myself as completely out of touch with these elements, and I think I can [nowadays] narrow it down to what I felt was a lack of creativity and creative thinking. How do creativity works? What elements constitute the creative process? and how does it arise on us? This shaped, to a large extent, my interest on the exploration of artificial intelligence within creative domains and for creative tasks.

%I studied game design for 6 years, and now I am in the brink of finishing my PhD on game design and the use of artificial intelligence to generate game content. The reasons why games might be less diffuse now, but why did I decided to be a programmer or research artificial intelligence in-depth? You see, I had no doubt I wanted to do games, but I had no particular creative skill that I could hone to make something out of this. I was in awe playing these games for their fun and enjoyable moments, design, art, stories, and in general, the experience they created. Yet, I saw myself as completely out of touch with these elements, and I think I can [nowadays] narrow it down to what I felt was a lack of creativity and creative thinking. This shaped, to a large extent, my interest on the exploration of artificial intelligence within creative domains and for creative tasks.

%It was the start of the millennium (


%Since I can remember, games have always been part of my life. I remember fondly the Christmas when my parents gave me my Nintendo 64 with Zelda: Ocarina of Time; I did not go to sleep until 06:00am, which was probably the first time I stayed awake so long [and perhaps the root of my sleeping pattern nowadays]. Likewise, I remember playing Metal Gear Solid 1 on my PSX, the emotion when passing each boss and scenario, sneaking around hoping to don't get the exclamation mark sound was just incredible. There was a caveat though, I did not have a Memory Card; thus, I had to finish the game without turning off my PSX. I hope that you, as a reader, can imagine the challenge that meant for an 8 years old (and for the 2000). Passed 3 days of non-stop playing, I finished the game. And with just the press of a button, the game restarted such as nothing happened. Yet, something happened; these experiences and the ones I leave out shaped my path. I studied game design for 6 years, and now I am in the brink of finishing my PhD on game design and the use of artificial intelligence to generate game content. The reasons why games might be less diffuse now, but why did I decided to be a programmer or research artificial intelligence in-depth? You see, I had no doubt I wanted to do games, but I had no particular creative skill that I could hone to make something out of this. I was in awe playing these games for their fun and enjoyable moments, design, art, stories, and in general, the experience they created. Yet, I saw myself as completely out of touch with these elements, and I think I can [nowadays] narrow it down to what I felt was a lack of creativity and creative thinking. This shaped, to a large extent, my interest on the exploration of artificial intelligence within creative domains and for creative tasks.

Creativity has always baffled me. Nowadays I believe creativity is an outstanding ability that we [all] possess. Some believe that they are not creative creatures, but in reality, and being candid about it, we are all creative and creative actions are taken all the time. We are all confronted with challenges and situations that require us to be creative on how we approach them. Whether this means being creative for artistic purposes such as painting or developing games, or to question your field and write a dissertation in political science, or to create variations on how to write your name; these all require creativity. Creativity varies on the task, process, outcome, and personal perception, which might be why people feel as non-creative~\cite{kaufman_beyond_2009}. Nevertheless, creativity as our other abilities, can be refined, developed, and fostered. The thesis that you are about to read, challenges the misconception that there are non creative people, but is not about creativity. This thesis explores a computational creative system that collaborates with humans in the exciting and creative area of game design to enhance, augment, and support these human's capabilities. However, in order to do so, this system needs to show creative output and to some extent, recognize the humans' creative process. Now, I agree that the embody part of creativity is essential to ``judge" and ``assess" creative content and process. But putting that aside, I present a computational creative system that I have used to study game design and, in that endeavour, analyze and research [computational] creativity, the very thing that baffles me.

\subsection*{This thesis and its implications}

%we\footnote{I use the pronoun ``we" since the work and research in this thesis would not be possible without my collaborators}

Throughout this thesis, I will explore Human-AI collaboration as an approach to co-create game content, and as a result, we explore how to approach game design facets, particularly level and narrative design, with AI algorithms in tandem with human designers. A \emph{Computational Designer} if you may. A core part of the research concerns how to establish an ``effective" collaboration and collaborative environment. This is addressed by studying the \emph{Computational Designer} in a collaborative system, the Evolutionary Dungeon Designer (explored in chapter~\ref{chap:edd}). Moreover, \textbf{collaboration} not \textbf{automation} is the thesis focus. 

Now, you may feel inclined to say that by studying this, modeling human designers and applying and using those models within the \emph{Computational Designer}, one might be able to effectively replace human designers, with the caveat that we need their design traces and examples. A parallel discussion goes on in Twitter every now and then. For instance, Open AI's DALLE-2 model and GPT-3, Google's Imagen, or Github's CoPilot are just some of the systems that are in the eye of the hurricane regarding the use of human creative output to create these models and the consideration and impact to human creativity and creative works. While you do not necessarily need to engage or agree in these discussions, they are relevant, important, and necessary. There is, of course, a possibility that these algorithms such as the \emph{Computational Designer} studied here, ends up working autonomously and does automated game design\footnote{For the interested reader, I refer to the awesome tutorial by Mike Cook: \url{https://www.youtube.com/watch?v=dZv-vRrnHDA}.}. Yet, our findings support and embrace collaboration, and refute this automated use; not because it couldn't be used since we have demonstrated ``high-performing" content can be generated, but because its use is useless without the human designer. This is especially true when we consider 1) the generation of subjective content where metrics must be reductionist and lose nuance, 2) the adequate assembly of content, 3) 

Now that we have addressed the elephant in the room (i.e., autonomous use of these techniques), we can move forward to some ethical considerations and implications I have come across these years. Some of these points are factual discussions over what we can already see and experience, while others are speculative in nature, to reflect, be thought-provoking, and challenge the status quo and raise awareness and discussion~\cite{fiesler_innovating_2021,klassen_run_2022}. For instance, what are the implications of implementing these systems in the workplace? Among the several points discussed by Mike Cook, one of them is the impact PCG algorithms have on the developers workload. Rather than reducing their workload (one of the main ideas with PCG), more is expected from the developers as PCG takes care of other work~\cite{cook_social_2021}. However, PCG takes care of asymmetrical workload; reducing repetitive and tedious tasks might mean that more laborious tasks awaits developers. Granted that these might be more interesting, they still require a larger labor and other intensity from developers. Rather than working on a ``wave" or ``sinusoidal" shape, where low (e.g., repetitive and tedious tasks) and high (e.g., creative output) intensity tasks are alternated, developers might end up working on a high intensity plateau.

On a tangential point, not much discussion has taken place regarding these repetitive and tedious tasks. On the one hand, we aim at reducing them ergo reducing the human workload to focus on what matters for them. On the other hand, these tasks might be necessary in the creative process, which is still very unknown to us~\cite{boden_creative_2004}. They might not be necessary for all, but perhaps novice designers requires them in order to try something different, workaround them, or simply because their grokloop is longer than those more experienced designers~\cite{compton_casual_2015}. Partlan et al. did a participatory design project to investigate how mixed-initiative tools and Computer Support Tool (CST) could be implemented and be in place, consulting with expert designers~\cite{partlan_design-driven_2021}. In their work, there is no discussion on these tasks, although, designers explicitly ask and discuss ``full control," ``anticipate requirements," or ``tool as a partner."

Moreover, there are two more points that I find relevant to discuss with human-AI collaborative tools, and their implementation. Both are related to the role humans and AI have, as well as the use of human-AI collaboration in the workplace. Throughout this thesis, the discussion mainly surrounds the role of the computer as a colleague, but what are the alternatives and how does the role affects the collaboration and the workplace. Partlan et al. work is relevant in this discussion, but under the same category ``Designers direct implementation as editors," there seems to be a representative duality; designers want full control, but want a tool as a partner~\cite{partlan_design-driven_2021}. This duality is explored in the thesis, where the aim is at establishing a colleague relationship, and it explores ways for designers to have control over the algorithm without hindering expressivity.\footnote{My Licentiate thesis is more directed and explicit in these properties~\cite{alvarez_exploring_2020}.} It also explores the use of designer modeling, where we try to model the designers creative and design process in multiple ways to personalize and adapt the \emph{Computational Designer}~\cite{liapis_designer_2013}. Yet, this thesis do not discuss the implications in the workplace for such a system as the work by Partlan et al.~\cite{partlan_design-driven_2021} or Lai et al.~\cite{lai_towards_2020,lai_mixed-initiative_2022} do. 

Nevertheless, we need to ask the question, how does the role affect the relationship and collaboration; and  what is the purpose of the mixed-initiative tool?

Now that I discussed the role and relationship, then we can ask what is the purpose of the mixed-initiative tool? This question differs from the previous as the role does not need be relevant here; rather the goal of the system might drastically change. Perhaps we are not interested in using mixed-intiiative systems as systems to produce work and help designers reach those goals as initially conceived~\cite{yannakakis_mixed-initiative_2014,liapis_mixed-initiative_2016}, but they might be autotelic and help explore creativity~\cite{compton_casual_2015} or reflect on the process~\cite{kreminski_reflective_2021}.


, and explores the idea of designer 

Moreover, what is the purpose of the mixed-initiative tool? This question differs from the 

For long, AI in games researchers have avoided ethical considerations and bias in the systems they develop (me included). While these systems are, of course, not exempt from these. For instance, Lepri et al. discusses three main points: Privacy, accountability and transparency, and fairness~\cite{lepri_ethical_2021}. These are all relevant regardless of the system, area, and application environment, and of course, many other challenges that I do not have the space here to cover~\cite{bender_dangers_2021}. Puck, a brand new automated game design system, was recently proposed 

Partlan et al. discuses 

First, the role these tools (and in consequence, the \emph{Computational Designer}), and the role human designers have.

I will discuss roles throughout the thesis with a focus on the role of the computer as a colleague. 

This might be good for developers that 

This might be something good for some developers that want the focus where it is needed

\begin{itemize}
    \item First, discuss my work considering its ethical implications. I mean, from work place point of view, understanding of these systems, their inclusion in different places and systems.
    \item Second, I would like to perturb the discussion and bring a hypothetical case, and discuss its implications? 
    \item Finally, I would like to bring interesting perspectives such as the responsibility paper from cook~\cite{cook_social_2021} to discuss games, but also AI in general. Maybe something from Gary Marcus? 
    \item Actually, it is also interesting to bring up the idea that employers and people will start adjusting to the "perfect colleague". Expecting humans to behave equally. That is an important point. Cook also bring this up as PCG should reduce the workload, but rather due to this ``reduction," more is expected from these developers.
    \item Also, what if doing those repetitive tasks, are parts where creativity or other abilities are developed/fostered? Damn, I  have not reflected on this.
    \item Adaptation, what is that?
    to have and create 
\end{itemize}

%Moreover, and moving on from the elephant in the room (i.e., autonomous use of these techniques); as I wrote this thesis and through my journey, certain more relevant implications and considerations with  the Human-AI collaboration area have arose. What are the challenges of adding 

%Not everyone engages in these discussions.. regarding Google's newest model DALL-E 2 and . Perhaps not everyone engages into the discussion, particularly researchers within areas such as human-ai collaboration and AI in general. Nevertheless, these discussions are relevant, important, and necessary. 

%there are three points we have made clear 




%---

% Throughout this thesis, we have explored Human-AI collaboration as an approach to co-create game content, and as a result, we explored how to approach game design facets, particularly level and narrative design, with AI algorithms in tandem with human designers. A \emph{Computational Designer} if you may. A core part of the research concerned how to establish an ``effective" collaboration and collaborative environment. This was addressed by studying the \emph{Computational Designer}

% Now, you may feel inclined to say that by studying this, modeling human designers and applying and using those models within the computational designer, one might be able to effectively replace human designers, with the caveat that we need their design traces. A parallel discussion goes on in Twitter regarding Google's newest model DALL-E 2 and . Perhaps not everyone engages into the discussion, particularly researchers within areas such as human-ai collaboration and AI in general. Nevertheless, these discussions are relevant, important, and necessary. There is, of course, a possibility that these algorithms such as the \emph{Computational Designer} studied here, ends up working autonomously and does automated game design\footnote{For the interested reader, I refer to the awesome tutorial by Mike Cook: \url{https://www.youtube.com/watch?v=dZv-vRrnHDA}.}. Yet, our findings support and embrace collaboration, and refute this automated use; not because it couldn't be used since we have demonstrated ``high-performing" content can be generated, but because its use is useless without the human designer. This is especially true when we consider 1) the generation of subjective content where metrics must be reductionist and lose nuance, 2) 

% %there are three points we have made clear 

% Moreover, and moving on from the elephant in the room (i.e., autonomous use of these techniques); through this thesis and journey, certain more relevant implications and considerations with the Human-AI collaboration area have arose. While


% Throughout this thesis,

I want to do three things.

