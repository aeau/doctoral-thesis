\section{BACKGROUND} \normalfont \label{background}

\setlength{\epigraphwidth}{2in} 
\epigraph{\textit{Snake? Snake? SNAAAKE!}}{Metal Gear Solid}

\setlength{\parindent}{0.0em}

This chapter offers an overview of the different fields surrounding the central subject of study in this thesis, i.e., the collaboration between AI and humans to co-create game content, and the related RQs. First,~\acrlong{pcg} is explored with multiple examples of the type of content that might be created. It is then presented the search-based approach, quality-diversity algorithms, and the~\acrlong{mi} paradigm as they are the main approaches and paradigms used throughout the thesis. Then, player and designer modeling is presented to give an overview of the concepts and the differences between them, and examples of each computational model. Finally, creativity and computational creativity are explored by briefly analyzing the field's goals with the most relevant literature and presenting examples within the computational intelligence in games research area.

\subsection{Procedural Content Generation}

Game content is the main component of any game, as it is what players interact with to achieve the designers' developed experience. Game content refers to anything within the game, from the game's rules, a hero's backstory, or the levels to be traversed by players. However, game engines and Non-Player Characters (NPC) behaviors are not considered the same type of game content as the former is used to create the games themselves, and the latter refers to the AI behavior in-game (e.g., movement or combat). Furthermore, as higher possibilities for more complex games are provided by technology, game engines, and platforms, and developers and players set higher requirements, games have increasingly become content-intensive entertainment mediums. 

To cope with this challenge and to relieve the burden and workload of game designers when creating all this content, several approaches have been proposed to create content under the field of~\acrlong{pcg}.~\acrshort{pcg} refers to the creation of content, mainly for games, using algorithms, autonomously or with the assistance of users~\cite{yannakakis_artificial_2018}. Content can be divided into game facets: audio, visuals, narrative, levels, rules, and gameplay~\cite{liapis_orchestrating_2019}, and have been categorized within the~\acrshort{pcg} field as \textit{Game Bits}, \textit{Game Space}, \textit{Game Systems}, \textit{Game Scenarios}, \textit{Game Design}, and \textit{Derived Content}~\cite{hendrikx_procedural_2013}.

There are plentiful of commercial games that utilize one way or another~\acrshort{pcg} such as The Binding of Isaac~\cite{bindingISAAC} or Civilization~\cite{civilization}, to the point that some games rely critically on these algorithms, providing experiences otherwise not possible such as Rogue~\cite{rogue}, Dwarf Fortress~\cite{dwarfFortress} or AI Dungeon~\cite{aidungeon}. However, there has been an increasing interest in~\acrshort{pcg} during the past decade in academia~\cite{liapis_10_2020}. There exist multiple approaches addressing different challenges in the creation of content, resulting in algorithms that can autonomously create game rule's~\cite{browne_evolutionary_2010,font_towards_2013}, narratives~\cite{ashmore_quest_2007,ammanabrolu_toward_2019}, levels~\cite{shaker_evolving_2012,sarkar_sequential_2020,green_mario_2020}, graphics~\cite{horsley_building_2017,pagnutti_you_2016}, and audio~\cite{scirea_metacompose_2016,hoover_functional_2014}. 

Another approach is to focus in the generation of content in multiple facets aiming at creating intertwined content; known as Holistic PCG~\cite{liapis_orchestrating_2019,salge_generative_2018}. By approaching content generation as a multi-faceted task, games and their content can be more coherent. There have been approaches focusing on generating complete games~\cite{browne_evolutionary_2010,guzdial_conceptual_2020,cook_angelina_2017}, but usually only some facets are targeted based on their criteria, requirements, and synergy. For instance, in~\cite{cook_rogue_2014} and~\cite{treanor_game-o-matic_2012}, mechanics, graphics, and levels are generated together, although through different processes. Two facets that are commonly associated with each other are the narrative and level facet~\cite{dormans_generating_2011,hartsook_toward_2011,ashmore_quest_2007,abuzuraiq_taksim_2019}. This is due to the fact that space requires context to make sense of it, and narrative requires space to develop. Nevertheless, these approaches usually follow a hierarchical procedure, where content in each facet is generated step-by-step such as in the work by Dormans~\cite{dormans_generating_2011}, which generates the mission graph that guides the level generation\footnote{Dormans and approach is also used in Dormans' games Unexplored and Unexplored 2.}. Yet, works like hte one by Hoover et al.~\cite{hoover_audioinspace_2015}, Holtar et al.~\cite{holtar_audioverdrive_2013} or Karavolos et al.~\cite{karavolos_multi-faceted_2019} present interesting results and examples of more intertwined content generation.

%there are examples of Holistic PCG where facets are intertwined with very interesting .. There are some examples of Holistic PCG, w-- the Game-O-Matic focuses on generating levels, mechanics, graphic ... For instance, levels, graphics, and mechanics can be intertwined such as that



%Other approaches have focus on creating content in multiple facets aiming at creating intertwined content~\cite{hoover_audioinspace_2015,cook_rogue_2014,treanor_game-o-matic_2012,holtar_audioverdrive_2013,karavolos_multi-faceted_2019}, and others on creating complete games~\cite{browne_evolutionary_2010,guzdial_conceptual_2020,cook_angelina_2017}.
% However, it is in academia where most of the work in~\acrshort{pcg} has originated and developed with a crescent interest during the past decade~\cite{Liapis2020-pcgWorkshop}. 

%Broadly, within the field of~\acrshort{pcg}, there exist [arguably]\footnote{While most algorithms belong or extend from the constructive or generate-and-test approaches, some techniques do not belong to either. For instance, Wave Function Collapse does not belong, presented as its own category, Constraint Solving~\cite{Karth2017-WFC}.} 

Furthermore, within the field of~\acrshort{pcg}, there exist [arguably] three main approaches to create content: constructive approach, generate-and-test approach, and search-based approach, each with their criteria~\cite{togelius_search-based_2011}. Constructive approaches focus on generating content following a set of predefined rules that can create valid content without evaluating the quality of the content after generating, rather the content is evaluated as it is being constructed~\cite{shaker_constructive_2016,green_two-step_2019,snodgrass_levels_2019}. Conversely, generate-and-test approaches focus on creating content iteratively that instead of being continuously tested as the content is constructed, it is tested after generation to satisfy a set of constraints or objectives. When tested, the process might iterate on the design. In this approach, the designer's focus is on creating the set of constraints to be satisfied~\cite{summerville_gemini_2018,volz_evolving_2018}. Search-based approaches are a specialized case of the generate-and-test approach that aims at using some type of search algorithm, mainly~\acrlong{ea}s, to generate content by exploring the generative space and through this process, encounter interesting individuals with non-trivial characteristics~\cite{hastings_evolving_2009,font_constrained_2016}.

%There exist other approaches to create content besides the ones previously described.
Besides these three main approaches, there exist other ones to generate content. For instance, Constraint Solving algorithms such as Wave Function Collapse (WFC), do not directly map their procedures to any of the aforementioned processes~\cite{karth_wavefunctioncollapse_2017,karth_addressing_2019}. Other examples are techniques within the~\acrshort{pcg} via~\acrshort{ml} approach~\cite{summerville_procedural_2018} such as approaches to repair unplayable generated content~\cite{zhang_video_2020} or generating content using learned probabilities from sample content~\cite{dahlskog_multi-level_2014}. However, this thesis focuses mainly on using a search-based approach to generate suitable content suggested to a designer in an interactive tool through~\acrshort{qd} algorithms~\cite{gravina_procedural_2019}. Our approach relies on exploring the generative space informed by a designer's design that helps focus the search in different areas of the space while still encountering diverse solutions for the designer.

% approaches using markov chains  generate content based on markov model on local and glboal.

%Besides these three approaches, other approaches to~\acrshort{pcg} exist. For instance, Constraint Solving algorithms such as Wave Function Collapse (WFC), do not directly map their procedures to any of the aforementioned processes~\cite{Karth2017-WFC}, sdladkasld. 

% Perhaps I can briefly discuss here PCG via ML, specialized version in PCG via RL, PCG via constraint solving, and 

%In this thesis, the focus is mainly on using a search-based approach to generate suitable content suggested to a designer in an interactive tool through~\acrshort{qd} algorithms~\cite{gravina2019procedural}. Our approach relies on exploring the generative space informed by a designer's design that helps focus the search in different areas of the space while still encountering diverse solutions for the designer.

\subsubsection{Search-based Approach}

The search-based approach is a specialization of the generate-and-test approach, where the aim is to use some search algorithm, being the most prominent,~\acrlong{ea}. However, essentially any metaheuristic algorithm and from the stochastic search algorithm family could be used as well and fall under the umbrella of search-based approaches. The main distinction between the search-based approach and the generate-and-test approach is that search-based approaches evaluate the generated solution with a quality estimator, e.g., fitness function or novelty behavior, providing a continuous evaluation of the generated content. Such evaluation drives the next generation steps, as the estimation helps the search to find promising paths.

Search-based approach has been widely used in~\acrshort{pcg} and basically for the generation of all the types of game content such as levels~\cite{dormans_generating_2011}, rules~\cite{font_towards_2013} or weapons~\cite{gravina_surprise_2016}. Moreover, the evaluation of the generated content is the most important part of search-based approaches, as well as the most challenging and complex. The used heuristics does not only need to be representative of the task at hand but also allows the expressive property of the search, as that is one of the main benefits of search-based approaches. Constraints to ensure quality [or playable] experiences are not enough, since that does not necessarily represent what a designer or player wants~\footnote{One of the challenges of generating games [and game content], is that it requires them to be fun and interacted as discussed in the previous section.}. However, evaluation functions come in all shapes and sizes, and they are all valid with their own set of ups and downs. For instance, they might come from  game design concepts such as design patterns~\cite{dahlskog_patterns_2015} or game level metrics~\cite{canossa_towards_2015}, or from aesthetic indicators such as symmetry~\cite{marino_empirical_2015}, or subjective evaluation from users~\cite{schrum_interactive_2020}, or even continuously adapting the evaluation based on gameplay~\cite{hastings_evolving_2009} or to the designer's preferences~\cite{liapis_adapting_2012}.

%\subsubsection{Holistic PCG}


\subsection{Quality Diversity} \label{sec:Backqd}

\acrfull{qd} algorithms are a family of algorithms under the approaches in~\acrlong{ec}, that focuses on combining the benefits and strengths of both convergence search, i.e., focusing on optimization and objective, and divergence search, i.e., disregarding objectives and searching for diversity~\cite{pugh_quality_2016,gaier_are_2019}. Through this,~\acrshort{qd} algorithms seek to generate a collection of high-performing solutions that are as diverse as possible\footnote{The following website serves as a database with research related to~\acrshort{qd}: https://quality-diversity.github.io/ maintained by Antoine Cully}. While convergence search refers mainly to the typical~\acrshort{ec} algorithms used for optimization, divergence search has increasingly being used to tackle many tasks that were previously dominated by convergence search. For instance, when the task or environment is deceptive, i.e., reaching the goal might be impossible or where plenty of local optima exist where a convergence search might get stuck. Lehman and Stanley proposed the Novelty Search algorithm, which introduces the idea of divergence search through ignoring objectives and searching for novel behaviors instead, with surprisingly good results~\cite{lehman_abandoning_2011,lehman_revising_2010}. From that moment onward, several divergent search algorithms have been proposed, such as surprise search~\cite{gravina_surprise_2016} or the Paired Open-Ended Trailblazer (POET)~\cite{wang_poet_2019,wang_enhanced_2020,dharna_co-generation_2020} to explore open-ended algorithms, as well as variations to novelty search such as constrained novelty search~\cite{liapis_constrained_2015} or~\acrfull{nslc}~\cite{lehman_evolving_2011}.

\acrshort{nslc} is an example of a~\acrshort{qd} algorithm that leverage on the divergent search to explore the space for novel behavior among solutions and on convergence search for preserving the high-performing individuals within the novel niches~\cite{lehman_evolving_2011}.~\acrfull{mape} is another algorithm in the~\acrshort{qd} family, and one that has gained considerable popularity in multiple areas such as games~\cite{charity_mech-elites_2020,alvarez_empowering_2019,fontaine_mapping_2019} and robotics~\cite{cully_robots_2015,tjanaka_approximating_2022}. As the other~\acrshort{qd} algorithms,~\acrshort{mape} explores the behavioral space for a collection of solutions that are both high-performing and diverse among each other, with the caveat that~\acrshort{mape} discretizes the behavior space as a grid of cells informed by a set of feature dimensions that illuminate the behavior space.~\acrshort{mape}' goal is to fill each cell belonging to a set of discrete feature dimension values with a high-performing individual encountered in the search and retain it until a higher-performing individual with similar characteristics is encountered~\cite{mouret_illuminating_2015}. This characteristic allows the exploration of features orthogonal to the fitness function that allow the discovery of diverse behavioral repertoires~\cite{cully_autonomous_2019,justesen_learning_2020,grillotti_discovering_2022}.

One major challenge with~\acrshort{mape} is the \emph{curse of dimensionality}, since each new feature dimension used adds a new dimension in the search space. Thus, some~\acrshort{mape} variation skip the grid architecture and focus on reducing the amount of feature dimensions or enabling the use of higher dimensions such as~\acrlong{cvtmape}~\cite{vassiliades_using_2018} or~\acrlong{ce}~\cite{vassiliades_comparison_2017}. Further, the~\acrlong{cmame} algorithm combines the effective adaptive search of~\acrlong{cmaes} with a map of elites, yielding large improvements for real-valued representations in terms of both objective value and number of elites discovered~\cite{fontaine_covariance_2019}. The work by Fontaine et al. was expanded into the~\acrlong{memape}, improving the quality, diversity, and convergence speed of~\acrshort{mape} in general~\cite{cully_multi-emitter_2021}. Other work within~\acrshort{mape} has focused on its robustness~\cite{justesen_map-elites_2019}, multi-objective tasks optimization~\cite{pierrot_multi-objective_2022}, or assessing its properties when coupled in interactive environments~\cite{alvarez_assessing_2021}.

%or its interactive use.

Moreover, within the field of games~\acrshort{qd} algorithms have started to be used extensively, especially~\acrshort{mape}, both for gameplay and agent behaviors~\cite{perez-liebana_generating_2021,zhang_deep_2022}, and the generation of content~\cite{gravina_procedural_2019}.~\acrshort{mape} has been used to create and find levels with just the right difficulty for a set of agents~\cite{gonzalez-duque_finding_2020}, to balance and create decks in hearthstone~\cite{fontaine_mapping_2019}, or create levels for puzzle games through crowdsourcing~\cite{charity_baba_2022}. Constrained~\acrshort{mape} introduced by Khalifa et al.~\cite{khalifa_talakat_2018}, combines~\acrshort{mape} with the~\acrfull{fi2pop} algorithm~\cite{kimbrough_feasibleinfeasible_2008}, to generate bosses for bullet hells games in Talakat. Since then, constrained~\acrshort{mape} has been used in other projects and experiments to benefit from its strengths, such as to generate game levels based on mechanics as feature dimensions in Mario~\cite{khalifa_intentional_2019,charity_mech-elites_2020}, and was combined with interactive evolution resulting in the~\acrlong{icmape}~\cite{alvarez_empowering_2019,alvarez_interactive_2020}. 

%Moreover, within the field of games and~\acrshort{pcg},~\acrshort{qd} algorithms have started to be used extensively, especially~\acrshort{mape}~\cite{gravina_procedural_2019}.

Thus far, the focus has been on discussing~\acrshort{pcg} and presenting algorithms that create content mostly autonomously. Automated game design is a complex task since it is required to create content (or full games) by itself with the help of heuristics, user models, and logic among the content created~\cite{togelius_experiment_2008,barros_who_2019,cook_angelina_2016,cook_getting_2020}. However, another paradigm within~\acrshort{pcg} is the mixed-initiative paradigm, where AI can collaborate with a designer to co-design games. Through this, we could leverage the strengths of both to create content.

% Furthermore, several computational designers and systems have been developed to create complete games such as Angelina~\cite{Cook2016-Angelina1}, Ludi~\cite{Browne2010-ludii}, a system to create card games~\cite{font2013-GenCardGames}, or a system that uses data from Wikipedia to create mystery games~\cite{barros2018-DATAeinstein}. Automated game design, as developed in those systems, show interesting and important advancements towards~\acrshort{cc} systems. However, not including the human designer creates constraints, challenges, and limitations in these systems, such as modeling fun, enjoyment, and interaction. Another interesting and promising path to explore is the~\acrlong{micc} paradigm within~\acrshort{pcg}, combining both~\acrshort{ai} and humans to co-create the game content, which is the focus of this thesis.


% it must model, to some extent, certain subjective human characteristics such as fun and enjoyment, use this ~\cite{Togelius2008-automaticGD,barros2018-DATAeinstein,Cook2016-Angelina1,Cook2020-automatedGDtutorial}. If these algorithms were to be introduced in a design tool, we could leverage on human knowledge and goals; thus, the algorithms could cope with the subjective constraint, in what is called the mixed-initiative paradigm. However, mixed-initiative tools do have their o
% until now?

% To better understand the content to be created, Content can be divided into game facets: audio,
 
% Games are composed of multiple types of content interacting with each other as specified by the developers, which is ultimately interacted and enjoyed by players. Content in games can be anything from the rules of the game to the levels that are traversed by the player. Liapis et al. classified this content in broader categories as game facets: audio, visuals, narrative, levels, rules, and gameplay~\cite{Liapis2019-OrchestratingGames}. To relieve some of the burden and workload of game designers when creating all this content, several approaches have been proposed to create the content under the field of~\acrlong{pcg}.~\acrshort{pcg} is the creation of content, mainly for games, using algorithms~\cite{Yannakakis2018}. Similarly to how Liapis et al. discuss game content in broader categories as facets, Hendrikx et al. categorized such content as \textit{Game Bits}, \textit{Game Space}, \textit{Game Systems}, \textit{Game Scenarios}, \textit{Game Design}, and \textit{Derived Content}~\cite{Hendrikx2013-pcgSurvey}.


\subsection{Human-AI Collaboration}

An alternative path to work with AI would also be Human-AI collaboration instead of AI automation. On the one hand, humans have qualities that are paramount in the collaboration such as subjective and domain knowledge, expert intuition, and a holistic perspective regarding external aspects. On the other hand, AI have many qualities that are favorable for Human-AI collaboration. It can, 1) go through big amount of data and learn representations from that, 2) explore the possibility spaces in several dimensions, which might be too prohibiting for humans, and 3) have a holistic perspective on multidimensional aspects with specific domains.

In general, using AI algorithms to generate content come with a set of requirements and properties that we would like them to have~\cite{shaker_procedural_2016}. For instance, the algorithm should be \emph{expressive:} it should generate diverse content; \emph{controllable:} it should be possible to steer the algorithm; or \emph{adaptive:} it should adapt to other content. Expressiveness has been more widely research within PCG mainly as a means to evaluate generators~\cite{smith_expressive_2012,horn_comparative_2014,kreminski_evaluating_2022}. Controllability, on the other hand, has been less explored but remains as a fundamental property for practitioners and designers~\cite{earle_learning_2021,madkour_towards_2022,partlan_design-driven_2021}. Furthermore, as we reshape tasks and environments for Human-AI collaboration, these properties become even more relevant since they are coupled with human interaction. For instance, \emph{adaptiveness} can be discussed by adapting to what the human creates, their criteria, or other generated content. Moving towards Human-AI collaboration raises up two other important properties as well, namely, \emph{explainability} and \emph{subjective evaluation}. \emph{explainability} is a long-term goal and research area in AI~\cite{doshi-velez_considerations_2018,holmberg_role_2021}, which is emphasized in these collaborative scenarios~\cite{zhu_explainable_2018}. \emph{Subjective evaluation} such as fun or interesting, is a hard task for optimization; thus, through Human-AI collaboration we can leverage human knowledge for assessment.  These properties and their possible benefits and tradeoffs are explored in this thesis in multiple ways discussed in section~\ref{sec:edd}. For instance, designers are given means to control various aspects of the computational designer to steer the generation, and in turn, we explore how this affect the algorithm's expresiveness.

%In this thesis, we explore these properties by adapting the computational designer towards the designer's design and criteria, and 

Moreover, the role that both human and AI play in this collaboration are as important as the properties that arise from it. Knowing, identifying, and setting the role for either agent sets the tone for the collaboration and what is expected from both. This was discussed by Guzdial et al., where designers perceived the collaboration differently depending on the assigned role for the AI, varying between: \emph{friend}, \emph{collaborator}, \emph{student}, or \emph{manager}~\cite{guzdial_friend_2019}. Guzdial et al.'s work is based on the colleague role introduced by Lubart, where there is a partnership between computer and humans. Lubart discussed three other roles that the computer might have to promote creativity; \emph{computer as nanny}: management of creative work; \emph{computer as pen-pal}: communication service between collaborators; and \emph{computer as coach}: Using creative enhancement techniques~\cite{lubart_how_2005}.


%somewhere here or somewhere else learning controllable generators~\cite{earle_learning_2021}.

%Control here as well~\cite{madkour_towards_2022}, this paper has not been published yet.


%Moreover, 
%Leveraging the human in the collaboration, tasks can be approached both b
%on the \emph{subjective evaluation}


%However, given the collaborative nature of the tasks and relation that human and AI have in these settings, there is the possibility that through Human-AI collaboration \emph{explainability} could be intrinsic to the collaboration. As humans explore and try different alternatives with these algorithms and models (e.g., virtual thinkering), better understanding and interpretation from how these models work could be achieved such as in~\cite{xie_interactive_2019}.



%Bring things about adaptability, controllability, and expressivity, and maybe explainability now that we are here.

%NOW LETS JUMP TO THE OTHER properties, but not fully.


%the relation and task that human and AI have in these collaborative settings, there is the possibility that Human-AI collaboration could 

%one possibility within Human-A However, Human-AI collaboration 

% and to some extent, what the human creates

%The collaboration between humans and AI give raise to a set of 

%As we reshape tasks and environments for Human-AI collaboration, there 

%Human-AI Collaboration refers to the 


% Paramount is the role of the computer agent in this interaction, as it would help establish the boundaries of the interaction, what is expected, and how creativity could be fostered. Lubart analyzed this interaction and examined the different ways computers could be involved in creative work to promote creativity. In his work, he proposed four roles: \emph{computer as nanny}: management of creative work; \emph{computer as pen-pal}: communication service between collaborators; \emph{computer as coach}: Using creative enhancement techniques; and \emph{computer as colleague}: partnership between computer and humans~\cite{lubart_how_2005}. Recently, this was explored by Guzdial et al., where designers perceived the AI collaborator with more or less value depending on their desired role for the AI, varying between: \emph{friend}, \emph{collaborator}, \emph{student}, or \emph{manager}~\cite{guzdial_friend_2019}.

%  The relationship that occurs between co-creators in a MI-CC tool is arguably one of the most important element within a system that aims to foster creativity.

%Establishing different roles such as colleague and collaborator might require some user model within the system. Designer modeling, as defined by Liapis et al.~\cite{liapis_designer_2013}, is a way to classify and predict a designer's style, goals, preferences, and processes. Preference models~\cite{alvarez_learning_2020,liapis_adapting_2012} have been built based on designers' choices and used as surrogate models to evaluate further generated content. Similarly, using the designers' creation, the designers' processes and styles could be modeled to inform other systems and adapt the generated content~\cite{liapis_designer_2014,alvarez_designer_2022,halina_threshold_2022}.

\subsubsection{Mixed-Initiative Paradigm}

\acrfull{mi} refers to the collaboration between Computer and Human to solve some task where both have a proactive initiative into solving the task regardless of the degree of such initiative~\cite{liapis_searching_2014}. Yet while this definition clearly separates~\acrshort{mi} approaches from others that ``simply'' assist humans in their tasks, it still remains a very disputed concept as: which agent initiates the ``conversation'', what task to be solved, and what initiative to take in each step remain unknown. Novick and Sutton discuss~\acrshort{mi} by analyzing a set of~\acrshort{mi} systems, and conclude that the initiative in~\acrshort{mi} is a multi-factor model, described as: \textit{1) choice of task}: describing the task; \textit{2) choice of speaker}: describing which agent is in control and how the interaction works; \textit{3) choice of outcome}: describing what is the outcome of the interaction~\cite{novick_what_1997}. Moreover, Allen describes~\acrshort{mi} systems as multi-agent collaboration scenarios. These need to have a flexible interaction strategy, leveraging each agent's strengths to solve the tasks, and that involves a continuous negotiation between agents to determine roles, i.e., initiative; thus, collaborating as a team~\cite{allen_mixed-initiative_1999}. The initiative will vary depending on which agent can solve a determined problem, providing solutions and taking the control while the other agents, e.g., a human or group of models, assist in the procedure~\cite{ferguson_mixed-initiative_2007}. Similarly, Horvitz discusses~\acrshort{mi} as a more natural collaboration between agents that explicitly integrate human control and manipulation, and [AI] automation strategies and their contributions to achieve some [shared] task~\cite{horvitz_uncertainty_1999,horvitz_principles_1999}.

% Mixed-Initiative was defined by Horvitz as the collaboration between humans and machine, leveraging in the strengths of both, to create some artifact, which requires a varied initiative from both in different development stages.

\subsubsection{Mixed-Initiative Co-Creativity}

\begin{itemize}
    \item nathan*~\cite{partlan_design-driven_2021}
    \item work by Gorm:~\cite{lai_towards_2020} and the new one from the journal!~\cite{lai_mixed-initiative_2022}.
\end{itemize}

Yannakakis et al. introduced the~\acrlong{micc} paradigm for the co-creation of creative content such as games, and regarding~\acrshort{pcg}, where machine and humans alternate initiative to co-design content~\cite{yannakakis_mixed-initiative_2014}. Their work and discussion on the capabilities of such interaction to foster creativity on both humans and machines is pivotal for understanding and develop~\acrshort{micc} tools that can reduce the designer's workload, foster their creativity, and in general, improve the design and creative process~\cite{liapis_can_2016,alvarez_fostering_2018}. 

Germinate~\cite{kreminski_germinate_2020} is a~\acrshort{micc} system to co-create rhetorical games using the constraint-based game generator Gemini~\cite{summerville_gemini_2018} under-the-hood. In Germinate, the designer can, in iterations, specify a set of constraints and properties they want games to have and which the generator will consider. The designer is then presented a set of games that they can play and inspect, and which they can use to modify the set of constrained previously set, improving their understanding of their own intent. Germinate focuses on accessibility by leveraging on the concept of Casual Creators~\cite{compton_casual_2015} within the~\acrshort{micc} paradigm, allowing through this iterative process, the designer to focus in the constraint that reflects their intent rather than any knowledge within game technology. 

Delarosa et al. presented an innovative~\acrshort{micc} system, where the computational designer is represented as three different agents with different representations trained using~\acrfull{rl}, suggesting specific changes to the designer as they create Sokoban levels~\cite{delarosa_mixed-initiative_2020}. Their approach is the first implementation of the work by Khalifa et al. that introduced a new approach to create content:~\acrshort{pcg} via~\acrshort{rl}~\cite{khalifa_pcgrl_2020}. In~\acrshort{pcg} via~\acrshort{rl}, the level creation process is set up as an RL problem, i.e., a sequential task, where the agent can learn policies to maximize the quality of the final level. Khalifa et al. approach use three different representations, i.e., different types of agents, to create levels: \textit{Narrow}: at each step the agent is located randomly in the level and can perform an action in such place; \textit{Turtle}: at each step the agent can move and change tiles in the way; and \textit{Wide}: at each step the agent has control of location and placement of tiles. Likewise, Delarosa et al. work includes the same agents and have an identical premise, i.e., level generation as an RL problem, with the caveat that these agents must now learn and adapt to a designer's design. The designer is suggested levels based on their own by each of the agents, which the designer might pick or disregard and continue editing. Their work was evaluated through thirty-nine sessions and showed that, on average, the levels created using AI suggestions were more playable and complex.

The Sentient Sketchbook is a tool where designers can co-create low-resolution sketches of strategy levels while being presented augmented information about their creation and suggested variations using multiple heuristics and objectives~\cite{liapis_sentient_2013}. In the Sentient Sketchbook, the designer focuses mainly on creating the sketch they envision, while the computational designer focuses on three main aspects. 1) Provide \textit{suggestions} adapted to the designer's current design using constrained novelty search~\cite{liapis_constrained_2015}. 2) Provide \emph{augmented information} on how the level is formed such as resource safety or navmesh. And 3) provide \emph{multiple levels of visualization} that transform the designer's sketch into usable levels. Further, the main feature of the tool and its most innovative one is the suggestions by means of an~\acrshort{ea} powered by three different search algorithms: objective-driven, objective-driven with diversity preservation, and novelty search~\cite{preuss_searching_2014}. The work by Liapis et al. is seminal to analyze and understand how~\acrshort{micc} systems have evolved and the benefits that they have for designers and AI likewise.

Cicero is a special kind of~\acrshort{micc} system, where the focus is on helping designers create complete games in the~\acrfull{gvgai} framework\footnote{http://www.gvgai.net/}~\cite{perez-liebana_general_2019} and~\acrfull{vgdl}~\cite{schaul_video_2013}, rather than individual game content~\cite{machado_cicero_2017}. In Cicero, the aim is to let the designer create the game they want while receiving suggestions on what content might be added next related to sprites, mechanics, interactions between entities, stats, or game's rules~\cite{machado_pitako-recommending_2019}. Technically, Cicero uses a recommender system (Pitako) that using the A-Priori algorithm, learned the multiple and common sequence of actions, sprites, and rules that compose all the database of games in the~\acrshort{gvgai} system. Thus, the suggestions that the designer receives are based on their creation and the statistics behind it in the system rather than exploring possible solutions as for instance, in the Sentient Sketchbook. Machado et al. evaluated Cicero in a user study with eighty-seven students demonstrating that it increased the users' levels of accuracy and computational affect when assisted, and supported one of the main benefits of~\acrshort{mi} systems, the decrease of participant's workload~\cite{machado_evaluation_2019}.

Tanagra presents a collaborative scenario where the designer can create platform levels together with an AI that focuses on menial tasks of the creation process, and which in any moment the designer can request to ``fill the blank''~\cite{smith_tanagra_2011}. Throughout the design process, the designer can place constraints with actual platforms. The AI using a reactive planner either creates a playable level considering the constraints or informs the designer that no level can be created satisfying the set of constraints. Through this, the design process shifted from focusing on the correct placement of platforms, respecting all the possible game rules, to focusing on providing subjective evaluation and exploring the generated content.

% Moreover, as part of the development of Tanagra, Smith and Whitehead presented a generic way to evaluate content generators through expressive range analysis~\cite{Smith:2010:Expressive-range} Tanagra is an early example of an~\acrshort{micc} tool 

While Tanagra presents an approach where the computational designer is designated to ``fill the blank'' based on the designer's design, more autonomy and initiative can be given to the computational designer for creating content in a continuous design process with the same premise. Morai Maker is a~\acrshort{micc} tool to co-create levels in the Mario AI framework~\cite{karakovskiy_mario_2012} (a Super Mario Bros.~\cite{mario} clone for AI research\footnote{Ahmed Khalifa is the current mastermind behind the Mario AI Framework: \url{https://github.com/amidos2006/Mario-AI-Framework}}) through turn-taking phases between designer and computational designer~\cite{guzdial_co-creative_2018}. The designer is initially in command of creating the first sketch of the level. Then by passing the turn, the computational designer can add content to the level and when finished, passes the turn and so on and so forth, until the designer is satisfied with their creation. One of the main innovations of the work by Guzdial et al. is that the computational designer is trained through~\acrshort{rl}, learning as it takes each turn since the designer can delete unwanted content created by the computational designer. Through this, the computational designer continuously learns to adapt to the designer's requirements and goals with positive and negative reinforcement.
% , akin to the creature in Black and White~\cite{BW2}.

% The computational designer in Morai Maker is trained through~\acrshort{rl}, learning 
% While extending such interaction to give more autonomy to the computational designer is 

Moreover, Lucas and Martinho presented 3Buddy~\cite{lucas_stay_2017}, a~\acrshort{micc} system to create dungeons in the game Legend of Grimrock 2~\cite{legGrim2}, where the computational designer acts as a colleague working in lockstep. Like Morai Maker and Tanagra, and with the idea of a conversation between agents, the designer is suggested variations to their current design when requested, which they can use to replace their design, discard it, or use parts of it. The computational designer uses an~\acrshort{ea} generating individuals in three different pools: \textit{convergence}: similarity between current design and generated individuals, \textit{innovation}: dissimilarity between current design and generated individuals, and \textit{guidelines}: following human-input constraints. The most interesting aspect of 3Buddy is that the designer can specify an area where they will work on and another where the computational designer should focus, thus working simultaneously on different areas of the dungeon.

Furthermore, Karth and Smith's approach uses a modified version of the WFC algorithm~\cite{karth_wavefunctioncollapse_2017}, which while not strictly a~\acrshort{micc} system; their approach focuses on the designer providing positive or negative examples to the algorithm, for it to use it to generate variations following such rules. Their novel approach presents a different design process somewhat similar to Morai Maker. Designers show the algorithm what they like and dislike to drive the algorithm's output to their goal~\cite{karth_addressing_2019}. 


% I am missing baba is y'all~\cite{charity2020baba}.

%Recently,~\acrshort{mi} was proposed to be used in the setting of teaching young children handwriting in a tool called Djehuty, which leverage the use of technology in developing countries to foment literacy. Djehuty continuously generates handwriting styles and suggest them as paths to the child~\cite{sarr_djehuty_2020}. Djehuty is another example of~\acrshort{mi}'s strengths, and as described above,~\acrshort{mi} can be used virtually in any collaborative scenario where agents can leverage in their strengths to contribute to a solution proactively. 

% It uses level design patterns to provide information to the designer and to drive the generation of suggestions for the designer~\cite{alvarez_empowering_2019}.

This thesis revolves around the~\acrfull{edd}, a~\acrshort{micc} tool to co-create adventure and dungeon crawler games, particularly, developing their narrative~\cite{alvarez_questgram_2021,alvarez_story_2022} and levels~\cite{alvarez_empowering_2019}.~\acrshort{edd} uses the~\acrlong{icmape}~\acrshort{qd} algorithm to continuously suggest adaptive, diverse, and high-performing solutions~\cite{alvarez_interactive_2020}. As~\acrshort{edd} is the main research tool developed and used in this thesis, a chapter is reserved for presenting the tool, all its features and algorithms, and discussing the main contributions around it.

However, while~\acrshort{micc} systems bring many benefits to design tools such as reducing workload, fostering creativity, providing adaptive experiences, learning design concepts, making game design tools more accessible, or creating various experiences, they have not being adopted by the game industry yet~\cite{partlan_design-driven_2021}. This is because, firstly,~\acrshort{micc} tools and common computer-aided design tools such as game engines (Unity, 2005; Unreal Engine 4, 2014), differ in their goals. In the former, the focus is on leveraging each agent strengths and where one's weakness, such as lack of knowledge in game design, can be supplied by the other agent. For instance, using game design patterns to help designers build levels~\cite{baldwin_mixed-initiative_2017,dahlskog_procedural_2014}; thus making these tools more accessible. In the latter, the focus is on providing a plethora of interconnected tools and systems unified in a system that relies on the designer having the complete initiative and expert knowledge to connect the bits that form the design of the game. Secondly, to have a natural dialogue and collaboration between AI and designers as discussed by Horvitz~\cite{horvitz_uncertainty_1999}, both need to understand each other design processes such as intentions and goals. Thirdly, to enable more autonomy in the interaction between human and machine, and give a varying degree of initiative to the machine to co-create the game content a game designer has as a goal, these tools are required to identify and use different designer's processes and design procedures. Therefore, the following section is devoted to discussing \emph{designer modeling}, an approach to achieve the before-mentioned third point, through modeling certain designer's processes and use them to drive the generation of content. 

\subsection{Modeling players and designers}

% Machine Learning (ML) has gained increased interest from game researchers, achieving remarkable success on training AI agents for very popular games, such as AlphaStar on Starcraft 2 \cite{alphastarblog} and OpenAI Five on Dota 2 \cite{berner2019dota}. Its combination with PCG has led to the raise of  Procedural Content Generation via Machine Learning (PCGML), defined as the generation of game content by models that have been trained on existing game content \cite{summerville2018procedural}, with applications to autonomous content generation, content repair, content critique, data compression, and mixed-initiative design. 

Player modeling relates to the study of players in-game to compose computational models on the player's characteristics that arise when interacting with games as cognitive, affect, and behavioral patterns~\cite{yannakakis_player_2013,thawonmas_artificial_2019}. Through this, the aim is to understand the player's experience when interacting with a game. Player modeling usually relies on data-driven and~\acrshort{ml} approaches with user-generated gameplay data, and have been used with a vast amount of goals. For instance, for automating playtesting~\cite{holmgard_automated_2019,gudmundsson_human-like_2018}, identifying player types (using Bartle's taxonomy~\cite{bartle_hearts_1996}) based on their playstyle~\cite{drachen_player_2009}, to understand and model in-game player's motivations~\cite{melhart_your_2019}, or for market purposes, to understand how players play and are engage in free-to-play games~\cite{saas_discovering_2016,rio_time_2020,guitart_forecasting_2019}.

%Using player data from \emph{Iconoscope}, a freeform creation game for visually depicting semantic concepts, Liapis et al. trained and compared several ML algorithms by their ability to predict the appeal of an icon from its visual appearance~\cite{liapis_modelling_2019}.  Furthermore, Alvarez and Vozaru explored personality-driven agents based on individuals' personalities using the \textit{cibernetic big five model}, evaluating how observers judged and perceived agents using data from their personality test when encountering multiple situations~\cite{alvarez_perceived_2019}.
%  using Bartle's player archetypes~\cite{bartle1996-taxonomy}

Furthermore, the combination of Machine Learning (ML) with PCG has led to the rise of Procedural Content Generation via Machine Learning (PCGML), defined as the generation of game content by models that have been trained on existing game content \cite{summerville_procedural_2018}. PCGML has been used for autonomous content generation~\cite{sarkar_towards_2020}, content repair~\cite{siper_path_2022}, mixed-initiative design~\cite{guzdial_co-creative_2018}, or content adaptation~\cite{gonzalez-duque_fast_2021}. The use of user models is essential for the generation of adaptive and tailored content, and when discussed in the context of PCG, usually relates to experience-driven PCG~\cite{yannakakis_experience-driven_2011}. 

Content adaptation can take place as players play or use the content online or offline, building models from collected data. For instance, Duque et al. adapt and adjust the difficulty of generated content as players play the game using bayesian optimization~\cite{gonzalez-duque_fast_2021}. Summerville et al. model players automatically and implicitly by learning from video traces; generating levels that correspond to the latent player models~\cite{summerville_learning_2016}. Player models could also be used to enhance and adapt design tools, specifically MI-CC tools~\cite{migkotzidis_susketch_2021,holmgard_automated_2019}. Further, training models on gameplay data from \emph{Tom Clancy's The Division} have also been used to model, and therefore find predictors of player motivation \cite{melhart_your_2019}, which renders a very valuable tool for understanding the psychological effects of gameplay. Former research followed a similar approach in \textit{Tomb Raider Underworld}, training player models on high-level playing behavior data, identifying four types of players as behavior clusters, which provide relevant information for game testing and mechanic design \cite{drachen_player_2009}. Melhart et al. take these approaches one step further by modeling a user's \textit{Theory of Mind} in a human-game agent scenario \cite{melhart_i_2020}, finding that players' perception of an agent's frustration is more a cognitive process than an affective response.

%Content adaptation is of interest in this thesis as using user models is an .. A promising PCGML usage is in the area of content adaptation, where using player and user models are essential to adapt the generated content~\cite{Duque2021-BayesianbasedPlayerModel,togelius2007-AutomaticPersonalisedRaceGames,Yannakakis2011-experiencedrivenPCG}. 

%Content adaptation can take place as players play or use the content online or offline, building models from collected data. For instance, Duque et al. adapt and adjust the difficulty of generated content as players play the game using bayesian optimization~\cite{Duque2021-BayesianbasedPlayerModel}. Summerville et al. model players automatically and implicitly by learning from video traces; generating levels that correspond to the latent player models~\cite{Summerville2016-LearningPlayerTailoredPlatformer}. Player models can also be used to enhance and adapt design tools, specifically MI-CC tools. Migkotzidis and Liapis use player models as surrogate models to generate content assisting game designers in the creation of more relevant content for specific players~\cite{Panagiotis2021-susketch}. Similarly, Holmgård et al. use player personas based on player archetypes as content critics to help designers adapt their content to different archetypes~\cite{Holmgard2019-proceduralPersonas}. Their work on player personas is similar to our proposed work, yet instead of personas based on player archetypes, we propose personas based on design style.

%Moreover, Yannakakis and Togelius discussed how the player experience could be modeled and used to drive the generation of new game content, and in this way, create content that is adapted to the experience and expectations of the player~\cite{yannakakis_experience-driven_2011}. Further, training models on gameplay data from \emph{Tom Clancy's The Division} have also been used to model, and therefore find predictors of player motivation \cite{melhart_your_2019}, which renders a very valuable tool for understanding the psychological effects of gameplay. Former research followed a similar approach in \textit{Tomb Raider Underworld}, training player models on high-level playing behavior data, identifying four types of players as behavior clusters, which provide relevant information for game testing and mechanic design \cite{drachen_player_2009}. Melhart et al. take these approaches one step further by modeling a user's \textit{Theory of Mind} in a human-game agent scenario \cite{melhart_i_2020}, finding that players' perception of an agent's frustration is more a cognitive process than an affective response.

\subsubsection{Designer Modeling}
% \subsubsection{Designer Modeling and Designer Understanding}

Understanding player behavior and experience, as well as predicting the player's motivation and intention, is key for mixed-initiative creative tools while aiming to offer in real-time user-tailored procedurally generated content. Nevertheless, the main user of~\acrshort{micc} tools are designers, and gameplay data is replaced by a compilation of designer-user actions and AI model reactions over time while both user and model are engaged in a mutually inspired creative process. A fluent~\acrshort{micc} loop should provide good human understanding and interpretation of the system, as well as accurate user behavior modelling by the system, capable of projecting the user's subsequent design decisions~\cite{compton_casual_2019}. In the same line, goal thirteen in the guidelines for Human-AI interaction \cite{amershi_guidelines_2019} highlights the importance of learning from user behavior and personalize the user’s experience by learning from their actions over time. 

Shifting towards a designer-centric perspective means that besides focusing on player modeling, it is necessary to focus on modeling the designers. Liapis et al.~\cite{liapis_designer_2013,liapis_designer_2014} introduced designer modeling for personalized experiences when using computer-aided design tools, with a focus on the integration of such in automatized and mixed-initiative content creation. The focus is on capturing the designer's style, preferences, goals, intentions, and iterative design process to create designer models. Through these models, designers and their design process could be understood in-depth, enabling adaptive experiences, further reducing their workload and fostering their creativity. Most of the focus has been on player modeling when generating content~\cite{migkotzidis_susketch_2021,holmgard_automated_2019}, but the nature of~\acrshort{micc} systems and its \emph{adaptiveness} goal require this designer-centric shift. Modeling preferences has been the focus on some recent work using as a proxy the designer's content and their choices~\cite{liapis_adapting_2012,alvarez_learning_2020,halina_threshold_2022}, but other modes such as style, underlying goals, or design process are interesting avenues~\cite{alvarez_designer_2022}. However, how to capture this content, how to operationalize it into working models, or where to apply them in the pipeline are open questions. 

%As a step towards designer modeling, preferences 

%Modeling individual designers preferences are a step towards designer modeling~\cite{liapis_adapting_2012,alvarez_learning_2020,halina_threshold_2022}   Halina and Guzdial explored the use of learned thresholds to retrain ML models that would suit the designer's preference~\cite{halina_threshold_2022}.

%Establishing different roles such as colleague and collaborator might require some user model within the system. Designer modeling, as defined by Liapis et al.~\cite{liapis_designer_2013}, is a way to classify and predict a designer's style, goals, preferences, and processes. Preference models~\cite{alvarez_learning_2020,liapis_adapting_2012} have been built based on designers' choices and used as surrogate models to evaluate further generated content. Similarly, using the designers' creation, the designers' processes and styles could be modeled to inform other systems and adapt the generated content~\cite{liapis_designer_2014,alvarez_designer_2022,halina_threshold_2022}.

As part of this thesis work, two approaches to model different designer's processes have been proposed, the designer's preference model~\cite{alvarez_learning_2020} (\textsc{paper v}), and design style cluster together with designer personas~\cite{alvarez_designer_2022} (\textsc{paper x}). The work presented in \textsc{paper v} introduced the Designer Preference Model, a data-driven solution that learns from user-generated data in the~\acrshort{edd}. This preference model uses an Artificial Neural Network to model the designer's preferences based on the choices they make while using~\acrshort{edd}, which is then used to drive the content generation. Moreover, The work presented in \textsc{paper x} uses data from the design process of 180 sessions to analyze the room styles created along the process, yielding twelve clusters representing such styles. The design process was again analyzed in function of these formed clusters, where we encountered four archetypical paths, i.e., designer personas, that were most commonly taken by designers with the aim to be used to drive the generation of content towards more adapted content. 

% \paragraph{Style}

% Style, while subjective for each designer, can be defined 

% \paragraph{Goals}

% While designers' main goal (or long-term goal) is to create a game and the game experience, they expect the player, the short- and mid-term goals are not straightforward. For instance, as the designer is creating a room, the goal of the room could be inferred by its neighbors. However, it is not until completed that some estimation can be given, and even then, the designer's goal might be completely different. Another example would be if a designer designs an enemy for a platform game, 

% Similarly, if a designer designs an enemy for 

% \paragraph{Intentions}

% Refers to 1) the intentions designers have when creating content, especially the ones that are hard to recognize or identify through heuristics. For instance, blocking the pass to a door in a room but creating access from a different area, or the placement of walls to create cover against enemies are just a couple of examples of non-intuitive designer's intentions that are hard to be recognized. and aligned with this, 2) the intentions designer

% 3) the expected intentions of the designer as they create content, for instance, creating a challenging room might mean that the designer would like to create more lenient rooms to balance. Aligned with this, 3) the designer's intentions for the player or the expected player/game experience, which in most cases is not easily identified in the created content. For instance, designers might focus on creating a vast amount of cover through walls in rooms for protecting players in combat whereas other designer might add the same amount of walls to create obstacles for players. 

% \paragraph{Design and Creative Process}

% The design and creative process is far from trivial, 

% Of course, the process is also constrained by what the tool allows you to do.

% Refers to the process most designers go through to create the content.

% \paragraph{Preferences}

% Preference is by far the most subjective of all the processes and procedures. Each designer develops preferences, not only as they gain more experience but even as they use the systems. For instance, one moment, the designer might be completely sold into preferring open areas with clear paths for players, and next (as they go through trial-and-error), they prefer smaller rooms with. Moreover, as they 

\subsection{Computational Creativity}

Creativity is ``the ability to produce work that is both novel (i.e., original, unexpected) and appropriate (i.e., useful, adaptive concerning task constraints)~\cite{sternberg_concept_1999}''. How creative processes occur, how an individual might come up with novel ideas, or how to assess creativity is very much an open research area~\cite{sternberg_handbook_1999,boden_creative_2004,sternberg_creativity_2005,csikszentmihalyi_creativity_1997}. Moreover,~\acrlong{cc} is a multidisciplinary field that studies computational systems that demonstrate human-like creative behaviors~\cite{colton_computational_2012}. As a multidisciplinary field,~\acrshort{cc} is not only interested in the algorithms or the outcome; it also aims to study the creative process and psychological causes of creative behaviors. Thus, through~\acrshort{cc}, some core concepts and research areas in creativity can be addressed. For instance, in \textit{the Creative Mind: Myths and Mechanism}, Boden studies and analyzes \emph{Creativity} and \emph{creative behaviors} with the use and help of~\acrshort{ai} through the lenses of~\acrlong{cc}. Boden discusses three forms of creativity: \textit{combinatorial}: combining existing knowledge in unfamiliar ways to produce new artifacts; \textit{exploratory}: exploring the conceptual space to encounter possible ideas;~\textit{transformational}: transforming the conceptual space, the imposed constraints, and the encountered ideas~\cite{boden_creative_2004}.

Within~\acrshort{cc}, games have been proposed as the optimal artifact to create to test the creative-like abilities of a~\acrshort{cc} system, since games are \emph{content-intensive}, \emph{multi-faceted content}, and should be~\emph{interacted with and experienced}~\cite{liapis_computational_2014}. As described above, game content relates to the main facets that represent any game: audio, visuals, narrative, levels, rules, and gameplay~\cite{liapis_orchestrating_2019}. Thus, creating systems that develop, to some extent, games poses an interesting application and challenge for~\acrshort{cc}, which can address some of the core questions in~\acrshort{cc}. For instance, investigating the creative process not only to create one type of content but the arrangement of such in a harmonious way as a team of humans creatively does, or the assessment of such content.

Using the combinatorial creativity form from Boden, Guzdial and Riedl proposed conceptual expansion. Conceptual expansion is an approach that combines neural networks trained to recognize or generate specific content to produce a \textit{combinet} that could be used to recognize or generate novel content, which lacks enough data to use it to train a new ml model~\cite{guzdial_combinets_2018}. Moreover, they applied their approach to the conceptual expansion of games, with the same idea of creating novel combinations of games from a set of models trained to produce content for specific games~\cite{guzdial_conceptual_2020}. In the same line, Sarkar et al. proposed the use of variational autoencoders (VAE) to create new levels by training the VAE with game levels from Super Mario Bros. and Kid Icarus. Through this, the VAE learns a representation of both game levels, and using~\cite{sarkar_controllable_2019,sarkar_generating_2021}.

Moreover, Mikkulainen discuses the use of~\acrlong{ec} to achieve creative AI, which refers to the use of AI not only to create and perform creative tasks such as generating games, but also to encounter creative solutions to complex multidimensional problems. In his work, he reflects on the aims of the~\acrshort{ai} field and discusses the use of search-based approaches for exploring complex multidimensional spaces filled with ``unknown unknowns'' with exciting results~\cite{miikkulainen_creative_2021}. This is further supported by the collection of Lehman et al.~\cite{lehman_surprising_2020} that presents creative examples by a myriad of~\acrshort{ec} researchers. Likewise, Sarkar discusses leveraging on creative AI techniques to approach game design, and with such demonstrated exploratory work on how it could be achieved, and the benefits from it~\cite{sarkar_game_2019}. Specifically, Sarkar discusses the co-design aspect that can be enabled through creative AI techniques, which is especially relevant for this thesis and the development of effective~\acrshort{micc} systems.

% conceptual  expansion,

% Guzdial and Riedl took the exploratory and combinatorial forms 

% Creativity is the main 

% Moreover, Mikkulainen discusses 
% Moreover, while Miikkulainen does not specifically refer to Computational Creativity, his recent work on Creative AI, reflects the aims of the AI field into using search-based approaches for exploring complex spaces filled with ``unknown unknowns", and reaching good results~\cite{miikkulainen2020-creativeAIEVO}. Likewise, Sarkar discusses ~\cite{sarkar2019-GameDesignCreativeAI}

% Similarly,~\acrlong{qd} algoritms were conceptualized under the same \textbf{pretext}, since a plethora of problems require more than convergence search, i.e. searches driven by objectives~\cite{stanley2015-mythObjective}. On the other hand, while divergent search has demonstrated many strengths, especially in deceitful environments~\cite{Novelty-Lehman2011}, when the search space is unconstrained in some dimensions, divergent search encounter challenges into finding optimal solutions~\cite{Lehman2010-MCNS}. Therefore,~\acrlong{qd} algorithms were conceptualized, where divergent and convergence searches are combined to highlight and use the strengths of both with many applications such as in robotics~\cite{Cully2015-qdRobotsAnimals} or games~\cite{gravina2019procedural}. Remarkably,~\acrfull{mape} is an algorithm with a simple process that has been demonstrated to work extremely well~\cite{Mouret2015}.



% unless the space is constrained in some way~\cite{Lehman2010-MCNS}

% \subsubsection{Evolutionary Computation for Computational Creativity}


% \subsection{Summary}

% In this chapter, it was described the main areas of concern and various research that are essential for the dissemination of this thesis. At the beginning, it was given an overview of the main research area~\acrlong{pcg}