\section{BACKGROUND} \normalfont \label{background}

This chapters offers an overview of the different fields that surround the central subject of study in this thesis, i.e. the collaboration between AI and humans to co-create game content, and the related RQs. First,~\acrlong{pcg} is explored with multiple examples on the type of content that might be created. Followed, it is presented the search-based approach, quality-diversity algorithms, and the~\acrlong{mi} paradigm as they are the main approaches and paradigms used throughout the thesis. Then, player and designer modeling is presented to give an overview of the concepts and the differences between them, and examples of each computational model. Finally, creativity and computational creativity are explored by briefly analyzing the goals of the field with the most relevant literature, and by presenting examples within the computational intelligence in games research area.


% In this section, the aim is to give a clear overview of the different fields that surround the central subject of study in this thesis, i.e. the collaboration between AI and humans to co-create game content, and the related RQs. Besides this, Through this section, we aim at providing a preliminary literature review of multiple~\acrlong{micc} systems that have been created so far, mostly focusing on the collaboration to create game content. This preliminary study is a separate contribution to the publications presented in this thesis, and will hopefully, trace the path on identifying gaps in the levels of interactions and collaborations that currently exist in ~\acrshort{micc} systems within academia to co-create game content. 

% \subsection{Games}

% Games

\subsection{Procedural Content Generation}

% \begin{itemize}
%     \item Type of content \checkmark
%     \item pcg definition \checkmark
%     \item example of games using pcg \checkmark
%     \item how about describing properties?
%     \item example of pcg algorithm used to create content \checkmark
%     \item division into the main branches and examples on them, ending on the paradigm of mixed-initiative co-creativity.\checkmark
% \end{itemize}

% Part of PCG's allure is the promise to produce game art and content faster and cheaper, as well as enabling innovative content creation processes such as player-adaptive games~\cite{shaker2012evolving, hastings_evolving_2009, dormansUnexplored2017}, data-driven content generation~\cite{Khalifa2018, Green2018}, and mixed-initiative co-creativity~\cite{Liapis2014}.

Game content is the main component of any game, as it is what players interact with to achieve the designers' developed expected experience. Game content refers to anything within the game, from the rules of the game, a hero's backstory, or the levels to be traversed by players. Furthermore, as higher possibilities for more complex games are provided by technology, game engines, and platforms, and higher requirements are set by developers and players, games have increasingly become content-intensive entertainment mediums. To cope with such a challenge and to relieve some of the burden and workload of game designers when creating all this content, several approaches have been proposed to create content under the field of~\acrlong{pcg}.~\acrshort{pcg} refers to the creation of content, mainly for games, using algorithms, autonomously or with the assistance of users~\cite{Yannakakis2018}. Content can be divided into game facets: audio, visuals, narrative, levels, rules, and gameplay~\cite{Liapis2019-OrchestratingGames}, and have been categorized within the~\acrshort{pcg} field as \textit{Game Bits}, \textit{Game Space}, \textit{Game Systems}, \textit{Game Scenarios}, \textit{Game Design}, and \textit{Derived Content}~\cite{Hendrikx2013-pcgSurvey}. 

There are plentiful of commercial games that utilize one way or another~\acrshort{pcg} such as The Binding of Isaac~\cite{bindingISAAC}, Civilization~\cite{civilization}, to the point that some games rely critically on these algorithms, providing experiences otherwise not possible such as Rogue~\cite{rogue}, Dwarf Fortress~\cite{dwarfFortress} or AI Dungeon~\cite{aidungeon}. However, it is in academia where most of the work in~\acrshort{pcg} has originated and developed with a crescent interest during the past decade~\cite{Liapis2020-pcgWorkshop}. Furthermore, there exist multiple approaches addressing different challenges in the creation of content, resulting in algorithms that can autonomously create game rule's~\cite{Browne2010-ludii,font2013-GenCardGames}, narratives~\cite{ashmore2007-questGeneratedWorld,ammanabrolu2019-towardQuestGeneration}, levels~\cite{Shaker2012EvolvingEvolution,sarkar2020-sequentialVAELVLGen,green2020-marioStitching}, graphics~\cite{Horsley2017-SpriteGenerator,Pagnutti2016-CreativeArtbot}, and audio~\cite{Scirea2016-MetaCompose,HooverPhD}. Other approaches have focus on creating content in multiple facets aiming at creating intertwined content~\cite{hoover2015-audioinspace,Cook2014-ARogueDream,Treanor2012Game-O-Matic:Ideas,Holtar2013Audioverdrive:Gameplay,karavolos2019multi}, and others on creating complete games~\cite{Browne2010-ludii,guzdial2020-conceptualGameExpansion,Cook2016-Angelina2}. 

% \footnote{with some exceptions such as Modl.ai~\cite{modl_ai} or Wave Function Collapse~\cite{wfc-}}

% Within the field of~\acrshort{pcg} there exist [arguably] three main algorithm branches~\cite{shaker_procedural_2016}, search-based algorithms~\cite{Togelius2011}, constructive algorithms, and test-and-generate algorithms, with PCG via ML~\cite{summerville2018procedural} as an upcoming one. Search-based approaches aim at using some type of search algorithm, mainly~\acrlong{ea}s in order to generate content by exploring the generative space and through this process, encountering interesting individuals with non-trivial characteristics~\cite{hastings_evolving_2009}. Constructive approaches are agnostic of the quality of  focus on using . Generate-and-test approaches focus on creating content that as it is generated, it is tested  so that it satisfy a set of constraints, hence, these approaches rely on not only creating quality content, but also content adapted to the specified set of constraints. In this approach, the designer's focus is on creating the set of constraints to be satisfied rather than anything else, and have shown promising results~\cite{Karth2019-pcgmlDiscriminativeLearning,Summerville2018-GEMINI}.

% As presented thus far, within the field pf~\acrshort{pcg} there exist many different approach to create content. However, there are three main approache

% Broadly, within the field of~\acrshort{pcg} there exist many differe three main approaches to create content, constructive approach, generate-and-test approach, and search-based approach, each with their own criteria~\cite{Togelius2011}.

Broadly, within the field of~\acrshort{pcg} there exist [arguably]\footnote{While most algorithms belong or extend from the constructive or generate-and-test approaches, some techniques do not belong to either. For instance, Wave Function Collapse do not belong, presented as its own category, Constraint Solving~\cite{Karth2017-WFC}.} three main approaches to create content, constructive approach, generate-and-test approach, and search-based approach, each with their own criteria~\cite{Togelius2011}. Constructive approaches focus on generating content following a set of predefined rules that can create valid content without evaluating the quality of the content after generating, rather the content is evaluated as it is being constructed~\cite{Karth2019-pcgmlDiscriminativeLearning,Snodgrass19-BSPExampleDriven}. Conversely, generate-and-test approaches focus on creating content iteratively that instead of being continuously tested as the content is constructed, it is tested after generation so that it satisfy a set of constraints or objectives. When tested, the process might iterate on the design. In this approach, the designer's focus is on creating the set of constraints to be satisfied~\cite{Summerville2018-GEMINI,Volz2018-GANevo}. Search-based approaches are a specialized case of the generate-and-test approach that aim at using some type of search algorithm, mainly~\acrlong{ea}s in order to generate content by exploring the generative space and through this process, encounter interesting individuals with non-trivial characteristics~\cite{hastings_evolving_2009,Font2016ConstrainedAlgorithms}.

% Perhaps I can briefly discuss here PCG via ML, specialized version in PCG via RL, PCG via constraint solving, and 

In this thesis, the focus is mainly on using a search-based approach to generate suitable content to be suggested to a designer in an interactive tool through~\acrshort{qd} algorithms~\cite{gravina2019procedural}. Our approach, rely on exploring the generative space informed by a designer's design that helps focus the search in different areas of the space, while still encountering diverse solutions for the designer.

\subsubsection{Search-based Approach}

The search-based approach is a specialization of the generate-and-test approach, where the aim is to use some type of search algorithm, being the most prominent,~\acrlong{ea}. However, essentially any metaheuristic algorithm and from the stochastic search algorithm family could be as well used and fall under the umbrella of search-based approaches. The main distinction between search-based approach and generate-and-test approach, is that search-based approaches evaluate the generated solution with a quality estimator, e.g., fitness function or novelty behavior, providing a continuous evaluation of the generated content. Such evaluation drives the next generation steps, as the estimation helps the search to find promising paths.

Search-based approach has been widely used in~\acrshort{pcg} and basically for the generation of all the types of game content such as levels~\cite{dormans2011generating}, rules~\cite{font2013-GenCardGames} or weapons~\cite{Gravina2016-WeaponGenSurpriseSearch}. Moreover, the evaluation of the generated content is the most important part of search-based approaches, as well as the most challenging and complex. The used heuristics does not only need to be representative of the task at hand, but also allow the expressive property of the search, as that is one of the main benefits of search-based approaches. Constraints to ensure quality [or playable] experiences are not enough, since that does not necessary represent what a designer or player wants~\footnote{One of the challenges of generating games [and game content], is that it requires them to be enjoyable and interacted as discussed in the previous section.}. However, evaluation functions come in all shapes and sizes. Whether the evaluation functions come from game design concepts such as design patterns~\cite{dahlskog2014-PCGDesignPatterns} or game level metrics~\cite{canossa2015-towardspcgEvaluation}, or from aesthetic indicators such as symmetry~\cite{Marinho2015-empiricalEvaluation}, or from subjective evaluation from users~\cite{Schrum2020-IE_GAN}, or even continuously adapting the evaluation based on gameplay~\cite{hastings_evolving_2009} or to the designer's preferences~\cite{Liapis2012-adaptiveVisual}, they are all valid with their own set of ups and downs.


% Evaluation functions can originate from different areas, game design concepts such as design patterns~\cite{dahlskog2014-PCGDesignPatterns} or , . In short, whether the  come in all shapes and sizes; whether 


% To ensure that quality and subjective measurements are  expresiveness, quality, and subjective 

% Moreover, content does not need to be generated by


% Moreover, content has been generated using different approaches


% a which helps drive the next generation steps continuously evaluate the generated solutions

% \begin{itemize}
%     \item Very brief introduction and examples since this is slightly covered before.~\cite{hastings_evolving_2009,Volz2018-GANevo,Schrum2020-IE_GAN,dormans2011generating,green2020-marioStitching, Colton2020-EvoArtCasualCreation}
% \end{itemize}

\subsubsection{Quality Diversity} \label{sec:Backqd}

\acrfull{qd} algorithms are a family of algorithm under the approaches in~\acrlong{ec}, that focuses on combining the benefits and strengths of both convergence search, i.e., focusing on optimization and objective, and divergence search, i.e., disregarding objectives and searching for diversity~\cite{Pugh2016,Gaier2019-QDSteppingstones}. Through this,~\acrshort{qd} algorithms seek to generate a collection of high-performing solutions that are as diverse as possible\footnote{The following website serves as a database with research related to~\acrshort{qd}: https://quality-diversity.github.io/ maintained by Antoine Cully}. While convergence search refer mainly to the typical~\acrshort{ec} algorithms used for optimization, divergence search has increasingly being used to tackle many tasks that were previously dominated by convergence search. For instance, when the task and/or environment is deceptive, i.e., reaching the goal might be impossible, or where plenty local optima exist where a convergence search might get stuck. Lehman and Stanley proposed the Novelty Search algorithm, which introduces the idea of divergence search through ignoring objectives and searching for novel behaviors instead, with surprisingly good results~\cite{Novelty-Lehman2011,Lehman2010-MCNS}. From that moment onward, several divergent search algorithms have been proposed such as surprise search~\cite{Surprise-Gravina2016}, as well as variations to novelty search such as constrained novelty search~\cite{liapis2015-ConstrainedNoveltySearch} or~\acrfull{nslc}~\cite{Lehman2011-NSLC}.

% One approach might be to use Novelty Search, which disregards the traditional fitness-based objective in lieu of producing novel individuals [19]. By encouraging agents to express different behaviors from other individuals, Novelty Search overcomes issues of deception [19] and has been shown to produce agents with general exploratory skills [9]. At a cursory glance, one might therefore expect Novelty Search to produce generalists. However, we argue that upon further reflection, it should be clear that Novelty Search actually produces specialists, in which each individual possesses a subset of skills, but no one organism possesses all skills.

\acrshort{nslc} is an example of a~\acrshort{qd} algorithm that leverage on divergent search to explore the space for novel behavior among solutions and on convergence search for preserving the high-performing individuals within the novel niches~\cite{Lehman2011-NSLC}.~\acrfull{mape} is another algorithm in the~\acrshort{qd} family, and one that have gained considerably popularity in multiple areas such as games~\cite{charity2020mech,Fontaine2019-hearhstoneDecks} and robotics~\cite{Cully2015-qdRobotsAnimals}. As the other~\acrshort{qd} algorithms,~\acrshort{mape} explores the behavioral space for a collection of solutions that are both high-performing and diverse among each other, with the caveat that~\acrshort{mape} discretises the behavior space as a grid of cells informed by a set of feature dimensions that illuminate the behavior space.~\acrshort{mape}'s goal is to fill each cell belonging to a set of discrete feature dimension values with a high-performing individual encountered in the search and retain it until a higher-performing individual with similar characteristics is encountered~\cite{Mouret2015}.
% \footnote{Since~\acrshort{mape} is one of the main algorithms explored and used, it is explained in more detail in section~\ref{sec:MAPE}}.

One major challenge with~\acrshort{mape} is the \emph{curse of dimensionality}, since each new feature dimension used, adds a new dimension in the search space. Thus, some~\acrshort{mape} variation skip the grid architecture and focus on reducing the amount of feature dimensions or enabling the use of higher dimensions such as~\acrlong{cvtmape}~\cite{cvt-mape2016} or~\acrlong{ce}~\cite{cluster-mape2017}. Further, the~\acrlong{cmame} algorithm combines the effective adaptive search of~\acrlong{cmaes} with a map of elites, yielding large improvements for real-valued representations in terms of both objective value and number of elites discovered~\cite{fontaine2019covariance}. The work by Fontaine et al. was expanded into the~\acrlong{memape}, improving the quality, diversity, and convergence speed of~\acrshort{mape} in general~\cite{cully2020-multiemitter}.

Moreover, within the field of games and~\acrshort{pcg},~\acrshort{qd} algorithms have started to be used extensively, especially~\acrshort{mape}~\cite{gravina2019procedural}.~\acrshort{mape} has been used to create and find levels with just the right difficulty for a set of agents~\cite{Gonzalez-Duque2020-DifficultyTrialError}, to balance and create decks in hearthstone~\cite{Fontaine2019-hearhstoneDecks}, or create levels for puzzle games through crowdsourcing~\cite{charity2020baba}. Constrained~\acrshort{mape} introduced by Khalifa et al.~\cite{Khalifa2018}, combines~\acrshort{mape} with the~\acrfull{fi2pop} algorithm~\cite{Kimbrough2008}, to generate bosses for bullet hells games in Talakat. Since then, constrained~\acrshort{mape} has been used in other projects and experiments to benefit from it's strengths such as to generate a game levels based on mechanics as feature dimensions in Mario~\cite{Khalifa2019-intentionalCompLevel,charity2020mech}, and was combined with interactive evolution resulting in the~\acrlong{icmape}~\cite{alvarez2019empowering}. 

% constrained mape~\cite{Khalifa2018}, creating levels for mario~\cite{Khalifa2019-intentionalCompLevel}, used to create levels based on mechanics as feature dimensions in mario~\cite{charity2020mech}, to balance and create decks in heartstone~\cite{Fontaine2019-hearhstoneDecks}, in a crowdsource mixed-initiative co-creativity system based on the game baba is you~\cite{charity2020baba}, and used to suggest rooms in the~\acrlong{edd}, a tool to create dungeons~\cite{Alvarez2020-ICMAPE}, and also to find levels with just the right difficulty~\cite{Gonzalez-Duque2020-DifficultyTrialError}. 

% MAP-Elites~\cite{Mouret2015}, Cully multi-emitter~\cite{cully2020-multiemitter} and variants~\cite{Khalifa2018,alvarez2019empowering,charity2020mech,charity2020baba,Khalifa2019-intentionalCompLevel}, etc. BUT REMEMBER THAT USAGE IS DIFFERENT THAN ACTUALLY THE APPROACHES.~\cite{gravina2019procedural}

% Novelty search with local competition (NSLC), POET~\cite{Wang2019-POET}, Minimal criterion coevolution~\cite{Brant2017-MnimalCriterion}, etc.~\cite{dharna2020-cogenerationPOET}. Finding game levels intelligent trial-and-error~\cite{Gonzalez-Duque2020-DifficultyTrialError}, ofc, the ones of heartstone and covariance~\cite{Fontaine2019-hearhstoneDecks,fontaine2019covariance}

Thus far, the focus has been on discussing~\acrshort{pcg} and presenting algorithms that create content mostly autonomously. Automated game design is a complex task since it is required to create content (or full games) by itself with the help of heuristics, user models, and logic among the content created~\cite{Togelius2008-automaticGD,barros2018-DATAeinstein,Cook2016-Angelina1,Cook2020-automatedGDtutorial}. However, another paradigm within~\acrshort{pcg} is the mixed-initiative paradigm, where AI can collaborate with a designer to co-design games. Through this, we could leverage in the strengths of both to create content.


% it must model to some extent, certain subjective human characteristics such as fun and enjoyment, use this ~\cite{Togelius2008-automaticGD,barros2018-DATAeinstein,Cook2016-Angelina1,Cook2020-automatedGDtutorial}. If these algorithms were to be introduced in a design tool, we could leverage on human knowledge and goals; thus, the algorithms could cope with thje subjective constraint, in what is called the mixed-initiative paradigm. However, mixed-initiative tools do have their o
% until now?

% To better understand the content to be created, Content can be divided into game facets: audio,
 
% Games are composed of multiple type of content interacting with each other as specified by the developers, which is ultimately interacted and enjoyed by players. Content in games can be anything from the rules of the game to the levels that are traversed by the player. Liapis et al. classified this content in broader categories as game facets: audio, visuals, narrative, levels, rules, and gameplay~\cite{Liapis2019-OrchestratingGames}. To relieve some of the burden and workload of game designers when creating all this content, several approaches have been proposed to create the content under the field of~\acrlong{pcg}.~\acrshort{pcg} is the creation of content, mainly for games, using algorithms~\cite{Yannakakis2018}. Similarly to how Liapis et al. discuss game content in broader categories as facets, Hendrikx et al. categorized such content as \textit{Game Bits}, \textit{Game Space}, \textit{Game Systems}, \textit{Game Scenarios}, \textit{Game Design}, and \textit{Derived Content}~\cite{Hendrikx2013-pcgSurvey}.


\subsection{Mixed-Initiative Paradigm}

% \begin{itemize}
%     \item introduce MICC~\cite{novick97-mixedInit,horvitz1999principles,Horvitz99-uncertainty,Allen99-MIinteraction}, and present relevant work in the area. \checkmark
%     \item Discuss related work on MI-CC:
%     \item Tanagra~\cite{smith_tanagra:_2011}, Sentient Sketchbook~\cite{Liapis2013-sentientsketchbook:}, Morai maker~\cite{guzdial-lvldsg-aiide-2018}, ropossum~\cite{shaker2013ropossum}, EDD~\cite{Baldwin2017}, Cicero~\cite{Machado2017}, Spaceship generator~\cite{Liapis2012-adaptiveVisual}, 3Buddy~\cite{lucas-3buddy-iccc2017}, Djehuty~\cite{Sarr2020-preschoolMixedInit}, Reinforce Learning brush~\cite{delarosa2020-RLbrushMixedinit}, GEMINI micc~\cite{kreminski2020-Germinate} NEED TO CHECK FOR MORE!
%     ~\cite{Karth2019-pcgmlDiscriminativeLearning} I think this is a valid example
% \end{itemize}

\acrfull{mi} refers to the collaboration between Computer and Human to solve some task where both have a proactive initiative into solving the task regardless of the degree of such initiative~\cite{LiapisPhD}. Yet while this definition clearly separates~\acrshort{mi} approaches from others that ``simply" assist humans in their tasks, it still remains a very disputed concept as: which agent initiates the ``conversation", what task to be solved, and what initiative to take in each step remain unknown. Novick and Sutton discuss~\acrshort{mi} by analyzing a set of~\acrshort{mi}systems, and conclude that the initiative in~\acrshort{mi} is a multi-factor model, described as: \textit{1) choice of task}: describing the task; \textit{2) choice of speaker}: describing which agent is in control and how the interaction works; \textit{3) choice of outcome}: describing what is the outcome of the interaction~\cite{novick97-mixedInit}. Moreover, Allen describes~\acrshort{mi} systems as multi-agent collaboration scenarios. These need to have a flexible interaction strategy, leveraging in each agent strengths to solve the tasks and that involve a continuous negotiation between agents to determine roles, i.e., initiative; thus, collaborating as a team~\cite{Allen99-MIinteraction}. Initiative will vary depending on which agent can solve a determined problem, providing solutions and taking the control while the other agents, e.g., a human or group of models, assist in the procedure~\cite{Ferguson2007-MICollaborativeProbSolv}. Similarly, Horvitz discusses~\acrshort{mi} as a more natural collaboration between agents that explicitly integrate human control and manipulation, and [AI] automation strategies and their contributions to achieve some [shared] task~\cite{Horvitz99-uncertainty,Horvitz99-mixedInit}.

% Mixed-Initiative was defined by Horvitz as the collaboration between human and machine, leveraging in the strengths of both, to create some artifact, which requires a varied initiative from both in different development stages.

\subsubsection{Mixed-Initiative Co-Creativity}

Yannakakis et al. introduced the~\acrlong{micc} paradigm for the co-creation of creative content such as games, and regarding~\acrshort{pcg}, where machine and humans alternate initiative to co-design content~\cite{yannakakis2014micc}. Their work and discussion on the capabilities of such interaction to foster creativity on both human and machine is pivotal for understanding computational creativity, and develop~\acrshort{micc} tools that can reduce the designer's workload, foster their creativity, and in general, improve the design and creative process~\cite{liapis2016-canmixedinitiative,Alvarez2018a}.

Germinate is a~\acrshort{micc} system to co-create rhetorical games using the constraint-based game generator Gemini~\cite{Summerville2018-GEMINI} under-the-hood, and where the designer can in iterations, specify a set of constraints and properties they want games to have and which the generator will consider~\cite{kreminski2020-Germinate}. The designer is then presented a set of games that they can play and inspect, and which they can use to modify the set of constrained previously set, improving their understanding of their own intent. Germinate focuses on accessibility by leveraging on the concept of Casual Creators~\cite{Compton2015-CasualCreators} within the~\acrshort{micc} paradigm, allowing through this iterative process, the designer to focus in the constraint that reflect their intent rather than any knowledge within game technology. 

% Through this iterative process, the designer focuses on the constraint that reflect their intent rather than any knowledge within game technology, and by analyzing the suggested games and .Germinate's focus on accessibility by leveraging on the concept of Casual Creators~\cite{ComptonPhD} exemplifies how~\acrshort{micc} system  evidences another of~\acrshort{micc}  let Gemini create the games, and through
% reassess reevaluate reconsider rethink reconceptualize

Delarosa et al. presented an innovative~\acrshort{micc} system, where the computational designer is represented as three different agents with different representations trained using~\acrfull{rl}, suggesting specific changes to the designer as they create Sokoban levels~\cite{delarosa2020-RLbrushMixedinit}. Their approach is the first implementation of the work by Khalifa et al. that introduced a new approach to create content:~\acrshort{pcg} via~\acrshort{rl}~\cite{khalifa2020-pcgrl}. In~\acrshort{pcg} via~\acrshort{rl}, the level creation process is setup as an RL problem, i.e., a sequential task, where the agent can learn policies to maximize the quality of the final level. Khalifa et al. approach uses three different representations, i.e., different types of agents, to create levels: \textit{Narrow}: at each step the agent is located randomly in the level and can perform an action in such place; \textit{Turtle}: at each step the agent can move and change tiles in the way; and \textit{Wide}: at each step the agent has control of location and placement of tiles. Likewise, Delarosa et al. work includes the same agents and have an identical premise, i.e., level generation as an RL problem, with the caveat that these agents must now learn and adapt to a designer's design. The designer is suggested levels based on their own by each of the agents, which the designer might pick or disregard and continue editing. Their work was evaluated through thirty-nine sessions and showed that on average the levels created using AI suggestions were more playable and complex.

The Sentient Sketchbook is a tool where designers can co-create low-resolution sketches of strategy levels, while being presented augmented information about their creation and suggested variations using multiple heuristics and objectives~\cite{Liapis2013-sentientsketchbook}. In the Sentient Sketchbook the designer focuses mainly on creating the sketch they envision, while the computational designer focuses on three main aspects, provide \textit{suggestions} adapted to the designer's current design using constrained novelty search~\cite{liapis2015-ConstrainedNoveltySearch}; provide \textit{augmented information} on how the level is formed such as resource safety or navmesh; and provide \textit{multiple levels of visualization} that transform the designer's sketch into usable levels. However, the main feature of the tool and its most innovative one, is the suggestions by means of an~\acrshort{ea} powered by three different search algorithms: objective-driven, objective-driven with diversity preservation, and novelty search~\cite{preuss2014-gooddiverseLVL}. The work by Liapis et al. is seminal to analyze and understand how~\acrshort{micc} systems have evolve and the benefits that they have for designers and AI likewise.

Cicero is a special kind of~\acrshort{micc} system, where the focus is on helping designers create complete games in the~\acrfull{gvgai} framework\footnote{http://www.gvgai.net/}~\cite{gvgaibook2019} and~\acrfull{vgdl}~\cite{Schaul2013-vgdl}, rather than individual game content~\cite{Machado2017}. In Cicero, the aim is on letting the designer create the game they want while receiving suggestions on what content might be added next in relation to sprites, mechanics, interactions between entities, stats, rules of the game, in general, any other game design element~\cite{machado2019pitako}. Technically, Cicero uses a recommender system (Pitako) that using the A-Priori algorithm, learned the multiple and common sequence of actions, sprites, and rules that compose all the database of games in the~\acrshort{gvgai} system. Thus, the suggestions that the designer receives are based on their creation and the statistics behind it in the system, rather than exploring possible solutions as for instance, in the Sentient Sketchbook. Machado et al. evaluated Cicero in a user study with eighty-seven students demonstrating that it increased the users' levels of accuracy and computational affect when assisted, and supported one of the main benefits of~\acrshort{mi} systems, the decrease of participant's workload~\cite{machado2019evaluation}.

% Using such a tool, Machado et al. supported through a user study what Yannakakis et al. described in relation to fostering creativy, and what Horvitz presented in relatwork described, in relation to  in addition to demonstrated and 

% Similarly to other, Ropossum is a~\acrshort{micc} tool to co-create levels in the game Cut the Rope~\cite{cutTheRope}, where designers 

Tanagra presents a collaborative scenario where the designer can create platform levels together with an AI that focuses on menial tasks of the creation process, and which in any moment the designer can request to ``fill the blank"~\cite{smith_tanagra:_2011}. Throughout the design process, the designer can place constraints with actual platforms and the AI using a reactive planner either creates a playable level considering the constraints or inform the designer that no level can be created satisfying the set of constraints. Through this, the design process shifted from focusing on correct placement of platform, respecting all the possible game rules, to focusing on providing subjective evaluation and exploring the generated content.

% Moreover, as part of the development of Tanagra, Smith and Whitehead presented a generic way to evaluate content generators through expressive range analysis~\cite{Smith:2010:Expressive-range} Tanagra is an early example of an~\acrshort{micc} tool 

While Tanagra presents an approach where the computational designer is designated to ``fill the blank" based on the designer's design, more autonomy and initiative can be given to the computational designer for creating content in a continuous design process with the same premise. Morai Maker is a~\acrshort{micc} tool to co-create levels in the Mario AI framework~\cite{Karakovskiy2012-MarioAI} (a Super Mario Bros.~\cite{mario} clone for AI research\footnote{Ahmed Khalifa is the current mastermind behind the Mario AI Framework: https://github.com/amidos2006/Mario-AI-Framework}) through turn-taking phases between designer and computational designer~\cite{guzdial-lvldsg-aiide-2018}. The designer is initially in command of creating the first sketch of the level and then by passing the turn, the computational designer is able to add content to the level and when finished, passes the turn and so on and so forth, until the designer is satisfied with their creation. One of the main innovations of the work by Guzdial et al. is that the computational designer is trained through~\acrshort{rl}, learning as it takes each turn, since the designer has the ability to delete unwanted content created by the computational designer. Through this, the computational designer learns continuously to adapt to the requirements and goals of the designer with positive and negative reinforcement from them, akin to the creature in Black and White~\cite{BW2}.

% The computational designer in Morai Maker is trained through~\acrshort{rl}, learning 
% While extending such interaction to give more autonomy to the computational designer is 

Moreover, Lucas and Martinho presented 3Buddy~\cite{lucas-3buddy-iccc2017}, a~\acrshort{micc} system to create dungeons in the game Legend of Grimrock 2~\cite{legGrim2}, where the computational designer acts as a colleague working in lockstep. Similar to Morai Maker and Tanagra, and with the idea of a conversation between agents, the designer is suggested variations to their current design when requested, which they can use to replace their design, discard it, or use parts of it. The computational designer uses an~\acrshort{ea} generating individuals in three different pools: \textit{convergence}: similarity between current design and generated individuals, \textit{innovation}: dissimilarity between current design and generated individuals, and \textit{guidelines}: following human-input constraints. The most interesting aspect of 3Buddy is that the designer can specify an area where they will work on and another where the computational designer should focus, thus working simultaneously on different areas of the dungeon.

Furthermore, Karth and Smith's approach uses a modified version of the wfc algorithm~\cite{Karth2017-WFC}, which while not strictly a~\acrshort{micc} system, their approach focuses on the designer providing positive or negative examples to the algorithm, for it to use it to generate variations following such rules. Their novel approach presents a different design process somewhat similar to Morai Maker, where designers show the algorithm what they like and dislike to drive the output of the algorithm to their goal~\cite{Karth2019-pcgmlDiscriminativeLearning}. 

% However, the conversation between AI and designer could be expanded to have explicit turn taking 

% A recent example of how the conversation between AI and designers to co-create content could be establish 

% I am missing baba is y'all~\cite{charity2020baba}.

Recently,~\acrshort{mi} was proposed to be used in the setting of teaching young children handwriting in a tool called Djehuty, which leverage in the use of technology in developing countries to foment literacy. Djehuty continuously generate handwriting styles and suggest them as paths to the child~\cite{Sarr2020-preschoolMixedInit}. Djehuty is another example of~\acrshort{mi}'s strengths, and as described above,~\acrshort{mi} can be used virtually in any collaborative scenario where agents can leverage in their strengths to proactively contribute to a solution. 

This thesis revolves around the~\acrfull{edd}, a~\acrshort{micc} tool to co-create dungeon levels that uses design patterns to provide information to the designer and to drive the generation of suggestions for the designer~\cite{alvarez2019empowering}.~\acrshort{edd} uses the~\acrlong{icmape}~\acrshort{qd} algorithm to continuously suggest adaptive, diverse, and high-performing solutions~\cite{Alvarez2020-ICMAPE}.As~\acrshort{edd} is the main research tool developed and used in this thesis, a chapter is reserved to present the tool, all its features and algorithms, and discuss the main contributions around it.

% augment the designer's information on what they are creating and that uses the~\acrlong{icmape} algorithm to continuously suggest adaptive, diverse, and high-performing solutions. 

% augment the designer's information on what they are creating and that uses the~\acrlong{icmape} algorithm to continuously suggest adaptive, diverse, and high-performing solutions. 

However, while~\acrshort{micc} systems bring many benefits to design tools such as reducing workload, fostering creativity, providing adaptive experiences, learning design concepts, making game design tools more accessible, or creating various experiences, they have not being adopted by the game industry yet. This is because, firstly,~\acrshort{micc} tools and common computer-aided design tools such as game engines (Unity, 2005; Unreal Engine 4, 2014), differ in their goals. In the former the focus is on leveraging on each agent strengths and where one's weakness such as lack of knowledge in game design can be supplied by the other agent, such as using game design patterns to help designers build levels~\cite{Baldwin2017a,dahlskog2014-PCGDesignPatterns}; thus making these tools more accessible. In the latter, the focus is on providing a plethora of interconnected tools and systems unified in system that rely on the designer having the complete initiative and expert knowledge to connect the bits that form the design of the game. Secondly, to have a natural dialogue and collaboration between AI and designers as discussed by Horvitz~\cite{Horvitz99-uncertainty}, both need to understand each other design processes such as intentions and goals. Thirdly, to enable more autonomy in the interaction between human and machine, and give a varying degree of initiative to the machine to co-create the game content a game designer have as goal, these tools require identify and use different designer's processes and design procedures. Therefore, the following section is devote to discuss \textit{designer modeling}, an approach to achieve such through modeling certain designer's processes and use them to drive the generation of content. 

%  Mixed-initiative co-creativity (MI-CC), a concept introduced by Yannakakis et al.~\cite{yannakakis2014micc}, refers to a creation process through which a computer and a human user feed and inspire each other in the form of iterative reciprocal stimuli. Some examples of this are \textit{Ropossum}~\cite{shaker2013ropossum}, \textit{Tanagra}~\cite{smith_tanagra:_2011}, \textit{CICERO}~\cite{Machado2017}, and \textit{Sentient Sketchbook}~\cite{liapis_generating_2013}. 

%  iterative refining, interactive evolution, designer modeling and coterminous design of alternatives
% to human designs. 
% '

% I think that I can use this space to already do some steps towards creating this analysis i want to do, so this licentiate in itself is the preamble of such work, and I can discuss it. So this thesis also contributes more than just being a comprehensive summary, but then again, I shouldn't spent that much time on this.

\subsection{Modeling players and designers}

% Machine Learning (ML) has gained an increased interest from game researchers, achieving remarkable success on training AI agents for very popular games, such as AlphaStar on Starcraft 2 \cite{alphastarblog} and OpenAI Five on Dota 2 \cite{berner2019dota}. Its combination with PCG has led to the raise of  Procedural Content Generation via Machine Learning (PCGML), defined as the generation of game content by models that have been trained on existing game content \cite{summerville2018procedural}, with applications to autonomous content generation, content repair, content critique, data compression, and mixed-initiative design. 

Player modeling relates to the study of players in game to compose computational models on the player's characteristics that arise when they interact with games as cognitive. affect, and behavioral patterns~\cite{Yannakakis2013-playermodeling,thawonmas2019artificial}. Through this, the aim is to understand the player's experience when interacting with a game. Player modeling usually relies on data-driven and~\acrshort{ml} approaches with user-generated gameplay data, and have been used with a vast amount of goals. For instance, for automating playtesting~\cite{Holmgard2019-proceduralPersonas,Gudmundsson2018-HumanLikePlayCandyCrush}, identifying player types (using Bartle's taxonomy~\cite{bartle1996-taxonomy}) based on their playstyle~\cite{Drachen2009-playerModellingTombRaider}, to understand and model in-game player's motivations~\cite{Melhart2019-ModellingMotivation}, or for market purposes, to understand how players play and are engage in free-to-play games~\cite{Saas2016-DiscoveringPlayingPatterns,delRio2020-PlayerConversion,Guitart2019-PlayerBehavioralData}.

% Player modeling, the ability to recognize general socio-emotional and cognitive/behavioral patterns in players \cite{thawonmas2019artificial}, has been appointed by the game research community as an essential process in many aspects of game development, such as designing of new game features, driving marketing and profitability analyses, or as a means to improve PCG and game content adaptation. Player modeling frequently relies on data-driven and ML approaches to create such models out of several sorts of user-generated gameplay data \cite{liapismodellingquality19,melhart2020feel,Drachen2009-playerModellingTombRaider,Holmgard2019-proceduralPersonas,Melhart2019-ModellingMotivation}.

Using player data from \textit{Iconoscope}, a freeform creation game for visually depicting semantic concepts, Liapis et al. trained and compared several ML algorithms by their ability to predict the appeal of an icon from its visual appearance~\cite{liapismodellingquality19}.  Furthermore, Alvarez and Vozaru explored personality-driven agents based on individuals' personalities using the \textit{cibernetic big five model}, evaluating how observers judged and perceived agents using data from their personality test when encountering multiple situations~\cite{Alvoz2019-PersonalityDriven}.
%  using Bartle's player archetypes~\cite{bartle1996-taxonomy}

Moreover, Yannakakis and Togelius discussed how player experience could be modeled and used to drive the generation of new game content, and in this way, create content that is adapted to the experience and expectations of the player~\cite{Yannakakis2011-experiencedrivenPCG}. Further, training models on gameplay data from \textit{Tom Clancy's The Division} has also been used to model, and therefore find predictors of player motivation \cite{Melhart2019-ModellingMotivation}, which renders a very valuable tool for understanding the psychological effects of gameplay. Former research followed a similar approach in \textit{Tomb Raider Underworld}, training player models on high-level playing behavior data, identifying four types of players as behavior clusters, which provide relevant information for game testing and mechanic design \cite{Drachen2009-playerModellingTombRaider}. Melhart et al. take these approaches one step further by modeling a user's \textit{Theory of Mind} in a human-game agent scenario \cite{melhart2020feel}, finding that players' perception of an agent's frustration is more a cognitive process than an affective response.

\subsubsection{Designer Modeling}
% \subsubsection{Designer Modeling and Designer Understanding}

Understanding player behavior and experience, as well as predicting the player's motivation and intention is key for mixed-initiative creative tools while aiming to offer in real-time user-tailored procedurally generated content. Nevertheless, the main user of~\acrshort{micc} tools are designers, and gameplay data is replaced by a compilation of designer-user actions and AI model reactions over time while both user and model are engaged in a mutually inspired creative process. A fluent~\acrshort{micc} loop should provide good human understanding and interpretation of the system, as well as accurate user behavior modelling by the system, capable of projecting the user's subsequent design decisions~\cite{ComptonPhD}. In the same line, goal thirteen in the guidelines for Human-AI interaction \cite{amershi2019guidelines} highlights the importance of learning from user behavior and personalize the user’s experience by learning from their actions over time. 

Shifting towards a designer-centric perspective means that besides focusing on player modeling, it is necessary to focus on modeling the designers. Liapis et al.~\cite{Liapis2013-designerModel,Liapis2014-designerModelImpl} introduced designer modeling for personalized experiences when using computer-aided design tools, with a focus on the integration of such in automatized and mixed-initiative content creation. The focus is on capturing the designer's style, preferences, goals, intentions, and iterative design process to create representative models of designers. Through these models, designer's and their design process could be understood in-depth, enabling adaptive experiences, further reducing their workload and fostering their creativity. 

As part of this thesis work, two approaches to model different designer's processes have been proposed, the designer's preference model~\cite{Alvarez2020-DesignerPreference}, and design style cluster together with designer personas~\cite{alvarez2020-designerpersonas}. The work presented in~\cite{Alvarez2020-DesignerPreference} introduced the Designer Preference Model, a data-driven solution that learns from user-generated data in the~\acrshort{edd}. This preference model uses an Artificial Neural Network to model the designer's preferences based on the choices they make while using~\acrshort{edd}, which is then used to drive the content generation. Moreover, The work presented in~\cite{alvarez2020-designerpersonas} uses data from the design process of 105 sessions to analyze the room styles created along the process, yielding twelve clusters representing such styles. The design process was again analyzed in function of these formed clusters, where we encountered four archetypical paths, i.e., designer personas, that were most commonly taking by designers with the aim to be used to drive the generation of content towards more adapted content. 

\paragraph{Style}

Style while subjective for each designer, can be defined 

\paragraph{Goals}

While the main goal (or long-term goal) of designers is to create a game and the game experience they expect for the player, the short- and mid-term goals are not straightforward. For instance, as the designer is creating a room the goal of the room could be inferred by it's neighbors but it is not until completed that some estimation can be given, and even then, the designer's goal might be completely different. Another example would be if a designer designs an enemy for a platform game, 

Similarly, if a designer designs an enemy for 

\paragraph{Intentions}

Refers to 1) the intentions designers have when creating content, especially the ones that are hard to recognize or identify through heuristics. For instance, blocking the pass to a door in a room but creating access from a different area, or the placement of walls to create cover against enemies are just a couple of examples of non-intuitive designer's intentions that are hard to be recognized. and aligned with this, 2) the intentions designer

3) the expected intentions of the designer as they create content, for instance, creating a challenging room might mean that the designer would like to create more lenient rooms to balance. Aligned with this, 3) the designer's intentions for the player or the expected player/game experience, which in most cases is not easily identified in the created content. For instance, designer's might focus on creating a vast amount of cover through walls in rooms for protecting players in combat whereas other designer might add the same amount of walls to create obstacles for players. 

\paragraph{Design and Creative Process}

The design and creative process is far from trivial, 

Of course, the process is also constrained by what the tool allow you to do

Refers to the process most designers go through to create the content

\paragraph{Preferences}

Preference is by far the most subjective of all the processes and procedures. Each designer develops preferences, not only as they gain more experience but even as they use the systems. For instance, one moment the designer might be completely sold into preferring open areas with clear paths for players, and next (as they go through trial-and-error), they prefer smaller rooms with. Moreover, as they 

\subsection{Computational Creativity}

% \begin{dissQuote}{\~{}Margaret Boden, The Creative Mind: Myths and Mechanisms, pp. 1}
%     Creativity is the ability to come up with ideas or artefacts that are \textbf{new}, \textbf{surprising} and \textbf{valuable}.
% \end{dissQuote}

% \begin{itemize}
%     \item introduce CC, and present relevant work in the area. 
%     \item Discuss related work on CC:
%     \item discuss related work on evolutionary algorithms? 

% \end{itemize}

Creativity is ``the ability to produce work that is both novel (i.e., original, unexpected) and appropriate (i.e., useful, adaptive concerning task constraints)´´~\cite{Sternberg1999-CreativityConcept}. How creative processes occur, how an individual might come up with novel ideas, or how to assess creativity is very much an open research area~\cite{sternberg1999-handbookCreativity,boden2004-creative,Sternberg2005-creativityCreativities,Csikszentmihalyi97-Creativity}. Moreover,~\acrlong{cc} is a multidisciplinary field that study computational systems that demonstrate human-like creative behaviors~\cite{Colton2012-CC}. As a multidisciplinary field,~\acrshort{cc} is not only interested in the algorithms or the final outcome; it also aims to study the creative process and psychological causes of creative behaviors. Thus, through~\acrshort{cc} some core creativity concepts and research areas can be addressed. For instance, in \textit{the Creative Mind: Myths and Mechanism}, Boden studies and analyzes \emph{Creativity} and \emph{creative behaviors} with the use and help of~\acrshort{ai} through the lenses of~\acrlong{cc}. Boden discusses three forms of creativity: \textit{combinatorial}: combining existing knowledge in unfamiliar ways to produce new artifacts; \textit{exploratory}: exploring the conceptual space to encounter possible ideas;~\textit{transformational}: transforming the conceptual space, the imposed constraints, and the encountered ideas~\cite{boden2004-creative}.

Within~\acrshort{cc}, games have been proposed as the optimal artifact to create to test the creative-like abilities of a~\acrshort{cc} system, since games are \emph{content-intensive}, \emph{multi-faceted content}, and should be~\emph{interacted with and experienced}~\cite{Liapis2014-gameCreativity}. As described above, game content relates to the main facets that represent any game: audio, visuals, narrative, levels, rules, and gameplay~\cite{Liapis2019-OrchestratingGames}. Thus, creating systems that develop to some extent games poses an interesting application and challenge for~\acrshort{cc}, which can address some of the core questions in~\acrshort{cc}. For instance, investigating the creative process to not only create one type of content but the arrangement of such in an harmonious way as team of humans creatively do, or the assessment of such content.

Using the exploratory and combinatorial creativity forms from Boden, Guzdial and Riedl proposed conceptual expansion. Conceptual expansion is an approach that combines neural networks trained to recognize or generate specific content to produce a \textit{combinet} that could be used to recognize or generate novel content, which lack enough data to use it to train a new ml model~\cite{guzdial2018-combinets}. Moreover, they applied their approach to the conceptual expansion of games, with the same idea of creating novel combination of games from a set of models trained to produce content for specific games~\cite{guzdial2020-conceptualGameExpansion}. In the same line, Sarkar et al. proposed the use of variational autoencoders (VAE) to create new levels by trainign the VAE with game levels from Super Mario Bros. and Kid Icarus. Through this, the VAE learns a representation of both games levels and using~\acrshort{ea}s they are able to generate levels satisfying some metric that drives the generation process~\cite{sarkar2019-controllableLvlBlending}.

Moreover, Mikkulainen discuses the use of~\acrlong{ec} to achieve creative AI, which refers to the use of AI to not only create and perform creative tasks such as generating games, but also to encounter creative solutions to complex multidimensional problems. In his work, he reflects on the aims of the~\acrshort{ai} field and discusses the use of search-based approaches for exploring complex multidimensional spaces filled with ``unknown unknowns" with exciting results~\cite{miikkulainen2020-creativeAIEVO}. Likewise, Sarkar discusses leveraging on creative AI techniques to approach game design, and with such demonstrate exploratory work on how such could be achieved, and the benefits from it~\cite{sarkar2019-GameDesignCreativeAI}. Specifically, Sarkar discusses the co-design aspect that can be enabled through creative AI techniques, which is especially relevant for this thesis and the development of effective~\acrshort{micc} systems.

% conceptual  expansion,

% Guzdial and Riedl took the exploratory and combinatorial forms 

% Creativity is the main 

% Moreover, Mikkulainen discusses 
% Moreover, while Miikkulainen does not specifically refer to Computational Creativity, his recent work on Creative AI, reflects the aims of the AI field into using search-based approaches for exploring complex spaces filled with ``unknown unknowns", and reaching good results~\cite{miikkulainen2020-creativeAIEVO}. Likewise, Sarkar discusses ~\cite{sarkar2019-GameDesignCreativeAI}

% Similarly,~\acrlong{qd} algoritms were conceptualized under the same \textbf{pretext}, since a plethora of problems require more than convergence search, i.e. searches driven by objectives~\cite{stanley2015-mythObjective}. On the other hand, while divergent search has demonstrated many strengths, especially in deceitful environments~\cite{Novelty-Lehman2011}, when the search space is unconstrained in some dimensions, divergent search encounter challenges into finding optimal solutions~\cite{Lehman2010-MCNS}. Therefore,~\acrlong{qd} algorithms were conceptualized, where divergent and convergence searches are combined to highlight and use the strengths of both with many applications such as in robotics~\cite{Cully2015-qdRobotsAnimals} or games~\cite{gravina2019procedural}. Remarkably,~\acrfull{mape} is an algorithm with a simple process that has been demonstrated to work extremely well~\cite{Mouret2015}.



% unless the space is constrained in some way~\cite{Lehman2010-MCNS}

% \subsubsection{Evolutionary Computation for Computational Creativity}


% \subsection{Summary}

% In this chapter, it was described the main areas of concern and various research that are essential for the dissemination of this thesis. At the beginning, it was given an overview of the main research area~\acrlong{pcg}