\section{INTRODUCTION} \normalfont 
\label{sec:intro}
% \addcontentsline{toc}{section}{\nameref{sec:intro}}

\epigraph{During the first three millennia, the Earthmen complained a lot}{\textit{John McCarthy}}

% \begin{dissQuote}{\~{}John McCarthy}
%     During the first three millennia, the Earthmen complained a lot.
% \end{dissQuote}

% \subsection{Tools}
John McCarthy was spot on, as we humans, are a main source of complain. Nevertheless, such complains makes us strive and search for solutions and better approaches to cope with our needs and objectives.\ Ironically, we would end up complaining about it, restarting the loop.

Since the dawn of time, we, humans have been in the search [and in the need] for tools to develop our ideas, or to execute mundane objectives. As time and technology advanced, more sophisticated types of assistance emerged to cope with the needs of humans, such as vehicles to traverse longer paths or ways to facilitate writing. With the invention of hardware and software, its ubiquity, and the raise of~\acrfull{ai}, a new avenue for human assistance opened up. Rather than having tools that facilitate our work by doing impossible things for us, e.g. move 500 km in a day, or simply assist us into doing repetitive work, they can now provide advance assistance with smart tools that allows us to work and explore in a different and more efficient way (Unity, 2005; Photoshop, 1990). However, tools that assist us in our tasks are not the only key factor; collaboration between humans have remained virtually unchanged as the essential way to move forward and to develop new experiences. Not only to achieve greater objectives as a group, but also to develop as individuals. While the current assistance tools to support the work and creative output of humans are valuable and helpful in many ways; this raises an essential question that holistically motivates and drives this thesis:

% From the dawn of time, we, humans have been in the search [and in the need] for tools to develop our ideas, e.g.\ the wheel, or to execute mundane objectives, e.g.\ the hammer. As time and technology advanced, more sophisticated types of assistance emerged to cope with the needs of humans, such as vehicles to traverse longer paths or ways to facilitate writing. With the invention of hardware and software, its ubiquity, and the raise of AI as the XXI century technology, a new avenue for human assistance opened up. Rather than having tools that facilitate our work by doing impossible things for us, e.g. move 500 km in a day, or simply assist us into doing repetitive work, they can now provide advance assistance with smart tools that allows us to work and explore in a different and more efficient way~\cite{unity,photoshop}. However, tools that assist us in our tasks are not the only key factor; collaboration between humans have remained virtually unchanged as the essential way to move forward and to develop new experiences, not only to achieve greater objectives as a group, but also to develop as individuals. Science is not an exception, and the hollywood portraits of ''mad scientists" working on their own to develop the next step forward [or backward], is far from reality \textbf{probably a footnote}. While the current assistance tools are valuable and help us in many ways, this raises an essential question that holistically motivates and drives this thesis:

% We, humans, from the early days have been in the search... With the invention of hardware and software, its ubiquity, and the raise of AI as the XXI century technology and software that is ubiquitous, suddenly, a new avenue for human assistance opened up, where rather than having tools that facilitate our work by doing impossible things for us, e.g. move 500 km in a day, or simply assist us into doing repetitive work, can now provide advance assistance with smart tools that allows us to work and explore in a different and more efficient way~\cite{unity,photoshop}.However, collaboration between humans have remained virtually unchanged as the essential way to move forward and to develop new experiences, not only to achieve greater objectives as a group, but also to develop as individuals. The sciences is not an exception, and the hollywood portraits of ''mad scientists" working on their own to develop the next step forward [or backward], is far from reality \textbf{probably a footnote}. While the current assistance tools are valuable and help us in many ways, this raises an essential question that motivates and holistically, drives this thesis:

% \blindtext

\begin{retQuestion}{}
    How can we create tools that no longer behave just as assistance tools to support our work, but can collaborate with us, to some extent, in the same way as human collaboration occurs? 
\end{retQuestion}

% \begin{kuestion}
%     How can we create tools that no longer behave just as assistance tools to support our work, but can collaborate with us, to some extent, in the same way as human collaboration occurs? 
% \end{kuestion}

% How can we create tools that no longer behave just as assistance tools, but can collaborate with us, to some extent, in the same way as human collaboration happens? 

% This question is not new and have been asked in other ways and approached by different disciplines. The concept used to describe this collaboration is~\acrfull{mi} 

This question in not new and have been approached by different disciplines, under the~\acrfull{mi} paradigm.~\acrshort{mi} refers to the collaboration between \emph{human} and \emph{computer} where both have some proactive initiative to solve some task.~\acrshort{mi} can be seen as a multi-agent collaboration scenario, where the interaction should be flexible, allowing for a continuous negotiation of initiative and leverage on each other's strengths to solve a task~\cite{Allen99-MIinteraction}. \emph{Initiative} was described as a multi-factor model that combines: choosing the task, choosing the agent in control and how the interaction is established, and choosing what is the expected outcome from the collaboration~\cite{novick97-mixedInit}. 

Moreover, Horvitz discussed such a question in terms of Intelligent User Interfaces~\cite{Birnbaum97-IUI}, describing mixed-initiative systems and interfaces as a more natural collaboration in a user interface that emerges from intertwining human control and manipulation, and automation by AI~\cite{Horvitz99-uncertainty}. Horvitz presented several principles of mixed-initiative interaction and it's challenges, many of which still exist~\cite{Horvitz99-mixedInit}, mainly describing this interaction as conversation systems between AI and humans~\cite{horvitz1999-conversationModel}. Moreover, Yannakakis et al. introduced the~\acrfull{micc} paradigm for the co-creation of creative content, where both AI and humans, alternate in the initiative to co-design and solve tasks~\cite{yannakakis2014micc}. Their work describes key findings and discussions for how MI-CC does not only help human designers solve tasks, but also fosters their creativity through an interactive feedback loop and lateral thinking~\cite{Liapis2016-CanComputersFosterCreativity,Liapis2014-gameCreativity,Alvarez2018}. 

Paramount as well is the role of the computer agent in this interaction, as it would help establish the boundaries of the interaction, what is expected, and how creativity could be foster. Lubart analyzed this interaction and examined the different ways computers could be involved in creative work to promote creativity. In his work, he proposed four roles: \emph{computer as nanny}: management of creative work; \emph{computer as pen-pal}: communication service between collaborators; \emph{computer as coach}: Using creative enhancement techniques; and \emph{computer as colleague}: partnership between computer and humans~\cite{LUBART2005-computerPartners}. Recently, this was explored by Guzdial et al. where designers perceived the AI collaborator with more or less value depending on their desired role for the AI, varying between: \emph{friend}, \emph{collaborator}, \emph{student}, or \emph{manager}~\cite{Guzdial2019-AISystemDesign-Creators}.

% and presented four roles how such interaction could enhance creativity and presented four roles for t: 

% This question is not new and have been asked in other ways and approached by different disciplines. Eric Horvitz, one of the pioneers in the area of~\acrfull{hci}, discussed such a question in terms of Intelligent User Interfaces~\cite{Birnbaum97-IUI}, describing mixed-initiative systems and interfaces as a more natural collaboration in a user interface that emerges from intertwining human control and manipulation, and automation by AI~\cite{Horvitz99-uncertainty}. Horvitz presented several principles of mixed-initiative interaction and it's challenges, many of which still exist~\cite{Horvitz99-mixedInit}, mainly describing this interaction as conversation systems between AI and humans~\cite{horvitz1999-conversationModel}. Moreover, Yannakakis et al. introduced the~\acrfull{micc} paradigm for the co-creation of game content, where both AI and humans, alternate in the initiative to co-design and solve tasks~\cite{yannakakis2014micc}. Their work describes key findings and discussions for how MI-CC, when used and applied correctly, does not only help human designers solve tasks, but also fosters their creativity through the interactive feedback loop and lateral thinking~\cite{Liapis2014-gameCreativity,Alvarez2018a}.

Furthermore, there has been a significant effort to analyze the possibilities of such a collaboration, resulting in multiple Ph.D. theses exploring such paradigm focusing on different aspects of it~\cite{SmithPhD,LiapisPhD,ComptonPhD,GuzdialPhD,MachadoPhD}. Thus, there is a growing interest on how to establish mixed-initiative interactions and new ways of humans and AI to collaborate. Earlier work focused on tools that supported the work of humans combining the strengths of both. Later on, research demonstrated that such interaction could enable the completion of tasks not previously able to be done by either alone, and enable creative work to be stimulated. 
% The role of the computer agent in this interaction is paramount as well, as it will help establish the boundaries of the interaction, what is expected, and how creativity could be foster. Lubart defined four roles: \emph{computer as nanny}, \emph{computer as pen-pal}, \emph{computer as coach}, \emph{computer as colleague}~\cite{LUBART2005-computerPartners}

% How the computer agent should ~\cite{LUBART2005-computerPartners}.

% Given this, there is a significant effort to analyze the possibilities of such a collaboration, even resulting in multiple Ph.D. theses exploring such paradigm~\cite{SmithPhD,LiapisPhD,CookPhD,ComptonPhD,GuzdialPhD,MachadoPhD}.
% There is a significant amount of work, which goal is to search for mixed-initiative interactions and new ways for the collaboration between humans and AI. Earlier work focused into tools that support the work of humans combining the strengths of both, but short after the understanding that such an interaction could enable the completion of tasks not previously able to be done by either alone, and enabling creative work to be stimulated~\cite{LUBART2005-computerPartners}.

% This give rise that There is a significant amount of work aiming at the search for mixed-initiative interactions between humans and AI, where earlier focus has been more into tools that support the work of humans combining the strenghts of both, the understanding that such an interaction could enable the completion of tasks not previously able to be done by either alone, and enabling creative work to be stimulated.

% Nevertheless, this collaborative approach divided into human control and AI automation as presented by Horvitz~\cite{Horvitz99-mixedInit} with multiple initiatives at different points of the development as discussed by Yannakakis et al.~\cite{yannakakis2014micc}, raises a controllability challenge for either actor: Which of the two should have the control at different steps of the development and over the goal? There is no real answer since it is very diffuse, and many factors need to be consider before appropriately indicating this. At the very least, some could say that depending on the task to be performed and the expertise of both, one or the other would clearly be the one in control of the development, whereas others would clearly position the human as the one in control. Yet, even if the answer would be clear, what happens to the expressivity of one of the sides as a consequence of the other controlling? The more control is given, the more constraint it exist, but is this a problem? Is it inevitable? At last, Boden explains it quite clear ´´...  We seek the imposed constraints [...], and try to overcome them by changing the rules."~\cite{boden2004-creative}

% Point out that this collaborative search for human control and AI automation with multiple initiatives at different points of the development, give raise to a controllability challenge for either two actors. Moreover, when we discuss content generation 

% If the aim of this research area is to push harder for mixed-initiative tools, where more autonomy is given to the AI, and for humans to consider the AI as a collaborator as described by Lubart~\cite{LUBART2005-computerPartners} and Guzdial et al.~\cite{guzdial2019friend} that can be taken serious and used its input as a key factor in the development of any type of content. Then we are required to develop AIs and tools that not only provides interesting and valuable input to the human, but also adaptive experiences that 

%  even if the answer would clearly be ´´the human" 

% solutions otherwise deemed impossible or hard to reach. With the same idea, the strengths of both complement each other 

% The advantage of having the human and AI collaborating with each other is analogous to humans collaborating with each other to reach solutions otherwise deemed impossible or hard to reach. With the same idea, the strengths of both complement each other 


% Moreover, human-in-the-loop AI
% Align with this, is the current focus on human-in-the-loop AI

% This collaboration may take different forms as studied by Lubbart~\cite{LUBART2005-computerPartners}, and recently explored by Guzdial et al.~\cite{guzdial2019friend}.



% This question is not new, Horvitz, one of the pioneers in the are of Human-Computer Interaction (HCI), described such a question as 
% The area of Human-Computer Interaction  T

% Through emulating this collaboration 

% \subsection{Games and Creative Outputs}

\subsection{Computational Creativity and Games}

Games, either digital or tabletop, are created through a complex creative process that couple together many different creative facets in different ways. Games contain a big amount of creative content carefully combined and intertwined to craft specific experiences, with the addition of rules that dictate how a player is to interact with it. In contrast with other creative content, games are multifaceted, content-intensive, and should be interacted, experienced, and enjoyed by others, which also creates a complex subjective task~\cite{Liapis2014-gameCreativity}. Usually, games are developed by more than a person (although, many exceptions exist~\cite{minecraft,undertale,stardewvalley}) reaching to hundreds and thousand of developers, with each developer specialized in different areas such as gameplay, AI, animation, concept art, etc. Each creates a specific part of the game and of the content following a road map, and collaborating with each other~\cite[Chapter~14]{fullerton2004-gamedesign}. However, no matter the size and talent of the team, the fact remains that developing games is a hard challenge~\cite{Blow2004-gamesHard}. As technology advances, the requirements increase substantially for any of the game facets, coupled with the users' increase demand, the higher competitiveness in the market, and the launch of many more platforms~\cite{Washburn2016-gamesPostmorten}.

Moreover,~\acrfull{cc} is one of the grand challenges of~\acrshort{ai}, where the quest is to study, develop, and build computational systems that demonstrate creative behaviors and can create different type of artifacts such as games or stories~\cite{Colton2012-CC}. Within the multiple content that can be created, games offer a unique property that distance them from other creative outputs, and which makes them more interesting to be analyzed and experimented on than others~\cite{Liapis2014-gameCreativity}. They offer a set of intertwined facets that represent any game: audio, visuals, narrative, levels, rules, and gameplay~\cite{Liapis2019-OrchestratingGames}, whereas other creative content focuses almost exclusively on a single aspect, e.g. music or dance. However, single aspect creative content has its own set of challenges, for instance, creating a system that creates music could mean the creation of each separated instrument, music sheet, arrangements, etc~\cite{HooverPhD}. Furthermore, Games must be interacted with and enjoyed by others than the developers, and the fact that these facets must fit each other in a natural manner poses them as a very exciting application to be researched and developed within the~\acrshort{cc} field.

\acrfull{pcg} is a field within computational intelligence in games, that focuses on the use of algorithms to create game content~\cite{Yannakakis2018}.~\acrshort{pcg} algorithms have been used to aid in the creation of a plethora of games such as No Man Sky~\cite{nomansky}, Spelunky~\cite{spelunky}, or Minecraft~\cite{minecraft}, to the extent that~\acrshort{pcg} and~\acrshort{ai} have enable experiences and interactions that were not possible before~\cite{aidungeon,rogue,elite}. Moreover, as one of the properties of~\acrshort{pcg} is to increase replayability by creating an abundance of well-made content~\cite{shaker_procedural_2016}, games are not the only beneficiaries of~\acrshort{pcg} methods. For instance, they have the opportunity to be used to increase the generality of~\acrfull{ml} approaches~\cite{Risi2020-pcgGeneralityML}, or a step towards open-endedness~\acrfull{ec}~\cite{clune2019-aigas}. 

% Several systems have been created have been done to develop computational designers and systems to create complete games~\cite{Cook2016-Angelina1,Cook2014-ARogueDream,font2013-GenCardGames,Browne2010-ludii}

Furthermore, several computational designers and systems have been developed to create complete games such as Angelina~\cite{Cook2016-Angelina1}, Ludi~\cite{Browne2010-ludii}, a system to create card games~\cite{font2013-GenCardGames}, or a system that uses data from wikipedia to create mystery games~\cite{barros2018-DATAeinstein}. Automated game design as developed in those systems show interesting and important advancements towards~\acrshort{cc} systems. However, not including the human designer creates constraints, challenges, and limitations in these systems such as modeling fun, enjoyment, and interaction. Another interesting and promising path to explore is the~\acrlong{micc} paradigm within~\acrshort{pcg}, combining both~\acrshort{ai} and humans to co-create the game content, which is the focus of this thesis.

% Several systems were created that developed computational designers and systems to create complete games~\cite{Cook2016-Angelina1,Cook2014-ARogueDream,font2013-GenCardGames,Browne2010-ludii}, which come with their own set of challenges and limitations when the human designer does not participate such as modeling fun and enjoyment. With this framing, another interesting and promising path to explore is the~\acrlong{micc} paradigm in~\acrshort{pcg}, combining both~\acrshort{ai} and human to co-create the game content, which is the focus of this thesis.

% However, another interesting and promising path to explore is the combination of AI and humans to co-create game content, as above presented.

% One of the main difference of games with other creative content, besides them being multifaceted and content-intensive, is that they should be interacted, experienced, and enjoyed by others, which also creates a complex subjective task~\cite{Liapis2014-gameCreativity}. 


% Games, either digital or tabletop, are created through a complex creative process that couple together many different creative facets in different ways.,  and their end-goal is to be interacted and experienced by players. Games contain a big amount of creative content carefully combined and intertwined to craft specific experiences, with the addition 

% Famously defined by Sid Meier: ``Games are a series of interesting decisions." Games contain a big amount of creative content carefully combined and intertwined to craft specific experiences for players. Games, either digital or tabletop, are carefully crafted experiences for players that contain a big amount of creative content 


% Games are usually developed by more than a person (although, many exceptions exist~\cite{minecraft,undertale,stardewvalley}) reaching to hundreds and thousand of developers, with each developer specialized in different areas such as gameplay, AI, animation, concept art, etc. Each creates a specific part of the game and of the content following a road map~\cite[Chapter~14]{fullerton2004-gamedesign}, and collaborating with each other. However, no matter the size and talent of the team, the fact remains that developing games is a hard challenge~\cite{Blow2004-gamesHard}. As technology advances, the requirements increase substantially for any of the game facets, coupled with the users' increase demand, the higher competitiveness in the market, and the launch of many more platforms~\cite{Washburn2016-gamesPostmorten}.

% % which can reach the thousands~\cite{bungie}; with each of the persons creating a specific part of the game and of the content following a road map plan´´master plan"~\cite{fullerton2004-gamedesign14}, and actively \textbf{collaborating with each other}. However, no matter the size and talent of the team, the fact remains that developing games is a hard challenge~\cite{Blow2004-gamesHard}. As technology advances, the requirements increase substantially for any of the game facets, coupled with the users' increase demand, the higher competitiveness in the market, and the launch of many more platforms~\cite{Washburn2016-gamesPostmorten}.

% \acrfull{pcg} is a field within computational intelligence in games, that focuses on the use of algorithms to create game content~\cite{Yannakakis2018}.~\acrshort{pcg} algorithms have been used to aid in the creation of a plethora of games such as No Man Sky~\cite{nomansky}, Spelunky~\cite{spelunky}, or Minecraft~\cite{minecraft}, to the extent that~\acrshort{pcg} and~\acrshort{ai} have enable experiences and interactions that were not possible before~\cite{aidungeon,rogue,elite}. Moreover, as one of the properties of~\acrshort{pcg} is to increase replayability by creating an abundance of well-made content~\cite{shaker_procedural_2016}, games are not the only beneficiaries of~\acrshort{pcg} methods, for instance having the opportunity to be used to increase the generality of~\acrfull{ml} approaches~\cite{Risi2020-pcgGeneralityML}, and perhaps a step towards open-endedness~\acrfull{ec}~\cite{clune2019-aigas}.

% Computational Creativity (CC) is one of the grand challenges of~\acrshort{ai}, where the quest is on producing, analyzing, enabling, and discussing~\acrshort{ai} that is creative (or at least, it is perceived as creative)~\cite{Colton2012-CC}. However, the idea of~\acrshort{ai} showing such a feature has been questioned, deemed as a human-only property~\cite{boden2004-creative}. Yet, it exists a high amount of research, tools, examples, and demonstration of~\acrshort{ai} exhibiting some kind of CC. 

% Moreover, games offer a unique property that distance them from other creative outputs, and which makes them more interesting to be analyzed and experimented on than others~\cite{Liapis2014-gameCreativity}. They offer a set of intertwined facets that represent any game: audio, visuals, narrative, levels, rules, and gameplay~\cite{Liapis2019-OrchestratingGames}, whereas other creative content focuses almost exclusively on a single aspect, e.g. music or dance. However, single aspect creative content has its own set of challenges, for instance, creating a system that creates music could mean the creation of each separated instrument, music sheet, arrangements, etc~\cite{HooverPhD}. Furthermore, Games must be interacted with and enjoyed by others than the developers, and that these facets must fit each other in a natural manner puts them as the final frontier for CC.

% \subsection{Mixed-Initiative Co-Creation}


% Interactive systems can help humans complete creative tasks that are tedious or require substantial human workload. Such tasks could be as simple as drafting a grammar corrected text (Word), to more complex scenarios such as creating visual artifacts (Photoshop) or designing entire video games (Unity). As above-mentioned,~\acrlong{micc} is a paradigm within human and AI collaboration, where both human and AI proactively collaborate to co-create and co-design some artifact. This thesis bases the research, exploration, and discussion on the work and concepts by Yannakakis et al.~\cite{yannakakis2014micc}. Through~\acrshort{micc}, the potential of interactive tools increases substantially.~\acrshort{micc} enables a new way of tackling creative tasks engaging in-depth humans and AI rather than just helping humans to complete tasks and reduce their workload. Furthermore, enabling a mutual feedback loop could also foster the creativity of both participants, create adaptive experiences for the users, or focus on achieving tasks with better and interesting results in a hybrid format.

% In this thesis, the focus is on exploring multiple approaches for the collaboration between AI and humans to co-create game content in an~\acrlong{micc} paradigm. The goal is to develop 

% In order to explore this collaboration, the main body of work presented in this dissertation is applied and evaluated through the~\acrfull{edd}, a~\acrlong{micc} system, where designers can create levels for rogue-like and adventure type of game such as Zelda~\cite{tloz} or The Binding of Isaac~\cite{bindingISAAC}.

% The research focuses on exploring multiple approaches for the collaboration between artificial intelligence (AI) and humans through interactive systems. Specifically, the project explores how humans and AI can participate in collaborative tasks to co-design and co-create content, in what is called a mixed-initiative co-creativity (MI-CC) paradigm. This paradigm is applied to the creation of videogames, one of the fastest-growing business worldwide, especially in Malmö. 

% The goal is to develop systems, techniques, and algorithms to tackle the different tasks in game design and development that could be assisted by AI. These are collaborative tasks where both AI and humans could collaborate to tackle the tasks by establishing a mutual feedback loop whereby the AI could reduce the workload, provide adaptive experiences, and foster creativity. furthermore, enabling a deep collaboration requires a trust relationship between humans and AI, for the latter to be seen as a collaborator. To reach this trust, is essential to develop interpretable and explainable systems that designers could understand, which simultaneously, would make these systems more available. 

%where such a collaboration does not only enhance human's capabilities and reduce the workload of humans to create an [creative] artifact, but also focuses on creating an iterative feedback loop fostering humans' creativity. 


% The~\acrfull{edd}, is a~\acrlong{micc} system, where designer can create levels for rogue-like and adventure type of game such as Zelda~\cite{tloz} or The Binding of Isaac~\cite{bindingISAAC}. As 

% games as the final frontier~\cite{Liapis2014-gameCreativity}. Orchestrating game facets~\cite{Liapis2019-OrchestratingGames}. key work: Boden~\cite{boden2004-creative}

% PCG and AI enables game experiences and interactions that were not possible before~\cite{aidungeon,nomansky},

\subsection{Problem Statement} \label{sec:problemst}

Interactive systems can help humans complete creative tasks that are tedious or require substantial human workload. Such creative tasks could be as simple as drafting a grammar corrected text (e.g., MS Word), to more complex scenarios such as creating visual artifacts (e.g., Adobe Photoshop) or designing entire video games (e.g. Unity Engine).~\acrlong{micc} is a paradigm within~\acrshort{pcg}, where both human and AI proactively collaborate to co-create and co-design games or creative content. Through~\acrshort{micc}, the potential of interactive tools increases substantially.~\acrshort{micc} enables a new way of tackling creative tasks engaging in-depth humans and AI rather than just helping humans to complete tasks and reduce their workload. Furthermore, enabling a mutual feedback loop could also foster the creativity of both participants, create adaptive experiences for the users, or focus on achieving tasks with better and interesting results in a hybrid format. Multiple approaches have been proposed as alternatives for creating systems that model the interaction between both AI and humans to create game content, and that use different techniques to study such interactions and its implications~\cite{Alvarez2020-ICMAPE,smith_tanagra:_2011,Liapis2013-sentientsketchbook,charity2020baba}. 

% Nevertheless, this collaborative approach divided into human control and automation as presented by Horvitz~\cite{Horvitz99-mixedInit} with multiple initiatives~\cite{Allen99-MIinteraction,novick97-mixedInit}, and whereas it could foster humans' creativity as discussed by Yannakakis et al.~\cite{yannakakis2014micc}, raises a controllability challenge for either actor:Which of the two should have the control at different steps of the development and over the goal? The question reflects the diffuseness of the challenge and situation, as many factors need to be consider before appropriately indicating this. At the very least, some could say that depending on the task to be performed and the expertise of both, one or the other would clearly be the one in control of the development, whereas others would clearly position the human as the one in control. Yet, even with a clear answer; what happens to the expressivity of one of the sides as a consequence of the other controlling? The more control is given, the more constraint it exists, but is this a problem? Is it inevitable? Boden explains it conspicuously ``...  We [humans] seek the imposed constraints [...], and try to overcome them by changing the rules.~\cite{boden2004-creative}". Therefore, constraints limit the space and as a consequence they are overcome by encountering creative solutions.

Nevertheless, this collaborative approach divided into human control and automation~\cite{Horvitz99-mixedInit} with multiple initiatives~\cite{Allen99-MIinteraction,novick97-mixedInit}, and whereas it could foster humans' creativity~\cite{yannakakis2014micc}, raises an \emph{initiative} challenge for either agent: Which agent should have the initiative at different stages of the development and over the goal? The question reflects the diffuseness of the challenge and situation, as many factors need to be consider before appropriately indicating this. At the very least, some could say that depending on the task to be performed and the expertise of both, either would clearly be the one taking the initiative of the development. Whereas others would clearly position the human as the one always in control. Yet, even with a clear answer; what happens in creative tasks to the expressivity of one of the sides as a consequence of the other taking the initiative? 

% Moreover, 

% Ironically, for the machine to be given more initiative from the de or at least

% Moreover, in design and creative tasks the human

% Moreover, as humans have the goal to create multiple 

% control mechanisms with a varying degree of control so that what is generated and produced by the computational designer is within the intentions and goals of the designer.

Moreover, in design and creative tasks, the designer usually has intentions in what they are creating and goals that they want to achieve with their design. Thus, to enable deeper~\acrshort{mi} levels to co-create content, some control mechanisms with a varying degree of control over the algorithms might be necessary for the designer. Through this, the designer could direct or constraint the generated content by the computational designer and oversee that it is within their intentions and goals. In this case, the control and expressive properties of each agent is at the expense of the other agents, as it constraints the space of possibilities~\cite{Baldwin2017}. This is specially relevant when the aim is a creative work such as games, where the creative expression needs to be fostered~\cite{Alvarez2018}. Yet it becomes particularly challenging when using mixed-initiative methods, where smart approaches needs to be in place for a natural conversation and successful collaboration. The more control is given, the more constraint it exists, but is this a problem? Is it inevitable? Boden explains it conspicuously ``...  We [humans] seek the imposed constraints [...], and try to overcome them by changing the rules.~\cite{boden2004-creative}". Constraints limit the space and as a consequence they are overcame by encountering creative solutions.

% Nevertheless, this collaborative approach divided into human control and automation~\cite{Horvitz99-mixedInit} with multiple initiatives~\cite{Allen99-MIinteraction,novick97-mixedInit}, and whereas it could foster humans' creativity~\cite{yannakakis2014micc}, raises an ``initiative'' challenge for either agent: Which of the two should have the initiative at different steps of the development and over the goal? The question reflects the diffuseness of the challenge and situation, as many factors need to be consider before appropriately indicating this. At the very least, some could say that depending on the task to be performed and the expertise of both, one or the other would clearly be the one in control of the development, whereas others would clearly position the human as the one in control. Yet, even with a clear answer; what happens to the expressivity of one of the sides as a consequence of the other controlling? The more initiative is given, the more constraint it exists, but is this a problem? Is it inevitable? Boden explains it conspicuously ``...  We [humans] seek the imposed constraints [...], and try to overcome them by changing the rules.~\cite{boden2004-creative}". Therefore, constraints limit the space and as a consequence they are overcame by encountering creative solutions~\cite{miikkulainen2020-creativeAIEVO}.

% In this case, the control and expressive properties of each agent is at the expense of the other, as it constraints the space of possibilities~\cite{Baldwin2017}. This is specially relevant when the aim is a creative work such as games, where the creative expression needs to be fostered~\cite{Alvarez2018}, but becomes particularly challenging when using mixed-initiative methods, where smart approaches needs to be in place for a natural conversation and successful collaboration. 

Fuerthermore, for this interaction to be fully fleshed, the human needs to understand the behavior of the AI through interpretable and explainable models and systems, and the AI needs to recognize and interpret the intentions of the humans seamlessly as they create their content. The former is the focus of~\emph{Interpretable} and \emph{Explainable AI}, which seeks to create or adapt models and systems for a better workflow between humans and AI, where humans could understand the AI's decision process to enable trust relationships and reach deeper interactions~\cite{Zhu2018-XAIDesignersMICC,Doshi-Velez2018,adadi2018peeking}. The latter would mean that the AI could adapt its behavior and functionality to the needs, expertise, and workflow of individual designers or specific group of designers. To do so, the AI is required to do an analysis on several design processes such as the designer's preferences, styles, and goals, which holistically is called \emph{Designer Modeling}~\cite{Liapis2013-designerModel,Liapis2014-designerModelImpl}. How to create these models and use them to develop adapted experiences is a complex challenge, and understanding the implications of its usability in the control-expressive properties, as well as other consequences, is non trivial.

In this thesis, the focus is on exploring multiple approaches for the collaboration between AI and humans to co-create game content in an~\acrlong{micc} paradigm. The goal is to develop systems, techniques, and algorithms that represent a [creative] computational designer to tackle the different tasks in game design and development. Tasks that could not only be assisted by AI, but rather AI could be a \emph{colleague} in the creative process. To tackle these shared tasks it could be established a mutual feedback loop, whereby AI and humans could inspire each other to explore unknown areas in the design landscape, and reach better and more creative solutions.


% To tackle these shared tasks both AI and humans could establish a mutual feedback loop
% Moreover, these are shared tasks where both AI and humans could work together to tackle the tasks by establishing a mutual feedback loop.  whereby the AI could reduce the workload, provide adaptive experiences, and foster creativity. 

To explore this, the main body of work presented in this dissertation is applied and evaluated through the~\acrfull{edd}, a~\acrlong{micc} system, where designers can create levels for rogue-like and adventure type of game such as Zelda~\cite{tloz} or The Binding of Isaac~\cite{bindingISAAC}. In~\acrshort{edd}, the human designer has the ability to quickly create interconnected rooms forming a dungeon to be experienced by players meanwhile the computational designer collaborates by providing suggestions using different algorithms and following multiple heuristics. The human designer can interact in several ways with the computational designer so that this adapts its output to whatever goal the human designer has, while still providing a diverse amount of alternatives and different experiences to the human designer.

% Collaboration between AI and humans to design and create content is a major challenge in AI, and the main focus of~\acrlong{micc} PCG~\cite{yannakakis2014micc}, as previously discussed. Multiple approaches have been proposed as alternatives for creating systems that model the interaction between both AI and humans to create game content, and that use different techniques to study such interactions and its implications~\cite{Alvarez2020-ICMAPE,smith_tanagra:_2011,Liapis2013-sentientsketchbook,charity2020baba}. 

% Nevertheless, this collaborative approach divided into human control and AI automation as presented by Horvitz~\cite{Horvitz99-mixedInit} with multiple initiatives at different points of the development and whereas this could foster humans' creativity as discussed by Yannakakis et al.~\cite{yannakakis2014micc}, raises a controllability challenge for either actor: Which of the two should have the control at different steps of the development and over the goal? The question reflects the diffuseness of the challenge and situation, as many factors need to be consider before appropriately indicating this. At the very least, some could say that depending on the task to be performed and the expertise of both, one or the other would clearly be the one in control of the development, whereas others would clearly position the human as the one in control. Yet, even with a clear answer; what happens to the expressivity of one of the sides as a consequence of the other controlling? The more control is given, the more constraint it exist, but is this a problem? Is it inevitable? Boden explains it conspicuously ``...  We [humans] seek the imposed constraints [...], and try to overcome them by changing the rules."~\cite{boden2004-creative}, therefore constraints limit the space and as a consequence they are overcome by encountering creative solutions.

% There is no real answer since it is very diffuse, and many factors need to be consider before appropriately indicating this. At the very least, some could say that depending on the task to be performed and the expertise of both, one or the other would clearly be the one in control of the development, whereas others would clearly position the human as the one in control. Yet, even if the answer would be clear, what happens to the expressivity of one of the sides as a consequence of the other controlling? The more control is given, the more constraint it exist, but is this a problem? Is it inevitable? Boden explains it conspicuously ``...  We [humans] seek the imposed constraints [...], and try to overcome them by changing the rules."~\cite{boden2004-creative}

% In this case, the control and expressive properties of each actor is at the expense of the other, as it constraints the space of possibilities~\cite{Baldwin2017}. This is specially relevant when the aim is a creative work such as games, where the creative expression needs to be fostered~\cite{Alvarez2018}, but becomes particularly challenging when using mixed-initiative methods, where smart approaches needs to be in place for a natural conversation and successful collaboration. 

%Understanding and modeling designers, especially when talking about modeling subjective information such as preference or style, is a complex challenge. 

% However, for this interaction to be fully fleshed, the human needs to understand the behavior of the AI through interpretable and explainable models and systems, and the AI needs to recognize and interpret the intentions of the humans seamlessly as they create their content. The former is the focus of~\emph{Interpretable and Explainable AI}, which seek to create or adapt models and systems for a better workflow between humans and AI, where humans could understand the AI's decision process to enable trust relationships and reach deeper interactions~\cite{Zhu2018-XAIDesignersMICC,Doshi-Velez2018,adadi2018peeking}. The latter would mean that the AI could adapt its behavior and functionality to the needs, expertise, and workflow of individual designers or specific group of designers. To do so, the AI is required to do an analysis on several design processes such as the designer's preferences, styles, and goals, which holistically is called \emph{Designer Modeling}~\cite{Liapis2013-designerModel,Liapis2014-designerModelImpl}. How to create these models and use them to develop adapted experiences is a complex challenge, and understanding the implications of it's usability in the control-expressive properties is non trivial.

% \emph{Designer Modeling}, akin to player modeling~\cite{thawonmas2019artificial}, refers to creating models of individual designers or group of designers informed by how they create various type of content to create personalized experiences and adapted tools. Liapis et al. described designer modeling as capturing the style, process, preferences, intentions, and goals of designers as they create content~\cite{Liapis2013-designerModel}. Furthermore, Liapis et al. implemented a prototype of such model in the Sentient Sketchbook~\cite{Liapis2014-designerModelImpl}, where the focus was on using the designer's current design and choice-based evolution to capture process, style, and goals. Alvarez and Font~\cite{Alvarez2020-DesignerPreference}, proposed the use of the behavior characteristics of the generated levels where a neural network was trained on the estimated preference of the designer using a similar approach as the choice-based evolution from Liapis et al.~\cite{Liapis2014-designerModelImpl}.

% An alternative method could be to identify a set of styles that most designers follow when creating content, and model how the designer traverse such a style space. For instance, rooms for a game like the binding of Isaac~\cite{mcmillen_binding_2011} could be classified based on multiple characteristics such as the room's objectives regarding enemies and treasures, access to different areas, or hidden encounter and treasures. Moreover, different designers could reach the same room style through different paths, where the focus along the creation could vary. Some designers would focus on the room's topology before anything else, whereas others would focus first on the objectives a player must achieve. These designer models could then be combined with search-based~\cite{Togelius2011} or other procedural generation methods~\cite{khalifa2020-pcgrl,Volz2018-GANevo} to suggest ways of getting to the next design style from the current one.

\subsection{Research Questions} \label{sec:RQS}

As motivated thus far, this thesis focuses on exploring different approaches for procedurally generating content for games or other creative content, specifically through the~\acrshort{micc} paradigm, where a human designer collaborates with an underlying AI to create creative content. Exploring the role of \emph{computers as colleagues} as defined by Lubart~\cite{LUBART2005-computerPartners}, this thesis delves into the use of~\acrshort{micc} tools and the multiple properties that emerge from the dynamic interaction between AI and Humans. The aim is at understanding how can we enable a rich, fruitful, and better feedback loop in these type of tools using and developing novel AI techniques in the field of~\acrlong{ec} and~\acrlong{ml} to improve the interaction and create adapted experiences. The thesis also analyzes and studies what are the requirements, challenges, and benefits of enabling in-depth collaboration, tailored experiences, and the properties that emerge (some seemingly competing properties) and their dynamics from such an interaction. Therefore, this thesis aims at addressing, discussing, and exploring the following research questions: 

% On of the goals of this thesis is to explore the role of computers as colleagues, and how it could be effectively established. 

%of state-of-the-art AI techniques and creating novel


%will analyze not only how can the interaction be better but also to what extent this interaction is needed, what are the requirements of human designers to enable in-depth collaboration, and the study of multiple properties that emerge (some of them seemingly competing properties) from enabling such interaction and it's dynamics. Therefore, this thesis aims to address and explore the following research questions: 

% the balance between two seemingly competing properties, controllability, and expressivity of the generated content.



% \begin{enumerate}
%     \item[] \textbf{RQ1.} How can we use and integrate quality-diversity algorithms into a mixed-initiative approach to help designers produce high-quality content and foster their creativity while allowing them to control, to a certain extent, the generated content?
% \end{enumerate}

\begin{retQuestion}{}
   \textbf{RQ1.} How can we use and integrate quality-diversity algorithms into a mixed-initiative approach to help designers produce high-quality content and foster their creativity while allowing them to control, to a certain extent, the generated content?
\end{retQuestion}

~\acrfull{qd} algorithms are a relative new family of algorithms, specifically aimed at tasks and environments that require the strengths of convergence and divergence search~\cite{Pugh2016}. Leveraging on~\acrshort{qd} algorithms to search for a surfeit of heterogeneous content while not losing-sight of the quality of the content could enable~\acrshort{micc} systems to explore a big area of the generative space producing more diverse and high-quality solutions. Through this, the system could propose a higher range of diverse solutions to the user, aiming at fostering the creativity of the human designer~\cite{liapis2016-canmixedinitiative}. Thus, how to integrate~\acrshort{qd} algorithms in~\acrshort{micc} systems that need to take into account the human work to provide valuable input is a promising open research area and one that this thesis explores. However, it is paramount to understand how to effectively use~\acrshort{qd} algorithms in these systems to fully leverage on their expressive power, while providing some type of control to human designers. 
% So that the output fits their objective, as well as accounting for the interaction with the human designers.

% To date, we have some examples of~\acrshort{qd} algorithms being used effectively to create content for games~\cite{Khalifa2018,Khalifa2019-intentionalCompLevel}, . However, a new algorithm called MAP-Elites~\cite{Mouret2015}
% ~\acrshort{qd} algorithms have been used effectively to create content for games 

% at the same time that the humans could control their output, which as a consequence, requires the exploration on the dynamics of interacting with such algorithms.   \textbf{ why is it important??}

% QD algorithms 

% \begin{enumerate}
%     \item[] \textbf{RQ2.} How can we use gameplay, player, and designer data to understand better players' and designers' actions and behaviors, in order to enhance their experiences?
% \end{enumerate}

\begin{retQuestion}{}
   \textbf{RQ2.} How can we use gameplay, player, and designer data to understand better players' and designers' actions and behaviors, in order to enhance their experiences?
\end{retQuestion}

Games and creative contexts are spaces where both players and designers are able to express themselves in different ways, producing data on how they both interact. Research areas such as Experience-driven PCG~\cite{Yannakakis2011-experiencedrivenPCG}, player modeling~\cite{Pedersen2010-modelingPlayerExp,Holmgard2019-proceduralPersonas} or designer modeling~\cite{Liapis2013-designerModel}, explore the use of such data to understand particular users~\cite{Liapis2013-designerModel,Drachen2009-playerModellingTombRaider,Melhart2019-ModellingMotivation} and to improve and enhance the experiences of players and designer. Especially focusing on enabling adaptive experiences~\cite{hastings_evolving_2009} and more accurate heuristics~\cite{Marinho2015-empiricalEvaluation,canossa2015-towardspcgEvaluation,summerville2017-understandingMario}. However, how to use the data (and even what to collect) is still an open research area, especially when applied to adaptive experiences for~\acrshort{micc} tools with only a few relevant examples~\cite{Liapis2014-designerModelImpl,Liapis2012-adaptiveVisual}. Furthermore, the importance of enhancing the experience of~\acrshort{micc} tools' users lies in the search for deeper understanding and collaboration between both humans and~\acrshort{ai}, which could enable a better experience for both.

%with their respective creative area. Research areas such as Experience-driven PCG~\cite{experiencedrivenpcg}, player modeling~\cite{playermodeling,Holmgard2019-proceduralPersonas} or designer modeling~\cite{Liapis2013-designerModel}, explore the use of such data to understand particular users~\cite{Drachen2009-playerModellingTombRaider,Melhart2019-ModellingMotivation} and to improve and enhance the experiences of users, mostly focusing on enabling adaptive experiences~\cite{hastings_evolving_2009} and more accurate heuristics~\cite{Marinho2015-empiricalEvaluation,canossa2015-towardspcgEvaluation,summerville2017-understandingMario}. However, how to use the data (and even what to collect) is still an open research area, specially when applied to adaptive experiences for mixed-initiative co-creative tools with only a few relevant examples~\cite{Liapis2014-designerModelImpl,Liapis2012-adaptiveVisual}.  \textbf{ why is it important??}

% \begin{enumerate}
%     \item[] \textbf{RQ3.} How can we model different designers' procedures and use them as surrogate models to anticipate the designers' actions, produce content that better fits their requirements, and enhance the dynamic workflow of mixed-initiative tools? 
% \end{enumerate}

\begin{retQuestion}{}
   \textbf{RQ3.} How can we model different designers' procedures and use them as surrogate models to anticipate the designers' actions, produce content that better fits their requirements, and enhance the dynamic workflow of mixed-initiative tools? 
   
    \begin{retQuestion}{}
        \textbf{RQ3.1} What trade-offs arise from modeling and using designer's procedures to steer the generation of content towards personalized content?
    \end{retQuestion}
   
   \begin{retQuestion}{}
        \textbf{RQ3.2} What constraints are created over the generative process when using designer models?
    \end{retQuestion}
   
\end{retQuestion}

The advantage of having the human and AI collaborating with each other is analogous to humans collaborating with each other to reach higher objectives and develop each other. Each one with its own set of strengths and weaknesses. However, mixed-initiative collaboration requires both human and AI to understand each other and the goals that the human aim to reach~\cite{Horvitz99-mixedInit,novick97-mixedInit}. This creates a particular problem where the AI needs to be able to identify certain processes and characteristics of the human. When employing~\acrshort{micc} to co-create creative artifacts, which in the case of this thesis focuses on games, this translates to design processes, style, preferences, intentions, and goals. This thesis aims at exploring how to model different designer procedures such as preference or style, using several~\acrlong{ml} methods and how to best use these as surrogate models to produce better content and enhance the experience of designers using~\acrshort{micc} systems.

% , and how to best use them as surrogate models for the~\acrshort{qd} algorithms discussed in RQ1.

Furthermore, RQ2 and RQ3 drive the research on how to gather and use different type of data, i.e. gameplay and designer data, and whereas designer modeling could be used in the~\acrshort{micc} feedback loop to create adaptive experiences. Through RQ3.1 and RQ3.2, the thesis focuses on exploring the trade-off of using designer modeling. Specifically, the interest lies on the benefits that designer modeling creates for the algorithms and designers, and the overall experience that the designer wants to create, i.e., the game. Moreover, the constraints that emerge from using these models as surrogate models to steer the content generation are non-trivial to address and are essential to study to understand and analyze their extent. Using these models will inevitably create constraints over the generation process as we aim to adapt the experience to each individual designer or group of designers. Therefore, RQ3.2 specifically aims at understanding: what are these constraints? what is constrained? and whether these constraints are positive or negative? 

%within the core~\acrshort{ai} 
%takes aims at exploring the 

%which when employed in a creative context, translates to design processes, style, preferences, intentions, and goals.  \textbf{more discussion here. specifically on how to model the procedures and use it as surrogate models, why is it important??}Furthermore, studying and understanding designers and players could reveal 

% \begin{enumerate}
%     \item[] \textbf{RQ4.} What type of challenges and benefits arise from trying to model designers' procedures and what type of constraints emerge from using these surrogate models to steer the generation towards personalized content?
% \end{enumerate}

% \begin{retQuestion}{}
%   \textbf{RQ4.} What type of challenges and benefits arise from trying to model designers' procedures and what type of constraints emerge from using these surrogate models to steer the generation towards personalized content?
% \end{retQuestion}

% While RQ2 and RQ3 drives the research on how to gather and use different type of data, i.e. gameplay and designer data, and whereas designer modeling could be used in the~\acrshort{micc} feedback loop to create adaptive experiences, through this RQ, the thesis focuses on exploring the tradeoff of using designer modeling. Specifically, the interest is on the benefits designer modeling creates besides the adaptive experiences for the algorithms, designers, and the overall experience that the designer wants to create, i.e. the game. Moreover, the constraints that emerge from using these models as surrogate models that steer the generation are non trivial to address and are essential to be studied to understand and analyze their extent. Using these models will inevitably create constraints over the generation process as we aim to adapt the experience to each individual designer or group of designers. Therefore, this RQ aims at understanding: what are these constraints? what is constrained? and whether these constraints are positive or negative? 

% Related to RQ2 and RQ3, modeling designer's procedures is needed to create adaptive experiences for designers when creating content, as they 
% Using models of designers multiple procedures such as style, design process, or preference, will inevitably create constraints over the generation process as we aim to adapt the experience to each individual designer or group of designers. but the real challenge, is understanding what are these constraints? where will they constraint? what will they constraint? and are these constraints positive or negative? 

\subsection{Pronouns, Style, and Clarification}

Throughout the thesis the pronoun ``we'' will be used in favor of ``I'', since the work and research achieved and presented in this thesis would not have been possible without the collaboration of my co-authors. 

When referring to a player or designer, this thesis chooses the pronouns ``they'' and ``their'' as to respect a gender-inclusive language. Moreover, throughout the thesis, it is referred to user and designer alike, as designer is the target user group within the possible user base of the systems and tools developed in this thesis. The player is referred as the end-user: the user that could and would experience the creations in the~\acrlong{micc} system.

% When referring to end-user, we refer to the user that could and would experience and play the creation designers create collaboratively in the~\acrlong{micc} system, i.e., the player. 

When discussing the participants in a mixed-initiative system, i.e., AI and Human (or group of both), this thesis uses the word ``agent'' when needed to refer to either, unless specifically discussing one in particular, as mixed-initiative systems have been described as multi-agent systems~\cite{Allen99-MIinteraction}.

Finally, this thesis will refer as ``computational designer'' to the overall AI system that interacts and collaborates with the human designer to create content through the~\acrshort{micc} system. 

% Finally, in the~\acrshort{micc} system presented, used, and developed in this thesis  there are several  AI systems fueling the AI that collaborates with designers in the~\acrshort{micc} system presented, used, and developed throughout the multiple publications. Thus, when not explicitly discussing individual algorithms, this thesis will refer to the overall AI system that collaborates with the designer as artificial designer.

% characteristics, properties
% The AI developed throughout the multiple publications and that 


% Finally, the content generated and suggested by the artificial designer is based on~\acrshort{ea}s; thus, when referring to this content it is discussed as individuals.