\section{INTRODUCTION} \normalfont 
\label{sec:intro}
% \addcontentsline{toc}{section}{\nameref{sec:intro}}

\epigraph{During the first three millennia, the Earthmen complained a lot}{\textit{John McCarthy}}

% \begin{dissQuote}{\~{}John McCarthy}
%     During the first three millennia, the Earthmen complained a lot.
% \end{dissQuote}

% \subsection{Tools}
% John McCarthy was spot on, as we humans, are a main source of complain. Nevertheless, such complains makes us strive and search for solutions and better approaches to cope with our needs and objectives. Ironically, we would end up complaining about it, restarting the loop.

John McCarthy was spot on, as we humans are a major source of complaint. Nevertheless, such complaints make us strive and search for solutions and better approaches to cope with our needs and objectives. Ironically, we would end up complaining about it, restarting the loop.

Since the dawn of time, we humans have been searching [and in need] for tools to develop our ideas or execute mundane objectives. As time and technology advanced, more sophisticated types of assistance emerged to cope with humans' needs, such as vehicles to traverse longer paths or ways to facilitate writing. With the invention of hardware and software, its ubiquity, and the raise of~\acrfull{ai}, a new path for human assistance opened up. Tools that were used to facilitate our work or assist us into doing repetitive work, could now provide advance assistance with smarter tools that allows us to work more efficiently. However, tools that assist us in our tasks are not the only key factor; the collaboration between humans has remained virtually unchanged as an essential way to move forward and to develop new experiences. Not only to achieve greater objectives as a group but also to develop as individuals. While the current tools to support humans' work and creative output are valuable and helpful in many ways; this raises an essential question that holistically motivates and drives this thesis:

% Since the dawn of time, we humans have been searching [and in need] for tools to develop our ideas or execute mundane objectives. As time and technology advanced, more sophisticated types of assistance emerged to cope with humans' needs, such as vehicles to traverse longer paths or ways to facilitate writing. With the invention of hardware and software, its ubiquity, and the raise of~\acrfull{ai}, a new path for human assistance opened up. Tools that were used to facilitate our work by doing impossible things for us, e.g., move 500 km in a day, or assist us into doing repetitive work, could now provide advance assistance with smart tools that allows us to work and explore differently and more efficiently (Unity, 2005; Photoshop, 1990). However, tools that assist us in our tasks are not the only key factor; the collaboration between humans has remained virtually unchanged as an essential way to move forward and to develop new experiences. Not only to achieve greater objectives as a group but also to develop as individuals. While the current tools to support humans' work and creative output are valuable and helpful in many ways; this raises an essential question that holistically motivates and drives this thesis:


% Rather than having tools that facilitate our work by doing impossible things for us, e.g., move 500 km in a day, or assist us into doing repetitive work, they can now provide advance assistance with smart tools that allows us to work and explore differently and more efficiently (Unity, 2005; Photoshop, 1990).

\begin{retQuestion}{}
    How can we create tools that no longer behave just as aid to support our work, but can collaborate with us, to some extent, in the same way as human collaboration function? 
\end{retQuestion}

This question is not new and has been approached by different disciplines, under the~\acrfull{mi} paradigm.~\acrshort{mi} refers to the collaboration between \emph{human} and \emph{computer} where both have some proactive initiative to solve some task.~\acrshort{mi} can be seen as a multi-agent collaboration scenario, where the interaction should be flexible, allowing for a continuous negotiation of initiative and leverage on each other's strengths to solve a task~\cite{Allen99-MIinteraction}. \emph{Initiative} was described by Novick and Sulton as a multi-factor model that combines: choosing the task, choosing the agent in control and how the interaction is established, and choosing the expected outcome from the collaboration~\cite{novick97-mixedInit}. 

Moreover, Horvitz discussed such a question in terms of Intelligent User Interfaces~\cite{Birnbaum97-IUI}, describing mixed-initiative systems and interfaces as a more natural collaboration in a user interface that emerges from intertwining human control and manipulation, and automation by AI~\cite{Horvitz99-uncertainty}. Horvitz presented several principles of mixed-initiative interaction and its challenges, many of which still exist~\cite{Horvitz99-mixedInit}, mainly describing this interaction as conversation systems between AI and humans~\cite{horvitz1999-conversationModel}. Moreover, Yannakakis et al. introduced the~\acrfull{micc} paradigm for the co-creation of creative content, where both AI and humans alternate in the initiative to co-design and solve tasks~\cite{yannakakis2014micc}. Their work describes key findings and discussions for how MI-CC does not only help human designers solve tasks, but also fosters their creativity through an interactive feedback loop and lateral thinking~\cite{Liapis2016-CanComputersFosterCreativity,Liapis2014-gameCreativity,Alvarez2018}. 

Paramount is the role of the computer agent in this interaction, as it would help establish the boundaries of the interaction, what is expected, and how creativity could be fostered. Lubart analyzed this interaction and examined the different ways computers could be involved in creative work to promote creativity. In his work, he proposed four roles: \emph{computer as nanny}: management of creative work; \emph{computer as pen-pal}: communication service between collaborators; \emph{computer as coach}: Using creative enhancement techniques; and \emph{computer as colleague}: partnership between computer and humans~\cite{LUBART2005-computerPartners}. Recently, this was explored by Guzdial et al., where designers perceived the AI collaborator with more or less value depending on their desired role for the AI, varying between: \emph{friend}, \emph{collaborator}, \emph{student}, or \emph{manager}~\cite{Guzdial2019-AISystemDesign-Creators}.

There has been a significant effort to analyze the possibilities of such a collaboration, resulting in multiple Ph.D. theses exploring and focusing on different aspects of~\acrshort{mi}~\cite{SmithPhD,LiapisPhD,ComptonPhD,GuzdialPhD,MachadoPhD}. Thus, there is a growing interest in how to establish mixed-initiative interactions and new ways of humans and AI to collaborate. Earlier work focused on tools that supported the work of humans, combining the strengths of both. Later on, research demonstrated that such interaction could enable the completion of tasks not previously able to be done by either alone, and enable creative work to be stimulated. 


\subsection{Computational Creativity and Games}

Games, either digital or tabletop, are created through a complex creative process that couple together many different creative facets in different ways. Games contain a large amount of creative content carefully combined and intertwined to craft specific experiences, with the addition of rules that dictate how a player is to interact with it. In contrast with other creative content, games are multifaceted, content-intensive, and should be interacted, experienced, and enjoyed by others, which also creates a complex subjective task~\cite{Liapis2014-gameCreativity}. Usually, games are developed by more than a person (although many exceptions exist~\cite{minecraft,undertale,stardewvalley}), reaching to hundreds and thousands of developers, with each developer specialized in different areas such as gameplay, AI, animation, concept art, etc. Each creates a specific part of the game and the content through collaboration and following a road map~\cite[Chapter~14]{fullerton2004-gamedesign}. However, no matter the team's size and talent, the fact remains that developing games is a hard challenge~\cite{Blow2004-gamesHard}. As technology advances, the requirements increase substantially for any game facet, coupled with the users' increase demand, the higher competitiveness in the market, and the launch of many more platforms~\cite{Washburn2016-gamesPostmorten}.

Moreover,~\acrfull{cc} is one of the grand challenges of~\acrshort{ai}, where the quest is to study, develop, and build computational systems that demonstrate creative behaviors and can create multiple types of artifacts such as games or stories~\cite{Colton2012-CC}. Within the various content that can be created, games offer a unique property that distances them from other creative outputs, making them more interesting to be analyzed and experimented on than others~\cite{Liapis2014-gameCreativity}. They offer a set of intertwined facets that represent any game: audio, visuals, narrative, levels, rules, and gameplay~\cite{Liapis2019-OrchestratingGames}, whereas other creative content focuses almost exclusively on a single-aspect, e.g., music or dance. However, single-aspect creative content has its own set of challenges. For instance, creating a system that creates music could mean creating each separated instrument, music sheet, arrangements, etc~\cite{HooverPhD}. Furthermore, Games must be interacted with and enjoyed by others than the developers, and the fact that these facets must naturally fit each other poses them as a very exciting application to be researched and developed within the~\acrshort{cc} field.

\acrfull{pcg} is a field within computational intelligence in games, that focuses on the use of algorithms to create game content~\cite{Yannakakis2018}.~\acrshort{pcg} algorithms have been used to aid in the creation of a plethora of games such as No Man Sky~\cite{nomansky}, Spelunky~\cite{spelunky}, or Minecraft~\cite{minecraft}, to the extent that~\acrshort{pcg} and~\acrshort{ai} have enabled experiences and interactions that were not possible before~\cite{aidungeon,rogue,elite}. Moreover, as one of the properties of~\acrshort{pcg} is to increase replayability by creating an abundance of well-made content~\cite{shaker_procedural_2016}, games are not the only beneficiaries of~\acrshort{pcg} methods. For instance, they have the opportunity to be used to increase the generality of~\acrfull{ml} approaches~\cite{Risi2020-pcgGeneralityML}, or a step towards open-endedness~\acrfull{ec}~\cite{clune2019-aigas}. 

Furthermore, several computational designers and systems have been developed to create complete games such as Angelina~\cite{Cook2016-Angelina1}, Ludi~\cite{Browne2010-ludii}, a system to create card games~\cite{font2013-GenCardGames}, or a system that uses data from Wikipedia to create mystery games~\cite{barros2018-DATAeinstein}. Automated game design, as developed in those systems, show interesting and important advancements towards~\acrshort{cc} systems. However, not including the human designer creates constraints, challenges, and limitations in these systems, such as modeling fun, enjoyment, and interaction. Another interesting and promising path to explore is the~\acrlong{micc} paradigm within~\acrshort{pcg}, combining both~\acrshort{ai} and humans to co-create the game content, which is the focus of this thesis.

\subsection{Problem Statement} \label{sec:problemst}

Interactive systems can help humans complete creative tasks that are tedious or require substantial human workload. Such creative tasks could be as simple as drafting a grammar corrected text (e.g., MS Word), to more complex scenarios such as creating visual artifacts (e.g., Adobe Photoshop) or designing entire video games (e.g., Unity Engine).~\acrlong{micc} is a paradigm within~\acrshort{pcg}, where both human and AI proactively collaborate to co-create and co-design games or creative content. Through~\acrshort{micc}, the potential of interactive tools increases substantially.~\acrshort{micc} enables a new way of tackling creative tasks engaging in-depth humans and AI rather than just helping humans to complete tasks and reduce their workload. Furthermore, enabling a mutual feedback loop could also foster both participants' creativity, create adaptive experiences for the users, or focus on achieving tasks with better and interesting results in a hybrid format. Multiple approaches have been proposed as alternatives for creating systems that model the interaction between AI and humans to create game content, and that use different techniques to study such interactions and its implications~\cite{Alvarez2020-ICMAPE,smith_tanagra:_2011,Liapis2013-sentientsketchbook,charity2020baba}. 

Nevertheless, this collaborative approach divided into human control and automation~\cite{Horvitz99-mixedInit} with multiple initiatives~\cite{Allen99-MIinteraction,novick97-mixedInit}, and whereas it could foster humans' creativity~\cite{yannakakis2014micc}, raises an \emph{initiative} challenge for either agent: Which agent should have the initiative at different stages of the development and over the goal? The question reflects the diffuseness of the challenge and situation, as many factors need to be considered before appropriately indicating this. At the very least, some could say that depending on the task to be performed and the expertise of both, either would clearly be the one taking the development initiative. Whereas others would position the human as the one always in control. Yet, even with a clear answer, what happens in creative tasks to the expressivity of one of the sides due to the other taking the initiative? 

Moreover, in design and creative tasks, the designer usually has intentions in what they are creating and goals that they want to achieve with their design. Thus, to enable deeper~\acrshort{mi} levels to co-create content, some control mechanisms with a varying degree of control over the algorithms might be necessary for the designer. Through this, the designer could direct or constraint the generated content by the computational designer and oversee that it is within their intentions and goals. In this case, each agent's control and expressive properties are at the expense of the other agents, as it constrains the space of possibilities~\cite{Baldwin2017}. This is especially relevant when the aim is a creative work such as games, where the creative expression needs to be fostered~\cite{Alvarez2018}. Yet, it becomes particularly challenging when using mixed-initiative methods, where smart approaches need to be in place for a natural conversation and successful collaboration. The more control is given, the more constraint it exists, but is this a problem? Is it inevitable? Boden explains it conspicuously ``...  We [humans] seek the imposed constraints [...], and try to overcome them by changing the rules.~\cite{boden2004-creative}". Constraints limit the space, and as a consequence, they are overcome by encountering creative solutions.

Furthermore, for this interaction to be fully fleshed, the human needs to understand the AI's behavior through interpretable and explainable models and systems, and the AI needs to recognize and interpret the intentions of the humans seamlessly as they create their content. The former is the focus of~\emph{Interpretable} and \emph{Explainable AI}, which seeks to create or adapt models and systems for a better workflow between humans and AI, where humans could understand the AI's decision process to enable trust relationships and reach deeper interactions~\cite{Zhu2018-XAIDesignersMICC,Doshi-Velez2018,adadi2018peeking}. The latter would mean that the AI could adapt its behavior and functionality to the needs, expertise, and workflow of individual designers or a specific group of designers. To do so, the AI must analyze several design processes, such as the designer's preferences, styles, and goals, which holistically is called \emph{Designer Modeling}~\cite{Liapis2013-designerModel,Liapis2014-designerModelImpl}. How to create these models and use them to develop adapted experiences is a complex challenge, and understanding the implications of its usability in the control-expressive properties, as well as other consequences, is not trivial.

In this thesis, the focus is on exploring multiple approaches for the collaboration between AI and humans to co-create game content in an~\acrlong{micc} paradigm. The goal is to develop systems, techniques, and algorithms representing a [creative] computational designer to tackle the different tasks in game design and development. Tasks that could not only be assisted by AI but rather AI could be a \emph{colleague} in the creative process. To tackle these shared tasks, a mutual feedback loop could be established, whereby AI and humans could inspire each other to explore unknown areas in the design landscape and reach better and more creative solutions.

To explore this, the main body of work presented in this dissertation is applied and evaluated through the~\acrfull{edd}, a~\acrlong{micc} system, where designers can create levels for rogue-like and adventure type of game such as Zelda~\cite{tloz} or The Binding of Isaac~\cite{bindingISAAC}. In~\acrshort{edd}, the human designer can quickly create interconnected rooms forming a dungeon to be experienced by players. Meanwhile, the computational designer collaborates by providing suggestions using different algorithms and following multiple heuristics. The human designer can interact in several ways with the computational designer so that this adapts its output to whatever goal the human designer has, while still providing a diverse amount of alternatives and different experiences to the human designer.

\subsection{Research Questions} \label{sec:RQS}

As motivated thus far, this thesis focuses on exploring different approaches for procedurally generating content for games or other creative content, specifically through the~\acrshort{micc} paradigm, where a human designer collaborates with an underlying AI to create creative content. Exploring the role of \emph{computers as colleagues} as defined by Lubart~\cite{LUBART2005-computerPartners}, this thesis delves into the use of~\acrshort{micc} tools and the multiple properties that emerge from the dynamic interaction between AI and Humans. The aim is to understand how we can enable a rich, fruitful, and better feedback loop in these types of tools using and developing novel AI techniques in the field of~\acrlong{ec} and~\acrlong{ml} to improve the interaction and create adapted experiences. The thesis also analyzes and studies the requirements, challenges, and benefits of enabling in-depth collaboration, tailored experiences, the properties that emerge (some seemingly competing properties), and their dynamics. Therefore, this thesis aims at addressing, discussing, and exploring the following research questions: 

\begin{retQuestion}{}
   \textbf{RQ1.} How can we use and integrate quality-diversity algorithms into a mixed-initiative approach to help designers produce high-quality content and foster their creativity while allowing them to control, to a certain extent, the generated content?
\end{retQuestion}

~\acrfull{qd} algorithms are a relatively new family of algorithms, specifically aimed at tasks and environments that require the strengths of convergence and divergence search~\cite{Pugh2016}. Leveraging on~\acrshort{qd} algorithms to search for a surfeit of heterogeneous content while not losing sight of the content's quality could enable~\acrshort{micc} systems to explore a big area of the generative space producing more diverse and high-quality solutions. Through this, the system could propose a higher range of diverse solutions to the user, aiming at fostering the creativity of the human designer~\cite{Liapis2016-CanComputersFosterCreativity}. Thus, how to integrate~\acrshort{qd} algorithms in~\acrshort{micc} systems that need to take into account the human work to provide valuable input is a promising open research area and one that this thesis explores. However, it is paramount to understand how to effectively use~\acrshort{qd} algorithms in these systems to fully leverage their expressive power while providing some control to human designers. 

\begin{retQuestion}{}
%   \textbf{RQ2.} How can we use gameplay, player, and designer data to understand better players' and designers' actions and behaviors to enhance their experiences?
   \textbf{RQ2.} How can we use player and designer data to understand better their behaviors and procedures to enhance and adapt~\acrlong{micc} systems?
   
%   understand better players' and designers' actions and behaviors to enhance their experiences?,
   
   
%   \textbf{RQ2.} What type of data is representative of player and designer 
   
%   How can we use gameplay, player, and designer data to understand better players' and designers' actions and behaviors to enhance their experiences?
\end{retQuestion}

Games and creative contexts are spaces where both players and designers can express themselves differently, producing data on how they both interact. Research areas such as Experience-driven PCG~\cite{Yannakakis2011-experiencedrivenPCG}, player modeling~\cite{Pedersen2010-modelingPlayerExp,Holmgard2019-proceduralPersonas} or designer modeling~\cite{Liapis2013-designerModel}, explore the use of such data to understand particular users~\cite{Liapis2013-designerModel,Drachen2009-playerModellingTombRaider,Melhart2019-ModellingMotivation} and to improve and enhance the experiences of players and designer. Especially focusing on enabling adaptive experiences~\cite{hastings_evolving_2009} and more accurate heuristics~\cite{Marinho2015-empiricalEvaluation,canossa2015-towardspcgEvaluation,summerville2017-understandingMario}. However, how to use the data (and even what to collect) is still an open research area, especially when applied to adaptive experiences for~\acrshort{micc} tools with only a few relevant examples~\cite{Liapis2014-designerModelImpl,Liapis2012-adaptiveVisual}. Furthermore, the importance of enhancing the experience of~\acrshort{micc} tools' users lies in the search for deeper understanding and collaboration between humans and~\acrshort{ai}, which could enable a better experience for both.

\begin{retQuestion}{}
   \textbf{RQ3.} How can we model different designers' procedures and use them as surrogate models to anticipate the designers' actions, produce content that better fits their requirements, and enhance the dynamic workflow of mixed-initiative tools? 
   
    \begin{retQuestion}{}
        \textbf{RQ3.1} What trade-offs arise from modeling and using designer's procedures to steer the generation of content towards personalized content?
    \end{retQuestion}
   
   \begin{retQuestion}{}
        \textbf{RQ3.2} What constraints are created over the generative process when using designer models?
    \end{retQuestion}
   
\end{retQuestion}

The advantage of having the human and AI collaborating is analogous to humans collaborating, each one with its own set of strengths and weaknesses to reach greater objectives and develop each other. However, mixed-initiative collaboration requires both human and AI to understand each other and the goals that the human aim to reach~\cite{Horvitz99-mixedInit,novick97-mixedInit}. This creates a particular problem where the AI needs to identify certain processes and characteristics of the human. When employing~\acrshort{micc} to co-create creative artifacts, which in this thesis focuses on games, this translates to design processes, style, preferences, intentions, and goals. This thesis aims to explore how to model different designer procedures such as preference or style, using several~\acrlong{ml} methods, and how to best use these as surrogate models to produce better content and enhance designers' experience using~\acrshort{micc} systems.

Furthermore, RQ2 and RQ3 drive the research on how to gather and use different types of data, i.e., gameplay and designer data. Whereas, designer modeling could be used in the~\acrshort{micc} feedback loop to create adaptive experiences. Through RQ3.1 and RQ3.2, the thesis focuses on exploring the trade-off of using designer modeling. Specifically, the interest lies in the benefits that designer modeling creates for the algorithms and designers, and the overall experience that the designer wants to create, i.e., the game. Moreover, the constraints that emerge from using these models as surrogate models to steer the content generation are not trivial to address and are essential to study to understand and analyze their extent. Using these models will inevitably create constraints over the generation process as we aim to adapt the experience to each designer or group of designers. Therefore, RQ3.2 specifically aims at understanding: what are these constraints? What is constrained? And whether these constraints are positive or negative? 

\subsection{Pronouns, Style, and Clarification}

Throughout the thesis, the pronoun ``we'' will be used in favor of ``I'', since the work and research achieved and presented in this thesis would not have been possible without my co-authors' collaboration. 

When referring to a player or designer, this thesis chooses the pronouns ``they'' and ``their'' to respect a gender-inclusive language. Moreover, throughout the thesis, it is referred to as user and designer alike, as a designer is the target user group within the possible user base of the systems and tools developed in this thesis. The player is referred to as the end-user: the user who could experience the creations in the~\acrlong{micc} system.

% When referring to end-user, we refer to the user that could and would experience and play the creative designers create collaboratively in the~\acrlong{micc} system, i.e., the player. 

When discussing the participants in a mixed-initiative system, i.e., AI and Human (or group of both), this thesis uses the word ``agent'' when needed to refer to either, unless specifically discussing one in particular, as mixed-initiative systems have been described as multi-agent systems~\cite{Allen99-MIinteraction}.

Finally, this thesis will refer to as ``computational designer'' to the overall AI system that interacts and collaborates with the human designer to create content through the~\acrshort{micc} system.

% Finally, in the~\acrshort{micc} system presented, used, and developed in this thesis  there are several  AI systems fueling the AI that collaborates with designers in the~\acrshort{micc} system presented, used, and developed throughout the multiple publications. Thus, when not explicitly discussing individual algorithms, this thesis will refer to the overall AI system that collaborates with the designer as artificial designer.

% characteristics, properties
% The AI developed throughout the multiple publications and that 


% Finally, the content generated and suggested by the artificial designer is based on~\acrshort{ea}s; thus, when referring to this content it is discussed as individuals.